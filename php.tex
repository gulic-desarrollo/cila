%Autor: Carlos de la Cruz (frodo@fmat.ull.es)

\chapter{PHP}

\section{Introducción}

PHP es  un lenguaje diseñado  para la generación dinámica  de páginas.
web Los  ficheros de PHP  se almacenan en  el servidor y  se ejecutan.
cuando un  usuario introduce en su  navegador la direccion de  una de.
estas páginas. Estas, a su vez, contienen código a partir del cual se.
generan páginas estáticas en formato HTML                            .

La utilidad de  esto es que podemos crear  automáticamente páginas web
que se actualicen solas, basándose en una fuente de datos.

Por  ejemplo, si  quisiéramos que  cada vez  que el  usuario accede  a
nuestra página, en esta apareciera la fecha y la hora, sería imposible
conseguirlo a partir de ficheros estáticos ``html''.

Por  tanto,  la  solución  sería  tener  en  el  servidor  un  fichero
``hora.php'',   mediante  el   cual,  cuando   alguien  accediera   a:
http://miservidor.com/hora.php, tuviera la hora del instante en que se
ejecutó el código php.

Cada vez que alguien le diera a ``recargar'' en el navegador, vería la
hora actual, sin embargo, como podremos  comprobar, si se hace clic en
el botón  ``ver código  fuente'' del  navegador, sólo  veríamos código
html conteniendo por algún lugar la hora en que se generó la página.

PHP tiene  todas las  ventajas que  tienen otros  lenguajes utilizados
para generación dinámica de contenido web como PERL y ASP, y otras que
ninguno de estos  permiten hacer de una forma flexible  y cómoda, como
por ejemplo la creación de ficheros de Macromedia Flash, imágenes (con
el módulo GDLib), informes en formato PDF, etc.

\section{Primeros Pasos}

Todo script php está delimitado por los identificadores \verb+<?php+ y
\verb+?>+. Dentro de estos símbolos, va  escrito todo el código php de
este.

La primera función  que veremos de PHP será ''echo'',  que nos permite
imprimir  texto o  tags HTML  en nuestra  página. La  función echo  va
siempre acompañada de unos paréntesis  y comillas(simples o dobles) en
caso  de que  vayamos a  imprimir una  o varias  cadenas estáticas  de
texto, y puede  ir o no acompañada  de estos en caso de  que vayamos a
imprimir sólo el contenido de una o varias variables.


Como con el resto de los  lenguajes, vamos a empezar viendo como sería
un "hola mundo" en PHP.

\begin{verbatim}
<?php

echo("<center><font size=7>Hola, Mundo!</font></center>");

?>
\end{verbatim}

Este sencillo  script nos permite  ver como diferenciar el  código PHP
del código HTML,  pues si introdujéramos código HTML  suelto dentro de
los delimitadores \verb+<?php+  y \verb+?>+, el módulo  PHP de nuestro
servidor web provocaría un error de interpretación.

PHP es  un lenguaje  muy flexible.  Nos permite  imprimir el  valor de
variables dentro de  cadenas de texto sin tener que  cerrar y volver a
abrir comillas, de la siguiente forma:

\begin{verbatim}
<?php

$limones=2;
$naranjas=3;

echo("Tenemos $limones limones y $naranjas naranjas.");
?>

\end{verbatim}

Fíjate  que  no es  lo  mismo  escribir: {\tt  echo("Tenemos  $limones
limones  y $naranjas  naranjas."); }  que {\tt  echo('Tenemos $limones
limones y $naranjas naranjas."); }.

La primera línea,  para $limones=2 y $naranjas=3  imprimirá "Tenemos 2
limones y  3 naranjas",  mientras que  la segunda  imprimirá ''Tenemos
$limones limones  y $naranjas naranjas".  Es decir, si  utilizamos las
dobles  comillas,  tendremos  que  los  valores  de  las  variable  se
despliegan dentro de estas. Si  utilizamos, por el contrario, comillas
simples, lo que haya entre ellas se imprimirá tal cual.

Es importante  saber que cuando  uno necesita utilizar comillas  en el
HTML resultante  de nuestro  script PHP,  conviene que  ''escape'' las
comillas, o de  lo contrario el interpretador mostrará  una página con
un error de compilación.

por ejemplo, si queremos mostrar el tag html:

{\tt   <a    href="http://www.fmat.ull.es/~frodo">},   tendremos   que
hacerlo   de    la   siguiente    forma   en   php:    {\tt   echo("<a
href=\"http://www.fmat.ull.es/~frodo\">");}.  (Fíjate  en  las  barras
invertidas que se anteponen a cada  una de las comillas que van dentro
del par de comillas de la función "echo".

Para la  inclusión de  código HTML previamente  escrito dentro  de una
página que se genera dinámicamente  (por ejemplo, supón que tienes una
cabecera, un pie  de página predefinidos para tu sitio  web, y quieres
que varíe sólo el cuerpo de la página, para darle un diseño uniforme),
utilizamos el comando {\tt include}.

Por ejemplo:

\begin{verbatim}
<?php

include "include/cabecera.php";
echo("Hola! este es el cuerpo de la página");
include"include/pie.php;

?>

\end{verbatim}


\section{Estructuras de control}

Para  el uso  de las  estructuras de  control PHP  es muy  parecido al
lenguaje ''C'', tanto por el uso  de llaves para delimitar los bloques
de código como por usar la misma sintaxis. Distinguiremos dos tipos de
estructuras de control:

\subsection{Condicionales: If, Switch}

If: Este condicional nos permite ejecutar  un bloque de código en caso
de que se  de una condición determinada. Por decirlo  de alguna forma,
la sintaxis podría ser:

\begin{verbatim}

If (condicion){
	linea1;
	linea2;
	linea_n;
}

\end{verbatim}

O, si es más corto lo que queremos hacer dada esa condición, podría ser:

\begin{verbatim}
	
\end{verbatim}



Veamos un ejemplo:

\begin{verbatim}
<?php

$a = 2;
$b = 10;

if ($a<$b) {
	echo("Hey!, parece que \$a vale $a, que es mayor que \$b, que vale $b");
	}
	
?>
\end{verbatim}



\subsection{Repetitivas: While,For}

\section{Como estructurar nuestro código en php}

Las funciones. 

\section{Manejo de formularios}


\section{Utilización  de bases  de  datos} Aunque  PHP soporta  varias
clases de servidores  de base de datos,  debido a que cada  uno usa un
interfaz distinto,  resulta bastante  incómodo tener que  aprender las
funciones  de PHP  para cada  uno  de ellos.  Es por  ello que  suelen
utilizarse librerías adicionales (no incluidas con PHP, pero sí con la
mayoría de las distribuciones Linux), como la que utilizaremos en este
texto/curso: AdoDB.

AdoDB nos  permite cambiar  de sistema de  bases de  datos simplemente
cambiando un parámetro al realizar la conexión a la Base de datos (BDD
de ahora en adelante) , de  forma que una aplicación PHP diseñada para
correr  con  Oracle,  puede  ser  portada  para  funcionar  en  MySql,
Informix, etc. con tan sólo cambiar una línea de código.


\section{Creación de imágenes dinámicamente}

\section {Bibliografía recomendada}


