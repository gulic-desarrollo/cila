%Autor: AgarFu, _ese & wizard--
%AgarFu: 10
%_ese: ¿?
%wizard--: 10

\chapter{Instalación de GNU/Linux}
\label{instalacion.tex}

\section{Introducción}

En este capítulo repasaremos de forma  genérica, para que nos sirva de
guía, una instalación de cualquier  distribución de Linux. Tenemos que
decir que lo que nos planteamos  conseguir no es tarea sencilla porque
todas  las  distribuciones  tienen  sus  pequeñas  diferencias  aunque
generalmente todas pasan en sus procesos de instalación por unas fases
que son comunes que aquí vamos a detallar. El orden en el que aparecen
las  diferentes  etapas  comentadas  en este  capítulo  no  tiene  por
que  ser de  forma  estricta el orden a  seguir  por todas  las
distribuciones, pero  es uno  de los  más usuales  en los  procesos de
instalación.

{\bf ¡Atención!} Antes de comenzar  con la instalación te recomendamos
que leas la sección referente a las particiones que se puede encontrar
en este mismo capítulo.

\section{Arrancando}

Normalmente cuando nos disponemos a instalar una distribución Linux lo
hacemos desde CD-ROM. Si nuestra placa base nos permite arrancar desde
CD-ROM no  tendremos problema  ya que simplemente  colocamos el  CD de
nuestra distribución  preferida en el  CD-ROM, le indicamos a  la BIOS
que lo primero  que intente arrancar sea el CD-ROM  y listo, tendremos
nuestro sistema de instalación de nuestra distribución esperando a que
le demos la orden de continuar.

\begin{figura}{bootsplash}{0.3}
\caption{Cargador de arranque de Mandrake Linux}
\end{figura}

Si nuestra placa no soporta el  arranque desde el CD siempre se pueden
hacer los  disketes para arrancar  desde la disketera. Todos  los PC's
que disponen de disketera pueden arrancar desde ella. Para arrancar el
sistema de instalación desde discos de 3,5" tendremos que buscar en el
contenido del CD de instalación de la distribución las imágenes de los
discos de  arranque. Normalmente están  en un directorio que  se llama
"images",  aunque  esto no es  una  norma fija.  Cuando  los
encontremos tendremos que copiar las  imágenes a los disketes. Esto se
hace  con una  herramienta  que normalmente  traen  también todos  los
CDs  de  las  distribuciones  que se  llama  rawrite.  Seguiremos  las
indicaciones  que  nos  dé  el  rawrite para  hacer  los  disketes  de
instalación y guardaremos con celo el trabajo realizado hasta ahora.

Una  vez  que  tenemos  esos discos en la mano tendremos que decirle a
nuestra  BIOS  que  intente  arrancar  en primer desde la disketera, 
reiniciamos y colocamos el diskete de instalación en  la  disketera.
Tras unos momentos de incertidumbre tendremos finalmente el sistema de
instalación de nuestra distribución listo para continuar.

\section{Idioma y teclado}

Normalmente después  de la bienvenida  tendremos que elegir  el idioma
para la instalación y la distribución del teclado que usaremos.

El idioma es a tu gusto, elige aquél  con el que más cómodo estés y la
distribución de teclado  se recomienda que uses la  distribución de tu
teclado, ya  que si  no lo  haces podrás  encontrarte que  pulsando la
tecla  \verb.ñ.  aparezca una  \verb.p.  o  algún otro  carácter.  Los
teclados españoles son del tipo qwerty-ES.

\begin{figura}{mandrake_idioma}{0.5}
\caption{Selección del idioma}
\end{figura}

\section{Particiones}

Hemos llegado  a un punto  delicado de la instalación,  normalmente es
peligroso porque  podemos dañar la información  almacenada en nuestros
discos duros.

Lo primero que nos planteamos es si  vamos a tener un ordenador con un
único  sistema operativo,  en nuestro  caso  Linux o  si serán  varios
sistemas operativos,  cuyo caso  más generalizado  es Linux  y Windows
conviviendo en el mismo ordenador.


\subsection{El caso más general: convivamos con Windows}

Comenzamos  por  el caso  más  general,  es  decir, los  dos  sistemas
operativos juntos. Suponemos que ya  está instalado Windows y que sólo
tenemos un disco  duro. En ese caso tendremos que  hacer sitio para la
instalación de  Linux en el  disco duro  que disponemos. Este  paso es
delicado, pero  si se  siguen las instrucciones aquí detalladas  no tendremos
problemas. Los  pasos a  seguir para  no perder  la información  de la
partición de nuestro Windows son los siguientes:

\begin{enumerate}
\item Desfragmentar la partición de Windows.

\item  Decidir cuánto  espacio vamos  a dejar  para la  instalación de
Linux.  Esta  decisión  al  final  corre  a  cuenta  del  usuario.  Se
recomienda que para una instalación completa de las distribuciones, se
reserve alrededor de 2 Gb de espacio en disco.

\item Reducir  el tamaño  de la  partición de  Windows. Para  llevar a
cabo  esta tarea  se  tienen varias  posibilidades. Podemos  encontrar
software\footnote{Uno de los paquetes de  software más usados en estos
menesteres  es  el  \emph{PartitionMagic} de  \emph{PowerQuest}}  para
Windows, que facilita la labor  permitiendo redimensionar el tamaño de
la partición  de manera  gráfica. Ésta es  la opción  que recomendamos
para los usuarios con poca experiencia en este tipo de prácticas. Para
usuarios  con  algo  más  de conocimientos  tenemos  utilidades  menos
amigables pero más fáciles de conseguir  como son el fips sobre msdos,
el fdisk, el parted de Linux  y el diskdruid. Tened cuidado con estas
dos  herramientas  porque, repetimos,  pueden  dañar  el contenido  de
vuestros discos duros.

\item Establecer   las  particiones   de   Linux. Se necesitarán   dos
particiones  en   lugar  de  una   como  se  piensa   inicialmente.  A
continuación se detallará este paso, común a la convivencia y a la vida en soledad.

\end{enumerate}

\begin{figura}{redhat_particiones}{0.8}
\caption{Preparación de las particiones}
\end{figura}


\subsection{La vida en solitario}

Trataremos el  segundo supuesto: en  nuestro disco duro sólo  habrá un
sistema operativo, Linux. Tendremos que hacer dos particiones, en este
punto hay  mucho que explicar.  Empezaremos por comentar los  tipos de
particiones que  normalmente se pueden  encontrar en un  PC doméstico.
Hemos de  tener claro lo  que son particiones primarias,  extendidas y
unidades lógicas.


\subsection{Tipos de particiones}

\begin{description}

\item[Particiones primarias:] es el tipo básico de partición. Podremos
disponer  como  máximo de  cuatro  particiones  primarias. Si  hacemos
uso  de todas  las  particiones  primarias ya  no  podremos crear  más
particiones de ningún otro  tipo\footnote{Como observación diremos que
en Windows  tendremos problemas a la  hora de trabajar con  más de una
partición primaria en FAT16 o FAT32}.

\item[Particiones  extendidas:]  en  caso  de  querer  más  de  cuatro
particiones  deberemos usar  unidades lógicas.  Para crear  una unidad
lógica tendremos  que eliminar  al menos  una partición  primaria para
crear  una  partición extendida.  Desde  este  momento podremos  crear
tantas  unidades  lógicas  como  deseemos  dentro  de  esta  partición
extendida.

\item[Unidades lógicas:] este tipo de particiones son las que se crean
a  partir de  la partición  extendida, nunca  podremos crear  unidades
lógicas dentro de una partición primaria.

\end{description}

Una vez  aclarados los tipos  de particiones que  podemos encontrarnos
vamos a determinar  la manera que tiene Linux de  asignarles un nombre
de dispositivo. En Linux todos los dispositivos son ficheros, desde la
pantalla  ({\tt  /dev/stdout})  hasta el  teclado  ({\tt  /dev/stdin})
pasando por los discos duros, CD-ROM, etc\ldots.

\subsection{Nomenclatura de particiones}

En primer  lugar hemos de decir  que los PC's domésticos  suelen tener
dos controladoras de disco  duro (IDE1 e IDE2) y en  cada una de estas
controladoras podremos conectar  dos dispositivos físicos (dispositivo
maestro  y  dispositivo  esclavo).  A  estas  controladoras  se  puede
conectar todo lo que tenga una  interfaz hardware IDE, esto es, discos
duros IDE,  CD-ROM's IDE, DVD's IDE,  unidades ZIP y DAT,  y una larga
lista de dispositivos.

La manera de nombrar los ficheros correspondientes a cada uno de estos
elementos de hardware es la siguiente:

\begin{center}
Controladora IDE1 $\left\{
\begin{tabular}{lcr}
Maestro & $\Rightarrow$ & {\tt /dev/hda} \\
Esclavo & $\Rightarrow$ & {\tt /dev/hdb} \\
\end{tabular}
\right.$
\end{center}

\begin{center}
Controladora IDE2 $\left\{
\begin{tabular}{lcr}
Maestro & $\Rightarrow$ & {\tt /dev/hdc} \\
Esclavo & $\Rightarrow$ & {\tt /dev/hdd} \\
\end{tabular}
\right.$
\end{center}

Por si  quedaran más  dudas, como casi  todos estamos  acostumbrados a
trabajar en Windows, si cada  disco forma una única partición, podemos
decir que {\tt /dev/hda} es el  equivalente a {\bf C:}, {\tt /dev/hdb}
es el equivalente a {\bf D:} y así sucesivamente\dots

Sigamos con  la nomenclatura de  las particiones utilizada  por Linux.
Las particiones  primarias y  las extendidas  se nombran  añadiendo al
nombre del  dispositivo, ya detallado,  un número que indica  el orden
lógico de la partición. La primera partición de uno de los discos será
{\tt  /dev/hd{\bf x}1},  la  segunda {\tt  /dev/hd{\bf  x}2} donde  la
{\tt\bf  x}  es  lo que  va  a  determinar  a  que disco  nos  estamos
refiriendo según la tabla anterior.

Las unidades lógicas también se nombran  utilizando el orden en el que
se definen  pero teniendo  en cuenta que  los cuatro  primeros números
están reservados  para las  particiones primarias,  por lo  tanto, las
unidades lógicas empezarán a numerarse  a partir del cinco.

Imaginemos por un momento que tenemos un
disco  duro, con  una partición  extendida,  tres primarias,  y en  la
partición extendida tenemos tres  unidades lógicas. Haremos referencia
a cada una de las particiones  de esta manera (Suponemos que el disco
está conectado como maestro a la primera controladora IDE).

\begin{center}
\begin{tabular}{rcl}
partición extendida & $\Rightarrow$ & {\tt /dev/hda1} \\
unidad lógica 1 & $\Rightarrow$ & {\tt /dev/hda5} \\
unidad lógica 2 & $\Rightarrow$ & {\tt /dev/hda6} \\
unidad lógica 3 & $\Rightarrow$ & {\tt /dev/hda7} \\
partición primaria 1 & $\Rightarrow$ & {\tt /dev/hda2} \\
partición primaria 2 & $\Rightarrow$ & {\tt /dev/hda3} \\
partición primaria 3 & $\Rightarrow$ & {\tt /dev/hda4} \\
\end{tabular}\\
\end{center}

Bien, sabemos  que esto es  bastante denso  al principio, cuando  no se
tiene práctica,  pero con el  uso nos  daremos cuenta que  tiene mucha
lógica y que aprendiéndonos un par de reglas todo va sobre ruedas.


\subsection{Esas dos particiones}

En  este punto  ya es  hora de  introducir el  significado de  las dos
particiones necesarias en un sistema Linux.  La primera de ellas es la
partición de swap. Esta partición lo utiliza el sistema operativo como
memoria de  intercambio. En el momento  en el que se  llena la memoria
principal  de nuestro  ordenador,  el sistema  utiliza  la memoria  de
intercambio  para continuar  ejecutando los  programas sin  problemas.
Esta partición es  de uso exclusivo del sistema, lo  que significa que
no la podremos  utilizar para guardar nuestro datos.  La buena noticia
es que  esta partición es  muy pequeña, normalmente es  suficiente con
dejar para la partición de swap 256 Mb de disco, no obstante puede ser
todo lo grande que se quiera, siempre que haya sitio para el resto del
sistema. El  tipo de sistema de  ficheros que tendremos que  asignar a
esta partición no tiene duda, Linux Swap.

La  segunda partición  es donde  vamos  a colocar  todos los  ficheros
del  nuevo sistema  operativo,  en ella  encontraremos desde  nuestros
directorios personales hasta los programas  de usuario, pasando por el
núcleo del sistema  operativo, por lo tanto,  esta partición tendremos
que dejarle  bastante sitio.  Como ya hemos  recomendado anteriormente
2Gb  serán suficientes  para una  instalación normal  de Linux.  Ahora
entremos en el fantástico mundo de  los sistemas de ficheros, que como
llevamos haciendo hasta el momento también daremos un breve repaso.

\subsection{Tipos de sistemas de ficheros}

Los  sistemas de  ficheros  se pueden  clasificar  de varias  maneras,
nosotros  lo haremos  atendiendo a  la  capacidad de  los sistemas  de
ficheros transaccionales  (journaling). El  journaling es  una técnica
que  implementan  algunos  sistemas  de ficheros  muy  parecida  a  la
utilizada  por las  transacciones en  las bases  de datos.  La técnica
consiste en que todas las operaciones  que se han de realizar sobre el
sistema de ficheros  se guardan en disco antes de  realizarse, y luego
se van llevando a cabo y se van eliminando de la lista a medida que se
concluye  su  ejecución. La  ventaja  que  nos  dan los  sistemas  con
journaling es  que si  nuestro sistema  se cae (por  ejemplo se  va la
luz)  todas las  tareas  que el  disco tendría  que  haber hecho  para
mantener la estructura lógica del  sistema de ficheros está guardada y
cuando restablecemos el  sistema tenemos el disco  en perfecto estado,
eliminando de esta  manera los famosos verificadores  de la estructura
del disco, el más conocido es scandisk.

Comencemos con los sistemas de ficheros que no implementan journaling,
éstos son:

\begin{description}

\item[FAT16:] sistema de fichero nativo  de Windows95 (sin política de
seguridad robusta).

\item[FAT32:] sistema de  ficheros nativo de  Windows98-WindowsMe (sin
política de seguridad robusta).

\item[ext2:] sistema de  ficheros nativo de  Linux 2.x y 4.x  (con una
buena política de seguridad).

\end{description}

Sistemas de ficheros que implementan journaling:

\begin{description}

\item[ext3:] Este sistema  de ficheros es  una evolución del  ext2 que
implementa journaling  además de  la política  de seguridad  basada en
permisos del ext2.

\item[ReiserFs:] Sistema de  ficheros con  journaling que  obtiene muy
buenos  resultados trabajando  con  ficheros de  gran tamaño,  también
cuenta con una muy buena política de seguridad.

\item[NTFS:] Sistema de  ficheros nativo  de WindowsNT,  Windows2000 y
WindowsXP,  tiene un  sistema de  seguridad  basado en  ficheros y  en
listas de control de acceso.  Desgraciadamente este tipo de sistema de
ficheros no es abierto, está sometido a continuos cambios por parte de
su dueño,  Microsoft, y por  lo tanto no se  tiene un buen  soporte en
Linux, de tal modo que sólo está soportado de manera segura la lectura
de este tipo de sistema de fichero.

\item[XFS:] sistema de  ficheros que implementan las  máquinas Silicon
Graphics.

\item[JFS:] sistema de ficheros que implementan los main frame IBM.

\end{description}

Con esta  brevísima descripción  de los tipos  de sistema  de ficheros
podrás elegir tú mismo cuál quieres poner. Nuestra recomendación es la
de  poner ext2  ya que  este  sistema de  ficheros está  en todas  las
distribuciones,  pero  si tu  distribución  te  da la  posibilidad  de
utilizar un sistema de ficheros con  journaling, escoge ext3 ya que en
las comparativas  obtiene mejores resultados en  sistemas de propósito
general que  los otros tipos de  sistema de fichero con  journaling, y
por  supuesto, la  elección entre  journaling  o no  journaling es  la
primera.  No solo  por  tener que  estar esperando  a  que termine  el
proceso de  fsck, sino también  por la  robustez y eficiencia  de este
tipo de sistemas de ficheros.

Vamos  a dar  por último  la  respuesta a  la pregunta  que muchos  se
estarán haciendo: ¿Podré utilizar la  partición de Windows desde Linux
y la de Linux desde Windows?.  Hay varias respuestas bastante claras y
tajantes. Sobre si  se podrán utilizar las particiones  de Linux desde
Windows,  la  respuesta  es  no de  escritura  (existen  problemas  de
licencias en el desarrollo) pero  sí de lectura (mediante una utilidad
llamada  {\em explore2fs}).  La  otra respuesta  depende  del tipo  de
sistema  de ficheros  que tengamos  en Windows.  Para los  sistemas de
ficheros del tipo FAT16 o FAT32 se podrán utilizar sin problemas. Para
los sistemas de ficheros con  NTFS, lo más recomendable es utilizarlos
en modo lectura, ya que el soporte de escritura no está completo debido
a que Microsoft no revela los cambios que introduce en el esquema del sistema
de ficheros, por
tanto,  es muy  peligroso  hacer operaciones  de  escritura ahí  desde
Linux. Esta limitación en ambos  sentidos es producto de la ocultación
de  información  sobre  la  que  se  basan  los  sistemas  cerrados  y
propietarios  y que  impide  que las  aplicaciones  abiertas y  libres
permitan realizarse.

% Para aclarar:  En un  caso (leer  linux desde win)  es porque  no se
% puede programar  un driver  de sistema de  archivo para  windows sin
% haber pagado una licencia (si  microsoft lo desarrollase tendría que
% ser código  gpl o tendría que  hacerlo desde cero). En  el otro caso
% (leer  ntfs  desde  linux),  las especificaciones  de  NTFS  no  son
% abiertas  y  muchas  de  las estructuras  internas  se  conocen  por
% documentos no  oficiales y  por mera prueba  y error.  Eso ralentiza
% mucho el desarrollo.

\section{Instalación del kernel y los módulos}

En este momento ya le hemos dicho al sistema de instalación dónde va a
tener que colocar  todos los ficheros. Ahora tendremos  que comenzar a
tomar decisiones  para indicar qué es  lo que queremos instalar  y qué
hardware tenemos en nuestra máquina.

Lo  primero  que tendremos  que  hacer  es  elegir qué  kernel  poner.
Normalmente  podemos  elegir entre  unas  pocas  opciones. Llegados  a
este  punto si  no sabes  cuál elegir,  escoge la  recomendada por  la
distribución. Ésta es  siempre una buena elección en caso  de duda. En
cambio si sabes el kernel que  quieres poner entre todas las opciones,
selecciónalo y continuemos con la instalación.

Una vez  decidido qué kernel  vamos a  usar tendremos que,  en algunas
instalaciones,  decir qué  hardware  tenemos instalado  y qué  drivers
queremos que use. Este paso se  hace cada vez de manera más automática
y no está recomendado para la gente que instala por primera vez Linux.
Sin embargo, no pasa nada  por mirar  tocar e intentar  cargar módulos
para  ver qué  es lo  que pasa.  Quienes  hayan  instalado un  sistema
Linux previamente  y conozcan  el tipo  de hardware  que poseen  y los
módulos  que  hacen falta  para  que  éste funcione  correctamente  es
éste el  momento de  seleccionarlos y  configurarlos para  su correcta
instalación.

\begin{figura}{lycoris_ethernet}{0.8}
\caption{Configuración de la red}
\end{figura}


Si hemos  instalado módulos  de hardware que  necesitan algún  tipo de
configuración, por ejemplo tarjetas de red, tras haber seleccionado el
módulo y  cargado con éxito  en el kernel, tendremos  que proporcionar
la  información necesaria  al  sistema para  que  se pueda  configurar
dicho hardware  correctamente, que  generalmente será  algún parámetro
adicional. En  el caso de la  tarjeta de red tendremos  que decirle si
coge la configuración por dhcp o  si la especificamos nosotros a mano.

\section{Estableciendo el arranque}

Este es otro de los pasos delicados de la instalación, menos que el de
particionar el disco duro pero también tendremos que tener cuidado con
lo  que hacemos.  Vamos a  seguir la  estructura del  apartado de  las
particiones, es decir, vamos a separar los sistemas donde conviven dos
sistemas operativos y los sistemas donde sólo tenemos Linux.

Para aquellas máquinas con arranque dual  Windows y Linux lo normal es
que no se disponga de un sistema gestor del arranque antes de instalar
Linux, por  lo tanto tendremos que  instalar el gestor de  arranque de
nuestra distribución  (LILO o GRUB) en  el master boot record  MBR. Lo
que lograremos de con esto es que nuestro cargador se ejecute antes de
cualquiera de  los sistemas operativos  y nos permita decidir  cual de
los dos queremos arrancar. No basta con instalar el gestor de arranque
en el  MBR, sino que,  tendremos que indicar que  particiones queremos
que  arranque.  Normalmente  esto  se  hace  de  manera  automática  y
simplemente tendremos que seleccionar las particiones a arrancar, pero
si no sucede de manera  automática haremos uso de nuestro conocimiento
en nomenclatura de  particiones para indicar que  arranque por ejemplo
la primera partición de nuestro IDE1 maestro ({\tt /dev/hda1}).

En  las  máquinas  donde  no  convivan  los  dos  sistemas  operativos
tendremos que instalar LILO en el master boot record pero no tendremos
que tener tanto cuidado  a la hora de indicar que es  lo que tiene que
arrancar  puesto que  el sistema  de instalación  configurará todo  él
solo.


\section{Instalar el sistema base}

En este momento las distribuciones  empiezan a diverger, en el sentido
de que no en todas este paso se realiza de la misma manera. En algunas
primero se instala el sistema base  y luego se selecciona qué software
extra se  quiere instalar. En  cambio hay otras distribuciones  en las
que el sistema base y el  software adicional a instalar se seleccionan
en un solo paso.

Si nuestra  distribución decide  instalar el sistema  base simplemente
haremos  caso,  en este  paso  normalmente  no  hay muchas  cosas  que
decidir. Simplemente  se instalan  los programas  básicos para  que el
sistema operativo pueda arrancar y poco más.

\begin{figura}{lycoris_usuario}{0.8}
\caption{Creación de usuarios durante la instalación}
\end{figura}


\section{Instalar el software}

Hoy en día las distribuciones traen un sistema gráfico de selección de
paquetes (software a instalar).  Normalmente se encuentran organizados
en categorías o ramas según el fin para el cual se usará ese software.
A continuación vamos a describir los más comunes:

\begin{description}

\item[Paquetes base] Tal  y como su propio nombre indica  se trata del
sistema  base nombrado  en el  párrafo anterior.  Entre otros  incluye
utilidades como {\bf bash} (un  tipo de consola, shell), {\bf adduser}
(utilidad para añadir  y eliminar usuarios), {\bf  gzip} (compresor de
ficheros) \ldots

\item[Paquetes  devel]  Esta  sección  de software  está  dedicada  al
desarrollo del  mismo, es  decir que  en esta  rama es  donde podremos
encontrar  herramientas  de   programación  tales  como  compiladores,
librerías, depuradores,  preprocesadores, etc.  Algunos de  ellos son:
{\bf g++}  (compilador de C++),  {\bf g77} (compilador  de Fortran77),
{\bf  ddd} (depurador  de código),  {\bf cvs}  (sistema de  control de
revisiones) \ldots

\item[Paquetes doc] Aquí podemos  encontrar todo tipo de documentación
relacionada con  los programas instalados  o el uso de  librerías. Sin
duda lo más  destacable de este apartado son las  llamadas páginas man
{\bf (manpages)}.

\item[Paquetes math] Software dedicado  a realizar tareas matemáticas.
Podremos  encontrar   programas  como  {\bf  GNUplot}   (generador  de
gráficas),  {\bf octave}  (programa  estrella de  esta sección,  sirve
para  realizar   todo  tipo   de  cálculos   numéricos,  simulaciones,
representaciones etc; un clónico de  MathLab), {\bf gnumeric} (hoja de
cálculo para GNOME) \ldots

\item[Paquetes X11]  Esta rama de  paquetes proporciona la  mayoría de
los componentes  del sistema  X-Window desarrollados por  {\em XFree86
Project}, junto a otros accesorios. Podremos encontrar el sistema base
de las X, KDE, GNOME, administradores  de sesiones como xdm, gdm, kdm,
distintos tipos de fuentes, servidor de fuentes (xfs), etc.

\end{description}

Los  paquetes  nombrados  previamente  son solamente  algunos  de  los
básicos.  Todas las  distribuciones  traen una  multitud de  programas
adicionales los cuales puedes instalar a tu gusto y dependiendo de tus
necesidades. Casi  todas las aplicaciones  se pueden encontrar  en los
CD's  de tu  distribución favorita,  pero algunas  de ellas  pueden no
estar en esos CD's.

\begin{figura}{kpackage}{0.8}
\caption{KPackage. Gestor de paquetes RPM y DEB para KDE}
\end{figura}

Para instalar esos programas bastará con descargar los paquetes de esa
aplicación e instalarlos. Hay que anotar que existen paquetes binarios
(compilados) para  las distintas  distribuciones. Los más  famosos son
los  paquetes  .RPM  (los  cuales  usan Red  Hat  y  Mandrake)  y  los
paquetes DEB (usados en las  distribuciones Debian). En la página {\em
www.rpmfind.net}  podremos encontrar  (casi)  todos  los paquetes  rpm
existentes,  con  lo  cual  nos  bastará con  buscar  el  programa  en
cuestión, descargarlo e  instalarlo ejecutando el comando  {\tt rpm -i
nombre-del-paquete.rpm}. Los usuarios de Debian lo tienen bastante más
fácil en  este aspecto gracias a  las herramientas APT. En  el caso de
que un paquete no se pueda instalar mediante esas herramientas podemos
descargar nuestro paquete .deb e instalarlo ejecutando el comando {\tt
dpkg -i nombre-del-paquete.deb}.

Existen  más  tipos   de  paquetes  dependiendo  de   la  multitud  de
distribuciones, pero  algunas veces no podremos  encontrar el software
deseado  en  forma de  paquete  binario.  En  ese caso  tendremos  que
descargar  el código  fuente  del programa,  compilarlo e  instalarlo.
Normalmente el código se encuentra en forma de paquetes .TGZ (paquetes
comprimidos), con lo cual habrá  que descomprimirlos antes de nada. La
secuencia de pasos para realizar lo anterior es la siguiente:

\begin{verbatim}
$ tar zxvf nombre-del-paquete.tgz
$ cd directorio-del-programa
$ ./configure
$ make
$ make install
\end{verbatim}

Esta secuencia  es la más  usual, pero  (como siempre) no  funciona en
todos  los casos,  para los  cuales  tendremos que  informarnos en  la
documentación del programa. 

Por último cabe destacar que la  mayoría de software hace uso de otros
programas o librerías  sin las cuales no se podrá  instalar o compilar
la  aplicación  en  cuestión.  La  solución  pasa  por  instalar  esas
dependencias previamente y a continuación el programa.


\section{Configuración del servidor X}

Dependiendo   de   la   distribución  que   estamos   instalando   nos
encontraremos  con distintas  utilidades  para  configurar el  sistema
X-Window.  Todas ellas  son muy  similares entre  sí pero  con ligeras
diferencias. Para no  concretar el proceso de  configuración según las
distintas distribuciones  usaremos la herramienta {\tt  xf86config} la
cual está presente en todas ellas.

El programa {\tt xf86config} es un programa de consola, en modo texto,
que modifica  los ficheros  de configuración del  servidor X,  para lo
cual necesitaremos  los privilegios  de superusuario. Para  invocar el
programa abrimos una  consola, accedemos como root  (usando el comando
su si procede) y  tecleamos {\tt xf86config}. Nos encontraremos con
una  pantalla  de descripción  de  la  aplicación. Cabe  destacar  que
la  ejecución  del programa  se  podrá  abortar en  cualquier  momento
presionando la combinación  de teclas Ctrl-C. Así que no  debemos temer a
equivocarnos ya que al interrumpir  esta herramienta no se afectan los
ficheros de configuración existentes.

\begin{figura}{xf86cfg}{0.4}
\caption{xf86cfg. Utilidad para configurar el servidor X}
\end{figura}

Para proceder  con la configuración  pulsaremos la tecla  {\tt Enter},
apareciendo  la   pantalla  de  selección  del   protocolo  de  ratón.
Introducimos el número de la opción elegida (normalmente es la 4. PS/2
Mouse) seguida de un {\tt Enter}. Se nos preguntará si queremos emular
el  tercer botón  (para aquellos  ratones  con sólo  dos botones).  Si
respondemos  afirmativamente al  pulsar el  botón izquierdo  y derecho
simultáneamente obtendremos  el mismo  efecto como cuando  pulsamos el
tercer  botón.  Respondemos  con  y/n y  apretamos  {\tt  Enter}  para
poder  llegar  a  introducir  el  nombre  del  dispositivo  del  ratón
(por  defecto  /dev/mouse, y  en  mayoría  de  los casos  nos  servirá
/dev/psaux). Posteriormente se nos  pedirá que introduzcamos el número
correspondiente a nuestro teclado de entre los listados en la pantalla
(los del tipo  generic en caso de duda). A  continuación tendremos que
elegir el  país (España  - 42). Seguidamente  podremos poner  un alias
para el idioma anteriormente elegido. Pulsando a {\tt Enter} se pondrá
la  opción por  defecto,  y  el programa  nos  preguntará si  queremos
activar las  opciones adicionales  del XKB  (en caso  de duda  {\tt n}
seguido de {\tt Enter}).

A  continuación accederemos  al apartado  de configuración  de nuestro
monitor. Observando las opciones mostradas  en la pantalla y eligiendo
las  adecuadas avanzaremos  dos apartados  y  en el  siguiente se  nos
pedirá que  escribamos un identificador  para el monitor.  Escribe una
cadena con  su nombre, descripción o  lo que quieras y  pulsa la tecla
{\tt Enter} para poder elegir nuestra tarjeta de vídeo.

Si leemos  la pantalla averiguamos que  al pulsar {\tt y  + Enter} nos
aparecerá un  listado de  tarjetas soportadas  (al pulsar  {\tt Enter}
continúa  la  lista)  del  cual tendremos  que  introducir  el  número
correspondiente  a  nuestro modelo  de  tarjeta.  Seguidamente se  nos
pedirá el tamaño de  la memoria de la tarjeta de  vídeo (4096K son 4M,
32768K son  16M \ldots). Al  igual que en el  caso del monitor  se nos
pedirá una cadena para identificar  la tarjeta de vídeo. Elegiremos la
profundidad de color  que deseamos que sea por defecto  y estaremos en
el último  paso que es el  de guardar el fichero  de configuración. Si
estamos seguros de haber  elegido las opciones adecuadas responderemos
afirmativamente  o en  el  caso contrario  abortaremos  el programa  y
volveremos desde principio.

El proceso anterior  a simple vista parece  demasiado complicado, pero
es sencillo con un poco de  paciencia. Destacar lo dicho al principio,
que cada distribución tiene su propia herramienta de configuración del
servidor  X,  las cuales  resultan  bastante  más sencillas  de  usar.
Algunos fabricantes de tarjetas  de vídeo como NVIDIA (www.nvidia.com)
proporcionan drivers de sus tarjetas  para Linux, así como los drivers
OpenGL y documentación sobre su instalación.

\section{Configuración del acceso a internet}

Este apartado  lo vamos a  dividir en tres subbloques  dependiendo del
tipo de nuestra conexión. Vamos a  hablar de como configurar el acceso
a internet mediante  un módem (externo o interno), una  línea ADSL con
router ADSL, y una conexión mediante un cable - módem.

\subsection{Acceso mediante modem}

Casi  todos los  modems externos  están  soportados por  el kernel  de
Linux; en  cambio hay  muy pocos  módem internos  para los  cuales hay
soporte en nuestro S.O. favorito.  Actualmente existe soporte para los
siguientes:

\begin{itemize}
\item Modems que hagan uso de los chipsets Lucent Apollo (ISA) y Lucent Mars (PCI)
\item Modems Intel V90 HaM e Intel 536ep
\item Modem IBM Mwave (usado en el Thinkpad 600E y posteriores)
\item Modems Conexant que hagan uso de los chipsets Conexant HCF y HSF
\item Modems ESS ISA
\end{itemize}

Toda  la documentación  sobre la  instalación y  los drivers  para los
modems anteriores se pueden encontrar en la página www.linmodems.org.

Antes de empezar con la configuración de nuestro módem nombrar como en
casos  anteriores que  existen herramientas  específicas de  distintas
distribuciones, escritorios  o aplicaciones (como  es el caso  de {\em
kppp} de KDE el cual es  intuitivo y bastante sencillo de configurar).
Otra  vez volveremos  a  usar  una aplicación  incluida  en todas  las
distribuciones llamada {\bf  pppd} y la utilidad  de configuración del
programa  anterior  {\bf  pppconfig}.  También nombrar  que  antes  de
empezar con el  proceso debemos tener instalado el pppd  y activado el
soporte para  el protocolo PPP ({\em  Point to Point Protocol})  en el
kernel, apartado {\em  Network devices}, el cual  suele estar activado
al instalar nuestra distribución.

\begin{figura}{kpp}{0.5}
\caption{KPPP. Interfaz GUI de pppd para KDE}
\end{figura}

Gracias  a   una  herramienta  como   {\em  pppconfig}  la   tarea  de
configuración es  muy sencilla,  basta con ejecutar  el programa  e ir
introduciendo  los  datos  requeridos  por  el  mismo.  Esta  utilidad
modifica  los ficheros  de  configuración, así  que  antes de  empezar
necesitamos acceder como root y a continuación escribir en la línea de
comandos {\tt  pppconfig}. Seleccionamos la opción  marcada (crear una
conexión nueva)  y posteriormente  introducimos un  identificador para
nuestra  conexión (como  ejemplo usaremos  'nueva-conexion'). Elegimos
nuestro tipo  de DNS  (casi siempre  estático, dependiendo  de nuestro
proveedor de acceso a internet), seleccionamos $<$OK$>$ y seguidamente
introducimos la  dirección del servidor  DNS primario y  secundario si
tenemos  (ambas direcciones  son  proporcionadas por  el proveedor  de
acceso).

En este  momento habrá que  seleccionar el método  de autentificación.
PAP suele funcionar  y si no es así prueben  CHAT o CHAP. Introducimos
el nombre  de usuario,  la contraseña, la  velocidad de  nuestro módem
(deja  la opción  por defecto  en caso  de duda  - 115200),  método de
marcación  (tonos o  pulsos)  y el  número de  teléfono  del acceso  a
internet.

Enciende  el módem  si es  externo porque  a continuación  el programa
intentará autodetectar el  puerto de nuestro módem. En el  caso de que
no lo  consiga tendremos que  introducirlo nosotros. En  algunos casos
estará en  /dev/modem. Si lo  anterior tampoco funciona  debemos saber
que los  puertos COM son dispositivos  que se encuentran en  /dev y se
relacionan mediante la siguiente tabla:

\begin{center}
	\begin{tabular}{c|c}
	Dispositivo & Puerto \\
	\hline
	/dev/ttyS0 & COM1 \\
	/dev/ttyS1 & COM2 \\
	/dev/ttyS2 & COM3 \\
	/dev/ttyS3 & COM4 \\
	\end{tabular}
\end{center}

Normalmente los modems  externos usan el COM1 o el  COM2, mientras los
internos usan el COM3 o el  COM4. Anotar que /dev/modem es simplemente
un enlace  a alguno de  los /dev/ttySx. Introduciremos  el dispositivo
correspondiente  (/dev/ttySx)  y  habrémos  acabado  con  el  proceso.
Guardamos los cambios y abandonamos al asistente pppconfig.

\begin{figura}{pppconfig}{0.9}
\caption{PPPConfig}
\end{figura}

Los ficheros de configuración por  defecto se guardan en el directorio
/etc/ppp. Buscamos la ruta del  fichero con el nombre 'nueva-conexión'
(u  otro introducido  como  identificador) y  ejecutamos el  siguiente
comando:

\begin{verbatim}
$ pppd file /ruta/del/fichero/nueva-conexion
\end{verbatim}

Lo cual normalmente corresponde a lo siguiente:

\begin{verbatim}
$ pppd file /etc/ppp/peers/nueva-conexion
\end{verbatim}

Podremos  observar  que  es  lo  que está  sucediendo  en  el  fichero
{\tt /var/log/messages} con el comando:

\begin{verbatim}
$ tail -f /var/log/messages
\end{verbatim}

Podemos  finalizar nuestra  conexión matando  el proceso  pppd con  un
comando kill ({\tt killall pppd}).

Por   último  recordar   que  existen   muchos  otros   asistentes  de
configuración algunos de  los cuales pueden simplificar  el proceso de
configuración.

\subsection{Acceso mediante ADSL y cable modem}

La configuración de acceso usando una  línea ADSL que vamos a explicar
en  este apartado  es usando  un  Router ADSL  el cual  debe de  estar
previamente  configurado. También  necesitaremos una  tarjeta Ethernet
que funcione en Linux.

Para poder  realizar el acceso  tenemos que configurar la  interfaz de
red. Normalmente  usaremos el  programa {\bf  ifconfig} con  una orden
similar a la siguiente (root):

\begin{verbatim}
$ ifconfig eth0 192.168.0.2 netmask 255.255.255.0 up
\end{verbatim}

Con  el comando  anterior  estamos configurando  la interfaz  Ethernet
eth0, asignándole una dirección IP  192.168.0.2 y la máscara de subred
255.255.255.0.  En internet  se pueden  encontrar varios  tutoriales y
manuales de  configuración de  red, así  que no  nos vamos  a extender
demasiado en este punto. El paso siguiente es establecer un camino por
defecto ({\bf gateway}). Lo haremos con este comando:

\begin{verbatim}
$ route add default gw 192.168.0.1 eth0
\end{verbatim}

donde 192.168.0.1 es la dirección IP de nuestro router ADSL.

Para  no tener  que repetir  este proceso  cada vez  que iniciamos  el
ordenador  podemos  poner la  información  sobre  nuestra interfaz  de
red  y el  gateway  en  un fichero  de  configuración que  normalmente
están  en el  directorio /etc  y tienen  nombres similares  a network,
networks  \ldots;  en la  distribución  Debian  se encuentra  en  {\em
/etc/network/interfaces} y cuyo aspecto es parecido al siguiente:

\begin{verbatim}
	auto eth0
	iface eth0 inet static
	   address 192.168.1.50
	   netmask 255.255.255.0
	   network 192.168.1.0
	   broadcast 192.168.1.255
	   gateway 192.168.1.1
\end{verbatim}

En  otras distribuciones  existen utilidades  de configuración  de red
como netcfg (RedHat) o netconfig (Slackware).

En un acceso a internet a  través de un cable módem la autentificación
de usuario  y la dirección IP  se obtiene de manera  dinámica a través
del servicio  DHCP. Además de  lo anterior  se obtendrá la  máscara de
subred, dirección de  gateway y las direcciones DNS; es  decir todo se
configura por si  mismo. Para su uso tendremos que  tener instalado un
cliente DHCP, por ejemplo dhcp-client.  En el fichero {\em interfaces}
bastará poner:

\begin{verbatim}
	auto eth0
	iface eth0 inet dhcp
\end{verbatim}

para  poder  usar  este  servicio.  En  los  distintos  asistentes  de
configuración  seleccionaremos el  uso  de DHCP  y  con esa  operación
tendremos configurada  la interfaz de red  y por lo tanto  el acceso a
internet a través de un cable módem.


\section{Configuración de la impresora: CUPS}

En Linux  podemos utilizar  varios sistemas de  impresión y  nos sería
imposible explicarte como se configuran  todos ellos, aquí nos vamos a
centrar en uno en concreto, CUPS, {\em Common Unix Printing System}.

Una de las  ventajas que tiene este  sistema frente a los  otros es su
facilidad  de configuración  ya que  podemos configurar  completamente
nuestras impresoras desde un navegador.

El cups está diseñado para que  podamos utilizar una misma impresora o
un grupo de éstas desde cualquier PC que esté en la red, es decir, que
un amigo  siempre que tu le  des permiso y estés  conectado a internet
podrá utilizar tu impresora para sacar  a tiempo ese trabajo que tiene
que entregar mañana y no le ha  dado tiempo de ir a comprar tinta para
la impresora o se ha quedado sin papel o mil cosas que suelen pasar.

Para  configurar  el  sistema   simplemente  tendremos  que  abrir  un
navegador   y   en   la   barra   de   dirección   colocaremos:   {\tt
http://localhost:631}, en ese momento,  si nosotros no tenemos permiso
para  configurar el  sistema  de  impresión nos  saldrá  un cuadro  de
diálogo en el  que tendremos que introducir el nombre  de usuario y el
password  de un  usuario del  sistema con  permiso para  configurar el
CUPS, por ejemplo root. Una vez  obtenido acceso a la configuración de
CUPS en el navegador aparecerá una página web como esta:

\begin{figura}{cups_inicio}{0.7}
\caption{Página web que CUPS sirve en el puerto 631}
\end{figura}

Para  comenzar   con  la   configuración  pulsaremos  sobre   {\tt  Do
Administration Tasks}  en ese  momento nos aparecerán  tres categorías
sobre las que podemos operar: clases, jobs, printers.

Las   clases  son   grupos   de  impresoras   con  unas   determinadas
características que  se unen para  hacer más  fácil la gestión  de los
documentos.  Por ejemplo,  si en  una oficina  se tienen  3 impresoras
láser, 2 chorro de tinta a color y 5 matriciales, podríamos hacer tres
grupos:  grupo láser,  donde  meteríamos las  impresoras láser;  grupo
tinta  donde colocaríamos  las impresoras  a chorro  de tinta  y grupo
matricial donde estarían las  impresoras matriciales. Con estos grupos
confeccionados  cuando quisiéramos  imprimir un  documento simplemente
tendríamos que  elegir uno de los  tres grupos y nuestro  documento se
imprimirá en la  impresora que antes quede libre de  trabajo del grupo
seleccionado.

Los jobs o trabajos son los trabajos  que hay en este instante en cola
para imprimir,  podremos eliminarlos,  detener su impresión,  etc. Los
trabajos que  ya han  sido terminados  también podremos  reanudarlos y
conocer varios datos sobre ellos.

Las  impresoras  son los  dispositivos  de  impresión que  tenemos  en
nuestro sistema podremos añadir impresoras  nuevas o gestionar las que
tenemos. Como nosotros vamos a  instalar nuestra impresora por primera
vez pulsaremos sobre {\tt Add Printer}.

Tendremos  que ir  facilitando  los  datos que  se  nos solicitan,  lo
primero es asignarle un nombre, decir  donde se encuentra y añadir una
breve descripción al nombre.

%\begin{figura}{cups_instalando1}{0.7}
%\caption{Configuración de una nueva impresora}
%\end{figura}

Lo  siguiente   es  seleccionar  donde  tendremos   conectada  nuestra
impresora,  normalmente tendremos  conectada  la  impresora al  puerto
paralelo ({\tt Parallel Port \#1})

%\begin{figura}{cups_instalando2}{0.7}
%\caption{Selección del puerto de la impresora}
%\end{figura}

En  la siguiente  página tendremos  que seleccionar  el fabricante  de
nuestra impresora.

%\begin{figura}{cups_instalando3}{0.7}
%\caption{Selección del fabricante de la impresora}
%\end{figura}

A continuación elegiremos el modelo de impresora.

%\begin{figura}{cups_instalando4}{0.7}
%\caption{Selección del modelo de la impresora}
%\end{figura}

y listo, nuestra  impresora ya está instalada,  simplemente faltan los
últimos  retoques. Pulsaremos  sobre el  nombre  que le  hemos dado  a
nuestra impresora y aparece la siguiente pantalla:

\begin{figura}{cups_configurando}{0.7}
\caption{Configuración de CUPS}
\end{figura}

Ahora pulsamos sobre  {\tt Configure Printer} y  seleccionamos el tipo
de papel  que queremos  usar, la  calidad de impresión  y ese  tipo de
cosas. Ya tenemos nuestra impresora lista para imprimir cualquier tipo
de  documento. Si  queremos comprobar  como imprime  nuestra impresora
podemos  pulsar en  la pantalla  anterior a  {\tt print  test page}  e
imprimirá la página de prueba de CUPS.

