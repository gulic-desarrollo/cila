%Autor: iCorrecam, aplatanad, wizard, amd77, miguev, faraox
%iCorrecam: 2
%aplatanad: 8 + 2
%wizard--: 3
%amd77: 3
%miguev: 3
%faraox: 2

\chapter{Aplicaciones para Internet}
\label{internet.tex}


\section{Navegadores de la World Wide Web}

Durante  muchos  años  {\sf  Netscape Communicator  4}  fue  el  único
navegador  multiplataforma  real,  dando  cobertura a  muchos  de  los
distintos UNIX comerciales  existentes. Puesto que Linux  no podía ser
menos, casi  desde que  Linux tiene interfaz  gráfico ha  existido una
versión del navegador de {\sf Netscape} para este sistema operativo.

{\sf Netscape  Communicator 4} proporciona soporte  para navegación de
páginas web con {\sf JavaScript} y {\sf Flash 5}, y permite visualizar
documentos PDF dentro  del navegador (mediante un plugin  para el {\sf
Adobe  Acrobat  Reader}).  También  nos permite  gestionar  el  correo
electrónico y componer páginas web.

Los  Linuxeros siempre  hemos  considerado que  el  navegador de  {\sf
Netscape}  consumía  demasiados recursos  en  Linux,  además de  tener
bastantes problemas de  estabilidad. Debido a éste y  a otros factores
importantes,  como  fueron la  forma  de  competir  con la  casa  {\em
Microsoft  Corporation},  {\em  Netscape  Communications  Corporation}
llegó  a la  sana conclusión  de que  la mejor  manera de  mantener su
navegador en el  mercado era liberando su código fuente.  Así nació el
proyecto {\sf Mozilla}.

Sin  embargo, {\sf  Mozilla} sigue  siendo un  navegador excesivamente
pesado para un  número importante de máquinas. Dentro  de la comunidad
del Software Libre,  se alzaron voces en contra de  ese desperdicio de
recursos, proponiendo  la creación  de navegadores  alternativos. Aquí
listamos algunas de las alternativas  que podemos encontrar en el área
de los navegadores web dentro del Software Libre:

\begin{description}

\item[Chimera  2] Navegador  web para  las {\sf  X}. Simple  y rápido,
todavía se  encuentra en un temprano  estado de desarrollo por  lo que
cuenta con numerosos errores.

\item[Netscape Communicator]  Bajo dicho  nombre podemos  encontrar el
navegador original de {\sf Netscape}. La última versión es la 4.77.

\item[Dillo] Navegador  multiplaforma pequeño, estable y  rápido. Está
basado en {\sf GTK} pero no requiere dispone de {\sf GNOME}.

\item[Encompass] Navegador libre para {\sf GNOME}.

\item[Galeon] Es un  navegador que utiliza el motor  de rendering {\em
Genko} de  {\sf Mozilla} para  mostrar el  contenido de la  World Wide
Web. Sin embargo, puesto que utiliza  las bibliotecas de {\sf GNOME} y
{\sf GTK}  es ligeramente más rápido  que {\sf Mozilla}, y  se integra
perfectamente con el resto de las aplicaciones {\sf GNOME}.

\item[Konqueror]  Gestor  de  ficheros,   navegador  web  y  visor  de
documentos del {\sf KDE}.

\item[Links]  Navegador  web para  la  consola.  Incluye soporte  para
visualizar tablas, marcos y descargas en segundo plano.

\item[Lynx] Navegador web para la consola.

\item[Mozilla] Es  un sofisticado navegador  gráfico de la  World Wide
Web  que soporta  un  gran  número de  tecnologías,  como por  ejemplo
soporte para HTML 4.0, CSS 2, {\sf JavaScript} y {\sf Java}. Además de
poder  ser utilizado  como un  sencillo visor  de HTML.  {\sf Mozilla}
está  basado en  parte del  código  de los  navegadores {\sf  Netscape
Communicator} y {\sf Netscape Navigator}.

\item[OpenOffice] Suite ofimática que incluye un buen navegador web.

\item[W3m] Visor de  la World Wide Web para la  consola. Dispone de un
excelente soporte para tablas y marcos.

\end{description}

Bueno, seguro que en el momento  de leer este apartado, habrán surgido
nuevos navegadores web dentro del mundillo del Software Libre.


\subsection{Mozilla}
\index{Mozilla}

{\sf  Mozilla} es  probablemente el  más completo  de los  navegadores
multiplataforma. Esto es debido a  que implementa soporte para un gran
número de tecnologías de la World  Wide Web, y se ciñe rigurosamente a
los estándares del W3C (nombre con el que se conoce al {\em World Wide
Web Consortium}, que  es el organismo encargado  de la estandarización
de las diferentes tecnologías presentes en la World Wide Web).

\begin{figura}{web_mozilla}{0.8}
\caption{Ventana del navegador web Mozilla}
\label{web_mozilla}
\end{figura}

Para centrarnos en  el manejo de {\sf Mozilla}  empezaremos mirando la
Figura \ref{web_mozilla}. En ella  podemos observar la clásica ventana
de navegación de  {\sf Mozilla} que es semejante a  la de otros muchos
navegadores. A continuación enumeramos los  elementos de la ventana de
arriba a abajo y de izquierda a derecha:

\begin{enumerate}

\item{{\em Barra de menús}}
\item{{\em Barra de herramientas de navegación}}
\item{{\em Barra de herramientas personales}}
\item{{\em Panel lateral}}
\item{{\em Área de visualización de la navegación}}
\item{{\em Barra de tareas}}

\end{enumerate}

La visualización de la mayor  parte de estos elementos puede activarse
o  desactivarse  desde  el  menú   {\tt  VER  $\rightarrow$  BARRA  DE
HERRAMIENTAS}  de  la  barra  de  menú  del  programa.

Adicionalmente  disponemos  de  una  barra denominada  {\em  barra  de
componentes}. Dicha  barra se muestra  como unos pequeños iconos  a la
izquierda de  la barra  de tareas  (parte inferior  de la  ventana del
programa). Dicha barra  nos permite lanzar de forma  sencilla y rápida
algunas  de  las  otras  herramientas de  Internet  que  acompañan  al
navegador web  {\sf Mozilla}. Entre dichas  herramientas disponemos de
un editor de HTML y de un lector de noticias y de correo electrónico.

\subsubsection{Navegación básica}

Para navegar por  la World Wide Web basta con  introducir la dirección
de  la máquina  o  recurso al  que  deseamos acceder  en  la barra  de
herramientas de navegación.  La misma barra dispone a  la izquierda de
botones para avanzar  o retroceder a través de  las páginas visitadas,
recargar la página  actual, o detener la descarga. A  la derecha de la
barra disponemos  de un botón de  acceso rápido al menú  de impresión,
con  el que  podemos imprimir  la página  actual. Para  simplificar el
aprendizaje, si dejamos  el puntero del ratón sobre  cualquiera de los
botones  durante  unos segundos  la  aplicación  nos informará  de  la
función de cada uno.

Además de las funciones básicas la barra de herramientas de navegación
nos  permite realizar  búsquedas de  términos en  Internet. Para  ello
basta  con  introducir  las  palabras  a buscar  en  la  propia  barra
y  a  continuación pulsar  en  el  botón  {\tt BUSCAR}.  El  navegador
consultará al buscador que tengamos  configurado (por defecto se trata
del  de {\em  Netscape Network})  y  nos mostrará  los resultados.  El
buscador utilizado puede ser  configurado en {\tt EDITAR $\rightarrow$
PREFERENCIAS...  $\rightarrow$  NAVIGATOR  $\rightarrow$  BÚSQUEDA  EN
INTERNET}.  Mientras   que  el  contenido   de  la  propia   barra  de
herramientas  de  navegación  puede  ser configurado  en  {\tt  EDITAR
$\rightarrow$ PREFERENCIAS... $\rightarrow$ NAVIGATOR}. Nos ocuparemos
de describir el cuadro de diálogo {\tt PREFERENCIAS} y las opciones de
configuración de {\sf Mozilla} más a delante.

\begin{figura}{web_mozilla_menunav}{0.75}
\caption{Menú del botón de retroceso de la barra de herramientas de
navegación}
\label{web_mozilla_menunav}
\end{figura}

Algunos  de los  botones de  la  barra de  herramientas de  navegación
disponen  de  un  pequeño  icono  con forma  de  flecha  en  la  parte
inferior-derecha de  los mismos. Dicho  icono suele desplegar  un menú
con  opciones adicionales.  En el  caso particular  de los  botones de
avance  y retroceso  dicho menú  nos permite  elegir a  que dirección,
de  entre las  ya  visitadas, queremos  avanzar  o retroceder  (Figura
\ref{web_mozilla_menunav}).  Hay  que  tener  en  cuenta  que  si  no
existiera el menú tendríamos que  retroceder o avanzar por las páginas
visitadas de una en una.

Durante nuestra  visita a la World  Wide Web la barra  de tareas suele
mantenernos informados de  las acciones que realiza  el navegador. Por
ejemplo, nos informa  de si estamos detenidos y de  cuanto se tardó en
descarga la página actual, o de si estamos descargando alguna página.

Además, la barra de tareas dispone de  una serie de iconos en la parte
derecha con  el objetivo  de mantenernos informados  del estado  de la
conexión. Si dejamos  unos segundos el puntero del  ratón sobre dichos
iconos  seremos  informados  de  su significado,  e  incluso  podremos
realizar alguna acción relacionada con dicha información.

\begin{description}

\item[Trabajar sin conexión] Mientras disponemos  de conexión a la red
y navegamos, {\sf Mozilla} descarga  las páginas que vamos visitando y
las almacena  en el  disco duro en  lo que se  denomina una  caché. La
existencia de está caché es  importante puesto que si visitamos varias
veces una misma dirección {\sf Mozilla} no tiene necesidad de volver a
repetir  la descarga.  En  su lugar  nos  proporciona directamente  la
página almacenada en la caché.  En ocasiones no disponemos de conexión
a  red por  lo  que  sería deseable  que  {\sf  Mozilla} no  intentará
descargarse la  páginas desde  la red y  nos mostrara  directamente la
copia  en la  caché.  También  es posible  que  aun habiendo  conexión
estemos interesados en que {\sf Mozilla} no haga uso de ella. Ese modo
de  trabajo que  andamos buscando  es el  denominado {\em  trabajo sin
conexión}. Existe  un icono que representa  dos enchufes desconectados
que  nos indica  que  estamos en  dicho modo.  Si  los enchufes  están
conectados significa que  estamos haciendo uso de la  conexión de red,
es decir {\em trabajo  con conexión}. El paso de un  modo de trabajo a
otro  se puede  realizar  pulsando  con el  ratón  sobre  el icono,  o
seleccionando el elemento de  menú {\tt ARCHIVO $\rightarrow$ TRABAJAR
SIN CONEXIÓN}.

\item[Cookies] En  ocasiones algunos servidores  de la World  Wide Web
requieren  que  los  navegadores  que  los  acceden  almacenen  cierta
información. En  muchos casos se  trata de información  sobre nuestras
preferencias que  dichos servidores han  recopilado, y que  desean que
guardemos para  volver a reclamarla  cuando nos volvamos a  conectar a
sus  páginas. Este  tipo de  comportamiento  puede ser  un agujero  de
seguridad en  potencia (o al menos  un problema de privacidad)  por lo
que {\sf  Mozilla} dispone  de un  filtro de  cookies que  nos permite
decidir  en  que servidores  confiamos  y  en  cuales no.  Si  nuestro
navegador ha aceptado una cookie para ser almacenada podremos observar
el  icono de  una  {\em  galleta}. Pulsando  sobre  la misma  podremos
averiguar la información  que contiene, así como  configurar el filtro
de cookies frente a posibles peticiones futuras.

\item[Seguridad] Es  importante recordar que  todo lo que  recibimos o
enviamos desde o hacia la  World Wide Web es fácilmente interceptarle.
Cuando accedemos a páginas donde la seguridad de las comunicaciones es
de vital importancia dicha comunicación se hace cifrando los datos.
Si disponemos de un icono con  un candado abierto en nuestro navegador
significa que todo lo que hagamos  en Internet puede ser observado por
otros.  Sin embargo,  si  dicho  candado está  cerrado  es porque  las
comunicaciones  son  seguras, por  lo  que  tenemos garantías  de  que
nuestros datos no pueden ser fácilmente interceptados. Al pulsar sobre
dicho icono  obtendremos la  información de  seguridad para  la pagina
actual.

\end{description}

Al igual que  en muchos otros navegadores {\sf Mozilla}  dispone de un
menú de  contexto activable  pulsando con el  botón derecho  del ratón
sobre alguno los  elementos del área de visualización.  Dicho menú nos
permite, por ejemplo, recargar la  página actual, descargar un enlace,
o  abrir  un enlace  en  una  ventana  diferente, entre  otras  muchas
posibles acciones.

Durante la navegación por la  World Wide Web resulta habitual disponer
de varias  ventanas de {\sf  Mozilla} abiertas en  páginas diferentes.
Sin embargo, todas  esas ventanas están ligadas a un  mismo proceso en
ejecución  del navegador  {\sf  Mozilla}. Es  importante conocer  esto
puesto  que {\bf  si seleccionamos  el elemento  de menú  {\tt ARCHIVO
$\rightarrow$ SALIR} TODAS las ventanas de {\sf Mozilla} se cerraran},
al  terminar  el proceso  que  las  gestionaba.  Si por  el  contrario
deseamos cerrar únicamente una de  las ventanas en particular, debemos
seleccionar  {\tt ARCHIVO  $\rightarrow$ CERRAR}  o utilizar  el botón
correspondiente de la barra de título del marco de la ventana.

\subsubsection{Pestañas}
\index{Mozilla!Pestañas}

Una de las  innovaciones tomada de otros navegadores  e introducida en
{\sf Mozilla} es  el uso de las {\em pestañas}  durante la navegación.
Tanto  si seleccionamos  la  opción {\tt  ARCHIVO $\rightarrow$  NUEVO
$\rightarrow$  PESTAÑA DE  NAVIGATOR}, como  si seleccionamos  en {\tt
ABRIR EN UNA PESTAÑA NUEVA} cuando pulsamos con el botón derecho en un
enlace, se nos abre una nueva área de visualización dentro de la misma
ventana  de  {\sf Mozilla}.  Podemos  disponer  de tantas  áreas  como
deseemos y en cada una visualizar una página diferente.

\begin{figura}{web_mozilla_tab}{0.8}
\caption{Navegación con pestañas en Mozilla}
\label{web_mozilla_tab}
\end{figura}

En  la Figura  \ref{web_mozilla_tab} podemos  observar cómo  cada área
está representada  por una pestaña  en la  parte superior del  área de
visualización. Seleccionando  un pestaña  u otra podremos  navegar por
una  página u  otra. Cuando  deseemos cerrar  la pestaña  seleccionada
bastará con que pulsemos en botón  a la derecha de todas las pestañas.
También  podemos arrastrar  un enlace  en la  pagina web  del área  de
visualización sobre una  de las pestañas. Con ello  conseguimos que en
dicha pestaña se cargue el recurso del la World Wide Web al que apunta
dicho enlace (por ejemplo, una página web, una imagen, etc).

Algunos   aspectos  del   comportamiento   de   las  pestañas   pueden
ser   configurados  en   {\tt  EDITAR   $\rightarrow$  PREFERENCIAS...
$\rightarrow$ NAVIGATOR $\rightarrow$ PESTAÑAS}.

\subsubsection{Marcadores}
\index{Mozilla!Marcadores}

{\sf Mozilla}  nos permite almacenar  de forma ordenada  y clasificada
las  direcciones de  los recursos  de la  World Wide  Web que  más nos
interesan.  Dicho  almacenamiento  se  hace  en forma  de  lo  que  se
denominan {\em marcadores}.

Añadir la dirección de la página  actual como marcador es tan sencillo
como seleccionar  {\tt MARCADORES $\rightarrow$ AÑADIR  A MARCADORES}.
También  podemos  pulsar con  el  botón  derecho  del ratón  sobre  un
enlace y  seleccionar la  opción correspondiente  para añadirlo  a los
marcadores.

En todo  caso el  marcador se crea  siempre al final  de la  lista. Si
deseamos  tener un  control  más  fino para,  por  ejemplo, añadir  el
marcador  a una  carpeta determinada,  podemos utilizar  el menú  {\tt
MARCADORES $\rightarrow$ ARCHIVAR MARCADOR}\dots Dicha opción del menú
muestra un cuadro de diálogo que nos permite seleccionar el nombre del
nuevo marcador, crear carpetas de  marcadores, y elegir en que carpeta
deseamos guardarlo. Los marcadores así creados son accesibles en forma
de menús en el menú {\tt MARCADORES} de la barra de menús.

En  ocasiones  es  necesario  realizar  sobre  los  marcadores  tareas
de  administración  mucho más  avanzadas.  La  opción {\tt  MARCADORES
$\rightarrow$ ADMINISTRAR  MARCADORES} despliega un cuadro  de diálogo
(Figura   \ref{web_mozilla_adm})  que   nos   permite  manipular   los
marcadores  a  nuestro  antojo.  Podemos  movernos  por  el  árbol  de
marcadores y copiarlos o pegarlos  con ayuda de ratón. También podemos
alterar  sus  propiedades  utilizando  el  menú  de  contexto  que  se
despliega con el uso del botón derecho de nuestro ratón.

\begin{figura}{web_mozilla_adm}{0.8}
\caption{Cuadro de diálogo {\tt ADMINISTRAR MARCADORES}}
\label{web_mozilla_adm}
\end{figura}

El administrador  de marcadores dispone  de opciones para  ordenar los
marcadores según  diversos criterios, para realizar  búsquedas, y para
exportar o importar hacia o desde los marcadores de otros navegadores.
Además de carpetas,  el administrador de marcadores  nos permite crear
separadores  que se  utilizan  para separar  elementos  dentro de  los
menús. Al  igual que  con el  resto de las  ventanas de  {\sf Mozilla}
debemos evitar  utilizar la  opción {\tt ARCHIVO  $\rightarrow$ SALIR}
puesto que ésta cierra todas la ventanas del programa.

Entre  las  carpetas  de  marcadores existe  una  con  un  significado
especial. La carpeta {\tt PERSONAL TOOLBAR FOLDER} representa la barra
de herramientas  personales, que  está situada debajo  de la  barra de
herramientas de  navegación. Todos los marcadores  insertados en dicha
carpeta aparecerán automáticamente como botones  en la citada barra de
herramientas.

\subsubsection{Panel lateral}
\index{Mozilla!Panel lateral,Mozilla!Sidebar}

A la izquierda del área de visualización está el {\em panel lateral} o
{\em sidebar}  de {\sf Mozilla}. Dicho  panel dispone de una  serie de
pestañas con funciones que ayudan  a la navegación. Utilizando el menú
{\tt PESTAÑAS} del panel lateral podemos decidir que pestañas vemos de
entra  la  disponibles.  También podemos  personalizar  nuestro  panel
añadiendo  nuevas  pestañas  con nuevas  funciones  descargadas  desde
Internet.

Para que  no interfiera con  la navegación podemos modificar  el ancho
del panel pulsando con el ratón en  la barra que lo separa del área de
visualización.  Si hacemos  una  pulsación sencilla  sobre la  pequeña
marca del centro podremos plegar y desplegar el panel de forma rápida.
También  podemos ocultarlo  de forma  permanente pulsando  en la  {\em
equis} de la parte superior-derecha del panel.

Por  defecto,  en  el  reducido  espacio  del  panel  lateral  podemos
consultar  el historial  y los  marcadores, y  realizar búsquedas  por
palabras o  por temas relacionados  en Internet. Sin embargo,  como ya
hemos comentado, dichas funcionalidades son completamente ampliables.

\subsubsection{Preferencias}
\index{Mozilla!Preferencias}

Muchas de las características de {\sf Mozilla} son personalizables. En
la  Figura  \ref{web_mozilla_prefs}  podemos  observar  el  cuadro  de
diálogo  de preferencias  de  {\sf Mozilla}.  A  dicho cuadro  podemos
acceder desde la opción {\tt EDITAR $\rightarrow$ PREFERENCIAS}\dots

Las preferencias de {\sf Mozilla} están clasificadas en categorías. Al
seleccionar una  categoría podemos observar  en el lado derecho  de la
ventana  las  opciones  de configuración  relacionadas.  Además,  cada
categoría puede  contener subcategorías cuya  lista se despliega  o se
pliega con una pulsación simple del ratón sobre dicha categoría.

\begin{figura}{web_mozilla_prefs}{0.7}
\caption{Cuadro de diálogo de preferencias de Mozilla}
\label{web_mozilla_prefs}
\end{figura}

\begin{description}

\item[APARIENCIA] Preferencias tales como los  colores, o los tipos de
letras utilizados por el navegador  en la visualización de las páginas
web son establecidas en esta categoría. También podemos seleccionar el
idioma y  el {\em tema}  de la  aplicación. Seleccionar un  nuevo {\em
tema} cambia  el aspecto  visual de los  botones, cuadros  de diálogo,
menú,  barras de  herramientas y  otros objetos.  En Internet  resulta
sencillo encontrar todo tipo de  {\em temas} para personalizar nuestro
navegador.

\item[NAVIGATOR]  Las  preferencias   específicas  del  navegador  son
establecidas en esta categoría. Entre  ellas contamos con la dirección
de la página  de inicio, el buscador por defecto,  o la administración
del historial de direcciones. En ocasiones el contenido de las páginas
web  está disponible  en  varios idiomas.  En  esta categoría  podemos
establecer los idiomas en los que preferimos ver dichas páginas web.

\item[COMPOSER] En esta categoría podemos  encontrar toda una serie de
opciones  relacionadas con  el editor  de  páginas web  que viene  con
{\sf Mozilla}.

\item[MAIL \& NEWS] En caso de haber instalado el lector de noticias y
de  correo  electrónico  de  {\sf Mozilla},  podemos  acceder  a  esta
categoría para configurarlo.

\item[PRIVACIDAD  Y  SEGURIDAD]  Las   preferencias  de  privacidad  y
seguridad  determinan   las  características  del  filtrado   de  {\em
cookies}  e  imágenes. Además,  nos  permite  configurar el  navegador
para  que guarde  contraseñas o  datos de  formularios que  utilizamos
habitualmente.  También   nos  permite  gestionar  los   protocolos  y
certificados utilizados  durante las conexiones encriptadas.  Debido a
que la mayor  parte de esta información confidencial  es almacenada en
el disco duro,  {\sf Mozilla} permite la encriptación  de dichos datos
bajo una contraseña maestra. Con  ello se consigue que esa información
no pueda ser accesible a terceros.

\item[AVANZADAS] En  esta categoría  se configuran  aspectos avanzados
como el soporte de {\sf Java} y {\sf JavaScript}, la configuración del
caché, o el acceso  a la red a través de proxies.  La opciones de esta
categoría nos permiten,  por ejemplo, establecer qué  permitimos y qué
no permitimos hacer a las páginas web a las que accedemos.

\item[SIN CONEXIÓN  Y ESPACIO  EN DISCO  DURO] Configuración  sobre el
comportamiento del  navegador en los  modos de trabajo con  conexión y
sin conexión.

\end{description}

Podemos utilizar  la ayuda para  obtener información más  detallada de
cada una de las opciones.

\section{Transferencia de ficheros (FTP)}

{\bf FTP (File Transfer Protocol)} es un protocolo que se utiliza para
transferir información, almacenada en  ficheros, de una máquina remota
a  otra local,  o viceversa.  Para  poder realizar  esta operación  es
necesario conocer la dirección  IP o el nombre de la  máquina a la que
nos queremos  conectar para realizar  algún tipo de  transferencia. Es
fundamental distinguir entre máquina local y máquina remota:

\begin{description}

\item[Máquina local] Es aquella desde  donde nos conectamos para hacer
la transferencia, es decir, donde ejecutamos el cliente de FTP.

\item[Máquina  remota]  Es  aquella  a  la  que  nos  conectamos  para
transferir la información.

\end{description}


\subsection{Cliente estándar}

El cliente estándar de FTP es una aplicación de la consola que ejecuta
a través del comando {\tt ftp}\index{ftp}. A continuación veremos cómo
se utiliza.


\subsubsection{Inicio de sesión FTP}

Para  realizar  transferencias  de   ficheros  por  protocolo  FTP  se
establecen conexiones (sesiones)  entre la máquina local  y la remota.
Estas sesiones comienzan por la validación del usuario y prosiguen con
las transferencias. Finalmente la sesión se cierra. Veamos un ejemplo:

\begin{verbatim}
$ ftp euler
Connected to euler.fmat.ull.es.
220 ProFTPD 1.2.0pre10 Server (Debian) [euler.fmat.ull.es]
Name (euler:miguev): 
\end{verbatim}

El servidor  nos preguntará  un nombre de  usuario, y  seguidamente la
contraseña.  El nombre  que daremos  debe  ser una  cuenta de  usuario
válida  en el  servidor al  que  intentamos acceder,  y la  contraseña
lógicamente  debe ser  correspondiente  a ese  usuario. En  servidores
públicos suele existir una cuenta de  acceso anónima sólo para leer (o
tal vez una carpeta donde poner cosas pero no leer). Para acceder a un
FTP como  usuario anónimo se  utiliza el  nombre {\tt anonymous}  y se
proporciona la dirección de correo electrónico como contraseña.

Una vez introducido el nombre y la contraseña el servidor nos recibirá
y el programa cliente de FTP nos mostrará un prompt (indicador de
órdenes), manifestando así que está preparado para ejecutar las
órdenes que le demos. A partir de aquí se realizan las tareas deseadas
mediante los comandos de FTP que veremos más adelante.

\begin{verbatim}
$ ftp euler
Connected to euler.fmat.ull.es.
220 ProFTPD 1.2.0pre10 Server (Debian) [euler.fmat.ull.es]
Name (euler:miguev): miguev
331 Password required for miguev.
Password:
230 User miguev logged in.
Remote system type is UNIX.
Using binary mode to transfer files.
ftz>
\end{verbatim}


\subsubsection{Comandos del FTP}

El  protocolo  FTP dispone  de  unos  comandos estándar  suficientes
para  las operaciones  de  transferencia de  ficheros. A  continuación
presentamos un resumen de las mismas:

\begin{description}

\item[ascii] Para hacer la transferencia  en formato ASCII (ésta es la
opción por defecto).

\item[binary]  Para  hacer la  transferencia  en  formato binario,  se
utiliza el comando.

\item[!  comando] Para  ejecutar  un comando  desde  el intérprete  de
comandos en la máquina local.

\item[cd directorio\_remoto] Para  moverse de un directorio  a otro en
la máquina remota.

\item[delete  fichero\_remoto] Para  borrar un  fichero en  la máquina
remota.

\item[delete  ficheros\_remotos] Para  borrar  varios  ficheros en  la
máquina remota.

\item[get   fichero\_remoto   fichero\_local]   Transfiere   el   {\tt
fichero\_remoto}  desde   la  máquina   remota  a  la   máquina  local,
guardándolo con el nombre {\tt fichero\_local} en la máquina local.

\item[help]  Proporciona una  lista  de  los comandos  del  FTP de  la
máquina local.

\item[help   comando]  Proporciona   información   sobre  el   comando
especificado, correspondiente a la máquina local.

\item[lcd directorio\_local] Para  moverse de un directorio  a otro en
la máquina local.

\item[lcd unidad:] Para  cambiar de una unidad de disco  a otra, en el
caso  particular de  que la  máquina  local esa  un PC  con Windows  o
MS-DOS.

\item[lls directorio\_local] Para listar el contenido de un directorio
en la máquina local.

\item[[ls|dir]  directorio\_remoto] Para  listar  el  contenido de  un
directorio en la máquina remota.

\item[mget lista\_ficheros\_remotos] Transfiere  los ficheros listados
desde la máquina remota a la máquina local.

\item[mkdir directorio\_remoto] Para crear un directorio en la máquina
remota.

\item[mput lista\_ficheros\_locales] Transfiere  los ficheros listados
desde la máquina local a la máquina remota.

\item[prompt] (Des-)activa  el modo interactivo de  las transferencias
de ficheros múltiples.

\item[put   fichero\_local   fichero\_remoto]   Transfiere   el   {\tt
fichero\_local} desde la máquina local a la máquina remota, guardándolo
con el nombre {\tt fichero\_remoto} en la máquina remota.

\item[pwd] Para saber  el directorio en el que se  está, en la máquina
remota.

\item[quit] Terminar la sesión actual y finalizar el programa.

\item[rmdir  directorio\_remoto]  Para  borrar  un  directorio  en  la
máquina remota.

\end{description}

Como ya hemos  comentado, una vez iniciada la  sesión podemos utilizar
estos  comandos  para realizar  todo  tipo  de operaciones  sobre  los
ficheros de la máquina remota.

Es importante  destacar que  por defecto  las transferencias  se hacen
en  formato  ASCII.  Es  decir, el  protocolo  considera  que  estamos
transfiriendo  archivos  de  texto  plano,  por  lo  que  realiza  las
conversiones oportunas entre el formato  utilizado en la máquina local
y el utilizado en la  máquina remota. Cuando transferimos archivos que
deben  ser copiados  {\em tal  cual},  es decir,  sin alteraciones  de
ningún tipo, como es el caso de ejecutables, imágenes y demás archivos
binarios, es  necesario especificar  que deseamos  el modo  binario de
transferencia.  Para ello  ejecutamos  el comando  {\tt binary}  justo
antes del comando que da comienzo a la transferencia.

\subsection{gFTP}
\index{gFTP}

Obviamente, el  comando {\tt ftp} no  es el único programa  cliente de
FTP disponible.  Existen multitud de  programas clientes de  FTP tanto
para la consola como para el entorno gráfico.

{\sf gFTP} es el cliente FTP del entorno de escritorio {\sf GNOME}. Se
trata,  por  tanto,  de  un  cliente en  modo  gráfico  diseñado  para
facilitar el acceso a  los recursos FTP. El uso de  un cliente en modo
gráfico nos permite  olvidarnos de los comandos del  protocolo FTP. En
su lugar todas  las operaciones se reducen a  sencillas acciones sobre
la interfaz gráfica.

\begin{figura}{ftp_gftp}{0.8}
\caption{Ventana de gFTP en una conexión a {\tt ftp.es.debian.org}}
\label{ftp_gftp}
\end{figura}

En la  Figura \ref{ftp_gftp}  podemos observar la  ventana de  un {\sf
gFTP} con  una conexión  a {\tt  ftp.es.debian.org}. Para  iniciar una
sesión de FTP con {\sf gFTP} debemos recurrir a la barra situada justo
debajo de  la barra de  menús. En  ella debemos especificar  los datos
requeridos para realizar la conexión.

\begin{description}

\item[SERVIDOR]  En  este  campo  debemos escribir  el  nombre  de  la
máquina remota  a la  que nos  vamos a  conectar. En  el de  la Figura
\ref{ftp_gftp} fue {\tt ftp.es.debian.org}.

\item[PUERTO] Cada servicio que se  ofrece en Internet tiene un puerto
asociado. El  puerto por  defecto para  el servicio de  FTP es  el 21,
aunque algunas máquinas dan dicho  servicio en otro puerto cualquiera.
En  este campo  se debe  especificar el  puerto en  el que  la máquina
remota está  esperando las peticiones  de FTP.  Si no se  indica nada,
como  es nuestro  caso, {\sf  gFTP} asumirá  que queremos  utilizar el
puerto por defecto, es decir, el 21.

\item[USUARIO] En este campo debemos  indicar el nombre de usuario con
el  que queremos  acceder  a  la máquina  remota.  Si  el servidor  es
público,  lo más  probable que  utilizando el  nombre {\tt  anonymous}
podamos acceder a través de la cuenta de usuario anónima.

\item[CONTRASEÑA]  La  contraseña  asociada  al  usuario  con  el  que
queremos acceder. En caso de acceder  a través de la cuenta de usuario
anónima debemos indicar nuestra dirección de correo electrónico.

\item[PROTOCOLO] El último campo nos  permite indicar el protocolo que
vamos  a utilizar  para nuestra  conexión.  {\sf gFTP}  no sólo  puede
utilizar el  protocolo FTP  para realizar transferencias  de archivos.
Por  ejemplo,  podemos  utilizar  SSH2  para  realizar  transferencias
de  archivos  encriptadas.  Esto  garantiza que  tanto  nuestro  datos
(nombre de  usuario, contraseña, etc)  como los de los  archivos estén
seguros frente a intentos de  interceptación de las comunicaciones. Es
importante destacar que el protocolo FTP no es un protocolo seguro, en
el  sentido de  que nuestra  contraseña y  todos los  demás datos  que
viajen por la conexión son fácilmente interceptables.

\end{description}

Tras completar  los datos podemos pulsar  en el botón de  la izquierda
para iniciar la  sesión. {\sf gFTP} se encarga de  enviar los comandos
necesarios para iniciar la conexión, evitándonos el engorroso problema
de tener que conocerlos. El botón  de la izquierda nos permite iniciar
una conexión si ésta  no ha sido iniciada, o terminarla  si ya ha sido
iniciada correctamente. Sin embargo, si queremos detener el intento de
conexión en curso debemos recurrir al botón de la derecha. Dicho botón
se utiliza para abortar cualquier tarea que {\sf gFTP} esté realizando
en el momento actual.

En la parte inferior de la  ventana de {\sf gFTP} podremos observar un
registro de la  conexión. En {\em rojo} veremos  información propia de
{\sf gFTP},  en {\em verde}  observaremos los comandos que  {\sf gFTP}
envía a la máquina remota, y  en {\em azul} podremos ver la respuestas
de la  máquina remota  a dichos  comandos. Sea cual  sea la  tarea que
estemos  realizando  siempre quedará  registrada  en  esa zona  de  la
ventana. Por eso  suele ser importante estar atento a  ella para saber
en  todo  momento lo  que  está  sucediendo.  Utilizando la  barra  de
desplazamiento de la derecha podremos ver los mensajes antiguos, o que
hayan pasado demasiado rápido.

El centro de la  ventana de {\sf gFTP} está dividido  en dos áreas que
podemos utilizar para navegar por el  árbol de directorios. El área de
la  izquierda  podemos  utilizarla  para  movernos  por  el  árbol  de
directorios de la  máquina local. Mientras que el área  de la derecha,
siempre y  cuando {\sf gFTP}  esté conectado, podemos  utilizarla para
movernos por  el árbol de directorios  de la máquina remota.  En ambos
lados podemos  utilizar el botón  derecho del ratón para  desplegar un
menú  de  contexto con  un  amplio  conjunto  de opciones.  Con  ellas
podemos crear directorios, renombrar  archivos, ver y editar archivos,
modificar sus atributos, borrar, etc.

Mientras que la  doble pulsación sobre una directorio  nos permite ver
su  contenido,  una  doble  pulsación  sobre  un  archivo  inicia  una
transferencia  para que  el archivo  sea enviado  al otro  lado de  la
conexión.  El mismo  efecto  conseguimos si  seleccionamos  uno o  más
archivos (p.  ej. pulsando con el  botón izquierdo del ratón  sobre el
archivo deseado mientras mantenemos pulsada la tecla {\tt Ctrl} o {\tt
Shift})  y los  arrastramos al  área  del otro  lado. También  podemos
conseguir el mismo efecto si pulsamos sobre los botones situados en el
centro  de las  dos  áreas.  Dichos botones  nos  permiten iniciar  la
transferencia de archivos en el sentido que más nos convenga.

Justo entre el área donde se  registran las tareas realizadas por {\sf
gFTP} y  la que utilizamos  para explorar los árboles  de directorios,
tenemos la cola de  transferencias. Todas las transferencias iniciadas
o  pendientes  se  muestran  en   dicha  área,  así  como  información
relacionada con ellas. Utilizando el menú de contexto que se despliega
con el  botón derecho de nuestro  ratón podemos detener o  iniciar una
transferencia,  o  alterar  el  orden  de la  cola  para  decidir  qué
transferencia se  iniciará después de  que se complete la  actual. Por
defecto,  las  transferencias  desde máquinas  remotas  diferentes  se
realizan en paralelo, mientras que  las transferencias desde una misma
máquina remota que se van realizando secuencialmente.

\section{Acceso remoto (SSH)}

En muchas ocasiones resulta interesante acceder a una máquina remota y
trabajar como  si estuviéramos  físicamente frente  a ella.  Es decir,
hablamos de  poder ejecutar  comandos en dicha  máquina sin  tener que
trasladarnos  a escribirlos  en  su teclado.  Este  tipo de  servicios
existe desde  los orígenes de  Internet pero siempre ha  entrañado sus
riesgos dado que  es necesaria la validación remota  del usuario. Para
entender  de  lo  que  estamos  hablando sólo  es  necesario  que  nos
aproximemos al servicio de TELNET (cuyo cliente suele estar disponible
bajo el comando del mismo nombre).

TELNET\index{TELNET} fue el primer  servicio de acceso remoto diseñado
para  Internet.  Básicamente el  comando  {\tt  telnet} toma  nuestras
pulsaciones de  teclado y las transmite  por la red hasta  el servidor
TELNET de la máquina remota a  la que estamos conectados. Éste toma la
salida  a terminal  de la  aplicaciones que  vayamos ejecutando  y las
envía  a  nuestro cliente  en  la  máquina  local. Ese  mecanismo  tan
sencillo  resuelve  el problema  del  acceso  remoto; excepto  por  un
pequeño problema. Toda  la comunicación se realiza en  texto plano. Es
decir, el texto se envía tal  cual, sin mecanismo alguno de compresión
y/o  encriptación  que  altere  las  cadenas de  texto.  Todo  lo  que
escribimos  (por  ejemplo,  comandos  del intérprete  de  comandos)  y
recibimos (por ejemplo, correo, informes,  cartas) puede ser visto por
quienes intercepten  nuestros paquetes. Esto es  especialmente crítico
durante  el proceso  de conexión,  momento  en el  que debemos  enviar
nuestro nombre  de usuario  y contraseña  para realizar  la validación
remota a la  que aludimos anteriormente. Es evidente que  en esa etapa
nuestra password, y con ella el acceso al servidor bajo nuestro nombre
de usuario, queda al descubierto.

A   efectos  prácticos   la   mayor  parte   de   los  protocolos   de
Internet  se  fundamentan en  TELNET.  Servicios  como la  World  Wide
Web  (HTTP\index{HTTP}),  el   correo  electrónico  (SMTP\index{SMTP},
POP\index{POP},  IMAP\index{IMAP}),  las  transferencias  de  ficheros
(FTP\index{FTP}), la administración de la red (SMNP\index{SMNP}), etc.
utilizan  protocolos  cuyas  especificaciones  indican  claramente  su
vínculo con TELNET. Si ya resulta  grave que en todas esas situaciones
nuestros datos queden al descubierto, más lo es aun cuando hablamos de
permitir o  denegar el  acceso a  un servicio tan  crítico como  es el
acceso remoto. Éste suele poner las  cosas más fáciles que ningún otro
para quien desee atacar el sistema.

Dos son las cosas que debemos recordar de todo lo anterior:

\begin{itemize}

\item {\bf  En condiciones normales  la mayor  parte de los  datos que
enviamos y/o  recibimos de Internet  están al descubierto.}  Es decir,
una  vez interceptados  pueden  ser leídos  directamente sin  requerir
ningún proceso intermedio.

\item  {\bf  No debemos  utilizar  TELNET  bajo ningún  concepto.}  El
comando {\tt telnet} puede ser una herramienta muy práctica en algunos
casos, pero peligrosa si la usamos para acceso remoto.

\end{itemize}

La  cuestión  ahora  es  cómo podemos  resolver  estos  problemas.  La
respuesta  es  utilizando  {\sf Secure  Shell  (SSH\index{SSH})}. Este
programa trabaja de forma similar  a TELNET solo que encriptando tanto
la información que  es transmitida a la máquina remota  como la que es
enviada por ésta. El resultado es que aunque la comunicación pueda ser
interceptada los  datos resultarán ininteligibles. En  realidad SSH es
un  paquete de  comandos  relacionados con  las transacciones  seguras
de  información en  la  red, que  utiliza  diferentes mecanismos  para
garantizar esa seguridad. De todos  ellos el comando {\tt ssh} esconde
el  programa de  acceso  remoto  del cual  la  forma  más sencilla  de
utilizarlo es:

\begin{verbatim} 
$ ssh usuario@máquina_remota 
\end{verbatim}

Por  ejemplo si el  usuario {\tt cila} quiere  acceder al servidor
{\tt gulic.org} ejecutaría lo siguiente: 

\begin{verbatim} 
$ ssh cila@gulic.org
\end{verbatim}

La primera vez que accedamos a una máquina el programa nos mostrará su
{\em huella dactilar} y pedirá que confirmemos que queremos establecer
la conexión. Esto  permite utilizar políticas de seguridad  en las que
podamos  verificar  la  autenticidad  de  la  máquina  a  la  que  nos
conectamos,  evitando  que  pueda  haber  sido  suplantada  por  otra.
Si  estamos seguros  de  la autenticidad  del  sistema remoto  debemos
contestar que  sí. La huella dactilar  es almacenada y vinculada  a la
dirección de la  maquina remota, por lo que nunca  más recibiremos una
mensaje de verificación  como el anterior. Exceptuando el  caso en que
la máquina haya cambiado, y con  ello su huella dactilar, situación en
la que probablemente estemos ante una posible suplantación. Si todo va
bien {\tt  ssh} nos  pedirá la  password del usuario.  En caso  de ser
autentificados dispondremos de  acceso remoto sobre el  sistema. Si al
ejecutar el  comando no especificamos  el nombre de usuario  ({\tt ssh
gulic.org}) el  programa utilizará por defecto  nuestro nombre
en la máquina local.

A la  hora de trabajar  con {\tt ssh}  debemos tener algunas  cosas en
cuenta. En  caso de  duda podemos  recurrir a  las páginas  del manual
({\tt  man ssh})  donde encontraremos  detallada información  así como
referencias  a  las  otras  herramientas del  paquete  SSH.  Si  acaso
destacar que debido a que el carácter {\tt \~{} } tiene un significado
especial  para el  {\tt ssh},  si  queremos escribirlo  en la  máquina
remota tendremos que pulsar {\tt \~{}\~{} } en nuestro teclado.

Aunque {\tt ssh} garantiza el  acceso remoto seguro no proporciona por
sí solo la  transferencia segura de archivos. Para ello  se utiliza el
comando {\tt scp}\index{scp} que tiene la siguiente forma:

\begin{verbatim}
$ scp usuario@máquina_origen:archivo_origen \
> usuario@máquina_destino:archivo_destino
\end{verbatim}

El  cual copia  {\tt archivo\_origen}  desde la  {\tt máquina\_origen}
hasta el {\tt archivo\_destino} en la {\tt máquina\_destino}. Si no se
especifica alguno  de los nombres de  máquina, el {\tt scp}  asume que
estamos  hablando  del  sistema  local.  El  siguiente  comando  copia
el  archivo  {\tt  mi\_archivo}  desde la  máquina  local  hasta  {\tt
gulic.org}:

\begin{verbatim}
$ scp mi_archivo gulic.org:
\end{verbatim}

Mientras que el siguiente comando hace lo contrario:
      
\begin{verbatim}
$ scp gulic.org:mi_archivo .
\end{verbatim}

Si al especificar  la ruta de archivo en la  máquina remota lo hacemos
de forma  relativa (o sea  sin usar {\tt /})  la ruta será  relativa a
nuestro directorio personal  en la maquina remota.  Esto significa que
el comando  anterior copia {\tt mi\_archivo}  desde nuestro directorio
personal en {\tt gulic.org}. Sin embargo, el siguiente comando
copia el  fichero {\tt /tmp/mi\_archivo}  (ruta absoluta) en  el mismo
servidor.

\begin{verbatim}
$ scp gulic.org:/tmp/mi_archivo .
\end{verbatim}

Otro comando interesante es {\tt sftp}\index{sftp} que nos proporciona
los  mismos servicios  que {\tt  ftp} solo  que de  forma segura.  Sin
embargo aún no está disponible en todos los sistemas.

Una de  las preguntas más  habituales de los  usuarios de Linux  es si
pueden acceder a  su servidor SSH en Linux desde  una máquina local en
otro  sistema operativo.  La  respuesta  es que  SSH  es un  protocolo
estándar por  lo que todo depende  de la disponibilidad de  un cliente
para su plataforma. En la  actualidad existen versiones libres para la
mayor  parte  de  sistemas  operativos. De  entre  ellas  destacaremos
{\em  PUTTY} %[pag. \pageref{putty}]  
como uno  de los  mejores clientes
TELNET/SSH para sistemas Microsoft® Windows®.


\section{Correo electrónico}

\subsection{Mutt }


{\sf Mutt}  es un programa  cliente de  correo electrónico, lo  que en
inglés  se denomina  un MUA  (Mail User  Agent, agente  de correo  del
usuario).  Es un  programa ``de  consola'',  lo que  significa que  no
necesita un  entorno de ventanas  para ejecutarse. Al igual  que otros
programas basados  en pulsasiones  de teclas,  {\sf Mutt}  resulta ser
poco intuitivo  al principio. Afortunadamente, se  encuentra traducido
al castellano y eso ayuda bastante.

Vamos a usar  {\sf Mutt} para familirizarnos un poco  él, verás que es
simple.  Abrimos una  ventana de  emulador de  terminal y  ejecutan el
comando {\tt mutt}.

\begin{verbatim}
      $ mutt
\end{verbatim}

Si nos fijamos  en la primera línea de la  pantalla vemos que aparecen
listadas  una  serie  de  teclas  con  sus  acciones  asociadas,  {\tt
q:Salir}, para salir, {\tt d:Sup} para suprimir un mensaje, etc. Vemos
un ejemplo de uso para hacernos una idea de las funciones básicas.

%        El  que se haya fijado  en las dos últimas  líneas habrá
%       visto que aparece lo siguiente: 
% 
% \begin{verbatim}
% ---Mutt: (ningún buzón) [Msgs:0]---(threads/date)-----------------------(all)---
% /var/spool/mail/miguev: No existe el fichero o el directorio (errno = 2)
% \end{verbatim}
 

% Esto significa que mutt está buscando  el correo del usuario en {\tt
% /var/spool/mail/miguev}  (en  el caso  del  usuario  miguev). No  se
% asusten. Lo que pasa es que el correo está en esa carpeta pero no en
% el terminal  donde están  sentados, sino en  el servidor  de correo.
% Para poder  usar el  correo con  Mutt hay que  entrar primero  en el
% servidor, lo que podemos hacer rápidamente  con lo que ya sabemos de
% SSH. Entramos en el servidor y ejecutamos mutt allí:


% \begin{verbatim}
%       $ ssh euler.fmat.ull.es
%       miguev@10.0.1.2's password: 
% \end{verbatim}
% 
% Una vez dentro del servidor podemos ya ejecutar mutt y utilizar
% el correo directamente desde el servidor. Esto que parece inútil
% tiene  su  utilidad.  Imagina  que  estás  en  un  ordenador  en
% cualquier lugar del mundo (con acceso a internet) y quieres leer
% tu correo en la facultad, pero no quieres bajártelo. Utilizas un
% programa  cliente de  SSH para  entrar  en el  servidor y  desde
% dentro usas el correo como  si lo tuvieras delante, aunque estés
% dentro del servidor.  El mayor problema que esto  presenta es la
% lentitud del protocolo SSH cuando la  conexión es a través de un
% módem de línea telefónica.

Ejecutamos el  comando {\tt mutt} y  vemos en el terminal  un programa
casi todas  las líneas  vacías, salvo  la primera  y las  dos últimas.
Probablemente en  el momento de  abrir {\sf  Mutt} por primera  vez no
veamos nada interestante, pero aquí tienen  un ejemplo de una lista de
mensajes vista desde {\sf Mutt}.

 \begin{verbatim}
q:Salir  d:Sup.  u:Recuperar s:Guardar m:Nuevo r:Responder g:Grupo
 43 Oct 27 Teresa Gonzalez ( 0)  *>Re: [l-gulic] CILA LLENO
 44 Oct 27 Lucas Gonzalez  ( 0)   >cvs
 45 Oct 27 Miguel Ángel Vi ( 0) Bienvenido al calendario de la 
 46 Oct 28 Teresa Gonzalez ( 0) Re: [l-gulic] Cambios en el CVS
 47 Oct 28 Pedro Gonzalez  ( 0)  *>
 48 Oct 28 Carlos de la Cr ( 0) La pu~etera introduccion :-)
 49 Oct 28 carlos de       ( 0) maldito texto sobre java
 50 Oct 28 frodo@fmat.ull. ( 0) MUY IMPORTANTE!!
 51 Oct 28 frodo@fmat.ull. ( 0) ahora te llega?
 52 Oct 28 frodo@fmat.ull. ( 0) ¡como no te llegue! ..grrr
 53 Oct 28 Administrador d ( 0) Re: instala esto
 
 
 ---Mutt: /var/spool/mail/miguev [Msgs:53 425K]---(threads/date)-------
\end{verbatim}

Vamos a enviar un email a alguien que esté con nosotros en el aula, de
esa forma  cada uno  enviamos un  correo y  recibimos otro.  Para ello
pulsamos  la tecla  {\tt m}  y  veremos como  en la  última línea  nos
pregunta por el destinatario del mensaje ({\tt To:}). Introducimos ahí
la dirección de email a la que enviaremos el mensaje:

\begin{verbatim}
To: frodo@fmat.ull.es
\end{verbatim}

Seguidamante {\sf Mutt} nos preguntará por el asunto del mensaje ({\tt
Subject:}).  Es importante  poner un  asunto al  mensaje, para  que el
destinatario  pueda tener  una idea  de qué  es ese  mensaje antes  de
abrirlo.  En  un  tiempo  en  que el  contagio  de  virus  por  correo
eletrónico es preocupantemente frecuente,  resulta muy molesto recibir
un mensaje de email sin asunto.

\begin{verbatim}
Subject: ¡Feliz cumpleaños!
\end{verbatim}

Una  vez que  {\sf Mutt}  ya  sabe el  destinatario del  mensaje y  el
asunto,  ejecuta el  editor que  tengamos  definido en  el fichero  de
configuración {\tt  \~{}/.muttrc}. Editamos el mensaje  que queramos y
salimos del editor {\bf guardando  el mensaje}, importante esto último
ya  que si  salimos  del  editor sin  guardar  el  mensaje {\sf  Mutt}
cancelará el  envío. Una vez  que salimos  del editor {\sf  Mutt} está
preparado para  enviar el mensaje,  pero nos ofrece la  posibilidad de
hacer aún varias cosas.


\begin{verbatim}
y:Mandar  q:Abortar  t:To  c:CC  s:Subj  a:Adjuntar archivo  d:Descrip
    From: Miguel Ángel Vilela <miguev@fmat.ull.es>
      To: frodo@fmat.ull.es
      Cc:
     Bcc:
 Subject: Hola pringao
Reply-To: Miguel Ángel Vilela <miguev@fmat.ull.es>
     Fcc:
     Mix: <no chain defined>
     PGP: En claro

-- Archivos adjuntos
- I     1 /tmp/mutt-euler-19795-2          [text/plain, 8bit, iso-8859-1


-- Mutt: Crear mensaje
\end{verbatim}

Como podemos apreciar  en el ejemplo, tenemos varias  opciones con sus
teclas asociadas  en la  primera línea.  Para cambiar  el destinatario
pulsaríamos {\tt t}, para enviar  una copia a alguien pulsaríamos {\tt
c}, para  editar el mensaje  de nuevo  pulsaríamos {\tt e},  etc. Para
enviar el mensaje pulsamos {\tt  y}. Entonces {\sf Mutt} nos devolverá
a la primera pantalla, pero  mostrando en la primera línea información
acerca del envío del mensaje. Debería aparecer:

\begin{verbatim}
Mensaje enviado.
\end{verbatim}

El resto  del manejo básico de  {\sf Mutt} es bastante  intuitivo y no
presenta dificultades. Tan  sólo hay que acostumbrarse  a trabajar con
pulsaciones de  teclas en lugar de  manejar el ratón. Si  en cualquier
momento  deseamos información  más  detallada acerca  de las  opciones
disponibles, pulsamos {\tt ?}.

\begin{verbatim}
i:Salir  -:PágAnt  <Space>:PróxPág  ?:Ayuda 
^B          M |urlview\n           call urlview to extract URLs out of 
^D          delete-thread          suprimir todos los mensajes en este 
^E          edit-type              editar el tipo de archivo adjunto
^F          forget-passphrase      borrar contraseña PGP de la memoria
<Tab>       next-new               saltar al próximo mensaje nuevo
<Return>    display-message        mostrar el mensaje
^K          extract-keys           extraer claves PGP públicas 
^N          next-thread            saltar al próximo hilo
^P          previous-thread        saltar al hilo anterior
^R          read-thread            marcar el hilo actual como leído
^T          untag-pattern          quitar marca de los mensajes que coi
^U          undelete-thread        restaurar todos los mensajes del hil
<Esc><Tab>  previous-new           saltar al mensaje nuevo anterior
<Esc>C      decode-copy            crear copia decodificada (text/plain
<Esc>V      collapse-all           colapsar/expander todos los hilos
<Esc>b      M /~b                  search in message bodies
<Esc>c      change-folder-readonly abrir otro buzón en modo de sólo lec
<Esc>d      delete-subthread       suprimir todos los mensajes en este 
<Esc>e      resend-message         usar el mensaje actual como base par
+                                  nuevo
Ayuda para index                                               -- (15%) 
\end{verbatim}

\subsection{Fetchmail}

{\sf Fetchmail} es una aplicación que nos permite descargar de nuestro
servidor  de  correo  nuestros  e-mails, puede  como  cliente  y  como
servicio. La funcion del {\sf  Fetchmail} es conectarse al servidor de
correo, bajarse  los e-mails  y luego pasarle  los mensajes  de correo
electrónico a  nuestro servidor  de correo  SMTP instalado  en nuestra
máquina (Debian por ejemplo instala  por defecto el {\sf exim}, aunque
existen otros como {\sf qmail} y {\sf postfix}) y luego el servidor lo
envía a los buzones de los usuarios.

Para  la  configuración del  {\sf  Fetchmail}  se utiliza  el  fichero
\verb|~/.fetchmailrc|. Vamos a utilizar un ejemplo sencillo.

\begin{verbatim}
poll pop.gulic.org proto POP3 
   user "faraox@gulic.org" is there with password "coche" is faraox here 
\end{verbatim}

La primera línea llama, {\tt  poll}, al servidor {\tt pop.gulic.org} e
indica  su  protocolo con  la  opción  {\tt proto},  seguidamente  del
protocolo, {\tt  POP3}. En la siguiente  línea se define el  uruario y
contraseña y  las opciones  del usuario.  Con la  opción {\tt  user is
there} definimos el nombre de  usuario, {\tt faraox@gulic.org} y luego
se define  con {\tt with  password} nuestra contraseña {\tt  XXXX}, la
opción {\tt is here}  nos indica el usuario al que  debe ir el correo,
{\tt faraox}.

% si alguien tiene  tiempo que añada algo sobre revisar  el correo con
% comandos, sin utilizar el archivo de configuración

Una utilidad interesante  si estamos manejando una  máquina con varios
usuarios es la utilización de el {\sf Fetchmail} como demonio. Creamos
el fichero  de configuración  y lo movemos  a /etc/fetchmailrc.  En la
cabecera usamos  la opción {\tt  set daemon}  seguida de el  número en
segundo de el intervalo de tiempo con el que queremos que se ejecute.

\begin{description}

\item[dns] Chequea las direcciones dns(por defecto).

\item[no  dns] No  chequea  direcciones  dns(recomendable, aumenta  la
velocidad).

\item[timeout] Especificamos el tiempo de inactividad del servidor, en
segundos.

\item[checkalias] Hace una comparación de IP.

\item[no checkalias] No hace comparación de IP(por defecto).

\item[folder]  Especifica la  carpeta  remota donde  se consultará  el
correo.

\item[mda]  Especifica   nuestro  programa   de  filtrado   de  correo
(mailfilter,etc).

\item[keep] No borra los mensajes del servidor.

\item [preconnect] Comando para ser ejecutado antes de la conexión.

\item[postconnect] Comando para ser ejecutado depues de la conexión.

\item[limit] Especifica el tamaño máximo de los mensajes.

\end{description}

Otra  utilidad  interesante es  {\tt  fetchmailconf}.  Es un  programa
gráfico  que  nos  permite   configurar  de  forma  intuitiva  nuestro
{\tt .fetchmailrc}.
