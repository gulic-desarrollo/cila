%Autor: amd77
%amd77: 7

\chapter{Documentacion y ayuda}
\label{documentacion.tex}
\label{documentacion}
\index{Documentación}
\index{Ayuda}

El primer  contacto con un entorno  nuevo como Linux siempre  se suele
hacer junto a alguien  que te oriente (al que de  ahora en adelante me
referiré con el término {\em gurú de turno}). Cuando ese gurú vuelve a
sus tareas cotidianas normalmente te quedas solo y te toca buscarte la
vida. En este capítulo veremos diferentes formas con las que conseguir
el conocimiento  suficiente para poderse  buscar la vida solo  y hasta
convertirte en un  gurú también, pues eso  es lo que él  ha hecho para
saber lo que sabe. Veremos que  una gran parte ya lo tienes almacenado
en tu disco  duro tras instalar Linux, y te  diremos cómo conseguirlo,
mientras que  el que  te falte  lo podrás  encontrar en  la red,  y te
diremos dónde buscarlo.

\section{Pasos para encontrar ayuda}

Estos pasos  son para cuando  no sabes  como funciona un  programa. Si
realmente lo  que pasa  es que  no entiendes  nada, deberías  pasar al
siguiente punto, Recursos, donde se detallan guías y tutoriales. Estos
pasos que aquí se detallan están en orden de dificultad creciente.

\subsection{Información en línea de comandos}
\index{Ayuda!--help}

Cada comando en Linux siempre está  documentado, ya sea de una forma u
otra. Cuando  conocemos un comando  y no  sabemos lo que  hace nuestra
primera  opción puede  ser llamar  al  comando con  el argumento  {\tt
--help} o  {\tt -h},  con lo  que en la  mayoría de  los casos  se nos
mostrará un pequeño resumen  de su uso y de la  forma de llamarlo, con
lo que nuestras  dudas comenzarán a disiparse. Este es  un estándar de
facto bastante extendido.

\subsection{Páginas del manual}
\index{Documentacion!man}

Si aún así  seguimos con dudas, o si queremos  ver una descripción más
pormenorizada de lo que hace cada uno de los parámetros de que dispone
el comando, probaremos si el comando dispone de {\em página de manual}
tecleando {\tt  man comando}.  En caso  afirmativo, nos  aparecerá una
descripción más detallada  del comando, comenzando por  un resumen, la
descripción de la sintaxis, del  funcionamiento, de las opciones y por
último  información  de  bugs,  ficheros  relacionados  y  referencias
cruzadas. Desde  los primeros UNIX las  páginas de manual han  sido el
principal  medio de  documentación  sobre comandos  concretos y  todas
coinciden en los mismos contenidos mencionados anteriormente.

Algo que debemos  tener en cuenta es que debido  también a la herencia
UNIX,  el primer  idioma en  que  se escriben  todos estos  documentos
técnicos suele ser  el inglés, por ser el idioma  de mayor difusión en
el mundo  de la informática. Cuando  el documento es un  documento muy
usado, entonces voluntarios lo traducen a su idioma nativo. El caso es
que las páginas de manual de  los comandos más usados seguro que están
ya  traducidos  al  español.  Si  nada  en  el  sistema,  ni  siquiera
aplicaciones, aparecen en español, entonces  es posible que haya algún
problema con las variables de entorno del idioma (para más información
consultar {\tt man setlocale}). Si hay aplicaciones que sí aparecen en
español,  entonces puede  ser que  estas  páginas de  manual no  estén
instaladas y tengas que hacerlo (depende de la distribución, en Debian
el paquete se llama {\tt manpages-es}).

Relacionados con el comando {\tt man} existen dos comandos. El primero
es  que {\tt  apropos}, que  busca la  palabra que  le decimos  en los
resúmenes  de las  páginas  de  manual instaladas  y  nos muestra  las
páginas que coinciden. El segundo es  de menor utilidad, se llama {\tt
whatis}, y simplemente muestra la línea de descripción del comando, no
la ayuda completa.

Con la llegada de los entornos gráficos, las páginas del manual se han
modernizado también y se han  integrado en los entornos más populares.
En  el caso  del entorno  de escritorio  {\sf KDE},  abriendo un  {\sf
Konqueror} y tecleando  en la barra de dirección  {\tt man:} obtenemos
un listado de todas las páginas de manual organizadas en categorías, o
bien  seguido  de una  barra  y  el  nombre  del comando  nos  aparece
directamente su página,  por ejemplo, {\tt man:/ls}. En  caso de tener
varios idiomas también  aparecen allí. En el caso  del escritorio {\sf
GNOME 1.2}, todo lo tenemos  en la aplicación {\sf Gnome-help-browser}
(sistema integral de ayuda).

\subsection{Páginas info}
\index{Documentacion!info}

Las  páginas  del manual  para  escribirse  utilizan un  preprocesador
llamado  {\tt  troff}. Con  la  llegada  de  nuevos y  más  versatiles
preprocesadores,  las páginas  del manual  tuvieron la  posibilidad de
vincularse unas con otras, es decir,  disponer de un índice del que se
va a otras páginas  y sobre estas, a su vez, puede  irse a otras, etc:
organizarlas como una obra de  consulta más que como páginas aisladas.
El primer  fruto de  esto fueron  las páginas {\tt  info}, que  son un
sistema de ayuda navegable.  Actualmente los preprocesadores como {\tt
sgmltools} permiten producir formatos de salida info, HTML, y otros, a
partir de  un único texto  de ayuda escrito  por el programador  de la
aplicación, y cualquier ahorro de tiempo es una ventaja.

El visualizador  de páginas info,  igual que man, ha  evolucionado con
los entornos gráficos.  El visualizador con interfaz en  modo texto se
ejecuta  tecleando con  {\tt info}  o bien  {\tt info  comando} si  ya
sabemos  lo que  deseamos ver.  Una  vez dentro,  con tabulador  vamos
recorriendo cada  uno de  los enlaces. Cuando  queremos entrar  en uno
pulsamos {\tt Enter}. La estructura  es jerárquica, en forma de árbol,
y podemos  movernos a los nodos  de arriba ({\tt u}),  siguiente ({\tt
n}) y anterior ({\tt p}).

En modo gráfico, en el escritorio  {\sf KDE}, al igual que las páginas
del manual, simplemente  tecleamos en la barra de  direcciones de {\sf
Konqueror} {\tt info:} para verlo todo,  o bien seguido de una barra y
el nombre  de un  tema concreto, como  por ejemplo,  {\bf info:/libc}.
Para {\sf GNOME 1.2},  la aplicación {\sf Gnome-help-browser} (Sistema
integral de ayuda) nos permite visualizarlo también.

\subsection{Documentación DVI/PS/PDF/HTML}
\index{Documentacion!dvi}
\index{Documentacion!ps}
\index{Documentacion!pdf}
\index{Documentacion!html}

El  siguiente  paso en  documentación  se  dió  con  \LaTeX y  con  la
evolución del {\tt sgmltools} hacia {\sf DocBook}. Con ambos sistemas,
escribir manuales y documentos se  facilita enormemente, y a partir de
un  único  documento  se  pueden  conseguir  versiones  en  diferentes
formatos:

\begin{itemize}

\item {\bf DVI (DeVice Independent file)}: Es el formato de salida del
procesador  \LaTeX.  Sólo  se  usa  en  documentos  sin  imágenes.  Se
visualiza, entre otros, con el programa {\tt xdvi}.

\item  {\bf  PS  (PostScript)}:  Es   un  formato  de  impresión  para
impresoras de  calidad profesional y  resulta muy interesante  pues se
ve  perfecto  tanto en  impresoras  profesionales  como en  impresoras
normales. La visualización en pantalla aparece tal cual va a salir por
impresora. Se puede visualizar con  {\tt gs}, {\tt gv}, {\tt gnome-gv}
({\tt ggv}, para {\sf GNOME}) y {\tt kghostview} (para {\sf KDE}).

\item {\bf  PDF (Portable  Document Format)}:  Es un  formato reciente
similar  al  PostScript,  con  la  desventaja de  que  no  se  imprime
directamente  en las  impresoras, sino  que es  necesario instalar  un
visor. Como  visualizadores libres tenemos  el {\tt xpdf} y  todos los
anteriores  visores  de PostScript.  También  existe  el {\em  Acrobat
Reader} ({\tt acroread}) que no es libre pero sí gratuito.

\item {\bf HTML (HyperText Meta  Language)}: El formato de archivos de
la WWW  resulta interesante para  manuales, pues se puede  publicar en
Internet y al  mismo tiempo lo puedes tener como  paquete instalado en
un directorio de  documentación de tu disco duro para  poder verlo sin
conexión. Suele contener  vínculos que te permiten  navegar por dentro
de todas  las relaciones en el  documento. Se puede ver  con cualquier
navegador de  las docenas de  ellos que  existen para Linux,  tanto en
modo gráfico  como en  modo texto.  Aparte, los  sistemas de  ayuda de
Gnome y  {\sf KDE}, así como  {\tt devhelp} (un sistema  de ayuda para
programadores),  aunque todos  tengan  una interfaz  integrada con  el
escritorio, resulta que usan HTML, por lo que también se pueden ver en
cualquier navegador.

\item {\bf  Texto plano}:  Este formato  puede parecer  obsoleto, pero
sigue  usándose  pues  no  hace  falta ningun  visor  para  verlo.  En
especial, es interesante cuando no tienes entorno gráfico, o cuando no
tienes  navegadores con  los que  visualizar el  resto de  formatos, o
simplemente por ahorrar espacio, pues  al carecer de todos los códigos
de  formato  ocupa notablemente  menos.  Se  visualiza con  un  simple
editor, o bien con {\tt less} o  {\tt zless} si tiene sufijo {\tt .gz}
(compresión {\tt gzip}).

\end{itemize}

\subsection{Documentación del paquete}
\index{Documentacion!del paquete}

Normalmente  usamos  Linux  después  de haber  instalado  una  de  las
distribuciones  existentes. Sea  cual  sea la  distribución, una  cosa
en  la que  se  coincide  es organizar  las  aplicaciones en  paquetes
individuales. Cuando alguien instala uno de esos paquetes, se instalan
los ejecutables y resto de  archivos necesarios para el funcionamiento
de la aplicación, y también las páginas  de manual y de info y también
suele ser común tener un directorio  con el resto de documentación del
paquete.

Recordemos que el  software en Linux es código  abierto, eso significa
que  el programador  distribuye  libremente el  código  fuente de  las
aplicaciones, con el deseo de que sea  útil para más gente. Y para que
esto sea  posible para  el máximo número  de personas,  el programador
siempre  suele  acompañar  junto  al  código  fuente  de  su  programa
instrucciones  de instalación  y  de  uso, así  como  manuales que  no
estén  en formato  man  o info.  Cuando la  gente  de la  distribución
crea  el  ejecutable  a  partir  del código  fuente,  éste  ya  no  es
necesario,  pero como  los archivos  de ayuda  del programador  tienen
información útil  pasan a  colocarse en un  directorio, que  suele ser
{\tt /usr/share/doc/nombredelpaquete}. Cuando  no conseguimos ayuda de
otras formas, no debemos olvidarnos de mirar aquí.

En la  distribución Debian existe  un paquete llamado {\tt  dhelp} que
busca documentos útiles  para el usuario por todos  los directorios de
documentación de paquete y te los indexa y muestra de forma organizada
para un fácil acceso a todos ellos.


\subsection{Recursos en Internet de la distribución}
\index{Documentacion!de la distribucion}

Cada distribución de  Linux construye los ejecutables a  partir de las
fuentes a su manera, y ordena los archivos de configuración y archivos
de ayudas  según su criterio. Posiblemente  las ayudas estén ahí  y no
seamos capaces de encontrarlas.  Simplemente pidiendo ayuda a usuarios
más  veteranos  de  nuestra  misma  distribución  consigamos  resolver
nuestro problema o  duda. Para esto, solo tenemos que  buscar foros en
Internet sobre nuestra distribución, o bien sobre Linux en general. Si
lo que queremos realizar es algo común, muy probablemente otra persona
lo  haya  preguntado  y  ya  ha  sido  contestado,  así  que  buscando
previamente  en  los  archivos  de las  listas  nuestra  pregunta  nos
ahorraremos preguntarlo y evitaremos hacer a la gente preguntas que ya
han  sido  tratadas  y  contestadas.  A  veces,  cuando  los  usuarios
veteranos están cansados de responder  siempre a las mismas preguntas,
se crean  los {\em  FAQ - Frequently  Asked Questions}  (traducido por
{\em  PUF -  Preguntas  de  Uso Frecuente}),  que  son colecciones  de
preguntas/respuestas muy típicas y útiles para cualquiera.

Por ejemplo,  para la distribución  Debian, el sitio donde  ponerse en
contacto con usuarios veteranos es  vía lista de correo (una dirección
de correo  electrónico a la  que se pueden suscribir  muchas personas)
{\tt  debian-user-spanish@lists.debian.org}  o  bien en  el  siguiente
foro:  {\tt  http://www.esdebian.org/}.  Para obtener  FAQ's  y  otros
documentos en: {\tt http://www.debian.org/doc/index.es.html}


\subsection{Recursos en Internet del programa}
\index{Documentacion!del programa}

En caso que nuestras pretensiones de conocer el programa sean mayores,
o que  las anteriores búsquedas  hayan sido infructuosas,  siempre nos
queda la  opción de buscar  la página web  de la aplicación.  Como las
aplicaciones  libres  viven de  su  difusión,  el programador  siempre
intenta disponer de  una página web donde se anuncie  su programa, las
nuevas versiones, su dirección de correo donde comunicar bugs, enlaces
con ayudas y  tutoriales, listas de correo donde  ser notificado sobre
las  actualizaciones  o  donde  los usuarios  de  su  programa  pueden
discutir, ayudarse  unos a los  otros o proponer mejoras  al programa.
Mirando en el directorio de  documentación del programa o tecleando el
nombre de la aplicación en un buscador nos puede llevar a esta página.


\subsection{Código fuente}
\index{Documentacion!codigo fuente}

Una  lectura del  código  fuente  puede a  veces  sacarnos de  ciertos
problemas. A  lo mejor  no está  a tu alcance,  pero cuando  los pasos
anteriores son infructuosos, éste es el último paso que te queda, y el
que te  convertirá en un gurú.  Por ejemplo, si nuestra  aplicación no
funciona  y  acaba con  un  mensaje  de error  extraño,  y  ni en  los
buscadores ni otros sitios obtenemos respuesta, siempre podemos buscar
la cadena  del mensaje en el  código fuente y una  vez lo encontremos,
mirando el  entorno del mensaje,  aunque no conozcamos el  lenguaje de
programación, a  lo mejor podemos  intuir que debe estar  pasando para
que nuestro  programa no funcione.  También, si el programa  tiene aún
partes en desarrollo o partes que no han sido probadas, el programador
puede indicarlo  poniendo un  texto con una  advertencia en  medio del
código. Si vemos que el programa  falla en algo inacabado, ya vemos que no
es problema nuestro. Este es el  poder de la fuente abierta. {\em ¡Usa
la fuente Luke!} Esto es lo  que distingue a los verdaderos {\em Jedi}
de Linux.


\section{Recursos}

\subsection{Linux Documentation Proyect}
\index{Ayuda!ldp}
\index{Ayuda!howtos}
\index{Ayuda!comos}
\index{Ayuda!faqs}
\index{Ayuda!pufs}

Este proyecto centraliza mucha información. Por su grado de elaboración,
se pueden subdividir en:

\begin{itemize}
\item Guides (Manuales/Tutoriales/Cursos): Explicación exhaustiva sobre
un conjunto de temas.
\item Howtos (Cómos): Explicación exhaustiva sobre un tema concreto.
\item Faqs (Pufs): Preguntas y respuestas sueltas sobre un determinado tema.
\end{itemize}
Si no están instalados ya en nuestra distribución, los sitios donde
podemos conseguirlos son los siguientes:

\begin{itemize}

\item Original, en inglés: // {\tt http://www.tldp.org/docs.html}

\item En español (proyecto Lucas): // {http://es.tldp.org/} 

\end{itemize}

No todos los  documentos están traducidos, y la fuente  más oficial es
en inglés, así que si no  encuentras lo que buscas en español, intenta
buscarlo en inglés.

En  la  distribución  Debian   existen  los  siguientes  paquetes  con
información en castellano que puedes instalar:

\begin{description}

\item      [{\tt      ldp-es-lipp}]      Linux      Instalación      y
Primeros    Pasos,    239    páginas,    se    encuentra    en    {\tt
/usr/share/doc/ldp-es/lipp/lipp-1.1.ps.gz}

\item    [{\tt    ldp-es-garl}]     Guía    de    Administración    de
Redes    con   Linux,    352   páginas,    se   encuentra    en   {\tt
/usr/share/doc/ldp-es/garl/nag.ps.gz}

\item  [{\tt ldp-es-gulp}]  Linux Programming  Guide, 151  páginas, se
encuentra en {\tt /usr/share/doc/ldp-es/gulp/lpg.ps.gz}

\item  [{\tt  ldp-es-glpu}]  Guia  de   Linux  para  el  Usuario,  169
páginas  (bash,  editores,  xwindows,   etc),  se  encuentra  en  {\tt
/usr/share/doc/ldp-es/glup/guide.ps.gz}

\item  [{\tt linux-tutorial-es}]  Tutorial de  Linux, se  encuentra en
{\tt /usr/share/doc/linux-tutorial-es/index.html}

\item   [{\tt  lucas-novato}]   De   novato  a   novato,  81   páginas
(instalaciones  debian 1.3  y  2.0, comandos  básicos, trucos...),  se
encuentra en {\tt /usr/share/doc/lucas/novato/novato-a-novato.ps.gz}

\item  [{\tt  doc-linux-es}]  HOWTOS  traducidos  al  castellano,  117
COMOS  diferentes  para  multitud  de temas,  se  encuentran  en  {\tt
/usr/share/doc/HOWTO/es/HOWTO/}

\item [{\tt doc-es-misc}] FAQ  de {\tt es.comp.os.linux}, 142 páginas,
se encuentra en {\tt /usr/share/doc/doc-es-misc/linux-faq.es.ps.gz}

\end{description}

\subsection{Buscadores: Google}
\index{Ayuda!buscadores}

Cada   proyecto  dispone   de  su   web  donde   ver  información   de
actualizaciones,  ultimas  versiones,  etc.  Para  encontrarla  puedes
usar  cualquiera de  los  buscadores de  Internet,  por ejemplo,  {\tt
www.google.com}. En  navegadores como {\sf Mozilla},  las búsquedas en
Internet están integradas  y facilitadas con un botón  que dice Buscar
en el propio navegador. En {\sf Konqueror}, para buscar en Google solo
tienes que  escribir en la  barra de  direcciones {\tt google:}  y las
palabras que quieras buscar separadas por espacios.


\subsection{Listas de correo ({\em mailing list})}
\index{Ayuda!listas de correo}

Una lista de  correo es una dirección  de correo ficticia a  la que te
suscribes y que si envías un  mensaje a esta dirección es recibido por
todas las  personas que  se hayan  suscrito. Las  listas de  correo se
crean con una temática en mente,  y hay multitudes para casi cualquier
tema, y  Linux no va  a ser menos en  este aspecto. Como  ejemplo, las
siguientes (la lista podría ser interminable):

\begin{itemize}

\item Conjunto de listas sobre  el proyecto Debian, donde los usuarios
y los desarrolladores se organizan: \\ {\tt http://lists.debian.org/}

\item  Listas   para  el  proyecto   de  escritorio  Gnome:   \\  {\tt
http://mail.gnome.org/}

\item  Listas  para el  proyecto  de  escritorio  {\sf KDE}:  \\  {\tt
http://lists.kde.org/}

\end{itemize}

\subsection{Grupos de news (red USENET)}
\index{Ayuda!grupos de news}

Los grupos de news o USENET son lo mismo que las listas de correo pero
usando  un  protocolo  específico  llamado  {\em  NNTP  (News  Network
Transfer Protocol})  Para verlas  necesitas usar un  cliente especial,
por ejemplo: {\sf Mozilla News},  luego {\tt knews} o {\tt knode}
para {\sf  KDE}, {\tt  sylpheed} o  {\tt pan}  para {\sf  GNOME}, {\tt
lynx}  y  {\tt trn}  para  consola,  y  muchos  más. Aquí  tienes  más
información:

\begin{itemize}

\item   Buscador    dentro   de   grupos   de    noticias:   \\   {\tt
http://groups.google.com/}

\item  Los  FAQS   de  muchos  grupos  de  news   españoles:  \\  {\tt
http://www.faqs.es.org/}

\item   Página  web   de  la   jerarquia  de   grupos  de   news  {\tt
es.comp.os.linux.{*}}: \\ {\tt http://www.escomposlinux.org/}

\end{itemize}

\subsection{Páginas de bugs}
\index{Ayuda!reporte de bugs}

Cuando  llevas un  tiempo  usando  un programa  y  te  das cuenta  que
haciendo una secuencia de acciones  consigues colgarlo, y cada vez que
repites esta secuencia  lo cuelgas, resulta que has  conseguido un bug
del programa (fallo, errata). Es  raro encontrar estos fallos de forma
tan determinista ya  que normalmente los fallos  suelen ser aleatorios
lo  que dificulta  su localización  y  obliga a  usar herramientas  de
programador para  descubrirlo (depuradores, core-dumps,  trazas, etc),
pero si  lo encuentras le habrás  hecho un gran favor  al programador.
Ahora solo  tienes que  decirle lo  que has  descubierto. Busca  en la
página de su programa el listado de  bugs, comprueba que el tuyo ya no
esté (lo siento ;-), y si no está, comunícalo. Para ello suele existir
un formulario web  donde escribes tu dirección de correo,  el bug y se
le asigna un número.  Luego, por ese número, podrás volver  a la web y
encontrar  los comentarios  del programador  o de  otros usuarios  con
respecto  a ese  bug. En  caso  de que  no disponga  de este  sistema,
simplemente  suscribiéndote  a su  lista  de  correo o  enviándole  un
dirección de correo te servirá.

Ejemplos de sitios solo para notificar y hacer seguimiento de bugs
son los siguientes:

\begin{itemize}

\item      Para     la      distribución      Debian:     \\      {\tt
http://www.debian.org/Bugs/}

\item    Para     el    navegador     {\sf    Mozilla}:     \\    {\tt
http://bugzilla.mozilla.org/}

\item   Para  el   entorno  de   escritorio  {\sf   GNOME}:  \\   {\tt
http://bugzilla.gnome.org/}

\item   Para   el  entorno   de   escritorio   {\sf  KDE}:   \\   {\tt
http://bugs.kde.org/}

\end{itemize}
