%Autor: aplatanad
%aplatanad: 12


\chapter{Programación en Bash}
\label{bash.tex}

En capítulos anteriores hemos visto algunos aspectos del intérprete de
comandos  (o shell  en la  literatura inglesa).  Como en  muchos otros
casos la variedad de intérpretes  de comandos existente es muy amplia.
Sin embargo  existe uno que ha  destacado sobre los demás,  o al menos
que  ha sabido  ocupar el  puesto  de estándar  de facto  en el  mundo
GNU/Linux. Estamos hablando del {\sf  Bourne Shell}, más conocido como
{\tt  bash}\index{bash},  que  suele  ser el  intérprete  de  comandos
instalado por defecto en nuestro sistema.

En  general esto  no  desmerece  en nada  las  posibilidades de  otros
intérpretes de comandos  como pueden ser el {\tt  tsh}\index{tsh} o el
{\tt ash}\index{ash}.  Todo lo contrario, los  intérpretes de comandos
del mundo  UNIX presentan una potencia  sin igual, en especial  si los
comparamos con sus equivalentes en Microsoft® Windows® (o sea el viejo
{\tt  COMMAND.COM} o  el  nuevo  {\tt CMD.EXE}).  Esto  es natural  si
tenemos en cuenta  que por la consola de los  sistemas UNIX han pasado
millones de  profesionales que han  contribuido con sus  comentarios o
con su esfuerzo a  que haya ido ganando en potencia a  lo largo de los
años. Cada usuario  puede tener su propio intérprete  de comando, pero
por sencillez,  y puesto que  es el  intérprete por defecto  en muchos
sistema GNU/Linux, no centraremos exclusivamente en el {\tt bash}.

En  realidad ya  hemos pasado  por  un capítulo  donde aprendimos  los
principios básicos del uso del  intérprete de comandos, ahora se trata
de utilizarlo  para generar  pequeños programas que  nos ayuden  en el
trabajo  diario. El {\tt  bash} no  sólo  permite la  ejecución de las
aplicaciones instaladas en el sistema;  sino que proporciona una serie
de comandos  internos así como  estructuras sintácticas de  control de
flujo semejantes a las existentes  en muchos lenguajes de programación
(p.ej: {\tt for}, {\tt case}, {\tt while}, {\tt until}).

%\vfill

\section{Ficheros de comandos}

Todas   las  características   que  veremos   pueden  ser   utilizados
interactivamente  introduciendo  los  comandos directamente  desde  la
consola. Esto es práctico para realizar tareas sencillas. Sin embargo,
para desarrollar  programas extensos o rutinas  ampliamente utilizadas
suele ser  más interesante escribir  nuestro programa en un  archivo a
modo  de {\em  script}. En  este último  caso podemos  invocar nuestro
programa escribiendo el comando:

\begin{verbatim}
$ bash mi_programa
\end{verbatim}

De esa  manera se iniciará  la ejecución de  una nueva copia  del {\tt
bash} que abrirá el script y lo ejecutará. En los sistemas Linux suele
haber  un  comando llamado  {\tt  sh}\index{sh}.  Dicho comando  suele
corresponderse con  el intérprete de  comandos por defecto  de nuestro
sistema. Por  lo tanto, podemos sustituir  {\tt bash} por {\tt  sh} si
estamos seguros de que el {\tt bash} es nuestro intérprete por defecto
o  de  que  nuestro  script  utiliza  características  estándar  entre
los  diferentes intérpretes  disponibles. Si  queremos garantizar  que
la  interpretación de  nuestro  script  la realice  el  {\tt bash}  es
conveniente indicarlo explícitamente en lugar de utilizar el {\tt sh}.
Resumiendo, podemos invocar nuestro programa de la siguiente manera:

\begin{verbatim}
$ sh mi_programa
\end{verbatim}

La verdad es que resulta mucho  más profesional y sencillo que nuestro
script se ejecute de forma semejante  a la de cualquier otro programa.
Es  decir, escribiendo  directamente  su nombre  en  el intérprete  de
comandos. Para ello sólo es necesario  que la primera línea de nuestro
script sea así:

\begin{verbatim}
#!/bin/bash
\end{verbatim}

O  sustituimos {\tt  bash}  por {\tt  sh} si  se  dan las  condiciones
comentadas anteriormente. El último paso  es habilitar los permisos de
ejecución y ya podemos utilizar nuestro script.

\begin{verbatim}
$ chmod u+x mi_programa
$ ./mi_programa
\end{verbatim}

Cada línea de  nuestro script debe contener un comando  a ejecutar por
el intérprete.  Si deseamos poner  varios comandos en una  misma línea
debemos  usar {\tt ;} para  separarlos. Por  lo tanto  la siguiente
secuencia de comandos:

\begin{verbatim}
whoami
pwd
date
\end{verbatim}

Es equivalente a:

\begin{verbatim}
whoami; pwd; date
\end{verbatim}

A la  hora de mostrar  texto por pantalla  se utiliza el  comando {\tt
echo}. Veamos el siguiente script:

\begin{ejemplo}{whoami.sh}{Ejemplo sencillo de programación en BASH}
Este código se limita a mostrar la información referente al usuario
en la máquina local, el directorio de trabajo y la fecha actual
del sistema.
\end{ejemplo}

Varios son  los elementos nuevos que  podemos ver en este  programa,
aparte del uso del {\tt echo}\index{echo}:

\begin{enumerate}

\item  Cuando el  intérprete  encuentra  un {\tt  \#}  ignora todo  el
contenido de la línea a partir de ese punto.

\item El  comando {\tt  who}\index{who} muestra información  sobre los
usuarios autentificados en el sistema.  Al añadir las opciones {\tt am
I}  estamos  indicando  que  queremos  que  sólo  muestre  información
referente a nosotros como usuarios.

\item El comando {\tt pwd}\index{pwd} informa del directorio actual de
trabajo.

\item El comando {\tt date}\index{date} muestra la fecha y hora actual
del sistema.

\end{enumerate}

Existen una serie  de variables que el propio intérprete  define en el
momento de  ejecutar el  script y que  contienen información  sobre la
línea de comandos  que se le ha pasado. Una  breve descripción de esas
variables la podemos ver en el Cuadro \ref{cmdline}. Veamos un ejemplo
sencillo:

\begin{table}[hbtp]
\centering
\begin{tabular}{|c|l|}
\hline
Variable		&	Contenido \\
\hline
{\tt \$0}		&	Nombre del fichero de comandos \\
{\tt \$1-\$9}		&	Argumentos del 1º al 9º. El primero es \$1, el segundo \$2, etc. \\
{\tt \$*}		&	Línea de comandos completa exceptuando \$0 \\
\hline
\end{tabular}
\caption{Variables de la línea de comandos}\label{cmdline}
\end{table}

\begin{ejemplo}{yorecuerdo.sh}{Argumentos de la línea de comandos}
Ejemplo de acceso a los argumentos de la línea de comandos especificados
al iniciar la ejecución de un script.
\end{ejemplo}

Al ejecutarlo obtenemos:

\begin{verbatim}
$ ./yorecuerdo A B C
El comando es ./yorecuerdo
El primer argumento es A
El tercer argumento es C
Todos los argumentos son A B C
\end{verbatim}
      

\section{Variables de entorno}

El  intérprete de  comandos  permite la  definición  de variables  que
puedan ser  utilizadas en nuestros  script. Algunas de  esas variables
tienen un significado particular en  nuestro sistema. Para conocer las
variables  actualmente  definidas  podemos utilizar  el  comando  {\tt
set}\index{set}. Si  lo usásemos  seguramente veríamos  variables como
{\tt \$HOME}\index{HOME},  que define  nuestro directorio  personal de
usuario (la  ruta {\tt  $\sim$/} tiene el  mismo significado),  o {\tt
\$PATH}\index{PATH},  que contiene  la lista  de directorios  donde el
intérprete buscará los  ejecutables cuando le indiquemos  el nombre de
alguno.

Definir nuevas variables es tan sencillo como realizar una asignación:

\begin{verbatim}
$ EDAD=24;
\end{verbatim}

Mientras que  acceder a su valor  se hace precediendo al  nombre de la
variable con el caracter {\tt \$}.

\begin{verbatim} 
$ echo Mi edad es $EDAD 
\end{verbatim}

Toda variable definida en un {\tt bash}  es local a esa copia del {\tt
bash} y por  tanto invisible para el resto de  programa. Si tenemos en
cuenta que cada  vez que ejecutamos un script se  inicia un {\tt bash}
nuevo, resulta evidente que todas las variables definidas en el script
serán destruidas cuando éste termine. Además si un script ejecuta otro
script,  uno no  podrá  ver  las variables  del  otro.  Para que  esté
garantizado  que una  variable pueda  ser vista  fuera del  {\tt bash}
donde fue definida es necesario {\em exportarla}.

\begin{verbatim}
$ EDAD=65
$ export EDAD
\end{verbatim}

A continuación vamos a añadir una nueva ruta al {\tt \$PATH}:

\begin{verbatim}
$ PATH=$HOME/bin:$PATH
\end{verbatim}

O a modificar nuestro prompt:

\begin{verbatim}
$ PS1='Le obedezco amo: '
\end{verbatim}

Como se puede apreciar hemos puesto  la frase entre comillas. Si no lo
hubiéramos hecho  así obtendríamos un  error a causa de  los espacios.
Esto nos lleva a intentar conocer  algunos caracteres que en el entono
del {\tt bash} tienen un significado especial.


\section{Metacaracteres}

A esos caracteres se los de nomina metacaracteres:

\subsection{Sustitución: {\tt * ?}}

El  primero  puede  ser  sustituido por  un  número  indeterminado  de
cualquier combinación de caracteres. El segundo sólo representa a {\em
un}  carácter  cualquiera. Se  explican  por  sí  mismos si  vemos  el
siguiente código de ejemplo en el que se listan todos los archivos del
directorio actual:

\begin{verbatim}
$ echo Introduzca un *
\end{verbatim}

O todos los que empiecen por {\tt a} y terminen por {\tt b}:

\begin{verbatim}
$ echo Introduzca un a*b
\end{verbatim}

O sencillamente todos los que empiecen por {\tt a} y terminen por {\tt
b} pero que sólo tengan tres letras:

\begin{verbatim}
$ echo Introduzca un a?b
\end{verbatim}
        

\subsection{Redirección: {\tt $>$ $>$$>$ $<$ \textbar}}

Podemos volcar  la salida  de un  comando a  un archivo,  borrándolo y
creándolo de nuevo si éste existe.

\begin{verbatim}
$ ls > lista_archivos
\end{verbatim}

O bien  si preferimos  añadir la  salida del comando  a un  fichero ya
existente:

\begin{verbatim}
$ ls -al >> lista_archivos
\end{verbatim}

O pasar dicha salida como entrada a otro comando:

\begin{verbatim}
$ ls | less
\end{verbatim}

O usar como entrada de un comando el contenido de un archivo:

\begin{verbatim}
$ less < lista_archivos
\end{verbatim}
        

\subsection{Ejecutar en segundo plano: {\tt \&}}

Si al  final de  un comando  añadimos {\tt \&},  éste se  ejecutará en
segundo plano.  Es decir, el {\tt  bash} no esperará a  que el comando
termine,  permitiéndonos seguir  ejecutando nuevos  programas mientras
este se ejecuta en paralelo. Evidentemente resulta muy práctico cuando
suponemos  que un  comando  va  a llevar  un  tiempo  de de  ejecución
prolongado y nosotros deseamos poder seguir utilizando el sistema.


\subsection{Separación de comando: {\tt ;}}

Como  ya hemos  visto, permite  indicar varios  comandos en  una misma
línea.


\subsection{Continuación de línea: {\tt $\backslash$}}

Permite dividir  una línea en  varias si  por algún motivo  no podemos
escribirla de una sola vez.

\begin{verbatim}
$ echo Quiero vivir en \
> canarias
\end{verbatim}
        

\subsection{Valor de una variable: {\tt \$}}

Como ya hemos visto, debe preceder  al nombre de una variable para que
el intérprete sepa que debe sustituir su valor.


\subsection{Otros: {\tt ( ) [ ] `}}

Los   corchetes  se   utilizan   en  la   evaluación  de   expresiones
condicionales  y  los  paréntesis  en  la  evaluación  de  expresiones
artiméticas. Veremos unos ejemplos más adelante.

En cuanto  al último  metacarácter, al ejecutar  un comando  entre las
comillas invertidas (tildes  francesas) se sustituye la  cadena por la
salida de dicho comando. Por ejemplo, si ejecutamos:

\begin{verbatim}
$ echo date
date
\end{verbatim}

Pero si ejecutamos:

\begin{verbatim}
$ echo `date`
mar oct 30 23:51:08 WET 2001
\end{verbatim}

Si deseamos  escribir estos metacaracteres  de forma literal  deben ir
precedidos de {\tt $\backslash$}.

\begin{verbatim}
$ echo \* \\ \[
* \ [
\end{verbatim}

También podemos evitar la sustitución si los escribimos entre comillas
simples o dobles:

\begin{verbatim}
$ echo "Enviar $100?"
Enviar $100?

$ echo "`minombre` es dulce"
`minombre` es dulce
\end{verbatim}

La diferencia  entre las comillas  simples y  las dobles es  que ambas
eliminan el significado  de todos lo metacaracteres  con la excepción,
sólo en el caso de las comillas dobles, de {\tt \$} y {\tt `}.


\section{Ficheros de comandos interactivos}

Aparte de ejecutar una  secuencia predeterminada de comandos nuestros
scripts  pueden ser  interactivos.  Es decir  solicitar  y leer  datos
desde la  consola de  usuario. Para  ello se  utiliza el  comando {\tt
read}\index{read} seguido por uno o más nombres de variable.

\begin{verbatim}
$ read NAME1 NAME2 NAME3
$ echo $NAME1
$ echo $NAME2
$ echo $NAME3
\end{verbatim}

El comando {\tt read} lee desde la entrada estándar hasta encontrar un
espacio  y  almacena lo  leído  en  la  primera variable  indicada.  A
continuación sigue  leyendo hasta el  siguiente espacio y  almacena la
nueva lectura en  la segunda variable. Así  sucesivamente, excepto que
la última variable siempre almacena  desde el último espacio alcanzado
con variable anterior hasta el final de la línea.


\section{Control de flujo del programa}

Al igual que en muchos  otros lenguajes de programación, disponemos de
sentencias de control de flujo del programa. Algunos ejemplos son:

\begin{description}

\item[{\bf if}\index{if}]
\begin{verbatim}
if condicion1; then 
elif condicion2; then comando;
[ ... ]
else comando;
fi
\end{verbatim}

\item[{\bf while}\index{while}] {\tt while condicion; do comando; done}

\item[{\bf until}\index{until}] {\tt until condicion; do comando; done}

\end{description}

Existen múltiples alternativas a la hora de establecer una condición
para el control del flujo. Una de ellas es la verificación del código
de error devuelto por un comando o programa al final su ejecución. Por
definición el código de error de un comando que ha terminado de forma
satisfactoria es 0, lo cual es considerado por el intérprete como
cumplimiento de la condición. En caso contrario el comando devolverá
un código distinto de 0 y el intérprete considerará que la condición
no se cumple. Para conocer en qué situaciones se devuelven los
diferentes códigos de error se hace necesario consultar la ayuda de
cada comando en particular. Por ejemplo en el siguiente código se
utiliza el programa {\tt grep}\index{grep} para buscar un nombre de
usuario en el archivo de contraseñas del sistema ({\tt /etc/passwd}).

\begin{ejemplo}{finduser.sh}{Utilización de la sentencia ``if''}
Ejemplo del uso de la sentencia {\tt if} para el control del flujo de
un programa diseñado para buscar un nombre de usuario en el archivo de
contraseñas del sistema.
\end{ejemplo}

Si  {\tt grep}  encuentra al  usuario en  {\tt /etc/passwd}  saldrá de
forma correcta,  por lo  que se  cumple la condición  y se  ejecuta el
comando que está dentro del cuerpo del {\tt if}.

Otras     sentencias     son      {\tt     case}\index{case},     {\tt
select}\index{select} y {\tt for}\index{for}. Una forma interesante de
esta última es:

{\tt for variable in lista; do comando; done}

Por ejemplo:

\begin{verbatim}
$ for a in pato gallo perro
> do
>   echo yo tenía un $a
> done
\end{verbatim}

Veamos algunos casos  más interesantes. Para ello vamos  a empezar por
el siguiente programa:

\begin{ejemplo}{minusculas1.sh}{Utilización de la sentencia ``for''}
Ejemplo del uso de la sentencia {\tt for} para el control del flujo de
un programa diseñado para renombrar todos los archivos de un
directorio. Esto se hace de forma que se sustituyen los caracteres en
mayúsculas por sus equivalentes en minúsculas.
\end{ejemplo}

Resulta sencillo apreciar  que debemos pasar como  primer argumento de
nuestro programa  un nombre de  directorio. Ese nombre se  almacena en
{\tt \$DIR}  para su uso posterior.  En la sentencia del  {\tt for} se
ejecuta el comando {\tt ls \$DIR}  que listará el contenido del citado
directorio. Al rodear el comando por comillas invertidas la salida del
mismo no va a la consola sino que  se almacena para su uso por el {\tt
for}. Éste  toma esa salida  como una lista  de elementos por  los que
tiene que pasar la variable {\tt \$a} en cada iteración. Por tanto, en
cada interación {\tt  \$a} contedrán el nombre de uno  de los archivos
del directorio {\tt \$DIR}.

Dentro del  bucle se usa  {\tt echo} para que  envíe el valor  de {\tt
\$a}  a la  salida estándar.  Pero  como usamos  el metacarácter  {\tt
\textbar}  en  realidad  pasa  a la  entrada  del  siguiente  comando,
es  decir  a  {\tt  tr}\index{tr}.  Éste  realiza  una  traducción  de
dicha  entrada  sustituyendo  los  caractéres  en  mayúscula  por  los
correspondientes en minúscula (o sea de {\tt A-Z} a {\tt a-z}) y envía
el resultado a la salida. Nuevamente el uso de las comillas invertidas
provoca que  la salida  se almacene  en la  variable {\tt  \$fname}. A
continuación se utiliza el comando  {\tt mv} para renombrar el archivo
{\tt \$a} a  {\tt \$fname}. Al finalizar el  bucle habremos renombrado
todos los archivos de {\tt \$DIR} pasando los caracteres de mayúsculas
a minúsculas.

El programa  finaliza con {\tt  exit 0}.  Ese comando no  es necesario
para finalizar un programa, pues éste termina cuando se llega al final
del archivo, pero resulta interesante para  fijar el código de error a
la salida.  En este caso se  indica que la ejecución  fue sin errores,
pues valores distintos de 0 se suelen usar para indicar condiciones de
error. Si no  se utiliza {\tt exit}\index{exit}  el programa terminará
con el código de error devuelto por el último comando ejecutado dentro
del programa.

El script anterior no funcionará si no pasamos en la línea de comandos
un nombre de  directorio válido. Para comprobar el  no cumplimiento de
una expresión podemos utilizar {\tt !}.  Por ejemplo, vamos a añadir a
nuestro programa la comprobación de que {\tt \$1} no esté vacío:

\begin{ejemplo}{minusculas2.sh}{Mejora del programa ``minusculas1''}
Mejora del programa {\tt minusculas1} para que se compruebe
que se haya pasado un nombre de directorio como argumento de la
línea de comandos.
\end{ejemplo}

Para aquéllos  que conozcan  C/C++ todo esto  resultará extremadamente
sencillo. Hemos utilizado los corchetes para encerrar la expresión que
debe ser  evaluada de forma  condicional. Al añadir {\tt  !} indicamos
que queremos comprobar el no cumplimiento de dicha expresión.

\begin{table}[hbtp]
\centering
\begin{tabular}{|c|l|}
\hline
Operador		&	Comprobación \\
\hline
\hline
{\tt -b fichero}	&	Comprueba si {\em fichero} es un dispositivo
orientado a bloques \\
\hline
{\tt -c fichero}	&	Comprueba si {\em fichero} es un dispositivo
orientado a caracteres \\
\hline
{\tt -d fichero}	&	Comprueba si {\em fichero} es un directorio \\
\hline
{\tt -f fichero}	&	Comprueba si {\em fichero} es un fichero
ordinario \\
\hline
{\tt -r fichero}	&	Comprueba si {\em fichero} es legible por el
proceso \\
\hline
{\tt -w fichero}	&	Comprueba si {\em fichero} es escribible por el
proceso \\
\hline
{\tt -x fichero}	&	Comprueba si {\em fichero} es ejecutable \\
\hline
\end{tabular}
\caption{Operadores de comprobación de {\tt[ ]}}\label{corchetes}
\end{table}

Aun así no somos capaces de comprobar  que se haya pasado un nombre de
directorio válido a nuestro programa.  Para ello se utilizan una serie
de operadores junto con  los {\tt [ ]}, de los  cuales podemos ver los
más empleados en el Cuadro \ref{corchetes}.

\begin{ejemplo}{minusculas3.sh}{Mejora del programa ``minusculas2''}
Mejora del programa {\tt minusculas2} para que compruebe
que el argumento pasado sea un nombre de directorio válido.
\end{ejemplo}

Puesto que las  modificaciones se explican por sí  mismas no estaremos
más en ellas. Un característica  interesante de nuestro programa sería
que comprobara la existencia del  archivo de destino antes de ejecutar
{\tt mv}. En  el estado actual si  el archivo de destino  ya existe el
programa muestra un  error. Esta modificación es sencilla  a partir de
nuestros conocimientos actuales.

\begin{table}[hbtp]
\centering
\begin{tabular}{|c|l|}
\hline
Operador		&	Acción \\
\hline
\hline
{\tt int1 -eq int2}	&	Verdadero si {\em int1} es igual a {\em int2} \\
\hline
{\tt int1 -ge int2}	&	Verdadero si {\em int1} es mayor o igual a {\em int2} \\
\hline
{\tt int1 -gt int2}	&	Verdadero si {\em int1} es mayor que {\em int2} \\
\hline
{\tt int1 -le int2}	&	Verdadero si {\em int1} es menor o igual a {\em int2} \\
\hline
{\tt int1 -lt int2}	&	Verdadero si {\em int1} es menor que {\em int2} \\
\hline
{\tt int1 -ne int2}	&	Verdadero si {\em int1} no es igual a {\em int2} \\
\hline
\end{tabular}
\caption{Operadores de comparación de enteros}\label{enteros}
\end{table}

Otra  característica  sería que  admitiera  poder  pasarle más  de  un
directorio  a  través  de  la  línea de  comandos.  Este  problema  es
ligeramente  más  complejo  puesto  que  necesitamos  conocer  cuántos
argumentos ha indicado  el usuario. Para averiguarlo  debemos usar una
variable especial del  intérprete, que éste prepara  para nosotros. Se
trata de {\tt \$\#} que contiene  el número de argumentos presentes en
la línea de comandos. Conocido el número de argumentos debemos  iterar
sobre  ellos utilizando la sentencia {\tt while}\index{while}. También
necesitamos  saber  como  comparar  números  enteros,   debido   a  la
necesidad de conocer el número de veces que hemos iterado en el bucle,
para así determinar  cuándo salir. En el  Cuadro \ref{enteros} podemos
ver algunos operadores de comparación de enteros.

\begin{ejemplo}{minusculas4.sh}{Mejora del programa ``minusculas3''}
Mejora del programa {\tt minusculas3} para que admita el paso
de más de un directorio como argumento de la línea de comandos.
\end{ejemplo}

Quizás el  único aspecto difícil de  comprender es el uso  del comando
{\tt shift}\index{shift}.  La forma general  de dicho comando  es {\tt
shift [n]}  donde, si no se  especifica, {\tt n} vale  1. Este comando
desplaza los argumentos de forma que el {\tt n+1} estará en {\tt \$1},
el  {\tt  n+2}  en  {\tt  \$2} y  así  sucesivamente.  Los  argumentos
anteriormente en  {\tt \$1}, {\tt  \$2}, etc se pierden.  Sabiendo eso
comprender el programa anterior es realmente sencillo.


\section{Funciones}

El {\tt bash}  también permite crear funciones. En este  caso el valor
de retorno de  la función será el valor de  retorno del último comando
ejecutado:

\begin{ejemplo}{funcsh.sh}{Ejemplo del uso de funciones}
Este programa define una función que muestra la fecha actual por la
salida estándar, y realizar la llamada a dicha función.
\end{ejemplo}

En general  el {\tt bash} es  un programa demasiado potente  como para
poder  ser  explicado  en  unas pocas  líneas.  Probablemente  con  lo
que  hemos  comentado  sea  posible  crear  nuestros  pequeños  script
para  automatizar algunas  tareas  tediosas, pero  en  caso de  querer
profundizar más lo mejor es recurrir  a la propia ayuda del intérprete
({\tt  man bash}).  Un tema  interesante a  consulta es  sobre el  uso
de  los  archivos {\tt  .bash\_profile}  y  {\tt .bashrc}  de  nuestro
directorio personal,  y sobre  cómo usarlos para  personalizar nuestro
entorno de trabajo.

