%Autor: aplatanad

\chapter{GNU Make}

\section{Introducci�n al Make}


Es evidente que utilizar la  l�nea de comandos resulta molesto incluso
cuando se dispone de unos pocos archivos de c�digo fuente. Para ayudar
en dichas tareas se utiliza el  comando {\tt make}. Con el objetivo de
que  funcione es  necesario  disponer  de un  archivo  de nombre  {\tt
Makefile} en el  directorio de nuestro c�digo fuente.  El archivo debe
contener  reglas  que  le  indiquen  al {\tt  make}  c�mo  generar  la
aplicaci�n. Un ejemplo de {\tt Makefile} puede ser el siguiente:

\begin{verbatim}
  # Makefile .- Ejemplo para el CILA 2001.

  CC = gcc
  CFLAGS = -g -Wall
  LFLAGS = -lm

  OBJECTS = main.o holafunc3.o
  INCLUDES = holafunc.h

  holamundo: $(OBJECTS)
	$(CC) $(LFLAGS) -o $@ $^

  $(OBJECTS): %.o : %.c $(INCLUDES)
	$(CC) -c $(CFLAGS) -o $@ $<
	
  clean:
	rm -f *~ $(OBJECTS)

  clean_all: clean
  	rm -f holamundo
\end{verbatim}

Las  primeras l�neas  se  utilizan para  definir  variables que  ser�n
utilizadas en el resto de  nuestro programa. Por ejemplo, definimos en
{\tt CC} el compilador a utilizar, es decir el {\tt gcc}, mientras que
las variables  {\tt CFLAGS}  y {\tt  LFLAGS} especifican  las opciones
para el  compilador y  el enlazador  respectivamente. En  nuestro caso
indicamos con {\tt  -lm} que queremos enlazar la  biblioteca libm, con
{\tt -g} que deseamos incluir el c�digo de depuraci�n en el ejecutable
de nuestro programa y con {\tt -Wall} que el compilador debe avisarnos
a  la m�s  m�nima sospecha  de un  posible error  en el  programa. Por
�ltimo se listan en {\tt OBJECTS}  el nombre de los archivos de c�digo
objeto {\tt (*.o)} que formar�n  nuestra aplicaci�n y en {\tt INCLUDE}
los  includes {\tt  (*.h)} de  nuestro c�digo  fuente. Las  siguientes
l�neas del {\tt Makefile} especifican los {\em targets} u objetivos de
la ejecuci�n del {\tt make}.

\begin{verbatim}
  holamundo: $(OBJECTS)
	$(CC) $(LFLAGS) -o $@ $^
\end{verbatim}

Indica  que el  programa {\tt  holamundo} depende  de disponer  de los
archivos  de  c�digo objeto  listados  en  {\tt  OBJECTS} y  que  para
generarlo debemos enlazarlos con el  programa especificado en {\tt CC}
y con las opciones de {\tt LFLAGS}. Para generar {\tt holamundo} basta
con ejecutar:

\begin{verbatim}
  $ make holamundo
\end{verbatim}

Puesto  que {\tt  holamundo} es  el primer  target del  {\tt Makefile}
tambi�n  es el  target por  defecto. As�  que se  obtienen los  mismos
resultados si ejecutamos:

\begin{verbatim}
  $ make
\end{verbatim}

Aunque  ya  hemos  especificado  la  dependencia  de  {\tt  holamundo}
respecto de  los archivos de  c�digo objeto se hace  necesario indicar
c�mo  se obtienen  dichos  archivos. El  siguiente  target indica  ese
proceso.

\begin{verbatim}
  $(OBJECTS): %.o : %.c $(INCLUDES)
	$(CC) -c $(CFLAGS) -o $@ $<
\end{verbatim}

El  �ltimo  target  permite  borrar  todos  los  archivos  intermedios
presentes  en el  directorio. Eso  incluye  a los  archivos de  c�digo
objeto que,  una vez creada  la aplicaci�n,  ya no tiene  utilidad, as�
como los archivos temporales que dejan algunos editores. Se ejecuta de
la siguiente manera:

\begin{verbatim}
$ make clean
\end{verbatim}

Tambi�n podemos borrar  todos los archivos generados  por nuestro {\tt
Makefile}, incluida la aplicaci�n.

\begin{verbatim}
$ make clean_all
\end{verbatim}

Evidentemente este {\tt Makefile}  puede extenderse con nuevos targets
que se  encarguen de  generar la documentaci�n  de ayuda,  que generen
bibliotecas de enlace din�mico con las funciones de uso m�s frecuente,
que  generen otros  ejecutables que  formen parte  de la  aplicaci�n e
incluso  que empaqueten  nuestro programa  y  lo dejen  listo para  su
instalaci�n en cualquier sistema.

