%Autor: _ese & iCorrecam
%_ese: 3
%iCorrecam: 4

\newcommand{\windows}{\mbox{Microsoft Windows}}
\chapter{Introducción a GNU/Linux}
\label{introduccion.tex}

\section{Un poco de historia}

\subsection{Orígenes de UNIX y sus versiones.}

{\bf UNIX}\index{UNIX} es uno de los sistemas operativos más populares
del mundo. Es  una marca registrada de {\bf  The Open Group}\index{The
Open   Group},  aunque   originalmente  fue   desarrollado  por   {\bf
AT\&T}\index{AT\&T}.

UNIX  es  un  sistema  operativo  real.  Por  sistema  operativo  real
se  entiende   que  debe   tener  como  mínimo   dos  características:
más   de   una   persona   puede    acceder   al   mismo   tiempo   al
ordenador   y,  mientras   lo   hacen,  cada   una   de  ellas   puede
ejecutar  múltiples   aplicaciones.  A  esto   se  le  llama   ser  un
sistema  operativo   {\em  multiusuario}\index{multiusuario}   y  {\em
multitarea}\index{multitarea}.  UNIX fue  diseñado originalmente  para
ser  ese tipo  de sistema  multitarea,  en  los años  70, y  para
que  se pudiera  ejecutar en  {\em mainframes}\index{mainframes}  y en
miniordenadores.

Con  UNIX, cada  usuario accede  al  sistema utilizando  un nombre  de
acceso. Opcionalmente,  aunque es  altamente recomendable,  el usuario
debe proporcionar una contraseña que asegura que la persona que accede
es quien dice ser. Este mecanismo de control permite limitar el acceso
de  los usuarios  a los  ordenadores, algo  especialmente útil  cuando
éstos están  conectados en  red. Dicho  proceso de  autentificación es
conocido coloquialmente como {\tt  login}\index{login}, puesto que ese
es el nombre del programa  que tradicionalmente se encarga del trabajo
de permitir o denegar el acceso a un usuario.

UNIX  funciona prácticamente  en  cualquier plataforma  que haya  sido
construida. Muchos  fabricantes han  adquirido el código  fuente (IBM;
Hewlett-Packard, Sun,  etc.) y  desarrollado sus propias  versiones, a
las que han incorporado su toque personal a lo largo de los años. Pero
no son  los únicos que  continúan modificando UNIX. Cuando  el sistema
se  desarrolló  por  primera  vez, el  código  fuente  se  proporcionó
gratuitamente a las universidades y a los institutos. Dos de ellas han
estado en  primera línea  desde el primer  momento: la  Universidad de
California en Berkeley y el Instituto Tecnológico de Massachussetts.

Como  podemos imaginar,  el desarrollo  de  UNIX se  produjo de  forma
bastante desordenada.  Gente de  todo el  mundo comenzó  a desarrollar
herramientas y mejoras para UNIX. Desgraciadamente, no existía ninguna
coordinación que  guiase todo el  desarrollo, lo cual  produjo grandes
diferencias entre las distintas versiones.  Aunque a día de hoy muchas
de esas diferencias  se mantienen, en la actualidad la  mayor parte de
las  implementaciones  de  UNIX  cumplen con  el  estándar  {\bf  IEEE
POSIX.1}\index{POSIX}.  Esto  simplifica  notablemente  el  desarrollo
de  aplicaciones, puesto  que  garantiza la  compatibilidad entre  las
diferentes implementaciones de UNIX.

El mayor inconveniente de UNIX es  que es muy grande. También es caro,
especialmente en sus versiones para PC. Y aquí es donde aparece Linux,
pues, como se explica con más  detalle un poco más adelante, se diseñó
para ser  pequeño, rápido  y barato. Hasta  ahora los  diseñadores han
tenido éxito.

Linux fue  creado originalmente  por {\em  Linus Torvalds}\index{Linus
Torvalds} en  la Universidad  de Helsinki,  Finlandia, allá  por 1991.
Linus basó el Linux en una  pequeña implementación de UNIX para PC con
fines didácticos  llamada {\em  Minix}\index{Minix}. A finales  de ese
año se  hizo público  Linux con  la versión 0.10.  Un mes  después, en
diciembre, apareció la  versión 0.11. Linus hizo que  el código fuente
fuera de libre disposición y animó  a otras personas a colaborar en su
desarrollo.  Lo hicieron.  Linux  continúa su  desarrollo  hoy en  día
gracias a un equipo mundial dirigido por él mismo que trabaja a través
de Internet.

Gran   parte   del  {\em   software}   desarrollado   para  Linux   se
creó   a  través   del  proyecto   GNU  de   la  {\bf   Free  Software
Fundation}\index{FSF,Free Software Fundation}.

\subsection{El {\em Software Libre} y la licencia GPL}

Llegados a este punto, nos encontramos  con un nuevo concepto: el {\em
Software Libre}. Tiene  su origen en el nacimiento  del {\em software}
en EE.UU.,  cuando la  informática era un  feudo reservado  a empresas
y  universidades,  y  los programadores  intercambiaban  trucos  ({\em
hacks}, en inglés)  que hacían brotar chispas en  los enormes cerebros
electrónicos. Por aquel entonces hacía  sus pinitos digitales un joven
programador, {\em Richard M. Stallman}\index{Richard M. Stallman} que,
al igual  que sus compañeros de  profesión, fue testigo de  la primera
gran transformación del mundo de la programación en industria cerrada.

Cuenta este  informático que, un  buen día, aparecieron por  la puerta
abogados que prohibieron a estos programadores compartir su código (el
código  de  sus  programas)  y les  obligaron  a  ocultar  celosamente
cualquier  información  que  pudiera  ser usada  por  la  competencia.
Además, decidieron  que las empresas  guardarían bajo llave  el código
fuente de  sus programas (la  secuencia original de  instrucciones que
los hace funcionar de una determinada manera) y sólo entregarían a sus
clientes  el  código  binario  (los  unos y  ceros  que  el  ordenador
entiende, pero  apenas pueden  interpretar las personas).  Por último,
obligaron a  los trabajadores a aceptar  la idea de que  quien violaba
estas normas no sólo cometía un  delito, sino también un pecado propio
de un loco, o de un {\em pirata}.

Años más tarde,  este dogma informático se  extendió hasta convertirse
en el  actual mercado  del {\em software},  donde comprar  un programa
significaba  adquirir el  derecho a  usarlo,  pero no  a abrirlo  para
saber cómo  funciona ni,  mucho menos, a  copiarlo o  modificarlo; una
prerrogativa que corresponde en exclusiva a la empresa fabricante.

Stallman, convencido de que a la sociedad se le había robado un debate
importante sobre  la evolución  de la tecnología  (la frase  ``{\it es
como si te vendieran un coche con  el capó sellado, para que no puedas
ver el motor}'' es una de  las analogías más usadas para explicar esta
realidad mercantil),  decidió dejar su  trabajo y emprender  una tarea
mucho  más altruista:  responder  al modelo  propietario  con un  {\em
software}  del que  nadie pudiera  apropiarse, con  un {\em  software}
libre.

Se trataba,  según su promotor, de  poner en marcha un  nuevo contrato
por el que los usuarios recibieran siempre el código fuente y, además,
el derecho inalienable a modificarlo a  su gusto. A este movimiento se
le  bautizó  con el  críptico  nombre  de {\bf  GNU}\index{GNU,General
Public Licence},  y para defenderlo  se creó la {\bf  Licencia Pública
General}  ({\bf GPL}\index{GPL},  sus siglas  en inglés),  un peculiar
contrato  mercantil  que,  a  diferencia  de  las  licencias  de  {\em
software} tradicionales, no sólo no restringe la posibilidad de copiar
y redistribuir los programas, sino que anima a los usuarios a hacerlo.

Este  nuevo  orden  informático  fue recibido  con  entusiasmo  en  la
entonces  incipiente  comunidad  de  programadores  que  pululaba  por
Internet, pero también  con cierta inquietud. De  hecho, el movimiento
GNU  fue visto  con  recelo  desde algunos  sectores  de la  población
estadounidense, que lo tacharon de ``izquierdoso'', por su tendencia a
compartir su  trabajo y por su  aversión al concepto de  propiedad que
había establecido la industria del {\em software}.

\subsection{¿Qué es Linux y GNU/Linux?}

Con ideología o sin  ella, el movimiento GNU se extendió  por la Red y
empezó a dar sus frutos, y así  nació Linux. Línea a línea, programa a
programa, el sistema operativo del  pingüino (la mascota de Linux que,
por cierto, responde al nombre de  Tux) se convirtió en poco tiempo en
el producto  más famoso del  código abierto y, paradójicamente,  en la
apuesta de más de una multinacional en el sector informático.

Lo   que    realmente   se   entiende    por   Linux   es    el   {\em
kernel}\index{kernel}, el  ``corazón'' de cualquier  sistema operativo
tipo UNIX.  Pero el  kernel por  sí solo no  forma todavía  un sistema
operativo.  Justamente para  UNIX  existe multitud  de {\em  software}
libre, lo  que significa que  también está disponible para  Linux. Son
estas  utilidades  las  que  realmente forman  el  sistema  operativo.
Entonces es cuando podemos hablar de {\bf GNU/Linux}\index{GNU/Linux}.

La gran  cantidad de de programas  de {\em software} libre  permite la
creación de  diferentes sistemas  Linux, cada uno  con un  conjunto de
programas, entorno  gráfico o  sistema de instalación  diferente. Cada
una de estas  agrupaciones particulares de software  entorno al núcleo
de  Linux es  lo que  denominamos distribuciones.  En general,  cuando
hablamos  de  Linux,  nos  estamos  refiriendo  a  cualquiera  de  sus
distribuciones. Las  más conocidas  son {\em  Debian}, {\em  Red Hat},
{\em SuSE}, {\em Mandrake}. De todas ellas,  la que más se acerca a la
filosofía del movimiento GNU es Debian  ya que es fruto del trabajo de
un gran grupo de voluntarios repartidos por el mundo, mientras que las
otras pertenecen a empresas privadas dedicadas al desarrollo de Linux.

\section{Distribuciones de Linux}

Lo que vulgarmente conocemos como Linux, debiera llamarse oficialmente
como GNU/Linux. El motivo no es otro sino que el corazón de un sistema
Linux está formado por un núcleo (Linux)  al que se le han añadido las
utilidades  desarrolladas  por  la  gente de  la  {\em  Free  Software
Foundation (GNU)}, o al menos así era en los primeros momentos.

Hoy en  día, la lista  de colaboradores en  el desarrollo de  Linux es
inmensa, estando  formada tanto  por personas como  usted, o  como yo,
como  por las  más grandes  compañías del  sector informático  actual.
Parece que  ha pasado una eternidad  (5 de Octubre de  1991), desde el
momento que Linus anunció la  primera versión ``oficial'' de Linux, la
0.02. Ya podía ejecutar bash (el shell de GNU) y gcc (el compilador de
C de GNU), pero no hacía mucho más.

En ese anuncio, puso frases como éstas:

\begin{quotation}
``{\em [\dots]¿Suspiráis al  recordar aquellos días de  Minix-1.1, cuando los
hombres eran hombres  y escribían sus propios drivers?  ¿Os sentís sin
ningún proyecto interesante y os  gustaría tener un verdadero S.O. que
pudierais modificar a  placer? ¿Os resulta frustrante el  tener sólo a
Minix? Entonces, este artículo es para vosotros.[\dots]}''
\end{quotation} 

Hoy día, a partir de esas frases (y de lo que implicaban), han surgido
distribuciones de  Linux para todos  los gustos. Algunas de  ellas son
las que figuran en este listado:

\begin{itemize}

% La URL no funciona (404)
% \item {\bf  Conectiva}: Una distribución  Brasileña de Linux.  \\ {\tt
% http://www.cfhl.com.br}

\item  {\bf Debian}\index{Distribuciones!Debian}:  La distribución  de
Linux más  libre disponible hoy  día, mantenida  por un gran  grupo de
voluntarios. \\ {\tt http://www.debian.org}

% La URL no funciona (404)
% \item {\bf  Enoch}: Un avanzado  y altamente optimizado GNU  Linux. \\
% {\tt http://www.swcp.com/\~drobbins/enoch/}

\item  {\bf  e-smith server  and  gateway}:  Software Open-source  que
convierte  un   PC  en  un   Servidor  de  Internet  Linux.   \\  {\tt
http://www.e-smith.net}

\item  {\bf   Mandrake}\index{Distribuciones!Mandrake}:  Una   de  las
distribuciones más  recientes, basada en  Red Hat y {\sf  KDE}.\\ {\tt
http://www.linux-mandrake.com}

\item {\bf  muLinux}: Distribución  de Linux,  totalmente configurable
y  minimalista,   casi  completa  y  orientada   al  usuario.  muLinux
reside  en  un  solo  diskette  reformateado  a  1722K  y  dispone  de
varios  diskettes adicionales.  Requisitos mínimos  PC 386-8M  \\ {\tt
http://sunsite.dk/mulinux/}

\item {\bf  NoMad}: Su principal  propósito es  ayudar a su  creador a
ser  feliz,  dándole algo  que  hacer  en  su  tiempo libre.  \\  {\tt
http://www.nomadlinux.com}

\item   {\bf  Project   Independence}:   Project  Independence   busca
facilitar  la   vida  a   los  recién  llegados   a  Linux.   \\  {\tt
http://independence.seul.org}

\item   {\bf  Red   Hat}\index{Distribuciones!Red  Hat}:   La  primera
distribución Linux en dirigirse seriamente hacia los usuarios finales.
\\ {\tt http://www.redhat.com}

\item {\bf SEUL (Simple End-User  Linux)}: Un proyecto basado en Linux
que  pretende  convertirse en  una  alternativa  viable frente  a  los
sistemas operativos comerciales. \\ {\tt http://www.seul.org}

\item {\bf Slackware}: Una distribución histórica: La Distribución. \\
{\tt http://www.slackware.com}

\item {\bf Stampede}: Una distribución compilada para Pentium. \\ {\tt
http://www.stampede.org}

\item  {\bf   S.u.S.E.}\index{Distribuciones!SuSE}:  Una  distribución
alemana,  orientada  al usuario  final.  Colaboran  activamente en  el
desarrollo del sistema {\sf X-Window}. \\ {\tt http://www.suse.com}

% La URL no funciona
% \item  {\bf The  LætOS Project}:  Ofrece una  distribución Linux  para
% el  público  en general,  ofreciéndolo  como  alternativa al  software
% propietario.   LætOS  incluirá   instalación  gráfica,   detección  de
% hardware, etc. \\ {\tt http://www.laetos.org}

\item  {\bf  Yellow  Dog  Linux}: Una  distribución  para  ordenadores
Macintosh PPC y G3. \\ {\tt http://www.yellowdoglinux.com}

\end{itemize}

\section{Paquetes de software en Linux}\index{paquetes}

El gran  volumen de  software disponible  en la  mayor parte  de estas
distribuciones suele hacer muy difícil  incluso las tareas más básicas
de administración.  Para simplificar  el proceso  se suele  recurrir a
{\em empaquetar} el software.

Un {\em paquete}\index{paquetes}  no es más que  un archivo comprimido
que contiene todo o parte de  los archivos necesarios para ejecutar un
determinado programa  de software. Además, los  paquetes contienen las
rutinas  necesarias para  la  correcta  instalación, desinstalación  y
actualización del  software que contienen.  El uso de los  paquetes es
tan potente que se ha extendido a otros elementos, tales como galerías
de fondos de escritorio, documentación, scripts, etc.

Los criterios  bajo los  cuales un determinado  grupo de  programas se
agrupa en un mismo paquete, o  un único software monolítico se reparte
entre  varios, son  particulares de  cada distribución.  Lo que  sí es
común  a todas  ellas  es que  la  gestión  de los  mismos  se hace  a
través  de una  compleja base  de datos.  Utilizando las  herramientas
adecuadas  podemos consultar  que paquetes  están instalados  o cuales
están  disponibles para  su  instalación. Con  la mismas  herramientas
podemos instalar, desinstalar o  actualizar un paquete garantizando la
integridad de nuestro  sistema. La base de  datos mantiene información
de las dependencia entre paquetes, por lo que podemos estar informados
en todo momento de que paquetes se necesitan para instalar uno dado, o
qué paquetes no podemos desinstalar porque son requeridos por otro que
si lo está.

Los formatos de paquetes más extendidos son los {\tt rpm}\index{rpm} y
los {\tt deb}\index{deb}. Los  primeros son utilizado mayoritariamente
por la distribución {\em Red Hat} y sus derivadas (como {\em Mandrake}
y {\em  SuSE}). Los  segundos son empleados  por la  distribución {\em
Debian} y sus derivadas.

El empaquetado  del software y  la gestión de  paquetes es una  de las
grandes ventajas que presentan las distribuciones Linux frente a otros
sistemas operativos. Una vez que te acostumbras a sus ventajas se hace
muy difícil vivir sin ellas.

Bueno, para terminar este rápido  repaso, comentar que en este listado
no se  encuentran, ni muchísimo menos,  todas las opciones de  las que
puede disfrutar  con Linux. Basta  dar una vuelta por  los principales
buscadores de  Internet, y  se dará  cuenta uno de  hasta dónde  se ha
llegado actualmente.

\section{GNU/Linux: cara y cruz}

\subsection{La cara}

Enumeramos las ventajas de Linux a continuación:

\begin{itemize}

\item {\bf  Multitarea total}. Se  pueden ejecutar varias tareas  y se
puede acceder a varios dispositivos al mismo tiempo.

\item {\bf Memoria virtual}. Linux puede  usar una porción de su disco
duro como memoria virtual, lo que aumenta la eficiencia del sistema al
mantener los procesos activos en la  memoria física (RAM) y al colocar
las partes  inactivas o usadas con  menos frecuencia en la  memoria de
disco.  La memoria  virtual  permite utilizar  la  máxima cantidad  de
memoria posible del sistema y permite que no se produzca fragmentación
de la memoria.

\item {\bf  Soporte multiusuario}.  Linux permite que  varios usuarios
accedan  a su  sistema simultáneamente  sin que  haya conflicto  entre
ellos, proporcionándole a cada su propio espacio de trabajo.

\item  {\bf Código  fuente  no  propietario}. El  kernel  de Linux  no
utiliza  código de  {\bf AT\&T}  ni ninguna  otra fuente  propietaria.
Otras organizaciones, como las  compañías comerciales, el proyecto GNU
y los programadores de todo  el mundo, han desarrollado {\em software}
para Linux.

\item {\bf Soporte mediante {\em  software} GNU}. Linux puede ejecutar
una amplia  variedad de  {\em \mbox{software}}, disponible  gracias al
proyecto GNU. Este {\em software} incluye de todo, desde desarrollo de
aplicaciones (GNU C y GNU C++)  a la administración del sistema (gawk,
groff, etc.) y juegos (GNU Chess, GnuGo, y NetHack).

\item  {\bf  Estabilidad}.  Linux  presenta una  gran  estabilidad  en
la  gestión de  sus  procesos  internos del  sistema.  Es muy  difícil
conseguir que Linux  se ``cuelgue'' y, por supuesto, jamás  se verá un
``pantallazo azul'' de los conocidos por \windows.

\item {\bf Gran oferta de software}.  Aunque Linux no sea tan conocido
por el público como lo es \windows, eso no quiere decir que no existan
aplicaciones  para el  usuario  medio. Por  el  contrario, cuando  uno
utiliza Linux, según  pasan los días, se tiene la  sensación de que no
necesita para  nada \windows\  porque todo  lo que  éste ofrece  ya lo
tiene Linux. Llega un momento en que no se echa de menos a \windows.

\item {\bf  Defensa contra los  virus}. Aunque  la mayor parte  de los
virus  que rondan  por Internet  son desarrollados  para \windows,  es
cierto que existen algunos para Linux, pero son más difíciles de crear
debido a  que Linux emplea un  sistema de permisos sobre  los ficheros
previniendo los  posibles desastres que se  ven todos los días  en los
entornos de \windows.  Si a eso añadimos que los  virus para \windows\
no se pueden  ejecutar en Linux salvo contadísimos  casos, nos podemos
hacer una idea del grado de seguridad con que cuenta Linux.

\item {\bf Relación con Internet}. Debido a que Linux creció gracias a
Internet, digamos que  ambos hablan en el mismo lenguaje  y por tanto,
se ve claramente que navegar por  Internet con Linux es más rápido que
con Microsoft  Windows. Obviamente, ésta  es una opinión  personal del
autor de este documento.

\item {\bf Entornos gráficos}. Hasta hace unos años, trabajar en Linux
sólo  era posible  desde  consola, ese  entorno  negro con  caracteres
blancos  (similar al  MS-DOS, pero  más  potente). Con  la llegada  de
Microsoft  Windows, la  comunidad Linux  se  vio de  forma obligada  a
desarrollar  nuevos  entornos gráficos  para  no  perder el  tren.  La
gran  diferencia  con Microsoft  Windows  es  que mientras  éste  sólo
dispone de un  escritorio, en Linux podemos elegir  con qué escritorio
queremos trabajar. Los más conocidos  son {\sf KDE}, {\sf GNOME}, {\sf
AfterStep}, {\sf  Enlightenment} y {\sf  Window Maker}; donde  los dos
primeros son los  más populares. Lo más curioso del  asunto es que con
ellos se ha iniciado una ``guerra dialéctica'' sobre cuál es mejor. Al
final, la  ventaja reside en la  variedad con la que  el usuario puede
decidir con cuál se siente más cómodo.

\item {\bf Servidores  caseros}. Parece mentira, pero  en casa podemos
tener un servidor  web, o un servidor FTP con  nuestro Linux. Sólo hay
que leer la documentación de cómo hay que hacerlo.

\item {\bf La comunidad Linux}.  A diferencia de \windows, Linux tiene
una comunidad de voluntarios con ganas  de ayudar a los que les cuesta
adentrarse  en este  mundo.  Y  todo por  afán  de  colaborar en  este
proyecto. Podemos decir  que esta comunidad tiene  un gran sentimiento
de solidaridad.

\end{itemize}

\subsection{La cruz}

Obviamente,  como todo  sistema  operativo, Linux  no  está exento  de
desventajas, como vemos aquí:

\begin{itemize}

\item {\bf Entorno  árido}. Aunque en los últimos años  el panorama ha
mejorado considerablemente,  no podemos olvidar que  para trabajar con
Linux,  sobre todo  si se  viene de  \windows, hay  que aprender  unas
cuantas nociones  si no se  quiere tener la  sensación de que  se está
perdido.  Todo ello  radica  en que  Linux no  es  tan intuitivo  como
Microsoft  Windows,  pero  afortunadamente  las  diferentes  compañías
comerciales  han aportado  herramientas que  facilitan esas  tareas al
usuario. Sólo hay que cambiar un poco el ``chip''.

\item {\bf Soporte de hardware}. Por desgracia, como se ve claramente,
el mercado de los sistemas informáticos de escritorio está orientado a
\windows. Prueba de ellos es que el 90\% de los PC lo tienen instalado
sin  usar otro.  Así  que, los  fabricantes de  hardware  sólo se  han
preocupado de  crear los controladores de  dispositivo compatibles con
\windows\  sin  pensar en  los  restantes  sistemas como  Linux,  Mac,
BeOS  y otros.  Ello ha  obligado a  que fueran  los propios  usuarios
programadores  los  que  desarrollaran sus  propios  controladores  de
dispositivo.  Con el  tiempo, dada  la demanda  creciente de  usuarios
decididos a utilizar Linux algunos  fabricantes ha comenzado a acceder
a las demandas de estos últimos.

\item {\bf Configuración del sistema operativo}. Hasta hace bien poco,
instalar y  configurar Linux en su  casa en condiciones era  una tarea
ardua y bastante complicada. Ello hacía  que mucha gente se rindiera y
siguiera  con Microsoft  Windows.  Las  diferentes distribuciones  han
tenido  en cuenta  estos  problemas y  han  aportado herramientas  que
ayudan al  sufrido usuario  a hacerlo  todo de  forma más  intuitiva y
automática. Entre estas herramientas están  las encargadas de la parte
de la instalación del {\em hardware} y {\em software}.

\item {\bf No todo el software necesario está presente}. Es cierto que
casi todo el software que un usuario necesita ya lo aporta Linux. Pero
hay  situaciones  en  las  que una  persona  necesita  una  aplicación
concreta  y ve  que  debe  utilizar Microsoft  Windows  para usar  esa
aplicación. De  todas formas, siempre surgen  desarrolladores con afán
de ayudar que terminan desarrollando esas aplicaciones.

\item  {\bf  Administración  de  Linux}.   Como  todo,  si  se  quiere
administrar Linux  de forma profunda,  ya no vale usar  los asistentes
gráficos como en \windows, sino que hay que leer mucha documentación y
experimentar. A  cambio se  gana una experiencia  que en  \windows\ es
difícil de conseguir.

\end{itemize}

\section{Recursos para GNU/Linux}

\subsection{Software}

Tal vez  por su escasa repercusión  en el ámbito doméstico,  uno pueda
pensar  que  la cantidad  de  software  es  bastante escasa.  Todo  lo
contrario. De  hecho se pueden  encontrar bastantes utilidades  en las
páginas webs propias de las distribuciones.

También se pueden encontrar aplicaciones en otras páginas o servidores
FTP con software propio, como  {\sf StarOffice}, {\sf KDE}, {\sf QCAD}
y {\sf VariCAD} (éstos dos últimos  son aplicaciones que aspiran a ser
serias alternativas  a {\sf AutoCAD}  aplicadas a los  entornos UNIX),
etc.

Si el  usuario proviene del entorno  \windows, tal vez pueda  tener la
sensación de que ``{\em Sí, trabaja como Windows, pero no es igual que
Windows\dots}'', y le gustaría encontrar las herramientas con el mismo
aspecto que este sistema operativo. Los desarrolladores han optado por
tres caminos:

\begin{itemize}

\item El primero  es el de dar las mismas  funcionalidades que las que
tienen esas  aplicaciones que existen  en \windows, aunque  su aspecto
externo no tenga nada que ver con el \windows.

\item El segundo camino añade además el aspecto externo original de la
aplicación que  funciona en \windows  con  el fin de que  el usuario se
encuentre en un entorno familiar.

\item El tercero  es el más drástico ya que  se crean herramientas con
aspectos y funcionalidades totalmente  diferentes a esas aplicaciones,
pero igualmente eficientes e inclusos superiores en algunos casos.

\end{itemize}

Para plasmar  en un  ejemplo de lo  que se acaba  de decir,  basta con
nombrar una de las aplicaciones más utilizadas en el entorno Microsoft
Windows que es  {\sf MS Word}. Del primer ejemplo  nos encontramos con
{\sf KWord}, desarrollado por el equipo que creó {\sf KDE}.

El  segundo es  {\sf  AbiWord}. El  autor reconoce  que  es casi  como
utilizar  el {\sf  MS  Word}, aunque  es evidente  que  aún le  faltan
bastantes opciones por desarrollar.

Y  el tercero  es el  \LaTeX, una  herramienta muy  potente, pero  que
quizás desconcierte  a los usuarios  noveles debido a  su complejidad.
Como anécdota  aclarativa, el libro que  el lector tiene ahora  en sus
manos  y está  leyendo se  ha escrito  usando precisamente  \LaTeX. Al
final la elección depende del propio usuario.

Una cosa está  clara: {\bf Hay software más que  suficiente para Linux
como para detener un tren}.

\subsection{Documentación}

La  documentación  básica  sobre  Linux se  puede  encontrar  en  {\tt
http://es.tldp.org/} (en castellano)  y {\tt http://www.tldp.org/} (en
inglés). En ellos  se pueden encontrar tutoriales y  cursos que pueden
ayudar a los usuarios a adentrarse en el mundo Linux.

Cabe destacar  que los documentos  más utilizados son los  COMOs ({\em
HOW-TO} en inglés) que son una guía bastante útil sobre alguna cosa en
concreto que se quiere hacer.

Otra  forma de  encontrar  información es  buscando  en las  múltiples
páginas webs referidas al tema, sin olvidar también echar un vistazo a

Por  último, si  se quiere  información  rápida de  algún programa  en
concreto, no hay  más que utilizar los manuales, es  decir, el comando
{\tt man}, las páginas  info con el comando {\tt info}  o buscar en la
documentación  del  programa  en  cuestión  en  los  directorios  {\tt
/usr/share/} y {\tt /usr/share/doc/}.
