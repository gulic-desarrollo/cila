
\chapter*{Agradecimientos}

\label{Agradecimientos}

Este  libro   lo  hemos   escrito  miembros   y  amigos   del  \GULiC;
informáticos,  estudiantes y/o  aficionados con  entrega, entre  todos
unimos nuestros esfuerzos para escribir  sobre aquello de lo que mejor
entendemos.

La  \FMAT~ de  la  \ULL~  ha colaborado  apasionadamente  (y lo  sigue
haciendo)  aportando  sus  aulas  de  ordenadores  para  realizar  las
sucesivas ediciones del \CILA~. Desde el \GULiC~ agradecemos además la
amabilidad con la que han colaborado en todo momento.

Pero este libro  no habría llegado a ser  lo que es de no  ser por las
siguientes personas. A todos ellos; muchas gracias.

\begin{itemize}

\item {\sc Sergio Alonso}, Vicedecano de Estadística en la \FMAT de la
\ULL.  Acogió amablemente  la propuesta  del  \CILA y  puso las  aulas
de  ordenadores a  disposición  del curso.  También  ha coordinado  la
publicación de este libro a través del Servicio de Publicaciones de la
\ULL.

\item {\sc  Manolo García Román}, Secretario  de la \FMAT de  la \ULL.
Brindó  una inestimable  ayuda  en las  cuestiones  {\TeX}nicas de  la
elaboración de estos  apuntes. También se entregó de  modo ejemplar en
la organización de  las distintas edicioens del curso en  el ámbito de
la  \ULL. Ha  coordinado la  publicación de  este libro  a través  del
\SPULL.

\item  {\sc  Miguel Ángel  Vilela  García}  escribió los  temas  ``The
GIMP'',  ``GNU  Fortran'',  ``Yacas'',  ``R'',  además  de  partes  de
``Aplicaciones   para   Internet'',  ``Editores''   y   ``Aplicaciones
diversas''.  También  retocó  la  traducción del  tutorial  de  \LyX{}
aportada por {\sc Sergio García Reus} para formar el tema ``\LyX''.

\item  {\sc  Jesús  Miguel  Torres  Jorge}  escribió  los  temas  ``El
entorno  X-Window'',   ``C/C++'',  ``GNU  Make'',  ``Bash''   y  parte
de  ``Aplicaciones  para  Internet''. Posteriormente  añadió  abudante
contenido a raíz de los cursos TIC de la \ULL.

\item {\sc Carlos Pérez Pérez} aportó el tema ``StarOffice''.

\item  {\sc Tomás  Bautista  Delgado} aportó  el  tema de  ``\LaTeX'',
resultado  de  modificar  su  estupendo manual  ``Una  descripción  de
\LaTeX''

\item  {\sc  Carlos  Alberto  Morales  Díaz}  escribió  los  temas  de
``Octave'', ``GNUplot'' y parte de ``Aplicaciones para Internet''.

\item   {\sc  Félix   J.   Marcelo  Wirnitzer}   escribió  los   temas
``Introducción a GNU/Linux'', ``El intérprete de comandos'' y ``HTML''
y parte de ``Administración básica'' e ``Instalación de GNU/Linux''.

\item {\sc Sergio García Reus}  aportó su traducción al castellano del
tutorial de \LyX.

\item {\sc Carlos de la Cruz  Pinto} escribió con los temas ``Emacs'',
``Java'', ``KDE'' y mejoró el tema de ``FreePascal''.

\item {\sc René Martín Rodríguez}  escribió los temas ``Instalación de
GNU/Linux'' y ``Depuradores''.

\item  {\sc  Edín Kozo}  colaboró  en  los temas  ``Aplicaciones  para
Internet'' e ``Instalación de GNU/Linux''.

\item   {\sc   Carlos   Mestre   González}   escribió   el   tema   de
``Administración básica'' y parte de ``Editores''.

\item  {\sc   Teresa  González  de   la  Fé}  colaboró  con   el  tema
``Aplicaciones diversas''.

\item {\sc Luis  Cabrera Saúco} colaboró con  los temas ``Introducción
a  GNU/Linux'',   ``Aplicaciones  para  Internet''   y  ``Aplicaciones
diversas''.

\item  {\sc Carlos  Alberto  Paramio Danta}  aportó  %los capítulos  de
%``autoconf'' y 
el tema ``gprof''.

\item {\sc Pedro A. Gracia Fajardo} colaboró con el tema ``The GIMP''.


\end{itemize}

También agradecemos a todas las personas  que nos han animado a seguir
con este proyecto:  alumnas y alumnos de la \FMAT~  y otras facultades
de la  \ULL, compañeros del  \GULiC, Maribel C. Salgado,  Pedro Reina,
gente de {\sc \mbox{HispaLinux}} y alguien  más que se nos queda en el
tintero (o en el teclado).


