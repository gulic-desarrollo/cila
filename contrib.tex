
\chapter*{Contribuciones}

\label{Contribuciones}

Estos  apuntes han  sido escritos  por miembros  y amigos  del \GULiC;
inform�ticos,  estudiantes y/o  aficionados con  entrega, entre  todos
unieron  sus esfuerzos  para escribir  sobre aquello  de lo  que mejor
entend�an.

La  \FMAT~ de  la \ULL~  colabor� apasionadamente  aportando aulas  de
ordenadores para  realizar el \CILA~, la  \PILA~ y el \TILA.  Desde el
\GULiC~ agradecemos adem�s la amabilidad  con la que han colaborado en
todo momento.



\begin{itemize}

\item {\sc Miguel  �ngel Vilela Garc�a}, estudiante  de Matem�ticas en
la \ULL. Encaj�  en retorno de la secci�n chicharrera  del \GULiC~ con
los �nimos de  colaboraci�n de la \FMAT, proponiendo  y organizando el
\CILA, la \PILA~ y el \TILA.  Tambi�n se encarg� del dise�o en \LaTeX~
de los apuntes, y redact� algunos  temas de los mismos. En definitiva,
el mayor culpable de todo esto.

\item  {\sc  F�lix  J.  Marcelo Wirnitzer},  (estupendo)  profesor  de
Inform�tica  en  un instituto.  Resucit�  la  secci�n chicharrera  del
\GULiC \ y dio un valioso apoyo moral al personal. Escribi� los temas
``Introducci�n a GNU/Linux'' y ``El int�rprete de comandos''.

\item {\sc  Manolo Garc�a  Rom�n}, profesor  en la  \FMAT de  la \ULL.
Brind�  una inestimable  ayuda  en  las cuestiones  \TeX  nicas de  la
elaboraci�n de estos apuntes.

\item {\sc Carlos  de la Cruz Pinto}, estudiante de  Matem�ticas en la
\ULL. Escribi� con los temas ``Emacs'', ``Java'' y ``Perl''.

\item {\sc Luis Cabrera Sa�co}, Presidente del \GULiC

\item {\sc Teresa Gonz�lez de la F�}
		Explicaci�n de las licencias GNU General Public License y
		GNU Free Documentation License

\item {\sc Ediz Kozo}
		Secci�n ``FTP'' (``Internet'')

\item {\sc  Ren� Mart�n  Rodr�guez}, estudiante  de Inform�tica  en la
\ULL. Escribi� los temas ``\LaTeX'' y ``\LyX''


\item {\sc  Carlos Alberto Morales D�az},
		Esp�ritu de aventura y secciones ``Octave'', ``GNUplot'' y
		``SciLab'' (``Matem�ticas'')

\item {\sc  Jes�s Miguel Torres Jorge}  \\
		M�s esp�ritu de aventura, cap�tulo ``El entorno X-Window'',
		secciones ``C/C++'', ``Bash'' (``Programaci�n'') y 
		``SSH'' (``Internet'')
	
\end{itemize}



