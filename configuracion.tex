%Autor: mojopikon

% MojoPikon les ruega que no le majen por las posibles desfiguraciones que este texto sufra
% en momentos concretos debido a un "lo pongo así y luego más adelante cuando lo tenga más
% perfiladito lo pongo bien puestito".
% GRACIAS.

\chapter{Configuración}\index{configuración}

\section{El Kernel de Linux}

\subsection{Introducción}:

Con  la  cantidad de  distribuciones  de  Linux  que  hay hoy  en  día
circulando por  la red,  escribir cualquier texto  sobre configuración
requeriría dedicar un libro entero para cubrir las distintas formas de
configuración que tienen cada una de estas: el SaX y el yast2 de SuSE,
los paneles  de control de mandrake,  y otras cosas por  el estilo son
cosas que  no abordaremos aquí,  por ser específicas, e  inútiles para
los usuarios que no tengan estas distribuciones.

Sin  embargo, en  esta  sección veremos  algunas  técnicas que  puedan
servirnos para cualquier distribución de linux.

¿Como configuro mi ...?

Para cualquier  periférico o dispositivo  de tu ordenador  que quieras
configurar, lo normal será que antes cargues en memoria el controlador
(driver) que lo va a manejar.  En otros sistemas operativos, lo normal
es que venga un ejecutable que se encargue de configurarlo todo.

En Linux, los controladores pueden venir  así (como por ejemplo, es el
caso de los controladores para modems  de Conexant), en cuyo caso sólo
hay que  ejecutar un pequeño  programita que se encarga  de instalarlo
todo, aunque por  desgracia no es el caso más  frecuente. Lo normal es
que tengas que seguir los siguientes pasos:

\begin{itemize}

\item Conectar el dispositivo

\item Fijarse en los mensajes de arranque del Kernel.  Pueden pasar dos cosas:

\begin{itemize}

\item Que el  kernel lo detecte por que el  kernel que estemos usando 
tenga soporte  para el dispositivo  que queremos. En este  caso, solo 
nos  quedaría configurar  los  programas que  vayan  a utilizar  este 
dispositivo o periférico.                                             

\item Que el kernel no lo  detecte. En este caso, tendremos que seguir
uno de estos pasos:

\begin{itemize}

\item  Si  nuestra distribución  de  linux  trae el  controlador  para
nuestro dispositivo como un módulo ya compilado, deberemos cargarlo.

\Large{FIXME:: Mira la sección ``Cargando Módulos''. ::FIXME}

\item ``recompilar  el kernel'' con los  controladores apropiados que 
se encarguen de manejar el dispositivo que queremos instalar.         

\end{itemize}

\end{itemize}

\item Configurar el/los programas que vayan a utilizar el dispositivo.	

\end{itemize}

\subsection{Recompilar el Kernel}

Aunque  pueda  parecer algo  muy  complejo,  por  lo raro  que  suena,
configurar  y   recompilar  el   kernel  no  suele   consistir  (salvo
dispositivos muy  nuevos),más que en  elegir algunas opciones  en unos
cuantos menús,  escribir unos pocos  comandos y esperar un  rato, cuyo
largo variará según la potencia de nuestra máquina.

En primer lugar, necesitamos asegurarnos de tener instaladas:

\begin{itemize}

\item El código  fuente del kernel. (Suele venir en  paquetes con nombres 
como  kernel-source,  o  kernel-source-2.4.18,  o  similar).  También 
puedes  bajar la  última versión  del kernel  en ftp.kernel.org,  que 
probablemente será lo más adecuado si tu dispositivo es muy nuevo.    

\item Las utilidades de desarrollo GNU (make,gcc,etc).

\item Si  queremos configurarlo  por medio de  menús desde  las X-window, 
necesitaremos  tener instaladas  las  librerías tk  (se instalan  por 
defecto en la gran mayoría de los casos).                             

\item  Si queremos  configurarlo por  medio de  menús en  modo texto, 
necesitaremos  tener  instalado  el paquete  libncurses5-dev  (en  tu 
distribución  podría  tener  un  nombre  distinto,  como  curses-dev, 
ncurses-devel, o  similar. Son  todos lo mismo).  \Large{FIXME:: Para 
más  información sobre  como instalar  paquetes, mira  la sección  de 
instalación. ::FIXME}.                                                

\end{itemize}

Una vez estamos seguros de tener todo esto instalado, hacemos lo siguiente:

\begin{itemize}

\item Abrimos un terminal o consola.
\item Vamos al directorio donde se hayan instalado los fuentes del kernel. (Normalmente {\tt /usr/src/linux}).

\begin{verbatim}
# cd /usr/src/linux
\end{verbatim}

También  es posible  que  los  fuentes una  vez  instalados estén  en 
formato  comprimido  .tar.gz,  o   .tar.bz2  en  el  directorio  {\tt 
/usr/src}. En ese caso,deberemos descomprimirlos con:                 

\begin{verbatim}
# cd /usr/src
# bzcat kernel-source-X.XX.tar.bz2 | tar xv
\end{verbatim}

o sustituyendo {\tt bzcat} por {\tt zcat} si el fichero es un .tar.gz.

Por cuestiones de compatibilidad con otras cosas que podemos instalar 
después, conviene  crear un enlace  simbólico llamado {\tt  linux} al 
directorio donde se encuentren los  fuentes. Por ejemplo, si acabo de 
descomprimir los fuentes a {\tt kernel-source-2.4.20}, estando dentro 
del directorio {\tt /usr/src}, usaría el comando                      

\begin{verbatim}
# ln -s kernel-source-2.4.20 linux
\end{verbatim}

\item Ejecutamos la  utilidad de configuración del  kernel, estando ya
dentro de {\tt /usr/src/linux}.

\begin{verbatim}
# make menuconfig
\end{verbatim}

Nota: Para utilizar menús en modo  X-window, en lugar de utilizar make
menuconfig, utilizaríamos el comando

\begin{verbatim}
# make xconfig
\end{verbatim}

No  obstante,  debido  a  que  podríamos  no  tener  X-window  cuando 
necesitemos configurar  nuestro kernel,  aquí utilizaremos  {\tt make 
menuconfig}.                                                          

\item Debería  aparecer un  menú lleno  de opciones,  si no  nos sale
antes un mensaje  de error diciéndonos que no  tenemos instaladas las
libncurses5-dev. (Ver más arriba para solucionar este problema). 

A  pesar de  que todas  estas opciones  puedan desconcertarnos  en un 
principio,  sólo necesitaremos  configurar  algunas  para que  nuesto 
sistema funcione  correctamente. Debido a  que ya hay  libros enteros 
que hablan sobre  cada una de estas opciones, sólo  mencionaré las de 
uso más común.

En estos menús, veremos muchas opciones  de las cuales cada una tiene 
3  estados: Y<>  M<>  N<>. Como  algunos podran  suponer,  Y<> y  N<> 
significan "Sí" o "No". Esto  significa que, cuando usemos este nuevo 
kernel que  vamos a compilar,  las opciones que hayamos  marcado como 
"Y",  estarán  disponibles,  y  por  tanto,  si  por  ejemplo,  hemos 
seleccionado  "Y" en  el  driver  para la  tarjeta  de sonido  "Sound 
Blaster", cuando  iniciemos linux con  ese kernel, este  detectará la 
tarjeta "Sound Blaster", y podremos utilizarla.                       

La opción de compilar como módulo nos permite tener los controladores
en un directorio.

\end{itemize}
