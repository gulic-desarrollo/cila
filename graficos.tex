%Autor: joseant

\chapter{Edici�n de gr�ficos}

\section{The GIMP}

{\tt The GIMP} es el Programa  de Manipulaci�n de Im�genes de GNU (GNU
Image Manipulation  Program). Es  un programa  libremente distribuible
�til  para  trabajos  como   retoques  de  fotograf�a,  composici�n  y
publicaci�n  de im�genes.  {\tt The  GIMP} ha  sido escrito  por Peter
Mattis y Spencer Kimball, y liberado bajo  la Licencia General P�blica
(GNU General Public  License). {\tt The GIMP} es  un programa bastante
com�n  en las  distribuciones de  GNU/Linux, y  muy popular  entre los
usuarios medios/avazanzados de Linux.

La primera vez que los  ejecutemos mostrar� un di�logo de instalaci�n,
lo  cual se  refiere realmente  a la  personalizaci�n (instalaci�n  de
usuario) del  programa. En este  di�logo nos preguntar�  acerca  de la
resoluci�n de la pantalla, que suele ser 72 dpi (dots per inch, puntos
por pulgada). Dado que el di�logo est� en espa�ol resulta muy f�cil de
seguir :-)

Cuando finalmente {\tt The GIMP} se presenta ante nosotros vemos cinco
ventanas. La  que aparece en primer  plano es la ventana  de ``consejo
diario'',  donde podemos  leer trucos  y consejos  que {\tt  The GIMP}
tendr�  la amabilidad  de ense�arnos.  Las  otras ventanas  son la  de
``selecci�n de  brocha'', la  de ``opciones  de herramientas'',  la de
``capas,  canales y  caminos'' y  la ventana  principal. Esta  ventana
principal contiene la barra de herramientas y los selectores de color,
gradiente y brocha.

Para aprender  a manejar GIMP s�lo  se necesitan ganas, osad�a  y unos
cuantos ratitos para sentarse y  ponerse a jugar con las herramientas,
filtros  y  scripts  que  proporciona.  Si  quieres  sacarle  el  jugo
puede  resultarte muy  �til  alg�n  libro sobre  {\tt  The GIMP}  como
\cite{gimpref}.


