%Autor: lasarux & miguev
%lasarux: 2
%miguev: 2

\chapter{Edición de gráficos}
\label{graficos.tex}

\section{The GIMP}

\subsection{Introducción}

{\sf The GIMP} es el Programa  de Manipulación de Imágenes de GNU (GNU
Image Manipulation  Program). Es  un programa libremente distribuible,
útil  para  trabajos  como   retoques  de  fotografía,  composición  y
publicación  de imágenes.  {\sf The  GIMP} ha  sido escrito  por Peter
Mattis y Spencer Kimball, y liberado bajo  la Licencia General Pública
(GNU General Public  License). {\sf The GIMP} es  un programa bastante
común  en las  distribuciones de  GNU/Linux, y  muy popular  entre los
usuarios medios/avazanzados de Linux.

La primera vez que los  ejecutemos mostrará un diálogo de instalación,
que realmente se refiere a la personalización (instalación de usuario)
del programa. En  este diálogo nos preguntará acerca  de la resolución
de  la pantalla,  que suele  ser  72 dpi  (dots per  inch, puntos  por
pulgada). Dado  que el diálogo  está en  español resulta muy  fácil de
seguir. Con aceptar la opción que nos ponga por defecto es suficiente.

Cuando finalmente {\sf The GIMP} se presenta ante nosotros vemos cinco
ventanas. La  que aparece en primer  plano es la ventana  de ``consejo
diario'',  donde podemos  leer trucos  y consejos  que {\sf  The GIMP}
tendrá  la amabilidad  de enseñarnos.  Las  otras ventanas  son la  de
``selección de  brocha'', la  de ``opciones  de herramientas'',  la de
``capas,  canales y  caminos'' y  la ventana  principal. Esta  ventana
principal contiene la barra de herramientas y los selectores de color,
gradiente y brocha.

\begin{figura}{gimp}{0.8}
\caption{Aspecto de The GIMP al usarlo por primera vez}
\end{figura}

\subsection{Los diferentes formatos de imágens}

Del  formato  del  fichero  que utilices  para  guardar  tus  imágenes
dependerá  la calidad  de  las  mismas (las  fotos  almacenadas en  un
formato que permita bastante compresión normalmente lo hace a costa de
su calidad)  y el espacio  que ocupe. Algunos además  pueden almacenar
información relativa al  tamaño de la foto para la  impresión (TIFF) e
información sobre su transparencia (GIF, PNG).

Existen muchos tipos de formato, cada uno para su uso.
The {\sf GIMP} tiene soporte para estos:

\begin{itemize}

\item  BMP: desarrollado  e impulsado  por Microsoft,  propietario del
mismo.  BMP es  una abreviatura  de Windows  BitMaP (Mapa  de Bits  de
Windows).

\item  CEL: este  formato es  originario del  Animator Studio.  Es muy
utilizado para guardar {\em sprites},  o mejor dicho imágenes pequeñas
para juegos.

\item FITS: es el formato standard en Astronomía.

\item FLI: el formato FLI fue originalmete creado por Autodesk para 
la realizacion de animacines virtuales con ordenador.

\item Fax G3: este formato se usa para poder procesar faxes.

\item GBR: es el formato para las Brochas de {\sf GIMP} (Gimp brush). 

\item GIF: El  Formato de Intercambio de Gráficos  (GIF) fue inventado
por CompuServe  y es uno estándares  de las imágenes en  la World Wide
Web. Sin embargo, las patentes de Unisys e IBM que cubren el algoritmo
de compresión LZW que es utilizado  para crear los archivos GIF, hacen
imposible tener software libre que genere GIFs adecuados.

\item GIH: este formato lo usa {\sf The GIMP} para guardar las brochas
animadas que  aparecen en  las herramientas. GIH  es el  acróninimo de
(GIMP Image Hose).

\item GIcon: este  es el formato nativo de los  iconos {\sf The GIMP}.
Este formato sólo permite escala de grises.

\item HRZ: formato siempre a 256x240  pixels y es (o mejor, era) usado
para edición de imágenes de TV. No tiene compresión.

\item Jpeg:  es acrónimo  de Joint  Photographic Experts  Group (Unión
de  Grupo  de  Expertos  en  Fotografía)  y  funciona  con  todas  las
profundidades de color. La compresión  de la imagen es ajustable, pero
altas compresiones dañarán la calidad final  de la foto, ya que es una
compresión con pérdidas.

\item  MPEG:  acrónimo  de  Motion Picture  Experts  Group  (Grupo  de
Expertos  en  Animación). Es  un  bien  conocido  de los  formatos  de
animación.

\item PAT: es el formato nativo de Patrones (Patterns) {\sf The GIMP}.

\item PCX: este formato gráfico fue  creado por ZSoft y difundido por 
la familia de programas de dibujo Paintbrush.                         

\item PIX: este es el formato usado por el programa Alias/Wavefront en
estaciones SGI  (Silicon Graphics). Sólo  permite imágenes a  color de
24-bits e imágenes en escala de grises de 8-bits.

\item PNG:  El formato PNG  (Portable Network Graphics) es  un formato
gráfico que usa compresión sin  pérdidas (loseless compression). Es el
formato actualmente  recomendado por  la organización W3C  (World Wide
Web Consortium) para imágenes sin pérdida de calidad.

\item PNM: Acrónimo de Portable  aNyMap. PNM permite paleta de colores
indexada, escala de grises e imágenes a todo color.

\item  PSD: formato  usado  por  el Adobe  Photoshop  ({\sf The  GIMP}
mantendrá las capas existentes).

\item  PSP:  formato  usado  por  el PaintShop  Pro  ({\sf  The  GIMP}
mantendrá las capas existentes).

\item PostScript (PS): PostScript fue creado por Adobe. Es un lenguaje
para describir  páginas, y  es usado  principalmente por  impresoras y
otros dispositivos de impresión. Es una manera estupenda de distribuir
documentos.  También  podemos leer  ficheros  PDF  (Acrobat) con  esta
opción.

\item  SGI: es  el formato  originalmente usado  por las  aplicaciones
gráficas de SGI.

\item  SUNRAS:  acrónimo de  SUN  RASterfile.  Este formato  es  usado
principalmente por las diferentes  aplicaciones de Sun. Permite escala
de grises, color indexado y todo color.

\item TGA:  este formato permite  compresión a 8,  16, 24, 32  bits de
profundidad.

\item TIFF:  acrónimo de  Tagged Image File  Format. Este  formato fué
diseñado para ser un estandar. Este es un formato de alta calidad y es
perfecto  cuando quieras  importar  imágenes de  otros programas  como
FrameWork o Corel Draw.

\item  URL:  acrónimo  de  Uniform Resource  Locator  (Localizador  de
Recursos  Uniforme).  Podrás  descargar   una  imágen  desde  internet
diréctamente al {\sf GIMP}. El formato  del nombre del fichero es {\tt
ftp://dirección/archivo} o {\tt http://dirección/archivo}.

\item WMF: acrónimo de Windows Meta File (Meta Fichero de Windows). Es
un  formato que  permite guardar  tanto gráficos  vectoriales como  en
mapas de bits.

\end{itemize}


% Para aprender  a manejar GIMP sólo  se necesitan ganas, osadía  y unos
% cuantos ratitos para sentarse y  ponerse a jugar con las herramientas,
% filtros  y  scripts  que  proporciona.  Si  quieres  sacarle  el  jugo
% puede  resultarte muy  útil  algún  libro sobre  {\sf  The GIMP}  como
% \cite{gimpref}.


%\section{DIA}

\section{QCad}

\index{QCad}

{\sf  QCad}  es un  programa  de  diseño  asistido por  ordenador  (en
inglés  CAD,  de  Computer  Aided Design)  para  trabajar  con  planos
bidimensionales.  No necesitas  conocimientos  de CAD  para empezar  a
trabajar  con {\sf  QCad}, sobretodo  si  ya has  trabajado con  otros
programas de CAD. {\sf QCad} es realmente fácil de utilizar si prestas
atención a la barra de estado de la parte inferior.

La siguiente  figura muestra {\sf  QCad} con  uno de los  ejemplos que
incluye,  el diseño  de  un tornillo  de banco.  Como  puedes ver,  la
prioridad en un plano no es la estetica sino la precisión.

Mira el apéndice de la página \pageref{recursos} para ver la dirección
de un tutorial para {\sf QCad}

\begin{figura}{qcad_main}{1}
\caption{QCad con un diseño de ejemplo}
\end{figura}

