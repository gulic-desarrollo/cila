%Autor: Félix J. Marcelo Wirnitzer
%email: fmarcelo@mencey.neuroomante.com
%chat: _ese en #gulic/irc.hispano.org

\chapter{El lenguaje HTML}
\label{html.tex}
\newcommand{\com}{``}
\newcommand{\elto}[1]{{\tt $<$#1$>$}}

Los orígenes de HTML se sitúan en la necesidad de proporcionar una
estructura a documentos que fueran accesibles a través de la red. Más
que definir la apariencia de un documento, HTML describe cómo debería
construirse. Con una metáfora visual, HTML le explica cómo se deben
encajar seis planchas de madera para crear un cubo, pero no dice nada
sobre si la madera debe ser de roble, pino o caoba, o si debe
barnizarse o pintarse.

A  lo  largo del  tiempo,  la  demanda  de  los usuarios  ha  motivado
la  inclusión  de  elementos  relativos   a  la  presentación  en  las
especificaciones  de  HTML, a  pesar  de  que  cada  vez se  veía  más
claro  que la  idea  original de  HTML era  definir  estructuras y  no
presentaciones. Tratemos de explicarlo mejor:  la idea original es que
cuando hacemos  una página  web con HTML,  deberíamos disponer  de dos
ficheros:  la propia  página web  con  el contenido  que nos  interesa
mostrar  y  un  fichero  de estilos.  Combinando  estos  dos  ficheros
tendremos  nuestra página  web.  Ahora  bien, con  el  tiempo, se  han
fundido estos dos ficheros en uno sólo creando un gran problema que es
de no seguir  los estándares que inicialmente se crearon.  Y ése es el
principal  problema de  ver páginas  webs desde  distintos navegadores
\footnote{No hay  más que  recordar cuando leemos  ``Página optimizada
por Internet Explorer'' o lo mismo para el Nestcape}

\section{Definición de la estructura de un documento}

Todos los documentos HTML están formados por cuatro partes:

\begin{enumerate}

\item Una línea declarando qué versión  de HTML se ha usado para crear
el documento.  Esta parte  es simplemente esta  línea:

\begin{verbatim}
<!DOCTYPE HTML PUBLIC = "-//W3C//DTD HTML 4.0 Transitional//EN"
"http://www.w3c.org/TR/REC-html40/loose.DTD"> 
\end{verbatim}

\item Un  elemento HTML  que describe el  documento como  un documento
HTML. Todos los  contenidos de un documento HTML, con  la excepción de
la declaración  del tipo de do\-cu\-men\-to,  \textbf{deben encerrarse
entre  las  etiquetas  \elto{HTML}   y  \elto{/HTML}}.  El  resto  del
documento se  encontrará entre las  etiquetas de apertura y  cierre de
HTML.

\item  Una  sección  que  declara el  encabezado,  con  las  etiquetas
\elto{HEAD} y \elto{/HEAD}. Esta sección contendrá \textit{metadatos},
que  son  datos  que  describen otros  datos  e  incluyen  información
tal  como  palabras  claves  o descripciones  cortas  que  usarán  las
herramientas de búsqueda  de la Red, autor  del documento, información
del  control de  versiones  y cualquier  otro dato  que  no pueda  ser
considerado como contenido del documento.

\item  El  cuerpo  principal,  donde se  encuentra  el  contenido  del
documento. Para  ello se pueden  utilizar las etiquetas  \elto{BODY} o
\elto{FRAMESET}.

\end{enumerate}

Por tanto, un  documento muy básico HTML que  contuviera un encabezado
con un título tendría el siguiente aspecto:

\begin{verbatim}
<!DOCTYPE HTML PUBLIC = "-//W3C//DTD HTML 4.0 Transitional//EN"
"http://www.w3c.org/TR/REC-html40/loose.DTD">
<HTML>
   <TITLE>
        Esto es un ejemplo en HTML.
   </TITLE>
</HTML>
\end{verbatim}

\section{¿Qué hacemos con esto?}

Con  este somero  ejemplo lo  que  se debe  hacer es  guardarlo en  un
fichero de texto con la extensión {\tt  .htm} o {\tt .html} y desde un
navegador (Konqueror,  Netscape o Galeon, según  se prefiera) abrir el
fichero con el nombre con que lo hemos guardado.
Desde ese momento ya podemos ver resultados.

Cada vez que cambiemos algo, lo guardamos e indicamos al navegador que
refresque la información.


\section{El cuerpo del documento}

Como su  propio nombre indica, en  el cuerpo, del documento  se sitúan
los contenidos del documento. Es  lo que normalmente se considera como
el ``contenido" del propio  documento o el  texto de un  libro sin
pensar en las cubiertas o tapas del libro.

Estos contenidos  deben encerrarse  entre las etiquetas  \elto{BODY} y
\elto{/BODY}, que deben estar situados {\bf inmediatamente} después de
\elto{/HEAD}.

\begin{verbatim}
<!DOCTYPE HTML PUBLIC = "-//W3C//DTD HTML 4.0 Transitional//EN"
"http://www.w3c.org/TR/REC-html40/loose.DTD">
<HTML>
   <TITLE>
        Esto es un ejemplo en HTML.
   </TITLE>

   <BODY>
        Aquí ya estamos dentro del propio cuerpo.
   </BODY>
</HTML>
\end{verbatim}

Cuando  abrimos  el  cuerpo  (\elto{BODY}),  podemos  establecer  unos
atributos o propiedades, que pueden ser los siguientes:

\begin{description}

\item[ALINK] Especifica el color de un enlace cuando se activa.

\item[BACKGROUND="imagen"]  Especifica un  gráfico que  se usará  como
fondo a modo de mosaico para el documento.

\item[BGCOLOR="nombrecolor"]   Especifica  el   color  de   fondo  del
documento.

\item[LINK="nombrecolor"]  Especifica  el  color de  los  enlaces  del
documento.

\item[TEXT="nombrecolor"] Especifica el color del texto del documento.

\item[VLINK="nombrecolor"]  Especifica el  color  de  los enlaces  del
documento que han sido visitados.

\end{description}

Los colores que se pueden usar son:
\begin{itemize}
        \item Acqua (Agua)
        \item Black (Negro)
        \item Blue (Azul)
        \item Fuchsia (Fucsia)
        \item Grey (Gris)
        \item Green (Verde)
        \item Lime (Verde Lima)
        \item Maroon (Marrón
        \item Navy (Azul Marino)
        \item Olive (Verde oliva)
        \item Purple (Púrpura)
        \item Red (Rojo)
        \item Silver (Plata)
        \item Teal (Azul Verdoso)
        \item White (Blanco)
        \item Yellow (Amarillo)
\end{itemize}

Por ejemplo,

\begin{verbatim}
<BODY BGCOLOR="Blue" Text="Yellow" >
        Esto es una página web con letras amarillas y fondo azul.
</BODY>
\end{verbatim}

\section{Estructura del contenido de un documento con encabezados}

Casi todos  los documentos  pueden dividirse  en distintos  bloques de
texto  o  apoyarse  con  información adicional  como  ilustraciones  o
fotografías. Incluso los documentos  HTML más básicos presentarán este
comportamiento.

Por ejemplo,  puede que su  documento tenga dos párrafos,  una imagen,
una  cita  y un  párrafo  final,  por lo  que  podría  decirse que  su
documento consta de cinco bloques diferentes.

Para  ello, echamos  mano de  los encabezados.  Éstos constan  de seis
niveles, desde \elto{H1}  hasta \elto{H6}, es decir, de  mayor a menor
importancia.  Lo que  entendemos por  nivel de  importancia no  es más
que  resaltar un  texto variando  su tamaño.  Evidentemente, un  texto
resaltará con un gran tamaño frente a otro de menor tamaño. Por tanto,
con \elto{H1} obtendremos el mayor tamaño y con \elto{H6} el menor.

\begin{verbatim}
<H1>
        Resaltado de nivel 1
</H1>

<H2>
        Resaltado de nivel 2
</H2>
\end{verbatim}
\dots
\begin{verbatim}
<H6>
        Resaltado de nivel 6
</H6>
\end{verbatim}


\subsection{Formato de texto}

\subsubsection*{Párrafo}

El tipo más básico de formatos de  texto es la agrupación de frases en
párrafos. Para ello se usa el elemento \elto{P}:

\begin{verbatim}
<P>El tipo más básico de formatos de texto es la agrupación de
frases en párrafos.</P>
\end{verbatim}

\subsubsection*{Énfasis}

En casi todos  los bloques existe una  parte que se le  quiere dar una
mayor importancia  que el resto del  texto. HTML marca esta  parte con
\elto{EM}:

\begin{verbatim}
Si te digo que no, es que quiero decir <EM>NO</EM>.
\end{verbatim}

\noindent
\subsubsection*{Énfasis fuerte}

Como todo en la vida, hay también grados de acentuar el énfasis en las
cosas. Esto lo hacemos con \elto{STRONG}:

\begin{verbatim}
Si te digo que no, es que quiero decir <STRONG>NO</STRONG>.
\end{verbatim}

\subsection{Listas}

Las listas en HTML se dividen en tres categorías básicas:

\begin{itemize}
\item \textit{Listas numeradas}
\item \textit{Listas no numeradas}
\item \textit{Listas de definición}
\end{itemize}

Las dos primeras son similares.  Una \textit{lista numerada} usará los
elementos  \elto{OL} y  \elto{/OL} mientras  que una  \textit{lista no
numerada} hará uso de los  elementos \elto{UL} y \elto{/UL}. Dentro de
estos elementos, para  marcar cada miembro de la lista,  se utiliza la
etiqueta común \elto{LI}.

\subsubsection*{Listas no numeradas}

Son aquéllas cuyos elementos no siguen un orden determinado, aunque se
muestran en el orden en el que se escriban en el archivo fuente.

Para crear una lista no numerada, se usa la etiqueta \elto{UL} seguido
de una colección de elementos con \elto{LI}:

\begin{verbatim}
<P>Coches que he tenido</P>
  <UL>
     <LI> 1993 - Seat Ronda </LI>
     <LI> 1996 - Volkswagen </LI>
     <LI> 1999 - Audi A4 </LI>
  </UL>
\end{verbatim}


\subsubsection*{Listas numeradas}

Son aquéllas  cuyos elementos deben  seguir un orden  determinado. Por
ejemplo, los  pasos que  deben seguir  en una receta  de cocina  o las
instrucciones para montar en bicicleta.

Para crear una lista numerada, se usa la etiqueta \elto{OL} seguido de
una colección de elementos con \elto{LI}:

\begin{verbatim}
<P>Cosas que tengo que hacer hoy</P>
  <UL>
     <LI> Mirar el correo </LI>
     <LI> Escribir un poco del libro </LI>
     <LI> Dar una clase </LI>
  </UL>
\end{verbatim}

\subsubsection*{Listas de definiciones}

Estas listas se  diferencian de las anteriores en que  cada entrada de
la lista  consta de dos partes:  el término definido, \elto{DT},  y la
propia descripción de la definición,  \elto{DD}. La lista comienza con
la etiqueta de las listas  de definiciones, \elto{DL}. Cada definición
debe tener obligatoriamente un término  y una descripción, tal como se
ve en este ejemplo:

\begin{verbatim}
  <DL>
     <DT>RAM
       <DD>Memoria de acceso aleatorio</DD>
     </DT>

     <DT>BIOS
       <DD>Sistema de entrada y salida básico</DD>
     </DT>
  </DL>
\end{verbatim}

\subsection{Tablas}

Se  utiliza  la  etiqueta  \elto{TABLE}, que  acepta  siete  atributos
principales:

\begin{description}

\item[SUMMARY:]  Contiene un  texto  que describe  el  contenido y  la
estructura de la tabla.

\item [WIDTH:] Valor que  indica la anchura que se le  quiere dar a la
tabla en proporción con la anchura total de la tabla del navegador (es
decir, una valor de 100 indica que  la tabla debe ocupar todo el ancho
disponible de la ventana, lo cual no equivale al término de ``pantalla
completa").

\item [BORDER:] Determina la anchura del  borde de la tabla, medido en
píxels. Un valor de 0 (cero) hace que no se ponga borde.

\item [FRAME:]  También se  posible colocar un  marco alrededor  de la
tabla. El  valor predeterminado es  `` \textit{void}" (vacío),  lo que
significa que no existe ningún  marco. Se admiten valores alternativos
para indicar cada uno de los lados  o una combinación de los lados del
marco.

\item [RULES:] Son separadores visuales entre filas, columnas o grupos
de celdas en el interior de la tabla. De este modo se puede incluir un
separador  vertical entre  las columnas  pero ninguno  que separe  las
filas entre sí. El valor predeterminado es ``\textit{none}" (ninguno),
que significa que no se mostrará ningún separador. Los separadores son
elementos visuales distintos a los bordes.

\item [CELLSPACING:] Espacio  entre las celdas, de la  tabla medido en
píxels, cuyo valor predeterminado es 0 (cero).

\item [CELLPADDING:] Espacio entre de  la celda y su contenido, medido
también en píxels, cuyo valor predeterminado es 0 (cero).

\end{description}

Lo  primero  que  hacemos  es  poner una  tabla  entre  las  etiquetas
\elto{TABLE} Y \elto{/TABLE}. Después se va  de línea en línea con las
etiquetas  \elto{TR} Y  \elto{/TR}.  Dentro de  ellas,  ya usamos  las
etiquetas  para  cada celdas  de  estas  líneas  que son  \elto{TD}  y
\elto{/TD}, que serán tantas como columnas queramos:

%\newpage

\begin{verbatim}
<TABLE BORDER=1>
        <TH>  ;Si queremos poner una fila a modo de cabecera ...
                <TD>Estamos en la primera celda de la cabecera</TD>
                <TD>Estamos en la segunda celda de la cabecera</TD>
                <TD>Estamos en la tercera celda de la cabecera</TD>
        </TH>
        <TR>  ;Primera fila
                <TD>Estamos en la primera celda de la primera fila</TD>
                <TD>Estamos en la segunda celda de la primera fila</TD>
                <TD>Estamos en la tercera celda de la primera fila</TD>
        </TR>
        <TR>  ;Segunda fila
                <TD>Estamos en la primera celda de la segunda fila</TD>
                <TD>Estamos en la segunda celda de la segunda fila</TD>
                <TD>Estamos en la tercera celda de la segunda fila</TD>
        </TR>
;Y así sucesivamente ...
</TABLE>
\end{verbatim}

En caso  de que no  queramos rellenar  nada en una  celda, simplemente
ponemos \elto{TD}\elto{/TD}

\subsection{Enlaces}

Antes que  nada, pasemos a  definir lo que  es un enlace.  Todos hemos
visto que cuando navegamos por una  página web, hay ciertas palabras o
gráficos que en el momento de  pasar el puntero del ratón se convierte
en  una mano  indicándonos que  si pinchamos  ahí nos  enviará a  otra
página web.  Pues bien, eso  es un enlace y  en el momento  de crearlo
dentro de un documento HTML tendrá el siguiente aspecto:

\begin{verbatim}
<A HREF="http://www.google.com">Este enlace nos lleva a la página del
Google</A>
\end{verbatim}

Donde el  texto que hemos escrito,  "Este enlace..." es el  área donde
debemos pinchar  con el ratón  para acceder a  esa página. No  sólo se
puede meter texto, sino también imágenes u otros objetos.

Con esto también podemos hacer un acceso a un correo electrónico:

\begin{verbatim}
<A HREF="mailto:cila@fmat.ull.es">Podemos probar a enviar un mensaje
a los autores de este libro</A>
\end{verbatim}

Con todo  lo explicado  en esta introducción  podemos empezar  a hacer
nuestros pinitos en el apasionante mundo del lenguaje HTML.
