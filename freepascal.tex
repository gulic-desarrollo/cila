%Autor: mojopikon
%mojopikon: 6

\chapter{FreePascal}
\label{freepascal.tex}
\index{FreePascal}

El lenguaje de  programación Pascal es sencillo  y bastante didáctico,
por  lo que  se suele  enseñar en  el primer  año de  algunas carreras
técnicas como  Matemáticas, Física  o Informática.  Normalmente, estos
cursos o asignaturas  de programación tratan de  enseñar al estudiante
los conceptos  básicos de la  programación de computadores  sin entrar
en  demasiados  detalles  acerca  del funcionamiento  interno  de  los
mismos. En este capítulo aprenderemos  las herramientas básicas que se
encuentran  disponibles  en GNU/Linux  para  programar  en Pascal.  El
compilador que  utilizaremos está siendo desarrollado  por el proyecto
{\bf Free Pascal},  que proporciona un buen compilador  de Pascal para
múltiples  plataformas, entre  éstas  GNU/Linux,  MS-DOS, MS  Windows,
Amiga, MaC OS y otras.

Comenzaremos  escribiendo  un  ejemplo  muy sencillo  de  programa  en
Pascal, el  típico ``Hola Mundo''.  En cualquier editor  escribimos el
siguiente código y lo guardamos con el nombre {\tt HolaMundo.pas}

\begin{ejemplo}{HolaMundo.pas}%
{Ejemplo 1 de Pascal }
Ejemplo 1 de Pascal 
\end{ejemplo}

Para  compilar un  programa escrito  en  Pascal con  el compilador  de
FreePascal utilizamos el comando {\tt ppc386} del siguiente modo:

\begin{verbatim}
$ ppc386 HolaMundo.pas
Free Pascal Compiler version 1.0.4 [2001/08/31] for i386
Copyright (c) 1993-2000 by Florian Klaempfl
Target OS: Linux for i386
Compiling HolaMundo.pas
Assembling holamundo
Linking holamundo
7 Lines compiled, 0.3 sec

$ ls
holamundo2    HolaMundo2.pas  holamundo2.o
\end{verbatim}

Como  podemos apreciar  en los  mensajes del  compilador, él  mismo se
encarga de compilar,  ensamblar y enlazar el programa  para generar el
fichero  ejecutable  {\tt  holamundo}.  Para  cambiar  el  nombre  del
ejecutable resultante se utiliza la opción {\tt -onombredelejecutable}
(sin dejar espacio entre la {\tt o} y el nombre del ejecutable.

\begin{verbatim}
$ ppc386 -oHolaMundo HolaMundo.pas
Free Pascal Compiler version 1.0.4 [2001/08/31] for i386
Copyright (c) 1993-2000 by Florian Klaempfl
Target OS: Linux for i386
Compiling HolaMundo.pas
Assembling HolaMundo
Linking HolaMundo
7 Lines compiled, 0.3 sec

$ ls
HolaMundo    HolaMundo2.pas  holamundo2.o
\end{verbatim}

Para ejecutar  el programa resultante,  hemos de recordar  que debemos
poner {\tt ./} delante del nombre del ejecutable:

\begin{verbatim}
$ HolaMundo
bash: HolaMundo: command not found
$ ./HolaMundo
Hola Mundo
\end{verbatim}

Veamos ahora  un ejemplo  del uso  de las  ``unidades'' en  Pascal. El
concepto de unidades en Pascal es equivalente al de librerías en C. Se
trata de  ficheros binarios  que obtenemos a  partir de  código fuente
separado y  luego enlazamos  con el  programa principal.  Esto permite
dividir el código de un programa  en varios ficheros y evita tener que
compilar todo el programa cada vez que se modifica una función. Con el
uso de unidades  basta con recompilar la unidad en  la que se modifica
el código  fuente y volverla a  enlazar con el programa.  A diferencia
del Borland  Pascal, el  compilador Free  Pascal utiliza  la extensión
{\tt  ppu} (en  lugar  de {\tt  tpu}) para  los  ficheros binarios  de
unidades.

Tenemos para  este ejemplo  dos ficheros de  código fuente  en Pascal,
{\tt HolaMundo2.pas} y {\tt saludos.pas}.

\begin{ejemplo}{HolaMundo2.pas}%
{Ejemplo 2 de Pascal }
Ejemplo 2 de Pascal 
\end{ejemplo}

\begin{ejemplo}{saludos.pas}%
{Ejemplo 2 de Pascal }
Ejemplo 2 de Pascal 
\end{ejemplo}

Debemos  tener cuidado  con un  detalle: Los  nombres de  las unidades
deben coincidir  con el nombre del  fichero en el que  están escritas,
i.e.  la unidad  {\tt  Saludos} no  se puede  escribir  en un  fichero
llamado  {\tt Otronombre.pas}.  Además,  los ficheros  en  los que  se
implementan las  unidades conviene que tengan  el nombre completamente
en  {\bf minúsculas}.  De lo  contrario, el  compilador FreePascal  no
encontrará la unidad al compilar un programa que la use, pero podremos
aún compilarla manualmente. El siguiente ejemplo ilustra la situación:

\begin{verbatim}
$ ls
HolaMundo2.pas  saludos.pas

$ mv saludos.pas otronombre.pas

$ ls
HolaMundo2.pas  otronombre.pas

$ ppc386 otronombre.pas
Free Pascal Compiler version 1.0.4 [2001/08/31] for i386
Copyright (c) 1993-2000 by Florian Klaempfl
Target OS: Linux for i386
Compiling otro.pas
otro.pas(4,6) Error: Illegal unit name: SALUDOS
otro.pas(10,1) Fatal: There were 1 errors compiling module, stopping

$ mv otronombre.pas Saludos.pas

$ rm *.o *.ppu

$ ls
HolaMundo2.pas  Saludos.pas

$ ppc386 HolaMundo2.pas
Free Pascal Compiler version 1.0.4 [2001/08/31] for i386
Copyright (c) 1993-2000 by Florian Klaempfl
Target OS: Linux for i386
Compiling HolaMundo2.pas
HolaMundo2.pas(6,16) Fatal: Can't find unit SALUDOS

$ ppc386 Saludos.pas 
Free Pascal Compiler version 1.0.4 [2001/08/31] for i386
Copyright (c) 1993-2000 by Florian Klaempfl
Target OS: Linux for i386
Compiling Saludos.pas
Assembling saludos
18 Lines compiled, 0.0 sec

$ ppc386 HolaMundo2.pas
Free Pascal Compiler version 1.0.4 [2001/08/31] for i386
Copyright (c) 1993-2000 by Florian Klaempfl
Target OS: Linux for i386
Compiling HolaMundo2.pas
Assembling holamundo2
Linking holamundo2
17 Lines compiled, 0.0 sec

$ ls
holamundo2    HolaMundo2.pas  Saludos.pas
holamundo2.o  saludos.o       saludos.ppu

$ ./holamundo2
¿Cómo te llamas? Pepe
Hola Pepe

$ mv Saludos.pas saludos.pas

$ rm *.o *.ppu

$ ls
HolaMundo2.pas  saludos.pas

$ ppc386 HolaMundo2.pas 
Free Pascal Compiler version 1.0.4 [2001/08/31] for i386
Copyright (c) 1993-2000 by Florian Klaempfl
Target OS: Linux for i386
Compiling HolaMundo2.pas
Assembling holamundo2
Linking holamundo2
18 Lines compiled, 0.0 sec

$ ls
holamundo2    HolaMundo2.pas  Saludos.pas
holamundo2.o  saludos.o       saludos.ppu

$ ./holamundo2
¿Cómo te llamas? Pepe
Hola Pepe
\end{verbatim}

En FreePascal para Linux, al contrario que la versión windows de deste
mismo compilador, hacer trazas a los programas es algo más complejo, y
requiere un programa adicional: GNU Debugger (GDB). Para utilizar este
programa con nuestros programas en Pascal, debemos añadir un parámetro
adicional en la línea de comandos:

\begin{verbatim}

ppc386 -g programa.pas

\end{verbatim}

Con esto, ``incrustamos'' el código fuente de nuestro programa en el
interior del ejecutable, de forma que si ejecutamos gdb pasando como
argumento el nombre del programa, podremos hacerle una traza.

Para saber  más acerca de  como hacer  una traza a  nuestros programas
escritos  en Pascal,  puedes consultar  la  sección sobre  GDB, en  la
página \pageref{gdb}.

Otros argumentos interesantes del compilador de FreePascal:

Sin entrar  demasiado en  las interioridades  de FreePascal,  creo que
estas opciones pueden  ayudarte a la hora de compilar  una práctica, o
de estructurar mejor tus proyectos. Ahí van:

{\tt -FUdirectorio}:  Esta opción  nos permite guardar  las ``units'',
una vez compiladas, en el directorio especificado. Por ejemplo:

\begin{verbatim}
ppc386 -FUunits unidad.pas
\end{verbatim}

Para utilizar units que estén en otro directorio que no sea el actual,
utilizamos el parámetro  {\tt -Fudirectorio} (nótese que  en este caso
la ``u'' es minúscula).

%\index{Depuradores!Turbo Pascal 7}

{\tt -So}: Usar esta opción nos permitirá comprobar la compatibilidad
de nuestro código con Turbo Pascal 7

{\tt -Un}: Esta opción nos permite llamar los ficheros de unidad (units)
con otro nombre distinto al que hemos especificado en el código tras
la palabra reservada "unit"

% Hey, sería bueno añadir unos cuantos switches más :-)

Por último, algunos editores que ``se llevan bien'' con FreePascal:

{\tt Anjuta} Tal vez, el mejor  editor para programadores (o al menos,
en opinión del autor de este tema). Soporta resaltado de sintaxis para
varios lenguajes, compilar y ejecutar desde el mismo editor, etc.

{\tt Emacs} A pesar de su aparente dificultad de manejo, el modo de Pascal
está muy trabajado, permitiendo indentar automáticamente todo el código
tan sólo pulsando la tecla tab en cada línea.

{\tt Vim} El autor de este tema  sólo ha usado de vim la funcionalidad
de resaltado de sintaxis mientras programaba en Pascal. Esta se activa
con la opción ":syntax on".

{\tt Scite} Es un editor muy  minimalista, y algo lioso de configurar,
pero es un editor para programadores  muy bueno, cómodo, y sobre todo,
compacto (cabe en un disco de  1'44), además de estar disponible tanto
para plataformas Win32 como GNU/Linux.

% Con esta  breve presentación  ya sabemos  lo suficiente  para compilar
% prácticas  en  Pascal,  sólo  queda   aprender  el  lenguaje  y  pasar
% muchas  horas  escribiendo  programas  para llegar  a  ser  auténticos
% programadores.
