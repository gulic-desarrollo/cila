%Autor: mojopikon

\chapter{FreePascal}


El lenguaje de  programaci�n Pascal es sencillo  y bastante did�ctico,
por  lo que  se suele  ense�ar en  el primer  a�o de  algunas carreras
t�cnicas  como Matem�ticas,  F�sica o  Inform�tica. Normalmente  estos
cursos o asignaturas  de programaci�n tratan de  ense�ar al estudiante
los conceptos  b�sicos de la  programaci�n de computadores  sin entrar
en  demasiados  detalles  acerca  del funcionamiento  interno  de  los
mismos. En este cap�tulo aprenderemos  las herramientas b�sicas que se
encuentran  disponibles  en GNU/Linux  para  programar  en Pascal.  El
compilador que  utilizaremos est� siendo desarrollado  por el proyecto
{\bf Free Pascal},  que proporciona un buen compilador  de Pascal para
m�ltiples  plataformas, entre  �stas  GNU/Linux,  MS-DOS, MS  Windows,
Amiga, MaC OS y otras.

Comenzaremos  escribiendo  un  ejemplo  muy sencillo  de  programa  en
Pascal, el  t�pico ``Hola Mundo''.  En cualquier editor  escribimos el
siguiente c�digo y lo guardamos con el nombre {\tt HolaMundo.pas}

\begin{verbatim}
{ Ejemplo 1 de Pascal para CILA }

Program HolaMundo;

Begin
  writeln ('Hola Mundo');
End.
\end{verbatim}

Para  compilar un  programa escrito  en  Pascal con  el compilador  de
FreePascal utilizamos el comando {\tt ppc386} del siguiente modo:


\begin{verbatim}
$ ppc386 HolaMundo.pas
Free Pascal Compiler version 1.0.4 [2001/08/31] for i386
Copyright (c) 1993-2000 by Florian Klaempfl
Target OS: Linux for i386
Compiling HolaMundo.pas
Assembling holamundo
Linking holamundo
7 Lines compiled, 0.3 sec

$ ls
holamundo2    HolaMundo2.pas  holamundo2.o
\end{verbatim}

Como  podemos apreciar  en los  mensajes del  compilador, �l  mismo se
encarga de compilar,  ensamblar y enlazar el programa  para generar el
fichero  ejecutable  {\tt  holamundo}.  Para  cambiar  el  nombre  del
ejecutable resultante se utiliza la opci�n {\tt -onombredelejecutable}
(sin dejar espacio entre la {\tt o} y el nombre del ejecutable.

\begin{verbatim}
$ ppc386 -oHolaMundo HolaMundo.pas
Free Pascal Compiler version 1.0.4 [2001/08/31] for i386
Copyright (c) 1993-2000 by Florian Klaempfl
Target OS: Linux for i386
Compiling HolaMundo.pas
Assembling HolaMundo
Linking HolaMundo
7 Lines compiled, 0.3 sec

$ ls
HolaMundo    HolaMundo2.pas  holamundo2.o
\end{verbatim}

Para ejecutar el programa resultante,  recordar que debemos poner {\tt
./} delante del nombre del ejecutable:

\begin{verbatim}
$ HolaMundo
bash: HolaMundo: command not found
$ ./HolaMundo
Hola Mundo
\end{verbatim}

Veamos ahora  un ejemplo  del uso  de las  ``unidades'' en  Pascal. El
concepto de unidades en Pascal es equivalente al de librer�as en C, se
trata de  ficheros binarios  que obtenemos a  partir de  c�digo fuente
separado y  luego enlazamos  con el  programa principal.  Esto permite
dividir el c�digo de un programa  en varios ficheros y evita tener que
compilar todo el programa cada vez que se modifica una funci�n. Con el
uso de unidades  basta con recompilar la unidad en  la que se modifica
el c�digo  fuente y volverla a  enlazar con el programa.  A diferencia
del Borland  Pascal, el  compilador Free  Pascal utiliza  la extensi�n
{\tt  ppu} (en  lugar  de {\tt  tpu}) para  los  ficheros binarios  de
unidades.


Tenemos para  este ejemplo  dos ficheros de  c�digo fuente  en Pascal,
{\tt HolaMundo2.pas} y {\tt saludos.pas}.


\begin{verbatim}
{ Ejemplo 2 de Pascal para CILA }
{    Fichero: HolaMundo2.pas    }

Program HolaMundo2;
Uses
  Crt, Saludos ;
Var
  nombre : string ;
Begin
  TextColor(13) ;
  write ('�C�mo te llamas? ') ;
  TextColor(15) ;
  readln (nombre) ;
  TextColor(14) ;
  Saluda (nombre) ;
  TextColor(7) ;
End.
\end{verbatim}

\begin{verbatim}
{ Ejemplo 2 de Pascal para CILA }
{    Fichero: saludos.pas       }

Unit Saludos ;

Interface

Uses Crt ;

Procedure Saluda ( mensaje : string ) ;

Implementation

Procedure Saluda ( mensaje : string ) ;
Begin
  writeln('Hola ', mensaje);
End;

Begin
End.
\end{verbatim}

Cuidado con  un detalle: Los  nombres de las unidades  deben coincidir
con el  nombre del fichero  en el que  est�n escritas, i.e.  la unidad
{\tt  Saludos}  no  se  puede  escribir en  un  fichero  llamado  {\tt
Otronombre.pas}.  Adem�s,  los  ficheros  en los  que  se  implementan
las  unidades conviene  que  tengan el  nombre  completamente en  {\bf
min�sculas}. De lo contrario el compilador FreePascal no encontrar� la
unidad  al  compilar  un  programa  que  la  use,  pero  podremos  a�n
compilarla manualmente. El siguiente ejemplo ilustra la situaci�n:

\begin{verbatim}
$ ls
HolaMundo2.pas  saludos.pas

$ mv saludos.pas otronombre.pas

$ ls
HolaMundo2.pas  otronombre.pas

$ ppc386 otronombre.pas
Free Pascal Compiler version 1.0.4 [2001/08/31] for i386
Copyright (c) 1993-2000 by Florian Klaempfl
Target OS: Linux for i386
Compiling otro.pas
otro.pas(4,6) Error: Illegal unit name: SALUDOS
otro.pas(10,1) Fatal: There were 1 errors compiling module, stopping

$ mv otronombre.pas Saludos.pas

$ rm *.o *.ppu

$ ls
HolaMundo2.pas  Saludos.pas

$ ppc386 HolaMundo2.pas
Free Pascal Compiler version 1.0.4 [2001/08/31] for i386
Copyright (c) 1993-2000 by Florian Klaempfl
Target OS: Linux for i386
Compiling HolaMundo2.pas
HolaMundo2.pas(6,16) Fatal: Can't find unit SALUDOS

$ ppc386 Saludos.pas 
Free Pascal Compiler version 1.0.4 [2001/08/31] for i386
Copyright (c) 1993-2000 by Florian Klaempfl
Target OS: Linux for i386
Compiling Saludos.pas
Assembling saludos
18 Lines compiled, 0.0 sec

$ ppc386 HolaMundo2.pas
Free Pascal Compiler version 1.0.4 [2001/08/31] for i386
Copyright (c) 1993-2000 by Florian Klaempfl
Target OS: Linux for i386
Compiling HolaMundo2.pas
Assembling holamundo2
Linking holamundo2
17 Lines compiled, 0.0 sec

$ ls
holamundo2    HolaMundo2.pas  Saludos.pas
holamundo2.o  saludos.o       saludos.ppu

$ ./holamundo2
�C�mo te llamas? Pepe
Hola Pepe

$ mv Saludos.pas saludos.pas

$ rm *.o *.ppu

$ ls
HolaMundo2.pas  saludos.pas

$ ppc386 HolaMundo2.pas 
Free Pascal Compiler version 1.0.4 [2001/08/31] for i386
Copyright (c) 1993-2000 by Florian Klaempfl
Target OS: Linux for i386
Compiling HolaMundo2.pas
Assembling holamundo2
Linking holamundo2
18 Lines compiled, 0.0 sec

$ ls
holamundo2    HolaMundo2.pas  Saludos.pas
holamundo2.o  saludos.o       saludos.ppu

$ ./holamundo2
�C�mo te llamas? Pepe
Hola Pepe
\end{verbatim}

Con esta  breve presentaci�n  ya sabemos  lo suficiente  para compilar
pr�cticas en  Pascal, s�lo queda  aprender el lenguaje y  pasar muchas
horas escribiendo programas para llegar a ser aut�nticos programadores
;-)


