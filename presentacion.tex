%Autor: miguev
%miguev: 7

\chapter*{Presentación}

\section*{Historia}

Desde el curso  académico 2001--2002 la \FMAT~ de la  \ULL~ dispone de
sistemas Debian  GNU/Linux en  las aulas  de ordenadores  destinadas a
impartir  clases prácticas.  Estas aulas  están destinadas  a que  los
alumnos  realicen sus  prácticas académicas,  tales como  programación
en  Pascal,  Fortran,  C  o Java,  análisis  estadísticos  o  cálculos
simbólicos. Sin embargo, los alumnos no disponían en su mayoría de los
conocimientos mínimos necesarios para  utilizar un equipo bajo sistema
operativo GNU/Linux.

Ante  tal  panorama, y  con  el  objetivo  urgente de  proporcionar  a
los  alumnos la  destreza  mínima que  necesitaban  para utilizar  los
ordenadores de  la facultad  con Debian  GNU/Linux, {\sc  Miguel Ángel
Vilela} (miembro  del \GULiC~ y  administrador de los  sistemas Debian
GNU/Linux de  las aulas)  propuso paralelamente  la realización  de un
{\bf\CILA} a {\sc Sergio Alonso} (Vicedecano de Estadística) y a otros
miembros del el \GULiC. En ambas  partes la idea tuvo buena aceptación
y en  sólo 10  días el \GULiC~  y la Facultad  de Matemáticas,  con el
patrocinio de IDE System Canarias S.L., organizaron la primera edición
del \CILA.

Tras el  éxito del  \CILA~ (30  inscritos, 27 asistentes  y más  de 50
alumnos  en lista  de espera)  se organizó  la primera  edición de  la
{\bf\PILA}, a la que no asistieron tantos alumnos debido (suponemos) a
que la asistencia implicaba traer  el ordenador personal. Sin embargo,
el balance final  de ambas actividades se consideró  muy positivo, por
lo que  el \GULiC~ se embarcó  en el siguente reto:  ampliar el \CILA~
para  aumentar  el temario  (incluyendo  la  \PILA). 

El \CILA~ se ha realizado ya dos años consecutivos en la \FMAT~ y para
los próximos años pretendemos  seguir realizándolo, ampliando su radio
de acción  a otras  facultades de  la \ULL. Este  libro es  una puerta
abierta  para que  el \CILA  sea  realizado en  otras universidades  y
centros docentes.

Parte  de  este proyecto  ha  sido  reformar  los apuntes  del  \CILA,
inicialmente  escritos  en {\sc  SGML}  (con  DTD de  LinuxDoc),  para
traducirlos a  {\LaTeX} y así  generar este libro, que  constituye los
apuntes  del  curso. Este  libro  es  una publicación  de  ``contenido
abierto'', lo  que significa que  está abierto a las  aportaciones que
pueda hacer cualquier persona y seguirá evolucionando con el tiempo.

\section*{Objetivos}

El objetivo del \CILA~ y de este libro es el aprendizaje del alumno (o
lector), a ser  posible de una forma amena. En  adelante trataremos de
tú  tanto al  alumno  como  al lector.  Esperamos  que  no te  sientas
ofendido.

Este libro  está dirigido  principalmente a  usuarios, con  un enfoque
fuertemente académico. Si hace poco  que llegaste al maravilloso mundo
de GNU/Linux  y necesitas una  guía práctica para empezar  a trabajar,
este libro puede  resultarte de utilidad. Si además  eres estudiante y
necesitas hacer tus prácticas en  ordenadores con GNU/Linux pero no te
orientas  aún,  este  libro  puede  ser tu  salvación.  Si  ya  tienes
experiencia con  GNU/Linux (no eres  un novato) no  encontrarás muchas
cosas nuevas aquí, aunque tampoco te vendrá mal leerlo.

La finalidad del \CILA~ es que los alumnos adquieran los conocimientos
necesarios  para  utilizar  los   ordenadores  con  GNU/Linux  en  sus
prácticas académicas, y tal vez se  animen a instalar GNU/Linux en sus
propios ordenadores  personales. Como se ha  mencionado anteriormente,
el  \CILA~ consta  de unas  clases teórico-prácticas,  en las  que los
alumnos toman apuntes  mientras prueban lo que van  aprendiendo con un
ordenador delante,  y unas jornadas  práctico-teóricas en la  que cada
alumno es invitado a traer  su propio ordenador personal para instalar
GNU/Linux  por  sí  mismo,  contando  siempre  con  la  ayuda  de  los
profesores.

Dada la posible variedad de los alumnos, la cantidad de materia que se
desea impartir y las limitaciones  de disponibilidad de puestos en las
aulas de  informática en  cuanto a  número y tiempo,  el \CILA~  se ha
dividido en varios  módulos, todos ellos opcionales.  

Las tareas rutinarias  de administración y mantenimiento  de un equipo
bajo sistema  operativo GNU/Linux  no se tratan  en este  libro, entre
otras razones porque suelen depender mucho  de la máquina en la que se
practiquen y  de las necesitades  de su(s) usuario(s). Esta  parte del
aprendizaje  de GNU/Linux  es demasiado  práctica para  un libro  como
éste. Para esto es mejor comprar algún libro de iniciación a Linux que
incluya  una distribución  de GNU/Linux  para instalar.  Este tipo  de
libros  se pueden  encontrar  en casi  cualquier  librería técnica  en
variedad para todos los gustos, desde pequeñas guías de 10 minutos con
CD de instalación rápida hasta libros gruesos de sabiduría condensada,
elige el que más te guste y reserva grandes ratos en tu agenda.

Uno de los motivos por los  que publicamos este libro bajo la licencia
GNU Free  Documentation Licence es  porque queremos que tenga  toda la
divulgación que  sus lectores quieran.  Este libro no es  copyright de
ninguna editorial, sólo  de los autores, y damos  nuestro permiso para
que este material sea  fotocopiado y redistribuido libremente. Nuestro
objetivo es acercar GNU/Linux y sus herramientas a cuantos más alumnos
y usuarios sea posible.

\section*{Contenidos}

Este  libro está  dividido  en  módulos atendiendo  a  la división  de
contenidos que hacemos a la hora  de impartir las clases, los cuales a
su  vez  están  divididos  en  temas,  cuyos  contenidos  resumimos  a
continuación.

\begin{description}

\item[\chaptername~ \ref{introduccion.tex}:  Introducción a GNU/Linux]
~\\ Cuando  hablamos de  GNU/Linux nos  referimos a  algo que  ha sido
formado por un sistema operativo llamado  Linux y una gran cantidad de
software  libre  proviniente  del  proyecto  GNU.  Daremos  un  repaso
histórico desde los orígenes de Linux y GNU hasta la actualidad.

\item[\chaptername~\ref{comandos.tex}: El intérprete  de comandos] ~\\
El primer mejor entorno de trabajo  que puedas encuentrar al entrar en
un sistema GNU/Linux es el intérprete  de comandos. Aunque cada día es
más frecuente  entrar a Linux  por las  ventanas, no debemos  dejar de
lado  la enorme  libertad que  nos da  un intérprete  de comandos.  Y,
aunque no lo parezca, es un entorno sumamente cómodo.

\item[\chaptername~\ref{xwindow.tex}:  El  entorno  X-Window]  ~\\  El
entorno  favorito de  casi todos  los usuarios  de computadores  es el
entorno de ventanas, en el que el ratón y el teclado se suman para dar
al usuario libertad y comodidad.  Verás que existe una amplia variedad
de entornos de  ventanas, y aprenderás trucos útiles para  cada uno de
ellos.

\item[\chaptername~\ref{editores.tex}: Editores: VIM,  GNU Emacs, Joe]
~\\ Una de las tareas más frecuentes que tendrás que hacer frente a un
ordenador es  escribir en  un fichero. Programas,  documentos, correos
electrónicos, páginas web, tal  vez configuraciones de programas. Este
tema te enseñará el manejo básico de los editores de texto más usuales
en GNU/Linux.

% Existen multitud de editores de texto,  pero en este tema aprenderás a
% manejar los dos más ampliamente utilizados.  Luego elige el que más te
% guste y a picar código.

\item[\chaptername~\ref{internet.tex}: Aplicaciones para Internet] ~\\
Hoy en  día un ordenador sin  conexión a Internet es  algo tristemente
aislado. En el caso de GNU/Linux esto se hace más notorio porque es un
sistema que ha  nacido y crecido a través de  Internet, se puede decir
que está hecho  en y para Internet. Un uso  básico de las herramientas
para Internet te  resultará imprescindible para seguir  adelante en tu
aprendizaje sobre GNU/Linux, ya que casi todas las respuestas están en
Internet.

% \item[\chaptername~\ref{aplicaciones.tex}: Aplicaciones  diversas] ~\\
% De  las miles  de aplicaciones  interesantes disponibles  en GNU/Linux
% haremos un breve repaso de las más destacadas.

\item[\chaptername~\ref{documentacion.tex}: Documentación y ayuda] ~\\
Cuando te sientas  perdido con GNU/Linux y sus  herramientas tendrás a
tu disposición una amplia ayuda y documentación, aprender a utilizarla
es la clave del éxito como usuario del software libre.

\item[\chaptername~\ref{instalacion.tex}:  Instalación  de  GNU/Linux]
~\\ ¿Te  animas a  instalar GNU/Linux  en tu  ordenador? ¡Felicidades!
Ahora bien, antes de atacar la  tarea hay unas cuantas cosas que debes
saber para que te salga bien.

\item[\chaptername~\ref{administracion.tex}:   Administración  básica]
~\\  Cuando por  fín  te  animes a  instalar  GNU/Linux  en tu  propio
ordenador tendrás que  enfrentarte con la tarea  más dura: administrar
tu  propio sistema,  para  mantenerlo funcionando.  Hay muchos  libros
dedicados exclusivamente  a esta  materia, así que  en este  tema sólo
aprenderás  lo fundamental  para  mantener un  ordenador personal  sin
grandes pretensiones. Para aprender más necesitarás dedicarle tiempo y
leer bastante.

\item[\chaptername~\ref{bash.tex}:    Bash]    ~\\   Cuando    quieras
automatizar algo sencillo  puede que sea suficiente un  script, que no
tendrás que compilar. Bash es  el intérprete de comandos más extendido
en GNU/Linux, y tiene su propio lenguaje de programación.

% \item[\chaptername~\ref{gimp.tex}: The  GIMP] ~\\  Fotografías, logos,
% animaciones, composiciones,  montajes y  mucho más  está a  tu alcance
% gracias a The  GIMP. Este tema te dará la  introducción necesaria para
% habituarte  al  entorno  de  trabajo  de The  GIMP  y  aprovechar  sus
% capacidades.

% \item[\chaptername~\ref{graficos.tex}:   Edición   de  gráficos]   ~\\
% Esquemas,    ilustraciones,     fotografías,    logos,    animaciones,
% composiciones,  montajes...   todo  esto  puede  hacerse   también  en
% GNU/Linux  utilizando software  libre. En  este tema  verás un  par de
% aplicaciones para ello.

\item[\chaptername~\ref{staroffice.tex}:  StarOffice]  ~\\ Si  piensas
que  no  podrás  sobrevivir  en  GNU/Linux  porque  no  tiene  Word  o
WordPerfect  estás equivocado.  StarOffice es  una suite  ofimática de
calidad, potente, que  responderás a tus necesidades.  Además, como es
libre, está disponible en varias plataformas además de GNU/Linux.

\item[\chaptername~\ref{html.tex}:  El lenguaje  HTML]  ~\\ Todos  los
ficheros  que aprenderás  a manejar  en este  curso están  en formatos
libres, de modo que cualquier otra persona puede utilizarlos (al menos
leerlos) sin  necesidad de  comprar ni piratear  programas especiales.
Con  el  aprendizaje  del  lenguaje HTML  estarás  en  condiciones  de
escribir documentos  que cualquier persona podrá  leer directamente en
la web con su navegador.

\item[\chaptername~\ref{lyx.tex}: \LyX] ~\\  Para escribir rápidamente
y  que  quede  bien  presentado  nada mejor  que  concentrarte  en  el
contenido de lo  que escribes y dejar al ordenador  que se encarge del
trabajo  de  formato  y  composición del  documento.  Lyx  está  hecho
exactamente para eso.

\item[\chaptername~\ref{latex.tex}: \LaTeX] ~\\ \TeX y \LaTeX~ son los
lenguajes  más potentes  que encontrarás  para componer  documentos de
calidad profesional,  aunque como  era de  esperar requiere  un cierto
esfuerzo de aprendizaje. Sin duda vale  la pena, valga como ejemplo el
este libro escrito íntegramente en \LaTeX.

\item[\chaptername~\ref{octave.tex}:   GNU  Octave]   ~\\  Para   casi
cualquier científico  se hace muy  necesaria la computación  rápida de
cálculos  matemáticos  aplicados  mediante  un  lenguaje  que  permita
programar  los  cálculos.   Octave  te  dará  la   solución  para  los
cálculos que  necesites para  física, ingeniería, matemáticas  y otras
disciplinas.

\item[\chaptername~\ref{gnuplot.tex}: GNUplot]  ~\\ La  elaboración de
gráficas  elaboradas es  otra tarea  fundamental  en el  campo de  las
ciencias, y GNUplot tiene amplias posibilidades para esto.

\item[\chaptername~\ref{r.tex}: GNU R] ~\\  Otro lenguaje de cálculo y
representación de datos, pero esta vez con capacidades especiales para
cálculos estadísticos.

\item[\chaptername~\ref{yacas.tex}:  Yacas]   ~\\  Este   lenguaje  es
diferente,  para  cálculo simbólico.  Esto  es  especialmente útil  en
Matemáticas,  donde a  veces  es necesario  manipular las  expresiones
matemáticas sin evaluarlas.

\item[\chaptername~\ref{freepascal.tex}: Free Pascal] ~\\ Pascal es un
lenguaje de  programación ampliamente utilizado en  las asignaturas de
programación de  primer año en  ingenierías y otras  carretas técnicas
como Matemáticas y Física.

\item[\chaptername~\ref{gnufortran.tex}:   GNU   Fortran]   ~\\   {\sf
Fortran}  es  un  lenguaje  de programación  específico  para  cálculo
numérico.  Aún se  utiliza  en entornos  científicos  como centros  de
cálculo o facultades de Matemáticas y Física.

\item[\chaptername~\ref{gnuc.tex}: GNU C/C++] ~\\ El lenguaje C es sin
duda ``El'' lenguaje  de programación por excelencia.  GNU/Linux se ha
construido mayoritariamente  sobre este lenguaje, por  lo que conviene
saber manejar  las herramientas de  programación para programar  en C.
Además es  el lenguaje  de programación  más ampliamente  utilizado en
prácticas académicas de programación.

\item[\chaptername~\ref{gnumake.tex}: GNU Make] ~\\ Cuando un proyecto
crece, sea  del tipo que sea,  siempre viene bien una  herramienta que
automatice las  labores de compilación  del código. Esto se  aplica no
sólo a programas, sino también a libros como éste.

\item[\chaptername~\ref{depuradores.tex}:  Depuradores]  ~\\ La  mayor
parte  del tiempo  que dedicamos  a  programar lo  pasamos buscando  y
corrigiendo errores en el código.  Esta tarea te resultará mucho menos
ingrata cuando conozcas el estupendo depurador DDD.

\item[\chaptername~\ref{gprof.tex}: GNU gprof] ~\\ Una herramienta muy
útil a la  hora de optimizar tu  código puede ser el  profiler de GNU,
que te ayudará  a encontrar cuellos de botella y  otros puntos débiles
de tus programas.

\item[\chaptername~\ref{java.tex}:  Java]  ~\\  El  lenguaje  Java  es
ampliamente utilizado  como lenguaje de  programación multiplataforma.
Es útil para aplicaciones y applets en documentos HTML.

\item[\chaptername~\ref{regex.tex}: Expresiones regulares]  ~\\ Una de
las mejoras  ayudas que  encontrarás a  la hora  de programar  son las
expresiones regulares.  Éstas te permitirán filtrar  texto para editar
tus  programas  de  forma  más eficiente,  o  convertir  ficheros  con
información  mal organizada  en tablas  de datos  de información  bien
organizada.

% \item[\chaptername~\ref{php.tex}:  PHP]  ~\\  es  un  lenguaje  ``Open
% Source''  interpretado  de  alto  nivel,  especialmente  pensado  para
% programación web.

\end{description}

\section*{Convenciones tipográficas}
\label{convenciones_tipograficas}

En la escritura de este libro hemos utilizado algunas convenciones
para facilitar su lectura. Usaremos los diferentes tipos de letra para
fines determinados:

\begin{itemize}

\item {\sf  sin adornos}: nombres de  programas, paquetes, protocolos,
lenguajes, etc. También para menús y botones.

\item {\tt máquina de  escribir}: comandos, código fuente, pulsasiones
y/o combinaciones de teclas; todo aquello  que haya que teclear o leer
de la pantalla.

\item {\em itálica}: términos nuevos o que haya que resaltar.

\item  {\bf  negrita}: elementos  de  una  descripción o  aquello  que
necesite especial atención.

\item  {\sc versalita}:  para  nombres de  personas, organizaciones  y
entidades en general.

\end{itemize}

Cuando hagamos  referencia a  combinaciones de teclas  utilizaremos la
siguiente notación:

\begin{itemize}
\item ``{\tt C-}'' indica la tecla {\tt Control}.
\item ``{\tt S-}'' indica la tecla {\tt Shift}.
\item ``{\tt A-}'' indica la tecla {\tt Alt} (a veces llamada {\tt Meta}).
\item ``{\tt F1}'' \dots ``{\tt F1}'' son las teclas de función.
\item ``{\tt ESC}'' indica la tecla de Escape.
\item ``{\tt TAB}'' indica la tecla del tabulardor.
\item ``{\tt Enter}'' indica la tecla del retorno de carro.
\end{itemize}

Además, cuando tengamos  que mostrar comandos y/o la  salida que éstos
produzcan lo haremos de la siguiente forma:

\begin{verbatim}
$ comando
salida
del
comando
\end{verbatim}

El símbolo del dólar \verb+$+ es lo que comúnmente se denomina el {\em
prompt}  del sistema,  lo que  el intérprete  muestra a  la espera  de
nuestras órdenes.  En las distribuciones  de GNU/Linux actuales  no es
normal que este  prompt sea únicamente el símbolo del  dólar, sino que
suele contener también el nombre de la máquina, el nombre del usuario,
o  el directorio  en  el  que nos  encontramos.  Estos detalles  serán
explicados en el tema ``El intérprete de comandos''.

En los temas  referentes a lenguajes de  programación, documentación o
cálculo incluiremos  algunos ejemplos  de código fuente  de programas,
scripts  o documentos.  Estos  ejemplos están  en  el directorio  {\tt
ejemplos} del  paquete que  contiene el código  fuente de  este libro,
cuya localización en internet te  facilitamos en la siguiente sección.
Estos  ficheros  se  utilizan  en  las  clase  teórico/prácticas  para
aprender  el  manejo  de  compiladores y  herramientas  de  cálculo  o
documentación.

\newpage

\begin{ejemplo}{ejemplo.txt}{Ejemplo de ejemplo}
Éste  es un  ejemplo de  cómo te  mostraremos los  ejemplos de  código
fuente. Éstos pueden  ser programas, scripts o  documentos escritos en
diferentes lenguajes  como C/C++, Fortran, Pascal,  HTML, PHP, \LaTeX,
Octave, GNUplot,  R, Yacas,  etc. Los ejemplos  que aparezcan  de este
modo estarán disponibles  por separado en sendos  ficheros cuyo nombre
se indica en la cabecera del ejemplo. Así te ahorras tener que teclear
los  ejemplos a  la hora  de  probar los  lenguajes. Cuando  necesites
buscar uno de  estos ejemplos puedes encontrar su página  en el índice
de ejemplos de la página \pageref{IndiceDeEjemplos}.
\end{ejemplo}

\section*{Sobre este documento}

Este libro ha sido escrito gracias a la iniciativa de algunos miembros
y amigos  del \GULiC, para  el \CILA~ organizado conjuntamente  por el
\GULiC~  y la  \FMAT~ de  la \ULL,  en Tenerife  (España). Mira  en la
página \pageref{Agradecimientos} para saber  quiénes participaron en la
elaboración de este libro.

Como  indica  la  nota  de copyright,  estos  apuntes  son  libremente
distribuibles bajo los  términos de la {\sf  Licencia de Documentación
Libre de GNU}, lo que a grosso modo significa que se permite su copia,
redistribución  y modificación  siempre que  se nombre  a los  autores
originales, no se reclame la autoría  de trabajo ajeno ni se pierda la
libertad de los contenidos.

En el apéndice de la  página \pageref{GFDL} encontrarás una traducción
{\bf  no oficial}  de la  {\sf Licencia  de Documentación  Libre GNU}.
Para  leer la  versión  original  de la  {\sf  GNU Free  Documentation
License}  acude a  la  página  oficial del  proyecto  {\sf GNU},  {\tt
http://www.gnu.org/licenses/fdl.html}

La   última  versión   de   estos  apuntes   se   mantendrá  en   {\tt
http://cila.gulic.org}, tanto  su código fuente como  las versiones en
PostScript y PDF, además de un paquete con los ejemplos por separado.

Este libro  es un proyecto {\sf  abierto} y en desarrollo,  por eso no
está completo  y posiblemente tarde  en estarlo (siempre hay  algo que
añadir, mejorar, corregir, actualizar,  etc.). Si quieres colaborar en
él ponte en contacto con nosostros enviando un correo electrónico a la
la  dirección {\tt  cila@listas.gulic.org}. Nos  gustaría recibir  tus
opiniones,  correcciones  y  sugerencias críticas  constructivas.  Las
críticas  destructivas serán  redireccionadas,  como  siempre, a  {\tt
/dev/null}

