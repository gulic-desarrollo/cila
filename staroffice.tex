%Autor: Sirma
%Sirma: 51

\chapter{StarOffice}
\index{StarOffice}
\label{staroffice.tex}

%%%%%%%%%%%%%%%%%%%%%%%%%
% Sección: Introducción %
%%%%%%%%%%%%%%%%%%%%%%%%%
\section{Introducción}

StarOffice  es   un  entorno  de   oficina  que  contiene   todas  las
herramientas  para  desenvolverse  en  un  ambiente  de  trabajo,  así
como  también hogareño.  Este  set de  herramientas  tiene sus  raices
en  StarOffice,  un  suite  de  oficina  que  desde  mediados  de  los
ochenta se desarrollaba  en Alemania, a cargo de  una compañía llamada
StarDivision,  cuyos  derechos  fueron   comprados  en  1999  por  Sun
Microsystems. La version 5.2 of StarOffice se lanzó en junio del 2000,
y la  versión 6.0 salió  en mayo del  2002. Lo más  extraordinario del
desarrollo de StarOffice 6.0 es  que se desarrolla mayoritariamente en
el  contexto del  proyecto Open  Source  de StarOffice.  A efectos  de
posibilitar dicho desarrollo, Sun hizo  público el código fuente de la
mayoría de los  módulos de StarOffice (excepto por  algunos módulos de
proveedores externos, que no han entrado en el proyecto), y estableció
el proyecto StarOffice. Esto no significa  sin embargo que Sun deja en
manos  de  voluntarios el  desarrollo  futuro  del suite:  la  mayoría
del  trabajo todavía  se  lleva  a cabo  por  desarrolladores de  Sun.
Sun  también  se encarga  de  los  costes operacionales  del  proyecto
StarOffice.

StarOffice es uno  de los grandes competidores del  paquete de oficina
\emph{Microsoft Office},  ya que el  StarOffice es compatible  con los
formatos de archivos del MS-Office,  además que existen versiones para
GNU/Linux, MS-Windows, y otros sistemas  operativos; y por último, una
característica muy importante: es gratis.

StarOffice es  una suite ofimática  que está compuesta  por diferentes
módulos que se pueden cargar  de forma independiente. Este capítulo no
está orientado a ser un manual exhaustivo de StarOffice, sino más bien
una  introducción  a  este  enorme  paquete,  donde  se  explicará  el
funcionamiento básico de  los módulos mas importantes  y se mencionará
la existencia de los demás módulos.

En la actualidad Sun Microsystems  ha liberado el código de StarOffice
5.2,  lo que  ha derivado  en  el desarrollo  de OpenOffice.org  ({\tt
http://www.openoffice.org}) mientras Sun  Microsystems sigue vendiendo
StarOffice 6.


%%%%%%%%%%%%%%%%%%%%%%%%%%%%%%%%%%
% Sección: StarOffice Writer %
%%%%%%%%%%%%%%%%%%%%%%%%%%%%%%%%%%

% Características generales:
%				Autocorrección
%				Impresión
%				Lectura de archivos Word
%
% Caracteristicas mas avanzadas:
%				DirectCursor
%				Autotexto

\section{StarOffice Writer: Procesador de textos}


\subsection{Introducción:}

StarOffice Writer es un procesador de textos\footnote{%
Un procesador de textos es un programa con el que se generan diferentes documentos
de texto como, cartas, informes o incluso el presente libro
} avanzado, con el que se pueden crear diferentes tipos de documentos en los
que se pueden insertar tablas, esquemas, gráficos, etc. Tambien cuenta con plantillas \footnote{%
Una plantilla es el esqueleto de un modelo de documento en el que vamos insertando
nuestro texto, esto se explica en el apartado \ref{Uso de Plantillas} 
} que simplifican la creación de: cartas, folletos, artículos, informes, curriculum
vitae, libros, etc. Además de las anteriores existen multitud de utilidades
que facilitan el uso del procesador, como son el autocorrector, el autoformato,
etc.

Un punto muy importante a favor de StarOffice Writer es la compatibilidad con los documentos
guardados con formatos propietarios como Microsoft Word, no obstante siempre
se podrán exportar documentos de cualquier procesador de texto guardandolo en
formato de texto Enriquecido (RTF\footnote{%
RTF: formato de texto enriquecido, es un formato estandar que puede ser entendido
por la practica totalidad de editores de texto.
}). StarOffice Writer posee un sisema de ayuda que puede ser consultado de forma contextual
(pulsando \boton{F1}) mientras se edita el documento o buscando en los temas de ayuda.
Se debe tener en cuenta que se puede acceder a todas las funcionalidades del
editor a través de los diferentes menús tanto con el ratón como con diferentes combinaciones
de teclas, denominadas teclas rápidas, que permitirán realizar cualquier tarea
de una forma mucho más rápida y eficaz.

\subsection{Como empezar:}

Este es el aspecto que tiene la pantalla del StarOffice Writer, antes de comenzar a crear
nuestro documento vamos a determinar para que sirven los principales menús que
encontramos en la aplicación.

\begin{figura}{StarWriter}{0.9}
\caption{Primer vistazo al StarOffice Writer}
\end{figura}



\subsubsection{Especificaciones del menú principal:}

%IMAGEN DE LA BARRA SUPERIOR

\begin{description}
\item [\textmd{Archivo:}]Este menú se utiliza para abrir, cerrar, imprimir o crear
asistentes para documentos, para cerrar StarOffice Writer hay pulsar en este menú la
opción cerrar.
\item [\textmd{Editar:}]Sirve para modificar el texto, cortar, pegar, seleccionar,
buscar y remplazar, etc.
\item [\textmd{Ver:}]Es posible ver el documento a distinta escala, tener acesso directo
a los archivos del usuario a traves del beamer, ver la regla, la barra de estado,
los submenús de las barras de herramientas.
\item [\textmd{Insertar:}]Con este menú se insertan campos, simbolos, encabezados,
indices, notas, referencias, tablas, imagenes, objetos.
\item [\textmd{Formato:}]Modificación de los parrafos, caracteres, páginas, el estilo
y aplicación del autoformato.
\item [\textmd{Herramientas:}]Aquí se encuentra el diccionario de sinonimos, la ortografia,
la numeración de capitulos y páginas, actualizar el documento y las macros\footnote{%
Una macro es una utilidad que creamos para automatizar tareas repetitivas.
}.
\item [\textmd{Ventana:}]Sirve para ir de una ventana a otra, es decir cuando se tiene
abierto más de un documento se tiene la posilididad de desplazarse de uno a
otro, verlos en cascada, azulejearlos que es verlos todos a la vez en la pantalla,
ver dos documentos en horizontal o en vertical, cerrar todas la ventanas a la
vez y ver cuantos documentos se tienen abiertos.
\item [\textmd{Ayuda:}]Para obtener ayuda sobre el contenido o sea sobre StarOffice Writer,
tener el asistente abierto mientras escribimos, o la ayuda activa que al posicionar
el cursor sobre un icono da una explicación de para que se utiliza (está sin
traducir al castellano).
\end{description}

\subsection{Uso de la ayuda: }


\subsection{El documento:}

Una vez que se ha abierto StarOffice, para abrir una sesión de StarOffice Writer hay
que abrir un documento ya sea nuevo o existente.


\subsubsection{Como crear un nuevo documento:}

Para crear un documento nuevo simplemente se hace doble click sobre el icono
Nuevo Texto y aparecerá el nuevo documento. Tambien se puede abrir un nuevo
documento seleccionando en el menú \menu{Archivo} la opción \menu{Nuevo} y elegir
\menu{Texto} .


\subsubsection{Como abrir un documento existente:}

Con StarOffice Writer existe la posibilidad de trabajar con documentos de formatos
muy diferentes, como pueden ser los documentos de Microsoft Word, html, TXT,
RTF, Winword, etc. Para abrir un documento existente se pulsa en \menu{Archivo}
y se elige \menu{Abrir} y aparecerá un cuadro de dialogo en el que aparece la
ubicación de los archivos en nuestro ordenador, se elige la ruta correcta y
el nombre del archivo y se pulsa abrir o utilizamos las teclas rápidas \boton{Ctrl-O}.

%Figura 5.3.2.: Abrir documento existente


\subsubsection{Guardar, guardar como:}

La primera vez que se guarda un documento hay que darle un nombre y decidir
su ubicación (donde se quiere guardar). Guardar como, se utiliza para guardar
el documento con un nombre de archivo, por ejemplo: Manual.doc, y en una localización
determinada en el sistema de archivos, por ejemplo: \comando{/home/usuario/StarOffice Writer/personal/trabajos/Manual.doc},
para Guardar Como se pulsa sobre \menu{Archivo} y después sobre \menu{Guardar como},
y se elige la ruta y el nombre del archivo en el cuadro de dialogo.

Guardar, se utiliza para conservar los cambios que se realizan en el documento
mientras se escribe, para guardar se pulsa sobre \menu{Archivo} y después sobre
\menu{Guardar} y todos los cambios serán guardados.

%Figura 5.3.3.: Guardar como.


\subsubsection{Edición y formateado de un documento:}

Como se ha comentado anteriormente el texto puede formatearse mientras se escribe
o una vez que se ha terminado la edición del documento. Si se quiere formatear
el texto mientras se escribe se activa la orden, por ejemplo activar \boton{negritas},
y una vez acabado se desactiva la opción. Para formatearlo una vez que terminado
hay que seleccionar el texto o la parte del texto que queramos modificar, para
señalarlo basta con colocar el puntero del ratón al comienzo del texto y pulsando
el botón derecho arrastrar el ratón hasta el final de la selección, tambien
se puede seleccionar mediante los cursores manteniendo la tecla de mayusculas
pulsada.

Para ver el documento en una escala mayor o menor se utiliza....Para cambiar
el tamaño y tipo de letra (una vez seleccionada) se pulsa \boton{Formato}......y
aparecerá un cuadro de dialogo en el que se puede elegir el tamaño de la letra,
el tipo de fuente, Negrita, cursiva, etc. Una vez seleccionados los cambios
pulsamos \boton{Aceptar} y se modificará el texto seleccionado. Para ver las marcas
del texto de principio o fin de párrafo, tabulaciones, sangrías, etc., se pulsa
en \menu{Ver} y después se selecciona ......... (Caracteres no Imprimibles)


\subsubsection{Trabajar con tablas:}

Para insertar una tabla en un documento se selecciona \menu{insertar} y se pulsa
\menu{tabla}, aparecerá un cuadro de diálogo.... 

Para modificar el aspecto de la tabla en el documento se selecciona \menu{Formato},
\menu{Tabla} y en el cuadro de diálogo se pulsa en la pestaña Tabla.......

%Figura 5.3.4.: Vista Previa.

Para cambiar el ancho de las columnas..... Para cambiar los bordes..... El grosor
o estilo de las líneas.....color..... sombras.......Como introducir texto en
las casillas.........., formato del texto de la tabla ......


\subsubsection{Imprimir un documento:}

Para poder imprimir correctamente un documento se debe tener previamente configurada
la impresora que se vaya a utilizar\footnote{%
La impresora se configura siendo Administrador y desde el escritorio de StarOffice
(Desktop)
}. Antes de imprimir el documento es conveniente que veamos el aspecto que va
ha tener una vez impreso, para previsualizarlo se pulsa en \menu{Archivo} y después
en \menu{Vista preliminar}, en la pantalla aparece una barra de herramientas
donde existen una serie de botones con los que podremos por ejemplo ampliar
el documento, mostrar varias páginas en pantalla, etc. Para ver el documento
se puede pulsar el icono con forma de prismáticos.

%Figura 5.3.4.: Vista Previa.

Una vez visualizado el documento, para imprimir se selecciona \menu{Archivo}
y después \menu{Imprimir}, se puede pulsar el icono de la impresora, abriendose
un cuadro de dialogo en el que se puede ver una serie de opciones de impresión,
impresora, propiedades de la impresora, donde elegiremos el formato del papel,
la orientación, el area de impresión, el número de copias, etc. También se
puede utilizar la serie de teclas \boton{Ctrl-P}

%Figura 5.3.5.: Cuadro de Dialogo de Impresión.


\subsection{\label{Uso de Plantillas}Uso de plantillas:}

%Figura 5.3.6.: Plantillas.


\subsection{Teclas rapidas:}


\subsection{Salir de StarOffice Writer:}

Para salir de StarOffice Writer, se pulsa \boton{Archivo} y se elige \boton{Salir}, pudiendose
utilizar también las teclas rápidas \boton{Ctrl-Q}. Si el documento no se ha guardado
previamente aparecerá un cuadro de dialogo
 que preguntando si se quieren guardarlos cambios realizados en el documento, seleccionando \boton{Si} los cambios serán
almacenados y se cerrara la aplicación, si por el contrario elegimos \boton{No}
los cambios no son guardados y se perderá todas las modificaciones realizadas
en el documento, finalmente si se pulsa \boton{Cancelar} se retornará al documento.

%Figura 5.3.6.: Salir de StarOffice Writer.

%Figura 5.3.7.: Cuadro de Dialogo para Salir
% \end{document}
%%%%%%%%%%%%%%%%%%%%%
% Sección: StarCalc %
%%%%%%%%%%%%%%%%%%%%%


\section{StarCalc: Hoja de Cálculo}

% La verdad que no sé que meter acá :-(

%Carlos Pérez Pérez {\\tt cperezperez@terra.es}

%*****************************************************
%                           Introducción             *
%*****************************************************

\subsection{Introducción}

Una hoja de cálculo\footnote{También denominada Planilla de Cálculo} es una
herramienta puesta al alcance del usuario para solucionar una cierta clase
de problemas. Estos problemas van desde llevar el cálculo de la contabilidad
del hogar hasta el ánalisis de datos estadísticos pasando por un sinfín de
posibilidades.
El uso de una hoja de cálculo queda supeditada a lo que el usuario quiera,
pero, hay situaciones en las que se hace imprescindible el uso de una hoja de
cálculo. Desde llevar la contabilidad de pequeñas empresas, en las que se
requieran sumar grandes cantidades de cifras que la hoja de cálculo puede
hacer automáticamente, utilizarlo para hacer regresiones estadísticas no
excesivamente complejas, llevar el control de pedidos, de ventas, de \ldots{}.
Un largo etcétera cuyo principal punto en común es el de evitar cálculos
manuales tediosos que puedan inducir a errores.
Definido el problema al que se puede enfrentar el usuario ahora queda la
elección del programa de hoja de cálculo. En el mercado hay una gran cantidad
de paquetes que incorporan hojas de cálculo. Aquí se expondrá el uso de la
StarCalc, la hoja de cálculo perteneciente al paquete ofimático StarOffice.
Pero lo aquí explicado puede ser aplicado sin esfuerzo al uso de otras hojas
de cálculo, como la Excel.
El lector debe tener en cuenta que aquí no se explican todas las opciones  y
características que incorpora StaCalc, sólo se trata de una introducción  al
uso de la misma así como una serie de prácticas en las que se tratará de
extender la parte teórica.

%**********************************************************
%                    Primeros Pasos                       *
%**********************************************************

\subsection{Primeros Pasos}

\subsubsection{Entorno de trabajo.}

El entorno de trabajo de la StarCalc se hace familiar para aquel
usuario que conozca el uso de las hojas de cálculo. Es un entorno
totalmente gráfico e intuitivo desde el principio. En el que podemos
observar una barra de menús donde se encuentran las operaciones que se
pueden hacer con la StarCalc. Debajo se hallan una serie de botones
que se usan para acceder más rápida y cómodamente a funciones y
acciones que se encuentran en los menús.\\

%<Gráfico principal>
\begin{figura}{StarCalc}{1}
\caption{La hoja de cálculo StarCalc}
\end{figura}


La zona de menús, allí se posicionan el menú \menu{Archivo},
\menu{Editar}, \menu{Ver}, \menu{Insertar}, \menu{Formato},
\menu{Herramientas}, \menu{Datos}, \menu{Ventana} y \menu{Ayuda}.
Dentro decada una  de ellas se agrupan las diferentes macros que la StarCalc
tiene predefinidas y que nos permiten realizar las tareas propias
de la hoja de cálculo. Ver figura \ref{fig:StarCalcMenus}.

%<Gráfico menu>
\begin{figura}{StarCalcMenus}{1}
\caption{Los menús de la StarCalc}
\label{fig:StarCalcMenus}
\end{figura}


La siguiente zona está conformada por una serie de botones que permiten el
acceso  rápido a funciones definidas. Este área también se extiende a la parte
izquierda de  la pantalla, que pueden verse en las figuras
\ref{fig:StarCalcBotones} y \ref{fig:StarCalcBotoensLaterales}.

%<Gráfico botones>
\begin{figura}{StarCalcBotones}{1}
\caption{Barra de botones superiores}
\label{fig:StarCalcBotones}
\end{figura}

\begin{figura}{StarCalcBotonesLaterales}{1}
\caption{Barra de botones laterales}
\label{fig:StarCalcBotoensLaterales}
\end{figura}


Luego se nos presenta la zona de trabajo, esto es, la parte destinada a la
introducción,  modificación, cálculo y presentación de datos, que puede
observarse en la figura \ref{fig:StarCalcTrabajo}.
 
%<Gráfico trabajo> 
\begin{figura}{StarCalcTrabajo}{1}
\caption{Zona de trabajo}
\label{fig:StarCalcTrabajo}
\end{figura}


La última zona está conformada por una línea en la que se presentan una serie
de datos  sobre características generales de la hoja en la que estemos
trabajando.   

%<Gráfico parte baja>
%\begin{figura}{StarCalcLinea}{1}
%\caption{Línea inferior}
%\end{figura}


%*************************************************************
%                   Creación de Nuevas Hojas                 *
%*************************************************************

\subsubsection{Creación de nuevas hojas}
Para crear una nueva hoja se puede recurrir a varias formas de hacerlo.
Por un lado se puede hacer doble click en el icono de
\comando{Nueva Hoja de Cálculo} dentro del escritorio del entorno StarOffice.
También se puede crear en el menú \comando{Archivo}, submenú \comando{Nuevo},
y elegir \comando{Hoja de cálculo}. Otra forma es utilizar las plantillas,
para ello se puede acceder al menú  \menu{Archivo}, submenú \menu{Nuevo},
y \menu{Plantillas}, o también presionar \boton{Ctrl+N}.

%Moverse por la hoja de Cálculo.
\emph{Selección de celda.}
Las celdas dentro de la hoja se pueden seleccionar bien usando el ratón 
o con el teclado. Con el ratón se presiona en la celda de inicio y, sin 
soltar, se extiende la selección moviendo el ratón hasta llegar a la celda 
final que se quiere seleccionar. Con el teclado, se posiciona con las teclas 
de dirección sobre la celda a seleccionar y se presiona la teclas \boton{Mayúsculas}. 
Sin soltar dicha tecla se seleccionan las celdas deseadas con las teclas de 
dirección.\\ 
Para seleccionar celdas que no sean contiguas se utilizará el ratón y la 
tecla \boton{Control}. Se selecciona el primer rango de celdas o una sola, y a 
continuación se seleccionan las celdas o rangos deseados sin soltar la  
tecla \boton{Control}.
Para el caso en el que la selección de celdas quede fuera de las ventanas, 
hay que hacer notar que el movimiento es automático y no se necesita 
preocuparse hasta dónde llegar. 
Las columnas y filas se pueden selccionar enteras si se presiona sobre 
el rótulo de fila o columna. 

\subsubsection{Abrir hoja desde archivo} 
%Las maneras de recuperar el trabajo (de otro o nuestro). 
Para recuperar un trabajo desde el lugar donde esté almacenado acudiremos
al menú \menu{Archivo} y \menu{Abrir} con lo que aparecerá el cuadro de
opciones para recuperar el archivo, como se aprecia en la figura
\ref{fig:StarCalcAbrir}.

\begin{figura}{StarCalcAbrir}{1}
\caption{Recuperar archivo}
\label{fig:StarCalcAbrir}
\end{figura}


Se puede acceder al cuadro de diálogo de abrir archivo presionando
\boton{Ctrl+O}.

%*****************************************************************
%                   Introducción de datos                        *
%*****************************************************************

\subsubsection{Introducción de datos.} 
 
%Manera adecuada de poner los datos en las celdas :) 
 
Se aceptan dos grandes tipos: constantes y fórmulas. 
Las constantes se dividen en: valores numéricos, de texto y de fecha y hora.
Para los valores numéricos y de texto se aceptan: 
Números del 0 al 9 y ciertos carácteres especiales, tales como +-E e () . , 
\$ \% y /. Las restantes entradas se considerarán de texto. Si se quiere 
utilizar un valor numérico como texto se antepondrá el signo ' y luego se 
introduce el valor. 

Para introducir datos nos situaremos en la celda adecuada presionando 
en ella y escribiremos aquello que queramos. 
 
StarCalc puede ayudarnos a introducir datos, por ejemplo, en series de 
números podemos poner el primer número y seleccionar la celda. En la parte
inferior derecha hay un punto negro, al seleccionarlo el puntero del ratón se 
convierte en una cruz, arrastrándolo hacia donde queramos poner los datos se
extenderá la selección y la StarCalc sumará automáticamente los siguientes datos.
En el caso de que los número no sean correlativos se pueden poner los dos
primeros números de la serie y la StarCalc hará el resto. 

Se puede hacer lo mismo con los nombres de los meses, si se introduce enero, y
siguiendo el procedimiento anterior se conseguirá que vaya escribiendo todos
los meses. 
La StarCalc muestra un rótulo al lado del puntero del ratón, de esta manera
sabremos hasta dónde llegar sin tener que hacer cálculos mentales.

\emph{Protección de datos.} Se pueden proteger libros enteros u hojas
individuales, de manera que  se permita acceso de sólo lectura o sin acceso. A
través del menú  \menu{Herramientas} y \menu{Proteger}. 
También se pueden proteger y desproteger celdas individuales. 
 
%*************************************************************************
%                            Construcción de Fórmulas                    *
%*************************************************************************

\subsubsection{Construcción de fórmulas.}
 
Creación de fórmulas para la transformación de los datos.\\ 
 
Se selecciona una celda vacía y se introduce el signo igual precedido de 
la expresión aritmética a utilizar. Por ejemplo, para sumar 3 y 6 pondremos:\\ 
\[=3+6\]
 
\emph{Precedencia de los operadores:}
\begin{itemize} 
\item Se procesan primero las expresiones entre paréntesis. 
\item Se ejecutan la multiplicación y la división antes de la suma y la resta. 
\item Los operadores del mismo nivel se calculan de izquierda a derecha. 
\end{itemize}

\emph{Los paréntesis.} 
Hay que incluir un paréntesis cerrado por cada uno que
se abra.\\
\emph{Referencias a las celdas.}
Dentro de una fórmula nos podemos referir a una celda de diferentes maneras. 
Escribiendo A1 haremos referencia a la celda de la fila 1 columna A, que
también puede ser seleccionada con el ratón. 
En este punto hay que diferenciar el tipo de referencia que se va a utilizar.\\ 
\emph{Referencias relativas.}
Se refieren a las celdas por sus posiciones en relacción a la celda que
contiene la fórmula. De esta manera la celda A1 se convertirá en B2 si
la fórmula se mueve una fila hacia debajo y una columna hacia la derecha.\\ 
\emph{Referencias absolutas.} 
Se identifica a la celda por su posición fija, la fórmula hará referencia 
siempre a la misma celda aunque se mueva la fórmula. La manera de hacer esto 
es utilizando el signo \$. Esto es, \$A\$1 hace referencia a la fila 1
columna A  en todo momento. A\$1 hará referncia a la fila 1 simpre y a la
columna se calculará  de forma relativa a la fórmula. El caso contrario
sería \$A que mantendría fija  la columna.  Nótese que se puede hacer
referencia a cualquier celda dentro del libro de  trabajo.\\
 
%<Ver si se puede en otros libros>. 

\emph{Uso de funciones.}
Las funciones tal y como las hemos visto pude que no tengan demasiado sentido 
si se trata, por ejemplo, de sumar una gran cantidad de celdas:\\
\[=A1+A2+A3+A4+A5+A6+A7+A8+A9+A10+A11+A12\]
por eso se puede utilizar la sintaxis que provee la StarCalc\\
\[=SUMA(A1:A12)\]
 
Se puede utilizar cualquier función definida dentro de la StarCalc pasándole
los parámetros adecuados. Sin embargo, no se pueden conocer o retener todas 
las funciones que tiene la hoja, por eso se puede utilizar insertar función 
que presentará un cuadro con las distintas funciones existentes, una pequeña 
explicación y los parámetros que se necesitan. 

%****************************************************************
%                        Formateo de la Hoja                    *
%****************************************************************

\subsubsection{Formateo de la hoja.} 
%Utilización de los formatos de datos correctos y adecuados.
Para poner el formato correcto de los datos podemos seleccionar
las celdas a formatear y acudir al menú \menu{Formato} \menu{Celda...}
y se nos presentará un cuadro con las diferentes opciones a
configurar, desde el tipo de cuadro hasta el idioma, la fuente, el
color, la justificación o la orientación de la escritura. Al finalizar
la configuración presionamos \boton{Aceptar}.

%****************************************************************
%                     Impresión y Presentación			*
%****************************************************************

\subsubsection{Impresión y presentación.}
%Imprimir documentos y presentación en pantalla
%(eso de los colores y las líneas).
El uso de la impresión dentro de la StarCalc sigue el mismo esquema que en el
resto de la suite StarOffice. Seleccionaremos el menú \menu{Archivo} e
\menu{Imprimir}, también podemos presionar las teclas \boton{Ctrl+P}. En
ambos casos se mostrará un cuadro de diálogo. En el caso de que no necesitemos
ajustar las propiedades presionaremos \boton{Aceptar} y StarCalc mandará el
trabajo a la cola de impresión.
 
%****************************************************************
%                        Salvando nuestro esfuerzo              *
%****************************************************************

\subsubsection{Salvando nuestro esfuerzo.} 
%Formas de grabar en un archivo lo que hayamos hecho. 
Para guardar los datos con los que estamos trabajando acudiremos a
\menu{Archivo} y \menu{Guardar} o \menu{Guardar como} si
queremos guardarlo bajo otro nombre. En el caso de que el archivo
sea nuevo se nos presentará el diálogo de guardar como para indicar
a la StarCalc bajo que nombre y tipo de archivo deseamos que sea
salvado nuestro trabajo.

%***************************************************************
%                     Análisis de Datos                        *
%***************************************************************

\subsection{Análisis de Datos}
%Análisis de datos. 
%Funciones Comunes. 
 
\subsubsection{Funciones comunes.} 
%Las funciones que ofrece la StarCalc 
Sintaxis de las funciones: Las funciones presentas dos partes
diferenciadas, por un lado, el nombre de la función, por el otro, sus
argumentos. Los argumentos de una función deben estar entre paréntesis
inmediatamente después del nombre de la función y su cantidad
dependerá del tipo de función que se use. También hay que destacar el
hecho que dentro de los argumentos se pueden incluir tanto otras
funciones, con sus propios argumentos, como valores numéricos, valores
de texto o valores lógicos, así como matrices y valores de error. La
forma más sencilla de insertar funciones es usándo el autopiloto de
funciones que se encuentra en el menú \menu{Insertar}, \menu{Función},
o presionando las teclas \boton{Ctrl+F2} o usando el botón que se
encuentra en las barras de botones, dónde se introducen las fórmulas.
Aquí no se van a exponer todas las funciones que existen e la
StarCalc, para ver todas las funciones que la hoja puede ofrecer puede
consultar el Autopiloto de Funciones, allí las encontrará agrupadas
según el uso que se les quiera dar. Ver figura \ref{fig:StarCalcFucniones}.

\begin{figura}{StarCalcFunciones}{1}
\caption{Autopiloto de funciones.}
\label{fig:StarCalcFucniones}
\end{figura}


\subsubsection{Fechas y horas.} 
%Manipulación de fechas para presentación y utilización en cálculos. 
Las fechas se introducen directamente en las celdas siguiendo alguno
de los siguientes formatos que son perfectamente legibles por
StarCalc: d/mm/aa; d-mmm-aa, d--mmm o mmm-aa. Para conseguir 5 de
Noviembre el año 2000, teclearemos \comando{5-nov-2000} y se nos
mostrará tal cual la hemos introducido. En aquellos formatos en los
que se omita alguna parte de la fecha se tratarán de distinta manera,
así si falta el día se introduce automáticamente el día primero, si es
el año se introduce el año en curso.  Nótese que no puede faltar el
mes en los formatos presentados. Para introducir series de fechas
utilizaremos lo aprendido en la sección de introducción de datos, esto
es, podemos extender la selección automáticamente con el ratón.  Para
la introducción de horas se pueden seguir los siguientes formatos:
h:mm AM/PM; h:mm;ss AM/PM; h:mm; h:mm:ss; mm:ss.0; [h]:mm:ss. El
lector interesado puede mirar el Autopiloto de funciones donde se
encuentran algunas funciones para utilizar y transformar fechas.
 
\subsubsection{Análisis financiero.} 
%Cálculo de TAE, VAN, TIR, rentabilidades, intereses, etc. 
En una hoja de cálculo no podían faltar las funciones referidas a
operaciones financieras, tales como el valor actual neto o la tasa
interna de retorno.  Estas funciones son muy utilizadas en entornos
empresariales y para el usuario doméstico que de esta forma puede
calcular el rendimiento de sus ahorros o si la inversión en la compra
de una casa es factible. En esta sección también se debe hacer
referencia a las funciones de depreciación, cálculo de la tasa de
retorno y análisis de valores bursátiles.  En este último caso,
desarrollaremos un método de análisis de acciones y opciones en la
parte teórica que no hace uso de las funciones que propone la
StarCalc.
 
\subsubsection{Análisis estadístico.} 
%Cálculos de medias, modas, varianzas, regresiones, etc. 
StarCalc utiliza funciones estadísticas predefinidas con las que puede
calcular multitud de aspectos sobre poblaciones. Así tenemos funciones
como media, modas, medianas, rango, histogramas, frecuencias
covarianzas, coeficiente de correalción, desviación típica,
distribuciones estándar, binomial, exponencial, normal, ji, poisson,
regresiones lineales, entre otras. Cuya utilización será de utilidad
para aquellas personas que requieran el uso de cálculos estadísticos.
 
%*********************************************************************
%                              Gráficos                              *
%*********************************************************************

\subsection{Gráficos} 
%Creación de gráficos para la mejor comprensión de los datos. 
%Gráficos.
\label{Starcalc-graficos}

Como dice el refrán, más vale una imagen que mil palabras. Por eso la
StarCalc provee una serie de herramientas que permiten convertir los
datos en gráficos para una mejor comprensión.

Creación de un nuevo gráfico. Lo primero es seleccionar los datos que
se desean representar para luego ir a \menu{Diagrama} en el menú
\menu{Insertar}. Se nos mostrará un cuadro pidiendo los datos, estos
serán los que hemos seleccionado o cualquier rango que podemos
ingresar manualmente.  En este punto podremos seleccionar si queremos
nuestro gráfico en la misma hoja o en una nueva hoja. Presionaremos
\boton{Avanzar} y se nos mostrará el cuadro de diálogo para
seleccionar el tipo de gráfico que vamos a utilizar.

\begin{figura}{StarCalcDiagramaSeleccion}{1}
\caption{Selección del tipo de gráfico}
\end{figura}

 
Una vez seleccionado el tipo de gráfico y la variante que vayamos a utilizar,
se nos presenta un cuadro en el que podremos introducir los títulos que van
a acompañar el gráfico. 

\begin{figura}{StarCalcDiagramaTitulos}{1}
\caption{Titulos del gráfico}
\end{figura}


Tras lo cual seleccionaremos \boton{Crear} y se nos presentará el gráfico
dentro de la hoja activa,  o en una hoja nueva si es lo que hemos seleccionado
previamente.  Podremos trabajar directamente sobre el gráfico utilizando el
botón derecho del ratón o el menú  de formato que nos permitirá personalizar
el gráfico. 

\emph{Personalización de gráficos.}
Presionando con el botón derecho elegiremos la opción del menú
\menu{Editar}. De esta manera  activaremos el gráfico y podremos utilizar
las funciones de personalización sobre  aquellas zonas del gráfico que
deseemos cambiar. Para ello seleccionaremos la parte  del gráfico a cambiar y
seleccionaremos la opción \menu{Propiedades del Objeto} que se nos  muestra
con el botón derecho del ratón o dentro del menú \menu{Formato}.  Ha de
hacerce notar que se pueden utilizar la hilera de botones verticales que se
encuentran  en la parte izquierda de la pantalla y que nos permiten acceder a
opciones del gráfico  de forma más rápida.

\begin{figura}{StarCalcDiagramaGraficoCambioPropiedades}{1}
\caption{Personalización de gráficos}
\label{fig:StarCalcDiagramaGraficoCambioPropiedades}
\end{figura}

% \ref{fig:StarCalcDiagramaGraficoCambioPropiedades}


%***********************************************************************
%              Gestión de bases de datos y listas                      *
%***********************************************************************

\subsection{Gestión de bases de datos y listas} 
%Utilizar la StarBase para usar con la StarCalc. Utilización de las 
%listas. 
La utilización de listas es muy común en las hojas de cálculo (clientes, teléfonos,
etc.). Las lista deben cumplir una serie de características para que su utilización
sea lo más efectiva posible.

\begin{itemize}
\item Cada columna debe contener el mismo tipo de información.
\item La primera fila deben ser rótulos de descripción del contenido.
\item No debería haber columnas o filas en blanco en la lista.
\item Para el uso de filtros no debería haber otra información en las filas que
ocupe la misma.
\end{itemize}

%Gestión de bases de datos y listas.
%Utilizar la StarBase para usar con la StarCalc. Utilización de las
%listas.

La utilización de listas es muy común en las hojas de cálculo (clientes,
teléfonos, etc.). Las lista deben cumplir una serie de características para
que su utilización sea lo más efectiva posible.

\begin{itemize}
\item Cada columna debe contener el mismo tipo de información.
\item La primera fila deben ser rótulos de descripción del contenido.
\item No debería haber columnas o filas en blanco en la lista.
\item Para el uso de filtros no debería haber otra información en las filas
que ocupe la misma.
\end{itemize}

Para crear una lista a partir de los datos de una hoja vamos al
\menu{Piloto de Datos} en el menú de \menu{Datos}.  Se nos mostrará el
cuadro de seleccionar fuente.

%<Gráfico>

Presionando aceptar se nos mostrará  el cuadro de Pilot de Datos
donde colocaremos por el método arrastar y soltar los datos que  queramos que
se muestren. En el botón de \boton{Opciones} podremos acceder a una serie de
características  para afinar aún más la lista.\\

\emph{Ordenación de listas.}
Para ordenar listas sólo tenemos que seleccionar el área a ordenar y elegir
\menu{Ordenar} dentro del menú  de \menu{Datos} e indicar el método de
ordenación y la columna por la que se ordenará, entre otras opciones  de
personalización. Se debe tener en cuenta que hemos de seleccionar la opción de
Área contiene  encabezamientos de columna para que no ordene los encabezados
junto con los datos. Para esto  seleccionaremos la solapa de Opciones dentro
del cuadro \boton{Ordenar} y marcaremos la opción \boton{Área} contiene 
encabezamientos de columna, a continuación aceptaremos y se creará la lista.   

\emph{Filtrado de listas.}
Con el uso de filtros dentro de las listas conseguiremos analizar aquellos datos
que interesan. Por eso, al crear la lista se añade un botón denominado
\boton{Filtro} que permite definir los  datos que se van a mostrar dependiendo
de aquellos datos que se busquen.   
\emph{Bases de datos.}
StarCalc puede trabajar conjuntamente con bases de datos profesionales. Se
pueden importar tablas desde bases de datos que StarOffice tenga registradas para trabajar
directamente con listas. En la versión 5.2 de StarOffice se instala por defecto el soporte
para Adabas siempre y cuando dicha base de datos se encuentre en el sistema.
El uso de bases de datos tiene su tratamiento en la sección de StarBase. 
 
%*************************************************************************
%                            Tablas dinámicas                            *
%*************************************************************************

\subsection{Tablas dinámicas} 
%Creación de tablas dinámicas para que el cálculo de datos 
%se automático. 
Es un tipo especial de tabla que resume la información de ciertos campos de
una lista o base de datos. 
Las tablas dinámicas están vinculadas a los datos de los que proceden. Cuando
dichos datos cambian, la tabla no se recalcula automáticamente, pero se puede
actualizar en cualquier momento.\\
La creación de tablas dinámicas en StarCalc sigue el mismo procedimiento que
el apartado anterior. Pero, esta vez, en lugar de seleccionar los datos en
la hoja, utilizaremos fuentes externas. Activamos el \menu{Piloto de Datos}
en el menú \menu{Datos}. Seleccionamos \menu{Fuente de datos registrada
en StarOffice}\footnote{En nuestro caso no podemos usar la Fuente
externa/interfaz porque no tenemos registrada ninguna base de datos.}
y presionamos \boton{Aceptar}. Seleccionaremos aquellas fuentes que deseemos
utilizar y persionaremos \boton{Aceptar}. Los siguientes pasos son los del
apartado anterior que el lector ya conoce.
%*************************************************************************
%                           Macros y StarBasic                           *
%*************************************************************************
\subsection{Macros y Starbasic}  
%Creación de Macros y utilización del lenguaje de macros StarBasic.  
%Esto se hará de forma bastante esquemática. :-(   

Una macro es un conjunto de intrucciones que se ejecutan en conjunto y que
permiten hacer tareas repetitivas y complejas.
Para crear una macro dentro de StarBasic tenemos que acceder a
\menu{Herramientas} y \menu{Macro...} se nos mostrá el cuadro Macro de la
figura \ref{fig:StarCalcMacro}.

\begin{figura}{StarCalcMacro}{1}
\caption{Cuadro de Macro}
\label{fig:StarCalcMacro}
\end{figura}


Lo primero será introducir el nombre de la macro a grabar y luego presionamos
\boton{Grabar}. Mientras estemos grabando la macro se almacenará cualquier
acción que ejecutemos, tanto con el ratón como a través del teclado. Para
finalizar presionaremos en el cuadro detener macro que se ve en la figura
\ref{fig:StarCalcMacroStop}

\begin{figura}{StarCalcMacroStop}{.1}
\caption{Parar grabación de la macro.}
\label{fig:StarCalcMacroStop}
\end{figura}


Una vez grabada la macro podremos depurarla para ajustar algunos parámetros
o para borrar aquellas partes en las que nos hayamos equivocado o queramos
modificar.
Las macros y sus modificaciones se hace utilizando el lenguaje que tiene
la StarCalc y que se denomina StarBasic. Es un lenguaje muy intuitivo y
que se parace al Visual Basic para Aplicaciones de la Excel.Figura
\ref{fig:StarCalcBasic}.

\begin{figura}{StarCalcBasic}{0.6}
\caption{Entorno de StarBasic}
\label{fig:StarCalcBasic}
\end{figura}


Cuando la se tenga la macro finalizada, se puede asignar dicha a macro a una
tecla o combinación de teclas. Para esto se accede al menú \menu{Herramientas},
\menu{Macro}. Dentro del cuadro de Macro presionaremos sobre
\boton{Asignar...} y seleccionaremos la combinación de teclas que quereamos
utilizar para activar la macro creada.
%**************************************************************************
%                         PARTE DE PRÁCTICAS                              *
%**************************************************************************

\subsection{De prácticas} 
En estas prácticas se pretenden enseñar rudimentos de la utilización 
de hojas de cálculo. Así como técnicas que no se enseñan en la parte 
teórica. 
Se presentan de menor a mayor dificultad. Exceptuando las dos últimas 
que son casos aparte.
 
\subsubsection{Control de ingresos y gastos.} 
Objetivo. Llevar la contabilidad doméstica. 
Don Rufino Pensante no sabía por dónde se le escapa el dinero que 
ganaba con su trabajo en una tienda de caramelos así que decidió 
montar un sistema de contabilidad doméstica para llevar un orden y
control de todo el dinero que pasaba por sus manos.\\

\begin{table}
\centering
\begin{tabular}{|l|r|r|r|r|r|}
\hline
Mes & Enero & Febrero & Marzo & Abril & Mayo\\
Sueldo & 145.000 & 145.000 & 145.000 & 145.000 & 145.000\\
Vivienda & 30.000 & 31.000 & 30.000 & 29.000 & 32.000\\
Comida & 40.000 & 43.000 & 46.000 & 39.000 & 41.000\\
Coche & 15.000 & 6.000 & 16.000 & 8.000 & 9.000\\
Agua & 7.000 & 8.000 & 6.000 & 4.000 & 9.000\\
Luz & 11.000 & 10.000 & 12.000 & 9.000 & 10.500\\
Varios & 10.000 & 18.000 & 11.000 & 8.000 & 11.000\\
\hline
\end{tabular}
%\caption{}
\end{table}

\noindent Lo  primero de  todo es  diseñar la hoja  para no  tener que
estar más tarde rediseñando todo  el trabajo. Vemos de que información
disponemos y  lo que  queremos conseguir. Esto  es, tenemos  series de
datos correspondientes a ingresos y gastos  y queremos ver el saldo al
final de cada periodo. Lo más lógico es pensar en tener una periocidad
mensual aunque no se descarta  que sea semanal para aquellas personas,
que, por ejemplo, tengan una nómina semanal.

El siguiente paso lógico es crear un libro nuevo con la StarCalc. Para
eso nos iremos a \menu{Archivo}, \menu{Nuevo}, \menu{Hoja de Cálculo}.
Pondremos los títulos de los encabezados para cada una de las columnas,
empezando en A1, insertaremos {\tt Descripción}. Para los rótulos de las
fechas introducimos la primera fecha y utilizaremos seleccionar y arrastrar
para que StarCalc complete el resto de las cabeceras.
Luego pondremos en la columna A todas las descripciones de los gastos e
ingresos, y debajo de cada fecha las cantidades correspondientes. Para seguir
con la claridad, insertaremos {\tt Gastos} para tener todos los gastos
agrupados e {\tt Ingresos} para los ingresos. De esta manera desglosaremos
los gastos e ingresos y sólo nos restaría añadir una línea para el cálculo del
saldo.

%Gráfico tabla completa.
\begin{figura}{StarCalcPracticas-1-1}{1}
\caption{Tabla de datos}
\end{figura}


Luego procedemos a retocar la estética para resaltar aquellos datos que nos
interesen, tal como los Ingresos, los Gastos y el Saldo. En este punto cada
cual puede poner su sello personal. Como guía, resaltaremos los rótulos en
negrita y las cuentas de Ingresos, Gastos y Saldo con una letra mayor, negrita
y usándo bordes.

Para el cálculo de los totales usaremos la función \comando{SUMA}, para lo que
elgiremos la columna correspondiente a enero y la fila que corresponde a
Gastos e introduciremos \comando{=suma(}. En este punto tendremos dos
opciones: ingresar manualmente el rango de celdas o seleccinarlas con el
ratón, en ambos casos hay que terminar cerrándo el paréntesis
\footnote{StarCalc es capaz de cerrar el mismo el paréntesis en el caso de no
escribirlo.}.

Para calcular el resto de meses, se procederá a copiar la fórmula anterior en
el resto de celdas de la fila correspondientes a los siguientes
meses\footnote{Una manera rápida de hacerlo es seleccionado la celda a copiar
y utilizando las macros. Esto es, \boton{Ctrl+C} para copiar, seleccionar el
resto de celdas y \boton{Ctrl+V} para pegar en todas las celdas a la vez}. La
StarCalc corregirá la fórmula para que coincida con los datos de cada uno de
los meses.

Hasta el momento tenemos los saldos de todas las cuentas de nuestro ejemplo.
Pero queremos que nuestros datos nos faciliten una mayor cantidad de
información. Podemos tener muchas cosas en mente, pero para esta primera
práctica vamos a ver máximos, mínimos, promedios y porcentajes.

Comenzaremos con los máximos y mínimos. Nos situaremos en la barra de rótulos
ingresaremos \comando{Máximo} en N2, \comando{Mínimo} en O2 y\comando{Promedio}
en P2, después de los meses. En la columna de Máximos utilizaremos la fórmula
\comando{MÁX()}, con la tilde incluida porque en otro caso no será reconocida.
Para el mínimo utilizaremos \comando{MÍN()}. La secuencia a ingresar sería: en
N3 \comando{=máx(b3:m3)} y en O3 \comando{mín(b3:m3}. En el caso de los
promedios, nos situaremos en P3 y teclearemos \comando{promedio( b3:m3)}.
Luego sólo nos quedará copiar las tres celdas construidas, en cada una de las
celdas correspondientes a las distintas cuentas.

La salida de los porcentajes se pueden hacer en una hoja distinta o en la misma
hoja. Nosotros lo haremos en la misma hoja, aunque lo que aquí se explique se
puede aplicar a otras hojas.\\
Primero hemos de construir los encabezados y las descripciones, esto es, los
meses y las cuentas. Para ello usaremos un truco que nos permitirá ahorrar
tiempo en el caso de que cambie alguno de los rótulos. Nos situaremos en A15
e introducimos \comando{=A2} y copiamos esta celda hasta M15 y luego repetimos
desde A3 hasta A25.\\
En la celda C16\footnote{No habrá salida para enero porque no tenemos datos
del mes anterior} introducimos \comando{=(c3-b3)/b3} y lo copiamos en el resto
de celdas. No se preocupe de la salida \comando{Err:503}, se produce porque no
hay datos correspondientes a esos meses. Para que los datos sean en porcentaje
tenemos que dar formato a las celdas. Seleccionaremos tadas las celdas que
contengan número, incluso las que contengan errores, y accederemos a
\menu{Formato}, \menu{Celda} y presionaremos las pestaña \boton{Números} si no
se encuentra activada, seleccionaremos \comando{Porcentaje} en Categoría.
Presionaremos \boton{Aceptar} y tendremos la salida con porcentajes.\\
De esta manera hemos obtenido las variaciones porcentuales. Pero, ¿no sería
interesante ver el porcentaje del total que significa cada cuenta?
Repetimos el paso anterior para copiar las etiquetas correspondientes a los
meses y a las cuentas, empezando en A28. Y añadimos en B30 \comando{=
b3/b\$9}, esta fórmula sólo se copiará hasta la cuenta Gastos, que tendrá una
salida del 100\%. En B35 introducimos \comando{=b10/b\$11} y lo copiamos en
todas las cuentas de Ingresos. Luego seleccionamos la columna de enero y la
pegamos en el resto de meses. De esta manera obtenemos el peso de cada una
de las cuentas en los distintos meses.

Por último, sólo nos queda añadir a las tablas efectos de resalte para destacar
aquellos datos que nos interese tener mejor visualizados. También puede
añadirse a la práctica una tabla de datos en la que se muestren los datos
acumulados \footnote{la fórmula a usar para la salida sería \comando{=b3} para
el primer mes y \comando{=b40+c3} y copiar para el resto de celdas esta última.
Nótese que b40 hace referencia a la primera cuenta del mes de febrero y que
podría ser otra en su ejemplo.} de mes a mes y, de esta manera, llevar un
control de los gastos e ingresos que llevamos para todo el año.

\subsubsection{Productos de financiación.}
Objetivo. Calcular las mensualidades del pago de un préstamo o el
rendimiento de los ahorros.

\subsubsection{Análisis de proyectos de inversión.}
Objetivo. Ayuda a la toma de decisiones en inversiones.

\subsubsection{Seguimiento de Acciones y opciones.}
Objetivo. Ver la evolución de los títulos de acciones y opciones que
tengamos. Modelo Black-Scholes.
Hoy en día no es extraño que una persona particular posea una pequeña cantidad
de acciones o de opciones. En esta práctica intentaremos hacer el seguimiento
de las rentabilidades de dichos títulos tratándo de anticipar posibles
cambios que nos indiquen pautas de compra o venta y así evitar riesgos. Aquí
haremos referencia al parqué de la bolsa madrileña por ser la que más
conicimientos poseeo. Para aquellas personas que conozcan la temática no
hace falta que se expliquen los conceptos. Para aquellos que no conozcan
los términos, el objetivo de la práctica son los conceptos de hoja de cálculo
y no los de economía financiera.

Para hacer el trabajo más fácil de hacer y de presentar se hará uso de
de cuatro hojas diferentes: una para las acciones, otra para las opciones,
otra para los cálculos de acciones y opciones y una cuarta para la salida
de gráficos.

El primer paso es crear un libro nuevo de StarCalc para lo que usaremos
el menú \menu{Archivo}, \menu{Nuevo} y \menu{Hoja de cálculo}. Una vez
creado el libro, comezaremos con la parte de acciones que es la más
fácil pues sólo se trata de controlar lo cotización diaria de los títulos.

La fórmula de Black-Scholes:

\begin{displaymath}
C(K,S,T-t,r,\sigma)=S·N(d_{1})-K·e^{-r(T-t)}·N(d_{2})
\end{displaymath}


\subsubsection{Análisis de tendencia.}

Objetivo. Calcular las previsiones para años futuros dada una serie de
datos con un modelo sencillo.
Los datos a utilizar son los siguientes y que se presentan de esta
manera por economía de espacio.

%****************************************************************
%                          CORREGIR
% La tabla se sale de los límites de la página.
\begin{table}
\centering
\begin{tabular}{|l|l|l|l|l|l|l|l|l|l|l|l|l|}
\hline
$t$   & 1 & 2 & 3 & 4 & 5 & 6 & 7 & 8 \\
$X_t$ & 99,93 & 103 & 103,41 & 112,79 & 109,52 & 113,57 & 110,5 & 120,42 \\
\hline
$t$   & 9 & 10 & 11 & 12 & 13 & 14 & 15 & 16 \\
$X_t$ & 118,23& 119 & 118,75 & 127,32 & 124,93 & 128,13 & 127,55 & 137,06\\
\hline
$t$   & 17 & 18 & 19 & 20 & 21 & 22 & 23 & 24\\
$X_t$ & 131,95 & 133,91 & 135,35 & 146,19 & 138,57 & 143,14 & 144,64 & 153,72\\
\hline
\end{tabular}
\end{table}
%******************************************************************

Se va a indicar la manera de hacer esta práctica paso a paso, por lo
que no será necesario el conocimiento previo ni de análisis de series
de datos ni del uso de la hoja de cálculo \footnote{El lector que haya
seguido este curso puede tratar de personalizar cada uno de los pasos,
o saltar aquellos que considere que no hacen falta.}\\
\begin{itemize}

\item El primer paso sería introducir los datos de la tabla anterior en una
columna que denominaremos \emph{X}.

\item Segundo paso, representar la serie X en el tiempo. Para eso seleccionamos
los datos de la columna X, y vamos a \menu{Insertar} \menu{Diagrama}. Seguiremos
los pasos de la sección \emph{Gráficos} en la página \pageref{Starcalc-graficos}.
\item Supongamos:

%****************************************************************************
%                                ERROR
%Los entornos matemáticos me dan errores pero se presentan bien en pantalla :-?
\[X_{t}=T_{t}+S_{t}+I_{t}\]
\[T_{t}=100+2_{t}\]
%\begin{displaymath}
%S_{t}=
%\left\{
%\begin{array}{cc}
%-1, & t=1,5,9,\ldots\\
%-1, & t=2,6,10,\ldots\\
%-2, & t=3,7,11,\ldots\\
%4,  & t=4,8,12,\ldots
%\end{array}
%\end{displaymath}
%*****************************************************************************

\item Obtener el componente tendencial, \(T_{t}\).\\
En la celda B1, ponemos \comando{t} introducimos \comando{1}, en B3 escribimos
\comando{=1+b2}. Copiamos la celda B3 desde B4 hasta B25. En la celda C1 escribimos
Tt. En C2, \comando{=100+2*b2} \footnote{Este es el valor para t=1}. Y copiamos
C2 desde C3 hasta C25.

\item Componente estacional, \(S_{t}\).\\
Definimos las variables cualitativas:

%****************************************************************************
%                                ERROR
%\begin{displaymath}
%D_{k,t}=
%\begin{array}{cc}
%1, & t \in \text{estación} k
%0, & \text{en otro caso}
%\end{array}
%;k=1,2,3,4.
%\end{displaymath}

%Luego:
%\begin{displaymath}
%S_{t}=-1D_{1,t}-1D_{2,t}-2D{3,t}+4D{4,t}
%\end{displaymath}
%****************************************************************************

\item Obteniendo las variables cualitativas estacionales.\\
En D1 escribimos \comando{d1}, en D2 ponemos \comando{1}, en las celdas D3, D4, D5 se
escribe \comando{0}. En E1, \comando{d2}, para E2, escribimos \comando{1}, y en E2, E4 y
E5, \comando{0}. Para F1, \comando{d3}, \comando{1} en F4, \comando{0} para F2, F3 y F5.
En G1 ponemos \comando{d4}, em G2, G3, G4 \comando{0}, para G5 \comando{1}.
Se seleccionan las celdas D2 hasta G5 y se copian desde D6 hasta G25.\\
En la celda H1 escribimos \comando{St}, y en H2 \comando{=-1*d2-1*e2-2*f2+4*g2},
copiaremos esta casilla desde H3 hasta H25.

\item Componente irregular, \(I_{t}\).
En la celda I1 se escribe \comando{It}. En la celda I2 \comando{=a2-c2-h2}. Copiaremos
I2 desde I3 hasta I25.\\
Calculamos el valor medio de la serie: en I26 insertamos \comando{=suma(i2:125)/24}.

\item El último paso que faltaría es representar gráficamente las series \(T_{t}\),
\(S_{t}\) e \(I_{t}\) frente al tiempo.

\end{itemize}


Para finalizar, lo más aconsejable es guardar los datos con \menu{Archivo} y
\menu{Guardar}.


%*********************************************************************
%                                 El final                           *
%*********************************************************************

%\subsection{Bibliografía}
%De donde he sacado todo lo de arriba.

%\subsection{Agradecimientos}
%Lo que proceda. Aunque no sé si corresponde aquí.
%%%%%%%%%%%%%%%%%%%%%%%%
% Sección: StarImpress %
%%%%%%%%%%%%%%%%%%%%%%%%
% \section{StarImpress: Creador y Visualizador de Presentaciones}
% \begin{figura}{StarImpress}{0.9}
% \caption{Presentaciones en StarOffice}
% \end{figura}

%%%%%%%%%%%%%%%%%%%%%
% Sección: StarDraw %
%%%%%%%%%%%%%%%%%%%%%
\section{StarDraw: Creador de dibujos}
\begin{figura}{StarDraw}{0.9}
\caption{El creador de ilustraciones de StarOffice}
\end{figura}


%%%% {Estudio de gráficos}
\subsection{Importancia de los gráficos}

La mayor parte de los documentos que se manejan en informática
contienen \textbf{texto y gráficos}. También se puede ver en libros,
folletos, carteles publicitarios, etc. que el uso combinado de texto y
gráficos resulta ser muy bueno para comunicar ideas.

Cuando se maneja un gráfico en informática, nos interesa el
\textbf{resultado final} y también la \textbf{facilidad de
manejo}. Con los dos tipos de gráficos que hay se puede conseguir la
misma calidad, pero el trabajo que demandan y la manera de manejar
cada uno hace que sea importante conocer las distintas características
de los dos tipos.

\subsection{Tipos de gráficos}

Existen dos tipos: los gráficos de \textbf{mapa de bits}, también
conocidos por su nombre en inglés de gráficos \emph{bitmap}, y los
gráficos \textbf{escalables} o \textbf{vectoriales}. Cualquiera de los
dos tipos permite manejar imágenes en \textbf{blanco y negro}, en
\textbf{escala de grises} y en \textbf{color}.

La diferencia entre los dos tipos es interna. En los gráficos bitmap,
la imagen se almacena como un conjunto de puntos, dispuestos en filas
y columnas. Cada punto se llama \textbf{píxel} y puede tener un color
o nivel de gris distinto. En los gráficos escalables lo que se
almacena es una descripción matemática de las rectas, curvas,
rellenos, etc. que definen cada elemento del gráfico; los elementos
pueden ser rectángulos, elipses, curvas, etc.

Veamos un ejemplo del mismo gráfico almacenado como bitmap y como
escalable:

El gráfico de la izquierda representa una línea cerrada compuesta por
tres segmentos rectos y uno curvo. Para poder apreciar bien los
píxeles, se ha creado un gráfico muy pequeño y luego se ha
ampliado. Normalmente, tendría muchos más puntos y por lo tanto mucho
mejor aspecto.

El gráfico de la derecha representa la misma figura pero como gráfico
escalable. Lo único que habrá que conocer son las coordenadas de los
puntos A, B, C y D para saber por dónde pasa la figura, y los puntos E
y F para conocer la curvatura del segmento AB. Con seis pares de
coordenadas es suficiente para representar la figura, pero los
programas que quieran representarla o imprimirla deberán calcular el
resto de los puntos.

\subsection{Gráficos bitmap}

Normalmente se obtienen gráficos bitmap cuando se digitalizan imágenes
reales con los aparatos llamados en inglés \emph{scanners}. Por eso a
los programas de alta gama que manejan este tipo de gráficos se les
llama programas de \textbf{retoque fotográfico}.

Si se quieren usar para imprimirlos deben ser de gran tamaño. Cuando
son pequeños se suelen usar para representar iconos en pantalla, ya
que se pueden representar con mucha rapidez.

Suele ser difícil manipular gráficos bitmap y uno de los mayores
inconvenientes que tienen es su pérdida de calidad cuando se reducen o
amplían. Vemos un ejemplo de un gráfico bitmap ampliado:

Se puede apreciar fácilmente la pérdida de calidad. Sobre todo en las
líneas diagonales y en las curvas se aprecia la aparición de <<dientes
de sierra>>. Esto se llama \textbf{efecto escalonamiento}.

\subsubsection{Resolución}

La resolución de un gráfico bitmap es la cantidad de píxeles que tiene
en cada dimensión. Es muy habitual que la resolución sea la misma que
la de una pantalla (640x480, 800x600, 1024x768,...) pero para imprimir
el gráfico con calidad hacen falta resoluciones mayores.

\subsubsection{Profundidad}

Se llama profundidad a la cantidad de bits que hacen falta para
representar el valor de cada píxel. Un dibujo en blanco y negro tiene
profundidad 1, ya que con un bit es suficiente para saber si el punto
es blanco o negro. Las imágenes en escala de grises y en
\textbf{paleta de color} tienen profundidades 4 (16 valores) u 8 (256
valores). Las imágenes en \textbf{color real} tienen una profundidad
de 24, ya que de cada punto se necesita saber su componente roja,
verde y azul, a 8 bits cada una.

\subsection{Gráficos escalables}

Son los que crean los diseñadores que trabajan con ordenador. El
gráfico se va creando figura a figura. Son muy fáciles de modificar,
pero si son complejos requieren muchos cálculos. Su principal ventaja
es la que les da nombre: se pueden cambiar de tamaño sin pérdida de
calidad; esto permite imprimirlos usando al máximo la resolución de la
impresora.

\subsubsection{Curvas de Bézier}

Uno de los métodos más utilizados en los gráficos escalables es la
descripción de las curvas como curvas de Bézier. Se puede ver un
ejemplo en el primer gráfico de esta hoja: el segmento AB es una curva
de Bézier. Los puntos E y F se llaman \textbf{puntos de
control}. Sirven como ``imanes'' que atraen a la curva y le dan su forma
característica.

\subsection{Conversión de gráficos}

Es posible convertir gráficos de tipo bitmap a escalable y viceversa,
pero los procesos son muy distintos:

De escalable a bitmap. El proceso se llama en inglés
\emph{raster}. Todos los programas lo saben hacer, ya que la única
manera de poder ver en pantalla o imprimir un gráfico escalable es
convertirlo previamente en bitmap.

De bitmap a escalable. Esto se llama \textbf{vectorización}. Es muy
difícil y suele dar malos resultados, a no ser que el dibujo tenga
bordes muy nítidos.

\subsection{Compresión de datos}

Como los ficheros de gráficos suelen tener gran tamaño, sobre todo los
bitmaps, se han desarrollado técnicas para almacenar la información
usando menos bytes. Existen muchas, aunque casi todas intentan
detectar la existencia de grandes zonas de imagen del mismo color: en
vez de almacenar todos sus puntos, se almacena su número y su
color. Dependiendo de la imagen, se puede convertir el fichero en
incluso la décima parte de su tamaño original. Existen dos tipos de
compresión: sin pérdidas y con pérdidas. El primer tipo es mejor para
imágenes artificiales y el segundo para las naturales.

\subsection{Formatos de ficheros}

Un mismo gráfico, sea bitmap o escalable, se puede almacenar en un
fichero de muchas formas distintas, que reciben el nombre de
\textbf{formatos}. Hay muchos programas que permiten convertir
imágenes de un formato a otro. De todas formas, los buenos programas
pueden leer y escribir ficheros en muchos formatos distintos. Algunos
de los más conocidos son:

\begin{description}
\item[BMP.] Usados ampliamente por Microsoft Windows y OS/2. Son bitmap.

\item[TIFF.] Típicamente obtenidos con escáneres. Son bitmap. Admiten
 muchos tipos de compresión y todo tipo de profundidad. Se pueden leer
 indistintamente en sistemas GNU/Linux, Windows, UNIX y Macintosh.

\item[EPS.] Significa \emph{Encapsulated PostScript}. El estándar en 
el mundo de la autoedición. Son escalables. Totalmente compatibles con 
impresoras y filmadoras PostScript.

\item[JPEG.] El formato más difundido con compresión con pérdidas. 
Ideal para manejar fotografías. Es posible controlar el grado de 
compresión: a mayor compresión, menor calidad de imagen.
\end{description}


%%%% {La ventana de Draw}
\subsection{Las ventanas}

En StarOffice cada ventana de documento tiene sus peculiaridades, para
poder ofrecer las funcionalidades que demanda cada módulo. Aunque
comparten muchas características, hay pequeños detalles que distinguen
las ventanas de cada tipo de documento.  En la ilustración de la
derecha se muestra una ventana de StarOffice que contiene un documento
de Draw. Normalmente se trabaja con la ventana del documento
maximizada, pero aquí se muestra en posición \emph{flotante}, para
poder apreciar mejor qué componentes pertenecen a la \textbf{ventana
de aplicación} de StarOffice y cuáles a la \textbf{ventana de
documento} de Draw.

\subsubsection{La ventana principal}

Recorriendo desde arriba hacia abajo la ventana principal, vemos:

\begin{itemize}
\item La barra de título.

\item El menú principal.

\item La barra de funciones.

\item La barra de objetos del desktop.

\item La zona de trabajo, en la que está la ventana del documento draw.sda.

\item La barra de tareas.
\end{itemize}

\subsubsection{La ventana del documento}

Si repasamos desde arriba hacia abajo la ventana del documento, nos
encontramos:

\begin{itemize}
\item La barra de título.

\item La barra de objetos de dibujo/imagen.

\item La regla horizontal.

\item La zona de trabajo (donde se prepara el dibujo).

\item La barra de desplazamiento horizontal, con los botones de vistas 
y las pestañas de los dibujos a la izquierda.

\item La barra de opciones, que por defecto no está activada.

\item La barra de colores, que por defecto presenta dos líneas de 
colores, pero que cambia el número de líneas arrastrando el extremo 
superior.

\item La línea de estado, con información sobre el documento.
\end{itemize}

Y si la repasamos de izquierda a derecha, tenemos esto:

\begin{itemize}
\item La barra de herramientas.

\item La regla vertical.

\item La zona de trabajo.

\item La barra de desplazamiento vertical.
\end{itemize}

\subsection{Las reglas y la línea de estado}

Aunque son muy útiles y normalmente se mantienen a la vista, se pueden
eliminar. En el menú \menu{Ver} se encuentran la opciones
\menu{Reglas} y \menu{Barra de estado} para regular su aparición.

\subsection{Determinación del papel}

Para determinar el tamaño y la orientación del papel que se va a
utilizar hay que elegir en el menú \menu{Formato} la opción
\menu{Página}, para obtener el cuadro de diálogo \textbf{Preparar
página}. En él se elige la ficha Página, como se ve aquí:

La lista desplegable de la sección Formato de papel presenta una
relación con algunos tamaños muy comunes (el más usado es el
\emph{A4}). La orientación se elige con los botones de opción Vertical
y Horizontal. Cuando todo está bien, se pasa a especificar otros
parámetros del cuadro de diálogo o se pulsa el botón Aceptar.

Si ninguno de los tamaños de la lista coincide con el deseado, se
puede especificar en los cuadros Ancho y Altura las dimensiones
exactas del papel.

Si no se elige ningún tamaño, el programa usa por defecto el
\emph{A4}.

\subsubsection{Márgenes generales}

Los márgenes se pueden definir con gran precisión y en varias unidades
distintas. Lo normal es escribirlos en centímetros con uno o dos
decimales. Por defecto, el programa presenta la característica de que
los márgenes tienen <<imán>>, de modo que cuando se desplazan los
objetos con el ratón, es muy fácil dejarlos alineados con los
márgenes.

Aunque son muy útiles y normalmente se mantienen a la vista, se pueden
eliminar. En el menú \menu{Ver} se encuentran la opciones
\menu{Reglas} y \menu{Barra de estado} para regular su aparición.


%%%% {Creación de objetos}
\subsection{Objetos básicos}

El principio de la creación de dibujos vectoriales es la acumulación
de objetos simples para formar el resultado final. En esta hoja se
presenta el modo de crear los objetos de dibujo más sencillos, dejando
para otras hojas su modificación, así como la creación de objetos más
complejos, como curvas de Bézier y objetos de texto. Los objetos
sencillos son sin embargo las piezas fundamentales de los diseños y
por tanto no hay que menospreciar su importancia.

\subsubsection{La barra de herramientas}

En los programas de dibujo en general, la barra de herramientas es un
componente fundamental, porque es la que decide el modo de trabajo: es
completamente distinto encontrarse seleccionando objetos que dibujando
rectángulos o polígonos, por ejemplo. La barra de herramientas de
StarOffice Draw presenta este aspecto cuando se elige la apariencia
flotante y orientación horizontal (normalmente está anclada a la
izquierda y vertical):

La primera herramienta por la izquierda es la que más se utiliza, la
de selección. Sirve para elegir uno o más objetos y
manipularlos. Cuando se utiliza otra cualquiera de las herramientas,
Draw devuelve el control a la de selección. Si se desea usar de manera
continuada alguna herramienta, hay que seleccionarla con una doble
pulsación.

\subsection{Método para crear}

Todas las herramientas que se van a ver a continuación funcionan de un
modo muy similar:

\begin{enumerate}
\item Se selecciona la herramienta.

\item Se pulsa en un punto y se arrastra hasta otro punto.

\item La figura está creada con la esquina superior izquierda en el 
primer punto y con la inferior derecha en el segundo.
\end{enumerate}

Y durante el arrastre del ratón se admiten estas acciones:

\begin{itemize}
\item Si se pulsa \boton{Esc} se anula la creación de la figura.

\item Si se arrastra pulsando \boton{Alt}, el primer punto es el 
centro de la figura y el segundo define el tamaño.

\item Si se arrastra pulsando \boton{Shift}, la figura es regular.

\item Si se arrastra pulsando \boton{Ctrl}, la posición y el tamaño 
encajan en una cuadrícula predeterminada con puntos cada 0,25 cm 
(aunque es configurable).
\end{itemize}

\subsubsection{Rectángulos}

A la derecha se ve la barra de herramientas que se usa para crear
rectángulos. Se pueden crear con o sin relleno y con o sin las
esquinas redondeadas. También se pueden crear cuadrados (que no son
más que un caso particular de rectángulo) directamente, sin recurrir a
pulsar \boton{Shift}.

\subsubsection{Elipses}

A la derecha se ve la barra de herramientas que se usa para crear
elipses. Se pueden crear con o sin relleno y completas o
parciales. También se pueden crear circunferencias (que no son más que
un caso particular de elipse) directamente, sin recurrir a pulsar
\boton{Shift}.

\subsubsection{Líneas y flechas}

A la derecha se ve la barra de herramientas de línea, que también
permite crear flechas, ya que para StarOffice Draw los objetos
\emph{línea} y \emph{flecha} pertenecen a la misma clase y siempre se
puede pasar de uno a otro.

\subsubsection{Objetos 3D}

A la derecha se ve su barra. Todas la figuras se crean igual, ya que
el usuario lo único que define al crearlas es el tamaño, luego el
programa se encarga de representar la forma.

\subsection{Conectores}

Estos objetos son un poco diferentes a los ya explicados, y se crean
de una manera ligeramente distinta. Los conectores son líneas que unen
otros dos objetos entre sí, con la peculiaridad de que si se modifican
estos objetos, el conector cambia automáticamente para acomodarse a la
nueva situación. Existe una gran variedad de ellos, como puede verse a
la derecha, en la barra de herramientas Conector.

Para crearlos, se sigue este método:

\begin{enumerate}
\item Se elige la herramienta.

\item Al acercar el puntero a un objeto, se marcan los posibles puntos
 de unión que ofrece el objeto; se pulsa el ratón y se mantiene.

\item Se arrastra el ratón hasta llegar al segundo objeto, que también 
mostrará sus puntos de unión. Se suelta el ratón en el punto deseado.

\item El conector queda dibujado.
\end{enumerate}

\subsection{Polígonos}

Para crear polígonos abiertos o cerrados se usan los cuatro iconos
centrales de la barra de herramientas Curvas. Los dos iconos de la
derecha crean polígonos con ángulos rectos o de 45°; los de la
izquierda, polígonos generales.

Para crearlos, se sigue este método:

\begin{enumerate}
\item Se elige la herramienta.

\item Se pulsa en el primer punto, sin soltar el ratón.

\item Se arrastra hasta llegar al segundo punto, en el que se 
suelta el ratón.

\item Se mueve el ratón al tercer punto, donde se pulsa o se 
pulsa y arrastra.

\item Se continúa como en el paso 4, añadiendo los puntos necesarios.

\item El último punto se define con una doble pulsación. El programa 
cerrará el polígono si se estaba preparando uno cerrado.
\end{enumerate}

\subsection{Modificación rápida}

Aunque más adelante se explicará con detalle cómo modificar los
objetos, ahora se dan unas pautas para poder experimentar sin más
dilación:

\begin{itemize}
\item Se selecciona un objeto pulsando sobre él con la herramienta 
de selección.

\item Se mueve arrastrándolo.

\item Se cambia su tamaño arrastrando los manejadores.

\item Se cambia la línea que lo rodea y el relleno usando la 
Barra de objetos de dibujo/Imagen:

\item Los colores de línea y relleno se pueden elegir en la barra 
de colores: con el botón izquierdo se elige el relleno y con el 
izquierdo la línea.

\item Se cambia su forma pulsando el botón Modificar puntos y 
arrastrando los cuadrados que aparecerán en varios puntos. Véase el 
ejemplo de la derecha.

\item Se elimina pulsando \boton{Supr}.
\end{itemize}

%%%% {Curvas de Bézier}
\subsection{Origen de las Curvas de Bézier}

El ingeniero aeronáutico francés Pierre Bézier trabajaba en la empresa
Renault y estaba diseñando la forma de los parachoques de los
coches. Necesitaba un sistema eficiente y flexible para representar
dicha forma, e inventó estas curvas. Hoy en día se utilizan en muchas
áreas de la informática, como tipografía e infografía, además de en
los programas de diseño.

\subsection{Nomenclatura de las Curvas de Bézier}

Aunque los conceptos que intervienen en una curva de Bézier son
claros, las palabras que los representan a veces no lo son
tanto. Además, cambian de programa en programa. En estas hojas se
utilizará la traducción al español utilizada en StarOffice, que no
coincide con la que se puede leer en otros lugares.

\subsection{Conceptos de las Curvas de Bézier}

Una curva de Beziér está formada por varios \textbf{segmentos}, pueden
ser \textbf{curvos} o \textbf{rectos}. La curva puede ser
\textbf{abierta} o \textbf{cerrada}. Éste es un ejemplo de curva de
Bézier, sobre el que se explicarán los distintos conceptos:

\begin{description}
\item[Puntos de apoyo] son los puntos extremos de los segmentos. 
Por ellos pasa la curva y siempre hay que definirlos. En general, 
cuando se crean curvas de Bézier se procura que haya la menor 
cantidad posible de puntos de apoyo. En el ejemplo, del A al G.

\item[Puntos de control] son los puntos a los que <<intenta acercarse>> 
la curva, aquéllos que definen su curvatura. Siempre hay que definirlos. 
La curvatura de cada segmento viene definido por un punto de control, 
dos o ninguno. En el ejemplo, del 1 al 7.

\item[Líneas de control] unen los puntos de apoyo con los puntos de 
control. Son meras referencias para ayudar en la creación de las 
curvas, luego no aparecen.
\end{description}

\subsubsection{Tipos de puntos de apoyo}
En StarOffice Draw cada punto de apoyo puede ser de estos tipos:

\begin{description}
\item[Punto de esquina.] Tiene dos líneas de control que forman un 
ángulo. La curva presenta un brusco cambio de curvatura al pasar 
por un punto de esquina. En el ejemplo, el F.

\item[Punto liso.] Tiene dos líneas de control que forman una 
línea recta; la distancia del punto de apoyo a los puntos de 
control no tiene por qué ser la misma. La curva tiene distinta 
curvatura a un lado o a otro del punto de apoyo. En el ejemplo, 
el E es el más claro.

\item[Punto simétrico.] Tiene dos líneas de control que forman 
una línea recta y la distancia del punto de apoyo a los puntos 
de control es la misma. La curva tiene la misma curvatura a uno 
y otro del punto de apoyo. En el ejemplo, el B.
\end{description}

\subsubsection{Tipos de segmento}

El tipo de un segmento es una característica determinada por el
primero de los puntos de apoyo que lo definen.

\begin{description}
\item[Segmento curvo.] Su curvatura estará definida por un punto 
de control (que corresponderá al primer punto de apoyo) o por 
dos (cada uno correspondiente a un punto de apoyo).

\item[Segmento recto.] No le corresponde ningún punto de control. 
En el ejemplo, el CD.
\end{description}

\subsection{Creación}

El modo de crear curvas de Bézier en StarOffice Draw presenta algunas
limitaciones que a veces impiden crear exactamente la curva deseada;
pero una vez creada, es muy fácil modificarla. Concretamente, tiene
estas limitaciones:

El primer segmento ha de ser curvo.

Sólo se puede definir el primer punto de control del último segmento,
ya que el segundo queda obligatoriamente oculto por el último punto de
apoyo.

Cuando se define el segundo punto de control de un segmento, se está
definiendo también el primer punto de control del siguiente
segmento. El punto de apoyo intermedio entre los dos segmentos queda
definido como liso, pero con las distancias como si fuera simétrico.

\subsubsection{Procedimiento}

\begin{enumerate}
\item En la barra de herramientas se elige Curva,rellena o Curva 
según se vaya a crear una curva cerrada o abierta.

\item Se pulsa en el primer punto de apoyo de la curva y no se 
suelta el ratón.

\item Se arrastra hasta llegar al primer punto de control del 
primer segmento, donde se suelta el ratón.

\item Ahora hay dos posibilidades:

\begin{enumerate}
\item Se pulsa y arrastra para definir el segundo punto de 
control.

\item Se pulsa para definir el segundo punto de apoyo. En ese 
caso, el siguiente segmento será recto.
\end{enumerate}

\item Se continua con los pasos 3 y 4 definiendo más puntos 
de apoyo y control.

\item En el último punto de apoyo, se hace una doble pulsación. 
El programa cerrará la curva si se estaba preparando una cerrada.
\end{enumerate}

\subsection{Edición}

Una vez creada una curva de Bézier se puede modificar la posición y
carácter de sus puntos de apoyo y de control. Se selecciona la curva y
se pulsa el botón Modificar puntos, de la barra de objetos, y ésta se
sustituye por la barra de objetos de Bézier, que aparece aquí:

\begin{itemize}
\item Para seleccionar un punto de apoyo basta pulsar sobre él. 
Para seleccionar más de uno, se pueden <<atrapar>> marcando un 
rectángulo que los contenga. También se pueden seleccionar o 
deseleccionar de uno en uno pulsando con \boton{Shift} pulsada.

\item Una vez seleccionados, se pueden desplazar arrastrándolos 
y cambiar su carácter con los botones de la barra de opciones.

\item Se pueden eliminar puntos de apoyo seleccionándolos y 
pulsando [Supr].

\item Se pueden añadir puntos de apoyo usando el botón Insertar 
puntos.

\item A veces hay un punto de control sobre un punto de apoyo. 
Se sabe cuál se va a manipular por la forma del puntero: con un 
cuadradito los de apoyo, con una curvita los de control.

\item Para terminar la edición de la curva se vuelve a pulsar 
el botón Modificar puntos.
\end{itemize}

\subsection{Trazados a mano alzada}

Se pueden definir curvas, tanto cerradas como abiertas, sin más que
dibujarlas arrastrando el ratón. Se elige una de las dos posibilidades
de la derecha de la barra de herramientas Curvas, se arrastra el
ratón, y al terminar el programa realiza unos cálculos y ofrece una
curva de Bézier que sigue la forma dibujada. Esta curva se puede
editar como cualquier otra.

%%%% {Texto}
\subsection{Tipos de texto}

En StarOffice Draw se pueden introducir cuatro tipos distintos de
texto, cada uno con una finalidad diferente. Para poder hacer
referencia a cada tipo, se usarán aquí unos nombres adaptados de la
denominación oficiosa presente en la ayuda del programa.

\begin{description}
\item[Texto normal.] Es el equivalente a textos distribuidos por 
párrafos que se puede encontrar en un procesador de textos. Se usa 
cuando hay que introducir una cantidad grande de texto, como en 
una explicación larga en un folleto de publicidad, por ejemplo.

\item[Texto ajustado a marco.] Es el más usado con fines decorativos. 
Se usa para textos cortos sobre los que haya que aplicar efectos. 
Por ejemplo, el título de una película en un cartel.

\item[Texto en leyenda.] Consiste en texto dentro de un cuadro y 
con una flecha que señala a algún lugar. Es muy útil para hacer 
anotaciones que expliquen la función de otros objetos de dibujo. 
Por ejemplo, para escribir los nombres de las piezas de una máquina.

\item[Texto en objeto.] Consiste en un texto contenido dentro 
de un objeto, que se desplaza y cambia con él. Permite hacer 
organigramas muy fácilmente, por ejemplo.
\end{description}

\subsection{Introducción de texto}

Para introducir texto normal, ajustado o en leyenda se usa la barra de
herramientas Texto, que se ve a la derecha. Para introducir texto en
objeto sólo es necesario tener previamente creado el objeto.

\begin{description}
\item[Texto normal.] Se elige su icono y luego pulsando y 
arrastrando en la zona del dibujo, se define un rectángulo, 
que contendrá al texto. A continuación aparece un punto de 
inserción, que invita a escribir el texto. Cuando se termina de 
escribir, se pulsa en algún punto vacío del dibujo.

\item[Texto ajustado a marco.] Se prepara igual que el texto 
normal, pero la importante diferencia es que al terminar de 
escribir, el texto se ajusta al rectángulo definido al principio.

\item[Texto en leyenda.] Se elige su icono y se pulsa en el punto 
donde debe aparecer la punta de la flecha; sin soltar el ratón, 
se arrastra hasta donde se quiere colocar el cuadro y se suelta 
el ratón. Para comenzar a escribir, se hace una doble pulsación 
sobre el cuadro. Cuando se termina de escribir, se pulsa en algún 
punto vacío del dibujo.

\item[Texto en objeto.] Se hace una doble pulsación sobre el 
objeto y aparece un punto de inserción en el centro del objeto, 
donde se escribe el texto. Cuando se termina de escribir, se 
pulsa en algún punto vacío del dibujo.
\end{description}

\subsection{Modificación de texto}

Para cambiar alguna característica de un texto, basta hacer una doble
pulsación sobre él. En ese momento se puede editar el texto y cambiar
las características de la parte del texto que se seleccione con el
ratón. El método más rápido es usar la \textbf{Barra de objetos de
texto/Draw}:

Con ella se puede elegir directamente la familia tipográfica, el
tamaño, algunas variedades, el color de los caracteres, la alineación
y separación de los párrafos y el interlineado.

Si se desea cambiar alguna de estas características a \emph{todos} los
caracteres a la vez, es necesario tener seleccionado el cuadro de
texto, pero no estar editándolo. Esto se consigue saliendo de la
edición pulsando \boton{Esc} en vez de pulsando fuera del cuadro.

\subsubsection{Los cuadros de diálogo}

Además de mediante la barra de objetos, se pueden cambiar las
características básicas de los caracteres y los párrafos mediante los
cuadros de diálogo \textbf{Caracteres} y \textbf{Párrafo}, a los que
se accede desde el menú \menu{Formato}, el menú de contexto o la barra
de objetos:
  
\subsection{Relación del texto con el cuadro}

Los textos siempre están contenidos en algún tipo de cuadro. La
relación entre los dos se puede configurar eligiendo en el menú
\menu{Formato} la opción \menu{Texto}, lo que abre el cuadro de
diálogo \textbf{Texto}, que se ve a la derecha. En la ficha Texto se
encuentran las opciones pertinentes en Draw (con la ficha Animación de
texto se accede a efectos espectaculares, pero se utiliza en el módulo
StarOffice Impress).

\begin{itemize}
\item La casilla de verificación Ajustar altura al texto permite 
que el cuadro aumente su altura automáticamente cuando se introduce 
más texto. Está marcada por defecto en el texto normal.

\item La casilla de verificación Ajustar al marco es la que permite 
el comportamiento de los textos ajustados a marco.

\item La casilla de verificación Ajustar al contorno permite que 
el texto tome la forma del objeto que lo contiene, por ejemplo 
tomando la curvatura de una elipse o siguiendo los ángulos de un 
polígono.

\item En la sección Distancia al marco se define el espacio en 
blanco que hay que reservar entre el borde del marco y el comienzo 
del texto. Si se especifica una cantidad negativa, el texto 
saldrá del marco.

\item En la sección Anclaje del texto se determina la alineación 
horizontal y vertical del texto.
\end{itemize}

\subsubsection{Observación de texto}

Cuando se selecciona cualquier cuadro de texto y se usa la barra de
objetos de dibujo/Imagen o la barra de colores, se modifican las
características del cuadro, no las del texto que contiene.

%%%% {Modificación de Objetos}
\subsection{Selección de objetos}

Para modificar y manipular objetos, primero hay que seleccionarlos. Es
posible seleccionar uno o más objetos. En la línea de estado quedará
reflejado qué tipo de objeto o cuántos se han
seleccionado. Visualmente se aprecia también por los ocho manejadores
que aparecen alrededor del objeto, o los objetos. Para seleccionar
objetos hay que usar, obviamente, la herramienta de selección.

\subsubsection{Un objeto}

Pulsando sobre el objeto, se selecciona. Si el objeto no tiene
relleno, es necesario pulsar sobre su línea.

Pulsando la tecla \boton{Tab} se van seleccionando todos los objetos
por el orden en que se han creado. Con \boton{Shift-Tab} se van
seleccionando en orden inverso al de creación.

Si hay varios objetos apilados, pulsando sobre ellos con \boton{Alt}
pulsada se van seleccionando de arriba hacia abajo y con
\boton{Shift-Alt} pulsada, de abajo hacia arriba.

\subsubsection{Varios objetos}

Arrastrando y soltando se marca un rectángulo, y todos los objetos
contenidos íntegramente en el rectángulo quedan seleccionados.

Pulsando sobre un objeto con la tecla \boton{Shift} pulsada, se añade
o elimina del conjunto de objetos seleccionados.

\subsection{Modificar puntos}

El botón del mismo nombre permite activar o desactivar esta
posibilidad. Si se selecciona un objeto, al marcar Modificar puntos se
puede modificar la forma de cada objeto, cada uno según su naturaleza:

\begin{itemize}
\item En un rectángulo se puede modificar el tamaño y la curvatura 
de las esquinas.

\item En las variedades de elipse, el tamaño y la posición de los 
puntos.

\item En los polígonos, la posición de los vértices.

\item En las curvas de Bézier, todas las características explicadas 
en la hoja <<Curvas de Bézier>>.
\end{itemize}

\subsection{Posición de objetos}

El método más sencillo para cambiar la posición de los objetos
seleccionados es arrastrarlos con el ratón.

Y el método más preciso es elegir en el menú Formato la opción
Posición y tamaño, para ver el cuadro de diálogo Posición y tamaño, en
el que se elige la ficha Posición, que se ve a la derecha.

\begin{itemize}
\item Las nueve casillas de opción permiten elegir respecto a qué 
punto se va a definir la posición.

\item La coordenada horizontal se marca en Posición X y la vertical 
en Posición Y.

\item Si se marca la casilla de verificación Proteger, ya no se 
podrá cambiar la posición del objeto con el ratón.
\end{itemize}

\subsection{Tamaño de objetos}

Arrastrando los manejadores de un objeto se puede cambiar su tamaño, y
están disponibles estas posibilidades:

\begin{itemize}
\item Si se pulsa \boton{Esc} se anula el cambio de tamaño.

\item Si se arrastra pulsando \boton{Shift}, se mantiene la proporción 
de las dimensiones.

\item Si se arrastra pulsando \boton{Alt}, el cambio se hará respecto 
al centro.

\item Si se arrastra pulsando \boton{Ctrl}, el tamaño cambiará en 
saltos de 0,25 cm. 
\end{itemize}

Sin embargo, el método más preciso es elegir en el menú \menu{Formato}
la opción \menu{Posición} y tamaño, para ver el cuadro de diálogo
Posición y tamaño, en el que se elige la ficha Tamaño, que se ve a la
derecha.

\begin{itemize}
\item Las nueve casillas de opción permiten elegir qué punto 
quedará fijo durante el cambio de tamaño.

\item Si se marca la casilla de verificación Igualar, las dimensiones 
siempre cambiarán respetando la proporción. Si el usuario cambia una,
 el programa calcula la otra.
\end{itemize}

\subsection{Rotación e inclinación de objetos}

Para rotar o inclinar objetos, una vez seleccionados, se elige en la
barra de herramientas Efectos el botón Rodar; aparecen ocho nuevos
manejadores alrededor del objeto y un punto de mira en el centro;
véanse las ilustraciones que aparecen un poco más abajo, a la
izquierda.

Para rotar el objeto con el ratón, se arrastra el punto de mira para
indicar el centro de giro que se desea usar y después se arrastra uno
de los manejadores de las esquinas.

Para inclinar el objeto, basta arrastrar alguno de los manejadores de
los lados
 
Si se desea más precisión, se elige en el menú Formato la opción
Posición y tamaño, para ver el cuadro de diálogo Posición y tamaño, en
el que se elige la ficha Rotación, que se ve un poco más arriba a la
derecha, o la ficha Inclinación/Radio de ángulo.

\subsection{Reflejo de objetos}

Es posible convertir un objeto en su simétrico respecto a una
recta. Lo más sencillo es el reflejo horizontal o vertical, al que se
accede directamente desde el menú \menu{Modificar}, submenú
\menu{Reflejar}, con sus dos opciones, \menu{Horizontal} y
\menu{Vertical}.

Si se desea que la simetría sea respecto a una recta cualquiera, el
proceso es un poco más largo:

\begin{enumerate}
\item Se selecciona el objeto.

\item Se elige en la barra de herramientas \textbf{Efectos} el botón 
\boton{Reflejar}.

\item Aparece una línea (roja en la pantalla) acabada en dos puntos 
de mira.

\item Arrastrando los puntos de mira se cambia la dirección de la 
línea; arrastrando ésta se cambia la posición.

\item Se arrastra alguno de los manejadores del objeto al otro lado 
de la línea (como se ve a la derecha) y se suelta.
\end{enumerate}

%%%% {Líneas y Rellenos}
\subsection{Barra o cuadros}

Casi todas las características que se van a explicar en esta hoja
están disponibles tanto en la barra de opciones como en los menús,
pero algunas características adicionales estarán disponibles en menús
y no en la barra.

\subsection{Líneas}

Para definir la línea que tienen los objetos se elige en el menú
\menu{Formato} la opción \menu{Línea}, lo que lleva al cuadro de
diálogo \textbf{Línea}, del que se muestran sus tres fichas:

La ficha Línea está dividida en tres secciones. En la de la izquierda
se definen las características de la línea; en la de la derecha se
definen los dos extremos de la línea, en los que se puede añadir
puntas de flecha; en la sección de abajo se ve el aspecto que tendría
la línea.

Las fichas Estilos de línea y Fin de línea permiten crear nuevos
estilos de línea y de puntas de flecha y guardarlos en archivos

\subsection{Rellenos}

Los objetos cerrados pueden tener varios tipos de relleno. Se elige en
el menú \menu{Formato} la opción \menu{Relleno} y aparece el cuadro de
diálogo \textbf{Relleno}, en el que se puede definir no sólo el
relleno, sino alguna característica más. Éstas son las tres primeras
fichas:

\begin{itemize}
\item La ficha Relleno es la básica, ya que en ella se elige entre 
los cinco tipos de relleno. Según el tipo que se elija, el resto 
de la ficha cambiará. El tipo Invisible permite que el objeto sea 
transparente y se pueda ver a través de él; el resto de los tipos 
se verá a continuación.

\item En la ficha Sombra se puede conseguir que el programa añada 
automáticamente una sombra con la misma forma que el objeto.

\item Con la ficha Transparencia se activa un bonito efecto, que 
consiste en que el objeto sea semitransparente, es decir, tiene un 
relleno, pero a pesar de todo se puede ver a través de él. Cuanto 
mayor sea el porcentaje de transparencia, más se verá a través del 
objeto.
\end{itemize}

\subsubsection{Gestión de rellenos}

Éstas son las cuatro últimas fichas del cuadro de diálogo:

En ellas se puede definir de un modo mucho más preciso cada tipo de
relleno. La idea es tener una lista de estilos, que se puede
modificar, guardar en archivos, leer de archivos, etc. y
posteriormente bastará elegir el estilo de la lista en la primera
ficha o en la barra de opciones.

\subsection{Fondo de la página}

Todo lo explicado sobre rellenos puede ser aplicado directamente a la
página completa. Se puede definir un relleno que ocupe todo el fondo
de la página: en el menú \menu{Formato} se elige \menu{Página} y en el
cuadro de diálogo \textbf{Página} se elige la ficha \textbf{Fondo},
que se ve aquí:

Como se ve, están disponibles los cinco tipos de relleno (el
establecido por defecto es Invisible). Al ir eligiendo cada tipo, la
ficha va cambiando.

%%%% {Relación entre objetos}
\subsection{Posición}

Los objetos están colocados sobre el dibujo como en una pila, unos por
encima de otros. Esto no se aprecia cuando los objetos están
separados, pero cuando se sobreponen, uno tapa parte del otro. La
posición relativa se puede cambiar en cualquier momento mediante siete
órdenes disponibles en el menú \menu{Modificar}, submenú
\menu{Posición}, o con la barra de herramientas Posición, que se ve a
la derecha.

\begin{itemize}
\item Las cuatro primeras hacen avanzar o retroceder una posición o 
colocar al principio o al final de la pila.

\item Las dos siguientes colocan el objeto seleccionado por delante 
o por detrás del objeto que se marque a continuación; el programa 
lo pide con un puntero en forma de mano.

\item La última se usa cuando hay seleccionados dos objetos y se 
desea invertir sus posiciones.
\end{itemize}

\subsection{Alineación}

En muchas ocasiones hay que ajustar las posiciones de los objetos en
el dibujo de modo que queden alineados entre sí. La alineación se
puede realizar de seis modos distintos, puesto que hay tres
posibilidades en horizontal y otras tres en vertical. Las seis son
accesibles desde el menú \menu{Modificar}, submenú \menu{Alineación},
o con la barra de herramientas Alineación, que se ve a la derecha.

\begin{itemize}
\item Si sólo se selecciona un objeto, la alineación se realiza 
respecto a los márgenes de la página.

\item Las alineaciones de varios objetos por el centro se realizan 
colocando los centros de todos los objetos en la línea media que 
definían los extremos de los objetos; por tanto, es posible que se 
muevan todos los objetos.

\item Las alineaciones de varios objetos por los lados se realizan 
colocando todos los lados requeridos alineados con uno de los 
objetos, que no se moverá, y es el que tenga ese lado más al extremo. 
Por ejemplo, al alinear por arriba, el objeto que esté más arriba 
no se moverá y los demás igualarán con él los lados superiores.
\end{itemize}

\subsection{Agrupación}

El modo habitual de trabajo consiste en crear un componente de un
dibujo a partir de varios objetos elementales; por ejemplo, la cara de
la derecha está compuesta de cinco elipses, un polígono y un
rectángulo. Una vez creados y colocados los objetos elementales, lo
que se hace es \textbf{agruparlos}, para formar el componente y así
poder trabajar con él de modo unificado. Se seleccionan los objetos y
en el menú \menu{Modificar} se elige \menu{Agrupar}. A partir de
entonces, el programa se refiere al grupo y no a sus componentes, y se
puede modificar como un objeto cualquiera.

En cualquier momento se pueden recuperar los objetos individuales, con
sólo seleccionar el grupo y en el menú \menu{Modificar} elegir
\menu{Desagrupar}.

\subsubsection{Edición}

Si hay que hacer algún cambio en algún componente de un grupo, se
elige en el menú \menu{Modificar} la opción \menu{Editar grupo}. En
ese momento, se pueden volver a seleccionar individualmente los
elementos del grupo, y ninguno más. Para terminar, se elige en el menú
\menu{Modificar} la opción \menu{Abandonar grupo}.

%%%% {Transformaciones}
\subsection{Combinar}

La combinación de dos o más objetos crea uno nuevo, con
características distintas a los originales. La propiedad más relevante
de la combinación es que el nuevo objeto puede presentar ``agujeros'';
también se puede formar un solo objeto que esté compuesto de varias
partes inconexas.

Para combinar dos o más objetos se comienza por seleccionarlos y en el
menú \menu{Modificar} se elige la opción \menu{Combinar}. En el
proceso los objetos pierden la condición que tuvieran y se convierten
en curvas de Bézier.

Y si un objeto está formado por varias curvas cerradas, es posible
usar la opción Descombinar para obtener dos o más objetos, uno por
curva.

\subsubsection{Ejemplo}

En la figura que aparece a la derecha se han combinado un rectángulo y
una elipse, ambos con relleno blanco. Obsérvese cómo después de la
combinación se puede ver a través de la elipse, ya que ahora no es tal
elipse, sino un agujero del anterior rectángulo.

\subsection{Convertir en curva de Bézier}

Los objetos básicos (con la excepción de los objetos 3D) y el texto
ajustado se pueden convertir en curvas de Bézier, lo que permite una
posterior modificación. Para hacerlo, basta seleccionar el objeto y en
el menú \menu{Modificar}, submenú \menu{Convertir}, elegir la opción
\menu{En curva}. En la ilustración se presentan un rectángulo con
bordes redondeados, un segmento de elipse y una letra que han sido
convertidos a curvas, y por tanto tienen nuevos puntos de apoyo.

\subsection{Formas}

Una de las maneras más utilizadas para obtener objetos nuevos consiste
en realizar ciertas operaciones con objetos existentes. En StarOffice
Draw hay tres operaciones disponibles, agrupadas en el menú
\menu{Modificar}, submenú \menu{Formas}: Unir, Substraer y Cortar.

Probablemente la mejor manera de entender las operaciones sea mediante
un ejemplo: se dibujan primero un rectángulo y posteriormente una
elipse, que se sitúan como se ve a la izquierda. A continuación se
puede ver el resultado que se obtendría con cada una de las tres
operaciones.

Hay que resaltar que la operación de substraer tendría un resultado
distinto si se hubieran creado las dos figuras en el orden inverso. Se
pide al lector que averigüe por sí mismo ese resultado.

%%%% {Efectos}
\subsection{Duplicar}

Cuando hay que repetir un objeto, lo más sencillo es usar copiar y
pegar. Pero si hay que obtener varias copias del mismo objeto, es
mucho mejor usar la orden ``duplicar''. Además, esta orden permite
crear fácilmente algunas figuras.

Para usarla, se selecciona el objeto, se elige en el menú
\menu{Editar} la opción \menu{Duplicar} y se ajustan los valores
deseados en el cuadro de diálogo \textbf{Duplicar}, que se ve más
abajo, a la izquierda. Por ejemplo, eligiendo un cuadrado transparente
y aplicando los valores que se ven, se obtiene la figura de abajo a la
derecha, que a su vez permite crear una estrella.

\subsection{Deformaciones}

En la barra de herramientas Efectos se encuentran tres opciones que
permiten distintas deformaciones de los objetos (se muestran sus
iconos a la derecha). Se llaman Posicionar en círculo (en
perspectiva), Posicionar en círculo (inclinar) y Distorsionar. Su uso
es muy sencillo: se selecciona el objeto, la opción y se arrastran los
manejadores del objeto.

\subsection{Disolvencia}

Este efecto también se conoce como \emph{morphing}. Consiste en que un
objeto se va transformando en otro en una serie de pasos. Para
aplicarlo, se seleccionan los dos objetos y en el menú \menu{Editar}
se elige la opción \menu{Disolvencia}; en el cuadro de diálogo
\textbf{Disolvencia} se decide cuántas etapas se desean y el
comportamiento que debe tener el programa con los atributos del
objeto. Más abajo aparece el cuadro de diálogo y el resultado de la
disolvencia entre un polígono y una elipse.

\subsection{Conversión a 3D}

Cualquier objeto plano, o grupo de objetos planos, se puede convertir
en tridimensional. Esto se conoce como <<extrusión>>. Para aplicar el
efecto, se selecciona el objeto y en el menú \menu{Modificar}, submenú
\menu{Convertir}, se elige la opción \menu{En 3D}.

\subsection{Cuerpos de rotación}

A partir de una figura plana se puede crear una figura tridimensional
mediante la rotación respecto a un eje. Los cuerpos así generados se
llaman <<cuerpos de revolución>>. En StarOffice Draw se pueden crear
eligiendo en el menú \menu{Modificar}, submenú \menu{Convertir}, la
opción \menu{En cuerpo de rotación 3D}. Si se hace con el botón del
mismo nombre de la barra de herramientas Efectos, será posible definir
el eje de rotación.

\subsection{Efectos 3D}

Las figuras tridimensionales tienen su propio rango de efectos en el
programa. La rotación es diferente a la que se realiza con figuras
planas, ya que es una rotación en el espacio. Pero el modo más
específico de manejar un objeto 3D es eligiendo en el menú
\menu{Formato} la opción \menu{Efectos 3D}, que abre el cuadro de
diálogo \textbf{Efectos 3D}, que se muestra a la derecha.

\subsection{FontWork}

Uno de los efectos más habituales es deformar un texto para que siga
el recorrido de una curva. Casi todos los programas de diseño tienen
este efecto, pero cada uno lo denomina de una forma distinta. En
StarOffice se llama FontWork. Se selecciona el texto, en el menú
\menu{Formato} se elige la opción \menu{FontWork}, y en el cuadro de
diálogo \textbf{FontWork}, de muy fácil manejo, se van definiendo
todos los parámetros, mientras se va viendo el resultado. Éste es un
ejemplo:

%%%% {Imprimir y exportar}
\subsection{Entrega de un trabajo}

Las ilustraciones creadas en un programa de diseño gráfico se deben
mostrar al exterior. Hacerlo en formato nativo, es decir, el propio
del programa, es lo mejor para preservar todos los efectos y poder
modificarlos, pero no es lo habitual. Normalmente los trabajos se
entregan en un formato que no admita apenas retoques. Y los más
comunes son el papel o ficheros gráficos bitmap. Para dar el trabajo
en papel hay que imprimir el documento y para darlo en formato bitmap
hay que exportarlo.

\subsection{Imprimir}

En el menú \menu{Archivo} se elige \menu{Imprimir} y aparece el cuadro
de diálogo \textbf{Imprimir}, que se muestra abajo, a la
izquierda. Las opciones que aparecen en él son las habituales, pero es
importante saber que pulsando el botón \boton{Opciones} se accede al
cuadro de diálogo \textbf{Opciones} de impresión, se ve abajo, a la
derecha.

En la sección \textbf{Opciones de página} se encuentran dos
posibilidades muy útiles:

\begin{itemize}
\item La casilla de verificación Ajustar al tamaño de la página 
permite imprimir el trabajo en un tamaño de papel diferente al 
que esté establecido para el dibujo, ya que el programa se encargará 
de escalar todos los objetos adecuadamente.

\item La casilla de verificación Páginas como azulejos permite 
imprimir un trabajo en el que el tamaño de página definido sea mayor 
que el realmente accesible; lo que hace el programa es imprimir el 
trabajo a escala real, pero en varias hojas, que luego se cortan y 
montan.
\end{itemize}

\subsection{Exportar}

En el menú \menu{Archivo} se elige \menu{Exportar}, lo que abre el
cuadro de diálogo \textbf{Exportar}. Tiene el mismo aspecto y uso que
el cuadro de diálogo \textbf{Guardar como}. Lo más importante es
elegir el formato con que se desea generar el archivo. Los formatos
disponibles están en la lista desplegable Tipo de archivo; aparecen
formatos escalables y bitmap, todos los tipos para los que se hayan
cargado los filtros correspondientes cuando se instaló el programa.

\subsubsection{Opciones por tipo de archivo}

Algunos de los tipos de archivo piden opciones de exportación
adicionales antes de generarse el archivo. Los datos que más
comúnmente se piden son el nivel y método de compresión, la
profundidad de color y la resolución. A continuación aparecen varios
de los cuadros de diálogo que piden esos datos:

\subsection{El problema de la resolución}

Cuando se exporta un gráfico vectorial a un formato bitmap es crucial
poder decidir el tamaño en píxeles del resultado. Según se acaba de
ver, esto es algo que StarOffice Draw sólo admite con algunos
formatos, la minoría. Realizando algunas pruebas, se descubre que el
programa utiliza la resolución de pantalla para calcular el
tamaño. Esta resolución se suele medir en puntos por pulgada
(abreviado a <<ppp>>), en inglés \emph{dots per inch} (abreviado
<<dpi>>). De modo que, conocida la resolución y el número de puntos que
se desea obtener, una sencilla operación matemática da las dimensiones
que hay que dar a la página. Además, con más pruebas, se ha visto que
al exportar sólo se genera la parte de la imagen que esté dentro de
los márgenes de la página, pero no la página completa.
%%%%%%%%%%%%%%%%%%%%%%%%%
% Sección: StarSchedule %
%%%%%%%%%%%%%%%%%%%%%%%%%
\section{StarSchedule: Agenda}
\begin{figura}{StarSchedule}{0.9}
\caption{Organizar las tareas con StarSchedule}
\end{figura}

%%%%%%%%%%%%%%%%%%%%%%
% Sección: StarChart %
%%%%%%%%%%%%%%%%%%%%%%
\section{StarChart: Generador de gráficas}
\begin{figura}{StarChart}{0.9}
\caption{Armado de gráficas en StarOffice}
\end{figura}

%%%%%%%%%%%%%%%%%%%%%%
% Sección: StarImage %
%%%%%%%%%%%%%%%%%%%%%%
\section{StarImage: Editor de Imágenes}
\begin{figura}{StarImage}{0.9}
\caption{Retocador de imágenes del starOffice}
\end{figura}


%**************************************************************
%                        Crear                                *
%**************************************************************

\subsection{Crear}
\subsubsection{Nueva imagen}

Cuando se comienza una nueva imagen en StarOffice y se arranca el módulo Image, lo primero
que aparece es el cuadro de diálogo \menu{Nueva imagen}, en el que hay que definir los dos parámetros
básicos de cualquier gráfico bitmap: las dimensiones y la profundidad de color.

\begin{figura}{StarImageNuevaImagen}{0.9}
\caption{Creación de una imagen}
\label{fig:StarImageNuevaImagen}
\end{figura}

% \ref{fig:StarImageNuevaImagen}

\subsubsection{La ventana de Image}
En la ilustración de la derecha se muestra una ventana de StarOffice que contiene un
documento de Image. Normalmente se trabaja con la ventana del documento maximizada, pero aquí 
se muestra en posición {\it flotante}, para poder apreciar mejor qué componentes pertenecen a la
\textbf{ventana de aplicación} de StarOffice y cuáles a la \textbf{ventana de documento} de Image.

\begin{figura}{StarImageNuevaImagenSin}{0.9}
\caption{Ventana nueva imagen}
\label{fig:StarImageNuevaImagenSin}
\end{figura}

% \ref{fig:StarImageNuevaImagenSin}

\subsubsection{La ventana principal}

Recorriendo desde arriba hacia abajo la ventana principal, vemos:
\begin{itemize}
\item La barra de título.
\item El menú principal.
\item La barra de funciones.
\item La barra de objetos del desktop.
\item La zona de trabajo, en la que está la ventana del documento {\it Sin nombre1}.
\item La barra de tareas.
\item La ventana del documento.
\end{itemize}

Si repasamos desde arriba hacia abajo la ventana del documento, nos encontramos:
\begin{itemize}
\item La barra de título.
\item La barra de objetos de Image.
\item La zona de trabajo (donde se prepara la imagen).
\item La barra de desplazamiento horizontal, con las pestañas de las imágenes a la izquierda.
\item La barra de colores.
\item La línea de estado, con información sobre la imagen.
\end{itemize}

Y si la repasamos de izquierda a derecha, tenemos esto:
\begin{itemize}
\item La barra de herramientas.
\item La zona de trabajo.
\item La barra de desplazamiento vertical.
\item Cambio de la profundidad de color.
\end{itemize}

\begin{figura}{StarImageColores}{0.9}
\caption{Selector de Colores}
\label{fig:StarImageColores}
\end{figura}

\begin{figura}{StarImageEscalaGrises}{0.6}
\caption{Conversión a grises}
\label{fig:StarImageEscalaGrises}
\end{figura}


Para cambiar el número de colores de la imagen se elige en el menú \menu{Colores} el submenú 
\menu{Modificar profundidad de color} y se toma una de sus posibilidades, ver figura 
\ref{fig:StarImageColores}.


%que aparecen en la ilustración de la derecha.\\
% Cambiar por la referencia a la figura.

Si el cambio consiste en cambiar una imagen a color en una imagen en escala de grises, se pueden
elegir más opciones tomando en el menú \menu{Colores} la opción \menu{Conversión} de escalas de grises
para abrir el cuadro de diálogo Escala de grises, que también se muestra a la derecha.

% \ref{fig:StarImageEscalaGrises}

\subsubsection{Cambio de dimensiones}
Para cambiar las dimensiones de la imagen se elige en el menú \menu{Modificar} la opción \menu{Modificar}
tamaño, que abre el cuadro de diálogo Modificar tamaño. Esta operación hay que evitarla, si se puede, ya que
siempre implica una pérdida de calidad. El programa escala la imagen, y para ello debe añadir o eliminar 
puntos, cosa que es imposible de realizar conservando todas las características.

\begin{figura}{StarImageModificarTam}{0.6}
\caption{Modificar tamaño}
\label{fig:StarImageModificarTam}
\end{figura}

% \ref{fig:StarImageModificarTam}

\subsubsection{Escala}
Además del cuadro de diálogo Escala, en Image se dispone de la barra de herramientas Escala para modificar 
el tamaño con que se ve la imagen en pantalla. De las opciones disponibles, la única que muestra la imagen 
como es exactamente es la opción \textbf{1:1}, con la que cada punto de la imagen se representa con un píxel
de la pantalla.

\begin{figura}{StarImageEscala}{0.4}
\caption{Barra de escala}
\label{fig:StarImageEscala}
\end{figura}

% \ref{fig:StarImageEscala}

\subsubsection{Manejo de archivos}
Image no dispone de un formato propio para almacenar imágenes, de modo que no se pueden proteger con 
contraseña, como ocurre con los formatos de texto y hoja de cálculo.
Image puede grabar y leer la mayoría de los archivos de imagen estándar. Muchos de estos formatos requieren 
información adicional, que habrá que dar en el momento de grabar.

%*****************************************************************
%                                Dibujar                         *
%*****************************************************************

\subsection{Dibujar}
\subsubsection{Herramientas de dibujo}
Image dispone de unas pocas herramientas que permiten dibujar de un modo intuitivo y sencillo. Se explicará 
brevemente cómo elegir los colores con los que dibujar y el manejo de las herramientas.

\subsubsection{Los colores}
Image maneja dos colores simultáneamente, lo que da mayor flexibilidad al uso de las herramientas. Un color 
se llama \textbf{color de primer plano} y el otro \textbf{color de fondo}. Ambos aparecen en la barra de opciones y en la de estado. Se pueden elegir en la barra de opciones (primer plano a la izquierda, fondo a la derecha) y en la de colores (cada uno con un botón del ratón).

\subsubsection{Pluma}
Sirve para dibujar a mano alzada arrastrando el ratón con cualquiera de los botones. La forma de la pluma 
se puede elegir entre siete posibilidades, en la barra de herramientas Plumas, que se ve a la derecha. La 
anchura del trazo se elige en la barra de opciones.

\begin{figura}{StarImagePlumas}{0.4}
\caption{Plumas}
\label{fig:StarImagePlumas}
\end{figura}

% \ref{fig:StarImagePlumas}

\subsubsection{Líneas}
Sirve para dibujar un segmento recto. Hay que pulsar en el punto inicial del segmento y arrastrar hasta el 
punto final, con cualquiera de los botones.

\subsubsection{Rectángulos y elipses}
Sirven para dibujar la figura correspondiente, rellena o no. Se dibujan pulsando y arrastrando el ratón 
con cualquiera de los botones. La anchura del borde se elige en la barra de opciones.

\subsubsection{Aerógrafo}
Sirve para dibujar con pequeñas manchitas de color; la densidad de las manchitas se regula en la barra de 
opciones y también con la velocidad de arrastre del ratón; el grosor se elige en la barra de opciones.

\subsubsection{Reflejos}
Para reflejar la imagen horizontal o verticalmente se usa el menú
\menu{Modificar}, submenú \menu{Reflejar}.  Las opciones también se
encuentran en la barra de herramientas Imagen. Si se desea reflejar
sólo una parte de la imagen, se puede seleccionar primero con la
herramienta de selección.

\begin{figura}{StarImageImagen}{0.4}
\caption{Reflejos}
\label{fig:StarImageImagen}
\end{figura}

% \ref{fig:StarImageImagen}

\subsubsection{Rotaciones}
Es posible rotar la imagen completa mediante las opciones del menú \menu{Modificar}, submenú \menu{Rodar},
que también se encuentran en la barra de herramientas Imagen. Para elegir el ángulo de giro hay 
que usar la opción \comando{Ángulo libre de rotación} para poder indicar el ángulo en el cuadro de diálogo 
Ángulo de rotación libre. Hay que saber que el uso de esta herramienta origina que el tamaño de 
la imagen aumente.

\begin{figura}{StarImageRotac}{0.5}
\caption{Rotación}
\label{fig:StarImageRotac}
\end{figura}

% \ref{fig:StarImageRotac}

\subsubsection{Recortar}
Consiste en dejar en la imagen sólo una parte seleccionada, con lo que la imagen disminuye de tamaño.
Se selecciona un rectángulo de la imagen y en el menú \menu{Modificar} se elige \menu{Recortar}.
Resulta sorprendente que en este programa el típico modo de trabajo de copiar y pegar no sirva para
reproducir una parte de la imagen (cosa que no se puede realizar de ninguna forma) sino que equivale
a recortar.

%*******************************************************************
%                          Modificar                               *
%*******************************************************************

\subsection{Modificar}
\subsubsection{Selección}
Todos los modos de modificación de imagen que se van a ver a continuación se pueden aplicar a toda 
la imagen o sólo a una parte de ella, según se seleccione o no la parte con la herramienta de
selección.

\subsubsection{Invertir}
Para invertir los colores se elige en el menú \menu{Colores} la opción \menu{Invertir}.

\subsubsection{Brillo y contraste}
Para modificar estas características se elige en el menú \menu{Colores} la opción \menu{Brillo/Contraste},
y en el cuadro de diálogo Brillo y contraste se regulan a voluntad.

\begin{figura}{StarImageBrillo_y_Contraste}{0.5}
\caption{Brillo y contraste}
\label{fig:StarImageBrillo_y_Contraste}
\end{figura}

% \ref{fig:StarImageBrillo_y_Contraste}

\subsubsection{Valores RGB}
Las imágenes que se forman por emisión de luz, como las que se ven en la pantalla de ordenador o la 
televisión, están formadas por tres componentes: rojo (red, R), verde (green, G) y azul (blue, B).
Se puede regular cada componente eligiendo en el menú Colores la opción Valores RGB y modificando ahí sus
cantidades.

\begin{figura}{StarImageRGB}{0.6}
\caption{Valores RGB}
\label{fig:StarImageRGB}
\end{figura}

% \ref{fig:StarImageRGB}

\subsubsection{Color}
La barra de herramientas Color, permite modificar el brillo, el contraste y los valores RGB, así como 
cambiar la profundidad de color.

\subsubsection{Filtros}
Los filtros son modos de modificación de la imagen que pueden basarse en cualquier característica: 
los hay artísticos, que simulan tipos de pintura; técnicos que ayudan a enfocar o desenfocar imágenes; 
decorativos, para dar más variedad a las imágenes, etc. En Image se pueden aplicar desde el menú Filtro
o desde la barra de herramientas Filtros, si bien desde la barra no se pueden ajustar algunos parámetros
y desde el menú sí. Ambos se muestran a la derecha. Para conocer sus efectos, lo mejor es utilizar una 
fotografía e invertir algo de tiempo en ir probando cada filtro.

\begin{figura}{StarImageFiltroMenu}{0.5}
\caption{Menú de filtros}
\label{fig:StarImageFiltroMenu}
\end{figura}

\begin{figura}{StarImageFiltros}{0.2}
\caption{Barra de filtros}
\label{fig:StarImageFiltros}
\end{figura}

% \ref{fig:StarImageFiltroMenu}
% \ref{fig:StarImageFiltros}


%******************************************************************
%                   Fin StarImage                                 *
%******************************************************************
%%%%%%%%%%%%%%%%%%%%%%%%%%%%%%%%%%%%%%%%%%%%
% Sección: Compatibilidad con el MS Office %
%%%%%%%%%%%%%%%%%%%%%%%%%%%%%%%%%%%%%%%%%%%%
%\section{Compatibilidad con el MS Office}
