%Autor: mojopikon
%mojopikon: 4

\chapter{Java}
\label{java.tex}

Java es un lenguaje de alto nivel y de propósito general. Al principio
fue desarrollado  en los laboratorios  de {\em Sun  Microsystems} para
servir en aparatos de electrónica de consumo, como videoteléfonos, set
boxes (descodificadores), o aparatos similares, pues pretendía hacerse
un lenguaje  lo más independiente posible  de la plataforma en  la que
fueran a ser ejecutados los programas hechos con él.

A  diferencia  de otros  muchos  lenguajes  compilados, el  compilador
de  java no  genera  ficheros ejecutables.  Genera  unos ficheros  con
extensión  {\tt  .class}  llamados  {\em  bytecodes}.  Estos  ficheros
posteriormente podrán ser ejecutados  mediante la {\em Máquina Virtual
Java} (JVM en  inglés), que es la encargada de  ejecutar los programas
de java.

Esto, que a primera vista puede  parecer tedioso e inútil, es una gran
ventaja para la portabilidad del código, pues cada plataforma tiene su
propia Máquina Virtual Java: Apple Mac OS tiene una, Linux tiene otra,
Windows otra, Amiga OS también, etc. Esto implica que si desarrollamos
un  programa en  un iMac  con  Linux, el  amigo que  tenga un  potente
servidor SUN podrá ejecutar nuestra aplicación hecha en Java, al igual
que nosotros  podremos utilizar el  complicado programa de  cálculo de
estructuras que otra  persona ha desarrollado en su PC  con Windows en
el trabajo.

Realmente, Java no es tan bonito  como lo estamos pintando. Uno de sus
principales  puntos  débiles es  que  no  es  demasiado rápido,  y  es
bastante  caprichoso en  cuestiones de  recursos de  hardware. Además,
existen  dos máquinas  virtuales Java,  la de  Microsoft y  la de  SUN
Microsystems, que presentan algunas incompatibilidades.

Pero su principal  ventaja es la impresionante  portabilidad, así como
su utilidad en campos como Internet (donde principalmente se usa en la
actualidad), en servidores web.

Los  programas   java  podemos   clasificarlos  básicamente   en  tres
categorías: {\em  applets}, {\em  servlets} y aplicaciones  {\em stand
alone}.

\begin{itemize}

\item  Los {\em  applets}  son pequeñas  aplicaciones  creadas con  el
propósito de ser incluidas en páginas web. Cuando el cliente, desde su
navegador web con  java incorporado pide esa página web,  el applet se
descarga  a su  ordenador donde  comienza su  ejecución. El  navegador
tiene la maquina virtual java incorporada.  Los applets son el tipo de
aplicaciones  java que  más restricciones  de seguridad  presentan. No
pueden acceder  al sistema de  archivos local fuera del  directorio en
que se  ejecutan, no pueden  abrir ventanas adicionales sin  que éstas
aparezcan señalizadas  con el indicativo: ``Warning:  applet window'',
ni  hacer muchas  cosas  que podrían  ser  perjudiciales para  nuestro
ordenador.

\item  Las  aplicaciones  que  a   nivel  de  seguridad  permiten  más
libertades que los applets.

\item Los  {\em servlets} son una  especie de applets que  se ejecutan
sólo en  el servidor  web cuando  uno pide una  página, y  que generan
dinámicamente  páginas  a  partir  de fuentes  como  bases  de  datos,
terceros programas que recopilan información, etcétera.

\end{itemize}

A nivel  visual, java tiene dos  grupos de controles, widgets,  o como
prefiramos  llamarlos (no  son  otra  cosa que  los  campos de  texto,
formularios, botones, etc.):

\begin{itemize}

\item El  Java AWT,  (Advanced Window  Toolkit), obsoleto  y mantenido
sólo por compatibilidad en versiones actuales de java.

\item  El conjunto  de widgets  {\bf  Java Swing},  que presenta  unos
controles mucho más estéticos y es mucho más flexible.

\end{itemize}

Ambos  conjuntos de  widgets  proporcionan  una presentación  uniforme
independiente de la plataforma en que sean ejecutados. Además, otro de
los principales atractivos de java es  que el compilador no nos cuesta
nada.  Se  encuentra  disponible  para bajarlo  de  java.sun.com  para
cualquiera de las plataformas más comunes en el mercado.

Veamos cómo se generan programas  básicos en java mediante un ejemplo.
En un  editor escribimos  el siguiente  código y  lo guardamos  con el
nombre de fichero {\tt Prueba.java}.

\begin{ejemplo}{HolaMundo.java}{Ejemplo mínimo de Java}
Ejemplo mínimo de Java
\end{ejemplo}

Es de vital importancia que el  nombre del fichero coincida con lo que
escribimos después  de ``{\tt  public class}''.  Asimismo, Java  es un
lenguaje muy exigente  en materia de mayúsculas y  minúsculas, como de
de espacios y  tabuladores. Para compilar este  programa utilizamos el
siguiente comando:

\begin{verbatim}
$ ls
HolaMundo.java

$ javac HolaMundo.java

$ ls
HolaMundo.java  HolaMundo.class

\end{verbatim}


Obtendremos  un fichero  con extensión  {\tt  .class} que  es el  {\em
bytecode}, el fichero que ejecutaremos con el siguiente comando:

\begin{verbatim}
$ java HolaMundo
Hola Mundo
\end{verbatim}

Lo que no tiene sentido es poner ``{\tt  java  Prueba.class}'', pues no
funcionaría, devolviendo el siguiente error:

\begin{verbatim}
$ java HolaMundo.class
Can't find class HolaMundo.class
\end{verbatim}

\begin{verbatim}
$ java HolaMunco.class
Exception in thread "main" java.lang.NoClassDefFoundError: HolaMundo/java
\end{verbatim}

Veamos ahora un ejemplo de {\em applet}. Tecleamos el siguiente código
y lo guardamos como {\tt HolaMundo2.java}.

\begin{ejemplo}{HolaMundo2.java}{Applet mínimo de Java}
Applet mínimo de Java
\end{ejemplo}

Compilamos  el  applet del  mismo  modo  que  hicimos con  el  ejemplo
anterior:

\begin{verbatim}
$ javac HolaMundo.java
\end{verbatim}

Con esto generamos  el fichero {\tt HolaMundo2.class}  que contiene el
applet. Ahora necesitamos una página  web que cargue el applet. Creamos
el fichero {\tt HolaMundo2.html} con el siguiente código:

\begin{ejemplo}{HolaMundo2.html}{Página HTML para incluir el applet de Java}
Página HTML para incluir el applet de Java
\end{ejemplo}

Para probar el applet utilizaremos  el método que utilizaría cualquier
visitante, cargarlo desde la página web que hemos creado a tal efecto.
Esto lo  hacemos con cualquier  navegador que soporte Java,  entre los
que recomendamos Netscape.

\begin{verbatim}
$ netscape HolaMundo2.html
\end{verbatim}

Veremos como el applet se ejecuta  dentro de la página web. Otra forma
de ejecutar un applet de  Java, por ejemplo cuando estamos programando
y sólo  deseamos probarlo  pero no queremos  ejecutar netscape,  es el
programa appletviewer que proporciona el JDK.

\begin{verbatim}
$ appletviewer HolaMundo2.html
\end{verbatim}

Podemos apreciar la  diferencia entre un {\em  applet}, cuya ejecución
está controlada por el navegador y se  limita a la página web desde el
que  es  cargado,  y  una  aplicación  intependiente  que  se  ejecuta
directamente en  la consola del  sistema sin más intermediario  que la
Máquina Virual Java.
