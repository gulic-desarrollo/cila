
\chapter{Licencia de Documentación Libre GNU}

\label{GFDL}

Versión 1.1, Marzo de 2000

Ésta es la GNU Free Document  License (GFDL), versión 1.1 (de marzo de
2000),  que cubre  manuales  y  documentación para  el software  de la
Free  Software  Foundation,  con  posibilidades en  otros  campos.  La
traducción  \footnote{N.  del T.  Derechos  Reservados  en el  sentido
de   GNU  {\tt http://www.gnu.org/copyleft/copyleft.es.html}}   no
tiene  ningún  valor  legal,  ni  ha  sido  comprobada  de  acuerdo  a
la  legislación de  ningún  país  en particular.  Vea  el original  en
{\tt http://www.gnu.org/copyleft/fdl.html}

Los autores de esta traducción son:

\begin{itemize}
\item Igor Támara ({\tt ikks@bigfoot.com})
\item Pablo Reyes ({\tt reyes\_pablo@hotmail.com})
\item Revisión: Vladimir Támara P. ({\tt vtamara@gnu.org})
\end{itemize}

\begin{quote}
Copyright $\copyright$ 2000  Free Software Foundation, Inc.\\
      59 Temple Place, Suite 330, Boston, MA  02111-1307  USA\\
  Everyone is permitted to copy and distribute verbatim copies
  of this license document, but changing it is not allowed.
\end{quote}
  Se permite la copia y distribución de copias literales
  de este documento de licencia, pero no se permiten cambios.
  
 
\section*{Preámbulo}

El propósito  de esta  licencia es  permitir que  un manual,  libro de
texto,  u  otro documento  escrito  sea  ``libre''  en el  sentido  de
libertad: asegurar a todo el mundo  la libertad efectiva de copiarlo y
redistribuirlo, con o sin modificaciones, de manera comercial o no. En
segundo término,  esta licencia  preserva para el  autor o  para quien
publica una manera de obtener reconocimiento por su trabajo, al tiempo
que no se consideran responsables de las modificaciones realizadas por
terceros.

Esta licencia  es una  especie de ``copyleft''  que significa  que los
trabajos derivados del documento deben a su vez ser libres en el mismo
sentido. Esto complementa la Licencia  Pública General GNU, que es una
licencia de copyleft diseñada para el software libre.

Hemos  diseñado esta  Licencia  para usarla  en  manuales de  software
libre,  ya que  el  software libre  necesita  documentación libre:  Un
programa  libre debe  venir con  los manuales  que ofrezcan  la mismas
libertades  que da  el software.  Pero esta  licencia no  se limita  a
manuales de software; puede ser  usada para cualquier trabajo textual,
sin tener  en cuenta su temática  o si se publica  como libro impreso.
Recomendamos esta  licencia principalmente para trabajos  cuyo fin sea
instructivo o de referencia.

\section{Aplicabilidad y definiciones}

Esta  Licencia se  aplica  a  cualquier manual  u  otro documento  que
contenga una  nota del pro\-pie\-ta\-rio  de los derechos  que indique
que  puede  ser distribuido  bajo  los  términos  de la  Licencia.  El
``Documento'', en adelante, se refiere a cualquiera de dichos manuales
o trabajos. Cualquier miembro del  público es un licenciatario, y será
denominado como ``Usted''.

Una ``Versión  Modificada'' del Documento significa  cualquier trabajo
que contenga  el Documento o una  porción del mismo, ya  sea una copia
literal o con modificaciones y/o traducciones a otro idioma.

Una  ``Sección Secundaria''  es  un apéndice  titulado  o una  sección
preliminar al prólogo  del Documento que tiene  que ver exclusivamente
con la  relación de quien publica,  o los autores del  Documento, o el
tema general del Documento(o asuntos relacionados) y cuyo contenido no
entra directamente en este tema general. (Por ejemplo, si el Documento
es en parte  un texto de matemáticas, una Sección  Secundaria puede no
explicar matemáticas.)  La relación  puede ser  un asunto  de conexión
histórica,  o  de  posición  legal,  comercial,  filosófica,  ética  o
política con el tema o la materia del texto.

Las ``Secciones Invariantes'' son  ciertas Secciones Secundarias cuyos
títulos son  denominados como  Secciones Invariantes,  en la  nota que
indica que el documento es liberado bajo esta licencia.

Los ``Textos de Cubierta'' son ciertos  pasajes cortos de texto que se
listan, como Textos de Portada o  Textos de Contra Portada, en la nota
que indica que el documento está liberado bajo esta Licencia.

Una  copia ``Transparente''  del Documento,  significa una  copia para
lectura  en máquina,  representada en  un formato  cuya especificación
está disponible al público general, cuyos contenidos pueden ser vistos
y  editados  directamente con  editores  de  texto genéricos  o  (para
imágenes compuestas  por pixeles) de  programas genéricos de  dibujo o
(para dibujos) algún editor gráfico  ampliamente disponible, y que sea
adecuado  para exportar  a formateadores  de texto  o para  traducción
automática  a  una variedad  de  formatos  adecuados para  ingresar  a
formateadores de  texto. Una copia hecha  en un formato de  un archivo
que no sea Transparente, cuyo formato  ha sido diseñado para impedir o
dificultar subsecuentes  modificaciones posteriores  por parte  de los
lectores no es  Transparente. Una copia que no  es ``Transparente'' es
llamada ``Opaca''.

Como ejemplos de formatos adecuados para copias Transparentes están el
ASCII plano sin formato, formato de  Texinfo, formato de LaTeX, SGML o
XML usando un DTD disponible ampliamente,  y HTML simple que sigue los
estándares, diseñado para modificaciones  humanas. Los formatos Opacos
incluyen PostScript, PDF, formatos  propietarios que pueden ser leídos
y editados unicamente en procesadores de palabras propietarios, SGML o
XML para los cuáles los DTD y/o herramientas de procesamiento no están
disponibles generalmente, y el HTML  generado por máquinas producto de
algún procesador de palabras sólo para propósitos de salida.

La ``Portada'' en un libro  impreso significa, la propia portada junto
con las páginas siguientes necesarias para mantener la legibilidad del
material, que esta Licencia requiere  que aparezca en la portada. Para
trabajos  en formatos  que  no tienen  Portada  como tal,  ``Portada''
significa el texto junto a la  aparición más prominente del título del
trabajo, precediendo el comienzo del cuerpo del trabajo.


\section{Copia literal}

Puede  copiar y  distribuir el  Documento en  cualquier medio,  sea en
forma comercial  o no, siempre  y cuando  esta Licencia, las  notas de
derecho de autor,  y la nota de licencia que  indica que esta Licencia
se aplica al Documento se reproduzca  en todas las copias, y que usted
no añada ninguna  otra condición a las expuestas en  en esta Licencia.
No puede usar medidas técnicas para  obstruir o controlar la lectura o
copia  posterior  de las  copias  que  usted  haga o  distribuya.  Sin
embargo, usted puede  aceptar compensación a cambio de  las copias. Si
distribuye un  número suficientemente grande de  copias también deberá
seguir las condiciones de la sección 3.

También puede prestar copias, bajo las mismas condiciones establecidas
anteriormente, y puede exhibir copias publicamente.

\section{Copiado en cantidades}

Si publica copias impresas del Documento  que sobrepasen las 100, y la
nota de Licencia del Documento  exige Textos de Cubierta, debe incluir
las copias  con cubiertas que lleven  en forma clara y  legible, todos
esos textos  de Cubierta: Textos  Frontales en la cubierta  frontal, y
Textos  Posteriores  de  Cubierta  en  la  Cubierta  Posterior.  Ambas
cubiertas deben identificarlo a Usted  clara y legiblemente como quien
publica  tales copias.  La  Cubierta Frontal  debe  mostrar el  título
completo  con todas  las palabras  igualmente prominentes  y visibles.
Además puede  añadir  otro  material  en  la cubierta. Las  copias con
cambios limitados  en las cubiertas,  siempre que preserven  el título
del Documento y satisfagan  estas condiciones, puede considerarse como
copia literal.

Si los textos requeridos para la cubierta son muy voluminosos para que
ajusten  legiblemente,  debe colocar  los  primeros  (tantos como  sea
razonable  colocar) en  la  cubierta  real, y  continuar  el resto  en
páginas adyacentes.

Si  publica o  distribuye copias  Opacas del  Documento cuya  cantidad
exceda las  cien, debe  incluir una copia  Transparente que  pueda ser
leída por una máquina  con cada copia Opaca, o entregar  en o con cada
copia Opaca una dirección  en red de computador públicamente-accesible
conteniendo  una  copia  completa   Transparente  del  Documento,  sin
material adicional,  a la cual el  público en general de  la red pueda
acceder a bajar  anónimamente sin cargo usando  protocolos de standard
público. Si usted  hace uso de la última opción,  deberá tomar medidas
necesarias, cuando  comience la distribución  de las copias  Opacas en
cantidad,  para  asegurar  que  esta  copia  Transparente  permanecerá
accesible  en el  sitio  por lo  menos  un año  después  de su  última
distribución de copias Opacas (directamente  o a través de sus agentes
o distribuidores) de esa edición al público.

Se solicita, aunque no es requisito,  que contacte con los autores del
Documento  antes  de redistribuir  cualquier  número  de copias,  para
permitirle  la  oportunidad de  que  le  suministren una  versión  del
Documento.

\section{Modificaciones}

Puede copiar  y distribuir una  Versión Modificada del  Documento bajo
las condiciones  de las seccions 2  y 3 anteriores, siempre  que Usted
libere la Versión Modificada bajo  esta misma Licencia, con la Versión
Modificada haciendo el rol del  Documento, por lo tanto licenciando la
distribución y  modificación de la  Versión Modificada a  quien quiera
que posea  una copia de  éste. Además, debe  hacer lo siguiente  en la
Versión Modificada:

\begin{itemize}

\item Uso  en la  Portada (y en  las cubiertas, si  hay alguna)  de un
título  distinto al  del  Documento, y  de  versiones anteriores  (que
deberían, si hay alguna, estar listados  en la sección de Historia del
Documento). Puede  usar  el  mismo  título  que  el  de las  versiones
anteriores al original siempre que quién publicó la primera versión lo
permita.

\item  Listar en  la  Portada, como  autores, una  o  más personas,  o
entidades  responsables por  la  autoría o  las  modificaciones en  la
Versión  Modificada, junto  con  por  lo menos  cinco  de los  autores
principales  del  Documento (Todos  sus  autores  principales, si  son
inferiores a cinco).

\item Estado  en la  Portada del  nombre de  quien publica  la Versión
Modificada, como quien publica.

\item Preservar todas las notas de derechos de autor del Documento.

\item  Añadir   una  nota  de   derecho  de  autor  apropiada   a  sus
modificaciones adyacentes a las otras notas de derecho de autor.

\item Incluir, inmediatamente después de  la nota de derecho de autor,
una nota  de licencia dando  el permiso  público para usar  la Versión
Modificada bajo los términos de esta Licencia, de la forma mostrada en
la Adición (LEGAL) abajo.

\item  Preservar  en esa  nota  de  licencia  el listado  completo  de
Secciones  Invariantes y  en  los  Textos de  las  Cubiertas que  sean
requeridos como se especifique en la nota de Licencia del Documento

\item Incluir una copia sin modificación de esta Licencia.

\item Preservar la sección llamada ``Historia'', y su título, y añadir
a esta una sección estableciendo al menos el título, el año,los nuevos
autores,  y  quién publicó  la  Versión  Modificada  como reza  en  la
Portada. Si no hay una  sección titulada ``Historia'' en el Documento,
crear una estableciendo el título, el año, los autores y quién publicó
el Documento  como reza  en la Portada,  añadiendo además  un artículo
describiendo  la Versión  Modificada como  se estableció  en el  punto
anterior.

\item  Preservar  la   localización  en  red,  si  hay,   dada  en  la
Documentación para  acceder públicamente a una  copia Transparente del
Documento,  tanto  como las  otras  direcciones  de  red dadas  en  el
Documento para  versiones anteriores  en las cuáles  estuviese basado.
Éstas pueden ubicarse  en la sección ``Historia''. Se  puede omitir la
ubicación en red para un trabajo que sea publicado por lo menos 4 años
antes  que el  mismo Documento,  o si  quien publica  originalmente la
versión da permiso explícitamente.

\item   En   cualquier    sección   titulada   ``Agradecimientos''   o
``Dedicatorias'', preservar el título de la sección, y preservar en la
sección  toda  la sustancia  y  el  tono  de los  agradecimientos  y/o
dedicatorias de cada contribuyente que estén incluidas.

\item  Preservar todas  las Secciones  Invariantes del  Documento, sin
alterar  su  texto  ni  sus  títulos. Los  números  de  sección  o  el
equivalente no son considerados parte de los títulos de la sección. M.
Borrar cualquier sección titulada ``Aprobaciones''. Tales secciones no
pueden estar incluidas en las Versiones Modificadas.

\item  Borrar  cualquier   sección  titulada  ``Aprobaciones''.  Tales
secciones no pueden estar incluidas en las Versiones Modificadas.

\item No  retitular ninguna sección existente  como ``Aprobaciones'' o
conflictuar con título con alguna Sección Invariante.

\end{itemize}

Si  la  Versión  Modificada  incluye  secciones,  apéndices  nuevos  o
preliminares  al prólogo  que califican  como Secciones  Secundarias y
contienen  material  no  copiado del  Documento,  puede  opcionalmente
designar  algunas  o  todas  esas  secciones  como  invariantes.  Para
hacerlo, añada sus  títulos a la lista de Secciones  Invariantes en la
nota de  licencia de  la Versión Modificada.  Tales títulos  deben ser
distintos de cualquier otro título de sección.

Puede   añadir  una  sección  titulada  ``Aprobaciones'', siempre  que
contenga únicamente  aprobaciones de su Versión  Modificada por varias
fuentes. Por ejemplo, observaciones de peritos  o que el texto ha sido
aprobado por una organización como un standard.

Puede  añadir un  pasaje  de hasta  cinco palabras  como  un Texto  de
Cubierta Frontal,  y un pasaje de  hasta 25 palabras como  un texto de
Cubierta Posterior, al  final de la lista de Textos  de Cubierta en la
Versión Modificada. Solamente un pasaje de Texto de Cubierta Frontal y
un Texto  de Cubierta Posterior puede  ser añadido por (o  a manera de
arreglos hechos por) una entidad. Si  el Documento ya incluye un texto
de cubierta  para la misma  cubierta, previamente añadido por  usted o
por  arreglo hecho  por la  misma entidad,  a nombre  de la  cual está
actuando, no puede añadir otra, pero puede reemplazar la anterior, con
permiso explícito de quien publicó anteriormente tal cubierta.

El(los) autor(es)  y quien(es)  publica(n) el  Documento no  da(n) con
esta Licencia  permiso para  usar sus nombres  para publicidad  o para
asegurar o implicar aprobación de cualquier Versión Modificada.

\section{Combinando documentos}

Puede combinar el  Documento con otros documentos  liberados bajo esta
Licencia, bajo  los términos definidos  en la sección 4  anterior para
versiones  modificadas, siempre  que incluya  en la  combinación todas
las  Secciones Invariantes  de  todos los  documentos originales,  sin
modificar,  y listadas  todas como  Secciones Invariantes  del trabajo
combinado en su nota de licencia.

El trabajo  combinado necesita  contener solamente  una copia  de esta
Licencia,  y  múltiples  Secciones Invariantes  Idénticas  que  pueden
ser  reemplazadas  por una  sola  copia.  Si hay  múltiples  Secciones
Invariantes con el  mismo nombre pero con  contenidos diferentes, haga
el título  de cada una de  estas secciones único añadiéndole  al final
de  éste, en  paréntesis,  el  nombre del  autor  o  de quien  publicó
originalmente esa sección,  si es conocido, o si no,  un número único.
Haga el mismo ajuste a los títulos de sección en la lista de Secciones
Invariantes en la nota de licencia del trabajo combinado.

En   la  combinación,   debe  combinar   cualquier  sección   titulada
``Historia''  de  los  varios   documentos  originales,  formando  una
sección  titulada ``Historia'';  de la  misma forma  combine cualquier
seción  titulada  ``Agradecimientos'',  y cualquier  sección  titulada
``Dedicatorias''.   Debe   borrar   todas  las   secciones   tituladas
``Aprobaciones''.

\section{Colecciones de documentos}

Puede hacer una colección consistente del Documento y otros documentos
liberados bajo esta Licencia, y  reemplazar las copias individuales de
esta Licencia  en los varios  documentos con  una sola copia  que esté
incluida en la colección, siempre que siga las reglas de esta Licencia
para una copia literal de cada  uno de los documentos en cualquiera de
todos los aspectos.

Puede  extraer  un solo  documento  de  una  de tales  colecciones,  y
distribuirlo individualmente  bajo esta Licencia, siempre  que inserte
una  copia de  esta Licencia  en el  documento extraído,  y siga  esta
Licencia en todos los otros  aspectos concernientes a la copia literal
de tal documento.

\section{Agregación con trabajos independientes}

Una  recopilación  del   Documento  o  de  sus   derivados  con  otros
documentos o trabajos separados o independientes, en cualquier tipo de
distribución o medio  de almacenamiento, no como un  todo, cuenta como
una Versión Modificada  del Documento, teniendo en  cuenta que ninguna
compilación de  derechos de autor  sea clamada por la  recopilación. A
tal  recopilación se  le llama  ``agregado'',  y esta  Licencia no  se
aplica a los otros trabajos  auto-contenidos y por lo tanto compilados
con el Documento, o a cuenta de haber sido compilados, si no son ellos
los mismos trabajos derivados del Documento.

Si  el requerimiento  de la  sección  3 del  Texto de  la Cubierta  es
aplicable a  estas copias del  Documento, entonces si el  Documento es
menor que un cuarto del agregado entero, Los Textos de la Cubierta del
Documento pueden ser colocados en cubiertas que enmarquen solamente el
Documento entre el agregado. De otra forma deben aparecer en cubiertas
enmarcando todo el agregado.

\section{Traducción}

Se considera a  la Traducción como una clase de  modificación. Así que
puede distribuir  traducciones del Documento  bajo los términos  de la
sección  4.  Reemplazar  las Secciones  Invariantes  con  traducciones
requiere  permiso  especial  de  los   dueños  de  derecho  de  autor,
pero  puede incluir  traducciones  de algunas  o  todas las  Secciones
Invariantes adicionalmente a las versiones originales de las Secciones
Invariantes. Puede incluir una traducción de esta Licencia siempre que
incluya también  la versión Inglesa  de esta  Licencia. En caso  de un
desacuerdo entre la traducción y la versión original en Inglés de esta
Licencia, la versión original en Inglés prevalecerá.

\section{Terminación}

No se puede copiar, modificar, sublicenciar, o distribuir el Documento
excepto por  lo permitido  expresamente bajo esta  Licencia. Cualquier
otro intento de copia,  modificación, sublicenciamiento o distribución
del Documento es nulo, y  sus derechos serán automáticamente retirados
de esa  licencia. De  todas maneras, los  terceros que  hayan recibido
copias  o derechos  de  su parte  bajo esta  Licencia  no tendrán  por
terminadas sus  licencias siempre  que tales  personas o  entidades se
encuentren en total conformidad con la licencia original.

\section{Futuras revisiones de esta licencia}

La  Free  Software  Foundation   puede  publicar  nuevas  versiones  o
revisadas  de  la Licencia  de  Documentación  Libre GNU  cada  cierto
tiempo.  Tales versiones  serán similares  en espíritu  a la  presente
versión, pero pueden  diferir en detalles para  solucionar problemas o
intereses. Vea {\tt http://www.gnu.org/copyleft/}

Cada  versión  de la  Licencia  tiene  un  número  de versión  que  la
distingue.  Si  el  Documento  especifica  que  una  versión  numerada
particularmente de esta licencia  o ``cualquier versión posterior'' se
aplica a  ésta, tiene la opción  de seguir los términos  y condiciones
de  la  versión especificada  o  cualquiera  posterior que  haya  sido
publicada(no como un borrador) por  la Free Software Foundation. Si el
Documento no especifica  un número de versión de  esta Licencia, puede
escoger  cualquier  versión  que  haya  sido  publicada  (no  como  un
borrador) por la Free Software Foundation.

\section{Addendum}

Para  usar esta  licencia  en  un documento  que  usted haya  escrito,
incluya una copia de la Licencia  en el documento y ponga el siguiente
derecho de  autor y nota  de licencia justo  después del título  de la
página:

\begin{quote}

	\copyright  Año  Su Nombre.

	Permiso para copiar, distribuir y/o modificar este documento
        bajo los términos de  la Licencia  de Documentación Libre  GNU,
        Versión  1.1 o cualquier  otra  versión posterior  publicada
        por  la Free  Software	Foundation; con  las Secciones Invariantes
        siendo LISTE SUS TÍTULOS, siendo LISTE  el texto de la  Cubierta
        Frontal,  y siendo LISTELO el texto de la Cubierta Posterior.

	Se incluye una  copia  de  la  licencia  en  la  sección  titulada
	``Licencia de Documentación Libre GNU''.

\end{quote}

Si   no  tiene   Secciones   Invariantes,   escriba  ``Sin   Secciones
Invariantes''  en vez  de decir  cuáles son  invariantes. Si  no tiene
Texto de Cubierta  Frontal, escriba ``Sin Texto  de Cubierta Frontal''
en vez  de``siendo LÍSTE el texto  de la Cubierta Frontal'';  así como
para la Cubierta Posterior.

Si su documento contiene ejemplos  de código de programa no triviales,
recomendamos liberar  estos ejemplos en  paralelo bajo su  elección de
licencia de  software libre, tal  como la Licencia de  Público General
GNU, para permitir su uso en software libre.

