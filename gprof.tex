%Autor: Carlos Paramio

\chapter{GNU gprof}\label{gprof}

Puede  sernos de  mucha  utilidad que  ejecutemos  nuestro software  a
trav�s de un  {\em profiler}, como puede ser {\tt  gprof}, ya que �ste
nos  proporcionar� suficiente  informaci�n  para  detectar cuellos  de
botella en nuestro programa, que ser�an susceptibles de optimizar.

\section{El profiler gprof}

Como ya  se ha dicho,  a menudo  deseamos optimizar el  rendimiento de
nuestro  programa,  pero  en determinadas  ocasiones  resulta  dif�cil
conocer d�nde  est�n localizados los  cuellos de botella.  Bien podr�a
ser una funci�n que consuma una  gran cantidad de tiempo de ejecuci�n,
como alguna otra que, aunque  muy r�pida, se ejecuta demasiadas veces.
Esto nos  permitir� optimizar  de manera adecuada  nuestra aplicaci�n,
sustituyendo por macros o funciones {\tt inline} aquellas m�s llamadas
(para eliminar el retardo de resoluci�n  de la llamada) y mejorando el
c�digo en s�.

Para   compilar  nuestra   aplicaci�n   de  forma   que  gprof   pueda
posteriormente extraer toda esta  informaci�n, simplemente a�adimos el
par�metro {\tt -pg} al compilador  GCC. Una vez hecho esto, procedemos
a ejecutar  el programa en  s�, y tras  salir correctamente de  �l (al
finalizar {\tt main()},  o tras un {\tt exit()}),  observaremos que se
ha generado un nuevo fichero en  el directorio de trabajo llamado {\tt
gmon.out}. S�lo nos queda pasar este archivo a trav�s de gprof, con el
siguiente formato: \\

\begin{verbatim}
gprof <opciones> <ejecutable> <gmon.out>
\end{verbatim}

En  nuestro  sempiterno  ejemplo  de  programa,  har�amos  {\tt  gprof
miprograma gmon.out}.

Hay una opci�n que merece la  pena destacar, que nos muestra el c�digo
fuente  de nuestro  programa junto  con un  conteo de  ejecuciones por
l�nea. Esta opci�n se activa con el par�metro {\tt -A}.

Una  vez  m�s,  ampliaremos  nuestros  conocimientos  acerca  de  esta
herramienta usando el comando {\tt info  gprof}, o bien con la lectura
del manual de GNU.% \cite{gprof}.

