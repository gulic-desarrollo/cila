%Autor: cparamio
%cparamio: 1

\chapter{GNU gprof}
\label{gprof}
\label{gprof.tex}

Puede  sernos de  mucha  utilidad ejecutar  nuestro software  a
través de un  {\em profiler}, como puede ser {\tt  gprof}, pues que éste
nos  proporcionará suficiente  información  para  detectar cuellos  de
botella en nuestro programa y así poder corregirlos.

\section{El profiler gprof}

Como ya  se ha dicho,  a menudo  deseamos mejorar  el  rendimiento  de
nuestro  programa,  pero  en determinadas  ocasiones  resulta  difícil
conocer dónde  están localizados los  cuellos de botella.  Bien podría
ser una función que consuma una  gran cantidad de tiempo de ejecución,
como alguna otra que, aunque  muy rápida, se ejecuta demasiadas veces.
Detectar esto nos  permitirá optimizar el código de nuestra aplicación,
sustituyendo por macros o funciones {\tt inline} aquellas más llamadas
(para eliminar el retardo de resolución  de la llamada) y mejorando el
código en sí.

Para   compilar  nuestra   aplicación   de  forma   que  gprof   pueda
posteriormente extraer toda esta  información, simplemente añadimos el
parámetro {\tt -pg} al compilador  GCC. Una vez hecho esto, procedemos
a ejecutar  el programa en  sí, y tras  salir correctamente de  él (al
finalizar {\tt main()},  o tras un {\tt exit()}),  observaremos que se
ha generado un nuevo fichero en  el directorio de trabajo llamado {\tt
gmon.out}. Sólo nos queda pasar este archivo a través de gprof, con el
siguiente formato: \\

\begin{verbatim}
gprof <opciones> <ejecutable> <gmon.out>
\end{verbatim}

En  nuestro  sempiterno  ejemplo  de  programa,  ejecutaríamos la orden
{\tt  gprof miprograma gmon.out}.

Hay una opción que merece la  pena destacar, que nos muestra el código
fuente  de nuestro  programa junto  con un  conteo de  ejecuciones por
línea. Esta opción se activa con el parámetro {\tt -A}.

Una  vez  más,  ampliaremos  nuestros  conocimientos  acerca  de  esta
herramienta usando el comando {\tt info  gprof}, o bien con la lectura
del manual de GNU. %\cite{gprof}.

