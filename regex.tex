%Autor: miguev, faraox
%miguev: 2 + 5
%faraox: 3


\chapter{Expresiones regulares}
\index{Expresiones regulares}
\label{regex.tex}

Las expresiones  regulares se comprenden  mejor desde su  utilidad que
desde su definición, así que vamos  a introducirlas con el ejemplo que
utilizan en el libro \cite{mastregex}.

Imagina que  necesitas una herramienta para  buscar palabras repetidas
en  documentos,  cosas  como   ``esto  esto'',  un  problema  bastante
frecuente en  documentos que son  editados una  y otra vez  (como este
libro). 

Tendrías que  escribir una solución que:  detectara palabras repetidas
de una línea a  la siguiente, cuando una palabra está  al final de una
línea y también al principio de  la siguiente línea; que no tuviera en
cuenta  si las  letras están  en minúsculas  o mayúsculas  y encontrar
palabras  repetidas en  medio de  código HTML,  \LaTeX, \LyX~  u otros
lenguajes de marcado usados en documentación.

Esto  es un  problema  real,  y el  autor  del libro  \cite{mastregex}
escribió  esta  solución y  se  vió  sorprendido  por la  cantidad  de
palabras  repetidas que  encontró en  el propio  libro. Este  problema
puede  resolverse  de   muchas  maneras  y  en   muchos  lenguajes  de
programación,  pero   las  expresiones  regulares  son   sin  duda  la
herramienta que hacen este tipo de  problemas más fáciles y rápidos de
resolver.

\section{Definición de expresión regular}

Una {\em expresión  regular} es un literal (cadena  de caracteres) que
define  una  serie de  reglas  que  debe  cumplir una  expresión  para
considerar que {\em encaja} con la expresión regular. Esto se consigue
utilizando caracteres con significados especiales para {\em describir}
un patrón con el que deben encajar las expresiones.

Los  caracteres   con  significados   especiales  se   denominan  {\em
metacaracteres}, y aunque  la mayoría son comunes  entre los distintos
estilos   de  expresiones   regulares  hay   diferencias  y   conviene
estudiarlos  desde la  herramienta en  la que  usemos las  expresiones
regulares. Así, nos encontramos con  expresiones regulares de UNIX, de
VI, de PERL, de Emacs, etc. y no tienen la misma sintaxis exactamente.

Dado  que  esto  no  es  un  curso  de  programación  no  veremos  las
expresiones regulares al  estilo de lenguajes de  programación sino al
estilo de los editores de texto.

\section{Expresiones regulares en VI \& VIM}
\index{Expresiones regulares!con VIM }

Una de las  funcionalidades más queridas de {\sf VI}  es la búsqueda y
\index{VIM!búsqueda y substitución}
sustitución de texto.  Para empezar pensemos en  simples palabras, por
ejemplo en substituir  la palabra ``coche'' por  la palabra ``carro''.
El  comando para  la substitución  es {\tt  :s} y  tiene el  siguiente
esquema:

\begin{verbatim}
:rango/busca/sustituye/[gc]
\end{verbatim}

Lo que pongas en {\tt rango}  determina la región del fichero que será
afectada por el  comando. Puedes poner dos números  de línea separados
por  coma para  afectar  a  todas las  líneas  entre  esas dos  (ambas
inclusive), o  bien substituir el  número de  la segunda línea  por un
\verb.$. para extender la región  afectada hasta el final del fichero.
Si en lugar de líneas poner \verb.%. afectará a todo el fichero.

Las opciones  {\tt g} y  {\tt c} del  final son opcionales.  La opción
{\tt g} hace  que se substituyan todas las ocurrencias  en cada línea,
ya que si  no la especificas sólo se substituye  la primera ocurrencia
de cada línea.  La opción {\tt c} hace que  {\tt VI} pida confirmación
antes de substituir cada ocurrencia.

La expresión {\tt busca} es el patrón que {\sf VIM} buscará. Puede ser
una  palabra,  o  más  en  general una  {\em  expresión  regular}.  La
expresión {\tt substituye}  es el patrón que {\sf  VIM} utilizará para
substituir en las ocurrencias, y  también puede ser una {\em expresión
regular}.

En  {\sf  VI}  las   expresiones  regulares  utilizan  los  siguientes
metacaracteres: \verb_. ^ $ [ ] \ * - /_

El punto  (\verb_._) encaja  con cualquier  caracter, \verb.$.  con el
final de la línea, \verb.^. con el  principio de la línea cuando es el
primero en la expresión regular, mientras  que en otro caso encaja con
cualquier caracter que  {\bf no} sea el que le  suceda. Los caracteres
\verb.-.,  \verb.[.  y  \verb.].  permiten  definir  rangos,  \verb.*.
extiende el  significado de la  expresión que  le preceda a  ninguna o
varias  ocurrencias de  la misma.  La barra  \verb./. se  utiliza para
separar  los  patrones de  búsqueda.  Para  utilizar alguno  de  estos
caracteres  de forma  que  encajen consigo  mismos  (por ejemplo  para
buscar un punto en el  texto) hay que {\em escaparlos} anteponiéndoles
una barra invertida (\verb.\.).

Otros  caracteres   adquieren  un  significado  especial   cuando  son
precedidos de la barra invertida.  Así \verb.\|. es el operador ``o'',
es decir la  expresión \verb.casa\|coche. encaja con {\tt  casa} y con
{\tt coche}.  Los símbolos de desigualdad  permiten delimitar palabras
en un patrón, usando \verb.\<. que  encaja con el principio de palabra
y \verb.\>. con  el final. Así podremos sustituir la  palabra {\tt vi}
por {\tt vim} sin que cambien palabras del texto como {\tt vivir}.

\begin{verbatim}
:s/\<vi\>/vim
\end{verbatim}

Podemos  utilizar el  carácter  \verb.^. para  indicar  que no  exista
ningun carácter antes  de la expresión regular  que queremos sustituir
en esa línea, y el carácter  \verb.$. para indica que no exista ningun
carácter despuéas  de la expresión  regular que queremos  sustituir en
usa línea. Vamos a ver unos ejemplos:

\begin{verbatim}
:s/^vi/vim/g
\end{verbatim}

En este  ejemplo, se cambia  la palabra {\tt vi}  por {\tt vim}  si no
existe ningun carácter delante de la palabra {\tt vi}.

\begin{verbatim}
:s/vi$/vim
\end{verbatim}

En el anterior ejemplo, se sustituye la palabra {\tt vi} por {\tt vim}
que se encuentre al final de una línea.

\begin{verbatim}
:s/^vi$/vim
\end{verbatim}

Los metacaracteres  nos permiten  una gran flexibilidad  a la  hora de
trabajar con expresiones regulares.

\begin{table}[htbp]
\centering
\index{Expresiones regulares!metacaracteres con VIM }
\begin{tabular}{|c|l|}
\hline
{\tt .}         & Cualquier caracter \\
{\tt \verb.\s.} & Espacios en blanco \\
{\tt \verb.\d.} & Digitos \\
{\tt \verb.\h.} & Cualquier letra \\
{\tt \verb.\w.} & Letras y digitos \\
{\tt \verb.\l.} & Letras minúsculas \\
{\tt \verb.\u.} & Letras mayúsculas \\
\hline
\end{tabular}
\caption{Metacaracteres en las expresiones regulares de VI}
\end{table}

Vamos a poner un ejemplo con este tipo de comandos. Si queremos buscar
en el texto fechas con el formato DD/MM/AAAA, lo que haríamos sería:

\begin{verbatim}
/\d\d\/\d\d\/\d\d\d\d
\end{verbatim}

O por ejemplo  si buscamos palabras que empiezen con  mayúsculas de un
modo sencillo podemos hacer:

\begin{verbatim}
/\<\u
\end{verbatim}

Ahora vamos a  introducir unos elementos nuevos  para hacer reemplazos
y  busquedas  más  ``profesionales''.  Uno   de  ellos  son  los  {\em
cuantificadores}.\index{VIM!expresiones  regulares!cuantificadores} Si
necesitamos  buscar   en  un  documento  grandes   grupos  de  números
de  9  cifras  por  ejemplo  nos  resultaría  muy  laborioso  escribir
\verb./\d\d\d\d\d\d\d\d\d..    Los   cuantificadores    son   diversas
expresiones que nos permiten definir  el número de veces que aparecerá
el  tipo de  caracter que  queremos  buscar. Los  más importantes  son
\verb.\+., relaciona  el caracter que aparece  una o más veces  en una
palabra, \verb.\=., relaciona el caracter que aparece 0, 1 o más veces
en  una  palabra.  Con  \verb.\{n,m}.  relaciona  el  caracter  si  se
encuentra entre  n y  m veces. La  expresión \verb.\{n}.  relaciona el
caracter si se  encutra n veces. Así en el  ejemplo anterior podríamos
usar:

\begin{verbatim}
/\d\{9}
\end{verbatim}

y nos mostraría las expresiones que contengan 9 digitos.

Podemos combinar los cuantificadores con los rangos de caracteres para
conseguir mejores resultados. Los  rangos de expresiones regulares son
definidos dentro  de \verb.[].. Por ejemplo,  \verb./[a-h]. buscará en
el  texto  todos las  letras  entre  la a  y  la  h, ambas  inclusive,
\verb./[A-H].  haría  lo  mismo   en  mayúsculas.  También  existe  un
valor  negativo  en los  rangos,  es  el  {\tt \verb.^.}.  El  comando
\verb./[^j-z]. mostrará todas las letras minúsculas anteriores a la j.

Para   buscar   mayúsculas   o   minúsculas   hay   algo   mejor   que
los   rangos   \verb.[a-z].  y   \verb.[A-Z].,   que   son  las   {\em
clases}\index{VIM!expresiones   regulares!clases}.   Las   clases   se
refieren mediante  palabras encerradas entre \verb.[[:.  y \verb.:]].,
por  lo  que para  buscar  una  expresión  mediante clases  el  patrón
de  búsqueda deberá  contener  \verb.[[:clase:]]. Las  clases que  nos
interesan son:

\begin{table}[htbp]
\centering
\begin{tabular}{|c|l|}
\hline
{\tt \verb.[[:lower:]].} & cualquier letra minúscula (incluida {\tt ñ}) \\
{\tt \verb.[[:upper:]].} & cualquier letra mayúscula (incluida {\tt Ñ}) \\
{\tt \verb.[[:alpha:]].} & cualquier letra \\
{\tt \verb.[[:digit:]].} & cualquier dígito \\
\hline
\end{tabular}
\caption{Metacaracteres en las expresiones regulares de VI}
\end{table}

Tambien   podemos   cambiar  el   orden   de   las  frases   de   todo
un   texto    o   de   zonas   específicas,    utilizando   los   {\em
agrupadores}\index{VIM!expresiones regulares!agrupadores}. Se trata de
crear  grupos de  expresiones  que luego  para sustituir  utilizaremos
expresiones  que  nos permitan  reordenarlos.  Para  definir un  grupo
se  utiliza \verb.\(\).,  el  texto  que se  encuentre  entre esos  el
paréntesis ``escapados''  será la expresion que  buscará. Por ejemplo,
estamos usando  un archivo con  varios nombres de ciudades,  pero para
poder utilizarlo necesitamos que todos esten al principio de la línea,
sin ningun espacio, podemos utilizar el siguiente comando:

\begin{verbatim}
:%s/^\(\s\+\)\(\w\+\)/\2/g
\end{verbatim}

El anterior comando busca en todo el documento (\verb.%.) que contenga
cualquier número espacios al principio de la línea (\verb.\(\s\+\).) y
que detrás contenga algún texto(\verb.\(\w\+\)). y lo sustituye por la
exresión(\verb.\2.), es  decir, solo  mantiene el grupo  \verb.\2., el
que contiene la palabra.

