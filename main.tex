%Autor: miguev

% Documento de clase 'book', papel A4 a doble cara con letra de 12 puntos.
\documentclass[a4paper,12pt,twoside,final]{book}

% Paquetes
% ========

% Paquete para poder escribir los caracteres especiales sin necesidad de
% comandos. La opci�n 'latin1' se corresponde con el standar ISO-8859-1
% (tambi�n es v�lido para ISO-8859-15, aunque no reconoce el �)
\usepackage[latin1]{inputenc}

% Paquete para castellanizar las variables del documento.
\usepackage[spanish]{babel}

% Paquete de s�mbolos de la AMS (American Mathematical Society)
\usepackage{amssymb}

% Paquetes para inclusi�n de gr�ficos Encapsulados PostScript
\usepackage{graphicx}

% Paquete para el s�mbolo oficial del �uro
\usepackage{eurosym}

% Paquete para construir los �ndices con makeindex.
\usepackage{makeidx}

% Paquete para hiperenlaces HTML
\usepackage{html}

% Paquete para filas m�ltiples (para tablas)
\usepackage{multirow}

% Paquete para vieguer�as con verbatim
\usepackage{verbatim}

% Estilo para los apuntes del CILA, 
% tomando c�digo de los paquetes fncychap y ull.
% Este paquete debe ser el �ltimo en ser incluido, 
% ya que sobreescribe definiciones de otros paquetes
% (p.ej. babel)
\usepackage{cila-html}

\frenchspacing

%%%
%%% Comando para crear el nombre de LyX tal como lo hace el propio LyX
%%% esta l�nea (con \providedcommand) es incluida por LyX al exportar a LaTeX
%%%
\newcommand{\LyX}{L\kern-.1667em\lower.25em\hbox{Y}\kern-.125emX\@}

%%%
%%% Comandos para siglas
%%%
\newcommand{\FMAT}{{\sc Facultad de Matem�ticas}}
\newcommand{\ULL}{{\sc Universidad de La Laguna}}
\newcommand{\GULiC}{{\sc Grupo de Usuarios de Linux de Canarias}}
\newcommand{\CILA}{Curso de Introduci�n a Linux para Alumnos}
\newcommand{\PILA}{Party de Instalaci�n de Linux para Alumnos}
\newcommand{\TIL}{Taller de Iniciaci�n a Linux}
\newcommand{\TILA}{Taller de Iniciaci�n a Linux para Alumnos}

%%%
%%% Anula la conversi�n a may�sculas
%%%
\renewcommand{\MakeUppercase}{}

%
% COMIENZO DEL DOCUMENTO
\begin{document}

% Modificaci�n de algunas variables para adaptarlas a nuestro uso particular
\renewcommand{\partname}{M�DULO}
\renewcommand{\chaptername}{TEMA}
\renewcommand{\listtablename}{�ndice de tablas}
\renewcommand{\tablename}{Tabla}

% �ndices general, de tablas, de figuras y de ejemplos
\tableofcontents\newpage\thispagestyle{empty}
\listoffigures\newpage\thispagestyle{empty}
\listoftables\newpage\thispagestyle{empty}
\listadeejemplos\newpage\thispagestyle{empty}

% Parte principal
\mainmatter

% Inclusi�n de los ficheros en el orden apropiado.
%\incluye{administracion}

% M�dulos
% I.   Entorno GNU/Linux (3 d�as, 15 horas en clases de 5 horas)
% II.  Party (2 d�as, 20 horas en jornadas de 10 horas)
% III. Documentaci�n (5 d�as, 20 horas, en clases de 4 horas)
% IV.  Matem�ticas (3 d�as, 12 horas, en clases de 4 horas)
% V.   Programaci�n (5 d�as, 20 horas, en clases de 4 horas)
% VI.  Programaci�n avanzada (?? horas)

% M�dulo I:   Entorno GNU/Linux
% 1. Historia de GNU/Linux
% 2. El entorno X-Window (Gnome, KDE)
% 3. El entorno del int�rprete de comandos (con mtools)
% 4. Usando internet (mozilla, konqueror, kmail, gftp, mutt, SSH)
% 5. Aplicaciones (gnotepad, abiword, openoffice)
% 6. Documentaci�n y sistemas de ayuda
\part{Entorno GNU/Linux}
\incluye{introduccion}
\incluye{comandos}
\incluye{xwindow}
\incluye{internet}
\incluye{aplicaciones}
%\incluye{documentacion}
\incluye{administracion}

% M�dulo II:  Party
% 1. Sistema base (particiones)
% 2. Hardware y kernel (make [menu|x]config && make-kpkg)
% 3. Software para los m�dulos.
% 4. Administraci�n b�sica (adduser, floppy, cdrom, APT)
% 5. Paquetes en fuentes (bajar, compilar, instalar Xine+dvdnav)
% 6. Software adicional (a gusto del consumidor)
\part{Instalaci�n de GNU/Linux}
\incluye{instalacion}

% M�dulo III: Edici�n y maquetaci�n de documentos
% 1. Edici�n de gr�ficos (DIA, QCad, The GIMP)
% 2. OpenOffice
% 3. HTML (lenguaje, bluefish, quanta)
% 4. LyX
% 5. LaTeX
\part{Edici�n de gr�ficos y documentos}
\incluye{graficos}
%\incluye{openoffice}
\incluye{html}
\incluye{lyx}
\incluye{latex}

% M�dulo IV:  Matem�ticas
% 1. GNUplot
% 2. Octave (SciLab)
% 3. R
% 4. Yacas
% 5. Maxima
\part{Herramientas matem�ticas}
\incluye{octave}
\incluye{gnuplot}
\incluye{r}
\incluye{yacas}

% M�dulo V:   Programaci�n
% 1. Editores de texto (gvim, xemacs)
% 2. Bash
% 3. GNU Fortran 77
% 4. FreePascal
% 5. GNU C/C++
% 6. Java
% 7. GNU Make
% 8. Depuradores (gdb y ddd)
%\thispagestyle{empty}
\part{Programaci�n}
\incluye{editores}
\incluye{freepascal}
\incluye{gnufortran}
\incluye{gnuc}
\incluye{gnumake}
\incluye{depuradores}
\incluye{java}
\incluye{bash}

%
% NO SE ENTRETENGAN CON ESTO TODAV�A
%
% M�dulo VI:  Programaci�n avanzada
% 1. PHP + [My|Postgre]SQL
% 2. Perl
% 3. Python (NumPy) 
% 4. GUIs toolkits: Tk, QT, Wx, Gtk, Glade
% 5. XML
%\part{Programaci�n avanzada}
%\incluye{php}
%\incluye{perl}
%\incluye{python}
%\incluye{toolkits}
%\incluye{xml}
%\indluye{ides}


% Ap�ndices y referencias
\appendix
\incluye{recursos}
\incluye{fdles}




% Parte posterior
\backmatter

% Bibliograf�a, generado por BIBTeX
\bibliographystyle{plain}
\bibliography{biblio}

% �ndice terminol�gico
\printindex


\end{document}
% FIN DEL DOCUMENTO
