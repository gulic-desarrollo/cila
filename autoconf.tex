%Autor: Carlos Paramio

\chapter{GNU Autoconf}\label{autoconf}

No hay  que olvidar que,  desde que  la proliferación de  los sistemas
UNIX ha ido en aumento, han surgido diferentes implementaciones; antes
de que  se estableciese el estándar  {\em POSIX}, esto dio  lugar a un
tremendo  caos,  que obligaba  a  los  desarrolladores interesados  en
hacer  su software  multiplataforma  a usar  diversos artificios,  que
permitiesen la migración  de su programa a otros sabores  de UNIX. Aún
cuando {\em POSIX}  se dio a conocer, los  diversos sistemas adoptaron
dicho estándar en distintas fechas,  y aún existen ligeras diferencias
que  serían  una  traba  para nuestro  propósito.  Fueron  necesarias,
entonces, el uso de ciertas herramientas que facilitaran la adaptación
de nuestro  código a  estas posibles diferencias,  y además  que dicha
tarea fuese  lo más automatizada  posible, para que el  futuro usuario
pudiese  compilar el  paquete de  software sin  necesidad de  realizar
complicados pasos.  Aunque existen muchos programas  que cumplen estas
especificaciones,  son de  destacar los  elaborados desde  el proyecto
GNU: {\tt autoconf}, {\tt automake} y {\tt libtool}.

\section{Introducción a autoconf}

La herramienta GNU autoconf, con el tiempo, se ha convertido en una de
las  más  útiles  y  necesarias  a la  hora  de  desarrollar  software
multiplataforma. Las diferencias entre diferentes arquitecturas obliga
al programador a conocerlas con  detalle. Con {\tt autoconf}, podremos
generar  un script  que nos  evite la  mayor parte  del trabajo.  Este
script, llamado {\tt configure}, generará unos ficheros {\tt makefile}
modificados para adaptarse a nuestro sistema.

El  modo de  funcionamiento de  {\tt autoconf}  se basa  en el  uso de
macros  m4 para  resumir  las operaciones  de  comprobación que  deben
hacerse. Estas macros comienzan todas con  el prefijo {\tt AC\_} y son
sustituidas por una porción de código  en el lenguaje de script de una
shell tipo Bourne.

Las  macros  se  definen  en  un  fichero  {\tt  configure.in},  y  al
ser  procesado  este  fichero  con  {\tt  autoconf},  se  generará  el
correspondiente   script  {\tt   configure}.  Sólo   dos  macros   son
imprescindibles para nuestro programa:

\begin{verbatim}
AC_INIT(fichero_unico_fuentes)
\end{verbatim}

La llamada a  {\tt AC\_INIT} comprueba que existe  el fichero indicado
para  asegurarse de  que se  encuentra en  el directorio  correcto. En
nuestro ejemplo, podríamos  usar {\tt main.c}. Se  utiliza al comienzo
del fichero de configuración.

La  otra llamada  se sitúa  al final  del fichero,  e indica  dónde se
recogeran los resultados  de la ejecución del  script. Normalmente, el
destino es el fichero {makefile}:

\begin{verbatim}
AC_OUTPUT(fichero_salida)
\end{verbatim}

Una  vez que  nuestro  script  esté listo,  tras  procesar el  fichero
{\tt  configure.in}   con  {\tt  autoconf},  éste   necesitará  de  un
fichero adicional  {\tt makefile.in}  que contenga nuestras  reglas de
compilación. Sólo que en este  fichero, usaremos expresiones como {\tt
CC=@CC@} que serán sustituídas por el  script con el valor adecuado al
generar  el  {\tt  makefile}  final. También  podemos  hacer  exportar
ciertas  definiciones a  un  fichero {\tt  config.h}, que  incluiremos
en  nuestros fuentes,  sin  más que  definir  el correspondiente  {\tt
config.h.in} y usar la  macro {\tt AC\_CONFIG\_HEADER} para establecer
dicho fichero  de cabecera.  Para facilitar la  tarea de  escritura de
la  cabecera  {\tt config.h.in},  tenemos  a  nuestra disposición  una
herramienta que viene con autoconf,  llamada {\tt autoheader}. Una vez
que nuestro {\tt  configure.in} esté listo, la usaremos  para crear la
cabecera de entrada.

En  {\tt  configure.in}  podemos introducir  comentarios,  simplemente
precediéndolos del modificador {\tt dnd} al principio de la línea.

Las macros  que hay  disponibles son  muy numerosas,  y como  dicta la
intuición, se  pueden crear  macros nuevas.  Para este  particular, es
recomendable leer la guía en línea del lenguaje de macros M4, con {\tt
info m4}.

Algunas macros de ejemplo son las siguientes:

\begin{description}

\item{AC\_PROG\_CC} :  Decide qué  compilador de  C usar,  seteando la
variable CC.

\item{AC\_CHECK\_PROG (prog)} : Comprueba que el programa existe en el
path actual.

\item{AC\_CHECK\_LIB (lib,  fun, action1,  action2} : Determina  si la
función fun existe  en la librería lib intentando  enlazar un programa
con lib. Ejecuta action1 si el test resultó ser correcto, o action2 en
caso contrario.

\item{AC\_CHECK\_HEADER  (head)}  :  Determina  la  existencia  de  la
cabecera head.

\item{AC\_C\_BIGENDIAN} :  Si los  words (palabra del  procesador) son
guardados  con  los  bits  de  mayor  peso  primero,  entonces  define
WORDS\_BIGENDIAN.

\item{AC\_C\_INLINE} : Si el compilador  no soporta las palabras clave
inline, \_\_inline\_\_  o \_\_inline, entonces define  inline como una
cadena vacía.

\item{AC\_CONFIG\_HEADER}  :  Fichero  de   cabecera  donde  irán  las
definiciones pertinentes para el código fuente de nuestro programa.

\end{description}

Como  siempre, una  de  las mejores  referencias  de documentación  se
encuentra en el comando {\tt info  autoconf}. %También en este caso, se
%elaboró  un  manual  técnico  \cite{autoconf}  que  es  perfecto  como
%material complementario.
