%\incluye{administracion}

% Módulos
% I.   Entorno GNU/Linux (3 días, 15 horas en clases de 5 horas)
% II.  Party (2 días, 20 horas en jornadas de 10 horas)
% III. Documentación (5 días, 20 horas, en clases de 4 horas)
% IV.  Matemáticas (3 días, 12 horas, en clases de 4 horas)
% V.   Programación (5 días, 20 horas, en clases de 4 horas)
% VI.  Programación avanzada (?? horas)

% Módulo I:   Entorno GNU/Linux
% 1. Historia de GNU/Linux
% 2. El entorno X-Window (Gnome, KDE)
% 3. El entorno del intérprete de comandos (con mtools)
% 4. Usando internet (mozilla, konqueror, kmail, gftp, mutt, SSH)
% 5. Aplicaciones (gnotepad, abiword, openoffice)
% 6. Documentación y sistemas de ayuda
\part{Entorno GNU/Linux}
\incluye{introduccion}
\incluye{comandos}
\incluye{xwindow}
\incluye{editores}
\incluye{internet}
%= \incluye{aplicaciones}
\incluye{documentacion}

% Módulo II:  Party
% 1. Sistema base (particiones)
% 2. Hardware y kernel (make [menu|x]config && make-kpkg)
% 3. Software para los módulos.
% 4. Administración básica (adduser, floppy, cdrom, APT)
% 5. Paquetes en fuentes (bajar, compilar, instalar Xine+dvdnav)
% 6. Software adicional (a gusto del consumidor)
\part{Instalación de GNU/Linux}
\incluye{instalacion}
% -- MUY BUGGY -- \incluye{configuracion}
\incluye{administracion}
\incluye{bash}

% Módulo III: Edición y maquetación de documentos
% 1. Edición de gráficos (DIA, QCad, The GIMP)
% 2. OpenOffice
% 3. HTML (lenguaje, bluefish, quanta)
% 4. LyX
% 5. LaTeX
\part{Edición de gráficos y documentos}
%= %= \incluye{graficos}
%== \incluye{gimp}
\incluye{staroffice}
%= \incluye{openoffice}
\incluye{html}
\incluye{lyx}
\incluye{latex}

% Módulo IV:  Matemáticas
% 1. GNUplot
% 2. Octave (SciLab)
% 3. R
% 4. Yacas
% 5. Maxima
\part{Herramientas matemáticas}
\incluye{octave}
\incluye{gnuplot}
\incluye{r}
\incluye{yacas}

% Módulo V:   Programación
% 1. Editores de texto (gvim, xemacs)
% 2. Bash
% 3. GNU Fortran 77
% 4. FreePascal
% 5. GNU C/C++
% 6. Java
% 7. GNU Make
% 8. Depuradores (gdb y ddd)
\part{Programación}
\incluye{freepascal}
\incluye{gnufortran}
\incluye{gnuc}
\incluye{gnumake}
\incluye{depuradores}
%= \incluye{autoconf}
\incluye{gprof}
\incluye{java}
\incluye{regex}

%
% NO SE ENTRETENGAN CON ESTO TODAVÍA
%
% Módulo VI:  Programación avanzada
% 1. PHP + [My|Postgre]SQL
% 2. Perl
% 3. Python (NumPy) 
% 4. GUIs toolkits: Tk, QT, Wx, Gtk, Glade
% 5. XML
%= %= \part{Programación avanzada}
%= \incluye{php}
%= %= \incluye{perl}
%= %= \incluye{python}
%= %= \incluye{toolkits}
%= %= \incluye{xml}
%= %= \incluye{ides}


% Apéndices y referencias
\appendix
\incluye{recursos}
\incluye{fdles}
