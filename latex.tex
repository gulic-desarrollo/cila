%Autor: tbd & miguev (\section{Herramientas})
%tbd: 50
%miguev: 1

\chapter{\LaTeX}
\label{latex.tex}

\index{\LaTeX}

\section{Introducción preliminar}

\LaTeX~\cite{llmanual}  es   un  sistema  de  composición   de  textos
especialmente orientado  a la  creación de documentos  científicos que
contengan fórmulas matemáticas. Además,  también se pueden crear otros
tipos  de documentos,  que  pueden ser  desde  cartas sencillas  hasta
libros completos. \LaTeX\ está organizado sobre \TeX~\cite{texbook}.

Este tema  está dedicado a \LaTeX\  y pretende versar sobre  lo básico
que se debe poder emplear en la mayoría de las aplicaciones de \LaTeX.
Existen diversos manuales~\cite{llmanual,companion} donde se encuentra
una descripción completa de \LaTeX.

\LaTeX\  está  disponible  para  la  mayoría  de  los  miniordenadores
y  microordenadores,  desde  IBM~PCs  en  adelante.  En  muchas  redes
universitarias de  ordenadores se encuentra instalado  para utilizarse
al instante. Normalmente existe una \guialatex\ donde podrás averiguar
cómo acceder a la  instalación de \LaTeX, cómo se opera  con ella y de
qué complementos se dispone.

El propósito  de este tema \emph{no}  es indicar cómo se  instala y se
mantiene un sistema  de \LaTeX, sino mostrar  cómo escribir documentos
para poderlos procesar con \LaTeX.

\bigskip
\noindent Esta descripción se divide en cuatro apartados:
\begin{description}

\item[El  primer  apartado]  muestra   la  estructura  básica  de  los
documentos de  \LaTeXe. También se  enseña un  poco de la  historia de
\LaTeX. Tras leer  este tema deberías tener una visión  muy escueta de
\LaTeX. Esta visión consistirá sólo de un pequeño ``marco de trabajo''
en el que  podrá integrar la información que se  proporcionamos en los
demás apartados  y la que podrás  hallar en otras fuentes  ---como los
manuales~\cite{llmanual,companion}---.

\item[El segundo apartado] incide en los detalles sobre la composición
de los  documentos. En este  aparatdos se  explican la mayoría  de las
instrucciones y  los entornos  básicos de \LaTeX.  Una vez  leído este
apartado serás capaz de escribir tus primeros documentos.

\item[En  el  tercer  apartado]  se explican  algunas  nociones  sobre
cómo  componer fórmulas  matemáticas con  \LaTeX. Aquí  te presentamos
varios  ejemplos  para ayudarte  a  entender  uno de  los  principales
potenciales de \LaTeX.

\item[El  cuarto apartado]  indica otras  posibilidades que  se pueden
obtener de  \LaTeX{} que, si  bien no  son esenciales, a  veces pueden
resultar muy útiles. Por ejemplo,  se muestra cómo incluir gráficos de
PostScript Encapsulado (EPS) en sus documentos o cómo añadir un índice
de materias en su documento.

\item[El último  apartado] resume  las herramientas básicas  que debes
saber manejar para trabajar con documentos \LaTeX.

\end{description}

\noindent  Se  recomienda  encarecidamente leer  estos  apartados  por
orden. También  sería altamente recomendable estudiarse  los ejemplos,
ya que  en ellos es  donde se encuentra  gran parte de  la información
útil.

\noindent Si  por alguna  razón hay  necesidad de  conseguir cualquier
material  relacionado   con  \LaTeX,  siempre  se   puede  recurrir  a
cualquiera de  los servidores  de archivos  de \texttt{CTAN}\@.  En la
República Federal de  Alemania es \texttt{ftp.dante.de} y  en el Reino
Unido  es  \texttt{ftp.tex.ac.uk}.  También existen  otros  servidores
replicados. En caso de no encontrarse  en uno de estos países, siempre
es mejor elegir el servidor más cercano.

\section{Lo que es fundamental saber}

\begin{intro} En  la primera  parte de  este apartado  presentamos una
visión general de  la filosofía e historia de  \textrm{\LaTeXe}. En la
segunda parte se incide en las  estructuras básicas de un documento de
\LaTeX. Tras leer  este tema, podrás tener un  conocimiento básico del
modo  de  funcionamiento  de  \LaTeX.  A  medida  que  se  avanza,  la
información de  este apartado resultará  de mucha ayuda  para integrar
toda la información  adicional que puedas obtener  sobre \LaTeX, tanto
en el resto de los apartados como de otros sitios. \end{intro}

\subsection{El nombre del juego}

\paragraph{\TeX}

\TeX\ es  un programa de ordenador  de Donald E.~Knuth~\cite{texbook}.
Está  orientado  a  la  composición  e  impresión  textos  y  fórmulas
matemáticas.

\TeX{}  se pronuncia  ``Tech'',  con  una ``ch''  como  en la  palabra
alemana ``Buch'' o  en la escocesa ``Loch''. Éste es  el sonido de una
`h'\ aspirada, como en la onomatopeya española ``argh''. En un entorno
\texttt{ASCII} \TeX{} se escribe \texttt{TeX}.

\paragraph{\LaTeX}

\LaTeX\ es  un paquete de macros  que nos permite componer  e imprimir
más  fácilmente  un documento  con  la  calidad tipográfica  de  \TeX,
empleando  para ello  patrones definidos.  Originalmente, \LaTeX{}  lo
escribió \index{\LaTeX!Lamport, Leslie}Leslie Lamport~\cite{llmanual}.
A grosso  modo, podemos decir  que se  utiliza el cajista  \TeX{} como
elemento de composición.

Desde diciembre de  1994, el paquete \LaTeX{}  está siendo actualizado
por  un equipo  llamado  \index{\LaTeX!LaTeX3@\LaTeX  3}\LaTeX 3,  que
dirige \index{\LaTeX!Mittelbach, Frank}Frank  Mittelbach, para incluir
algunas de  las mejoras que se  habían solicitado desde hace  tiempo y
para reunificar  todas las versiones  retocadas que han  surgido desde
que  apareciera la  primera versión  \index{\LaTeX!LaTeX 2.09@\LaTeX{}
2.09}\LaTeX{}  2.09  hace  ya   algunos  años.  Para  distinguirla  de
la  vieja,  a  la  nueva   versión  se  le  llama  \index{\LaTeX!LaTeX
2e@\LaTeXe}\LaTeXe. Lo que aquí presentamos es \LaTeXe.

\LaTeX{} se pronuncia ``Lei-tegh'',  aunque entre los hispanohablantes
se ha aceptado  ``La-tegh''. Para referirnos a \LaTeX{}  en un entorno
\texttt{ASCII}  escribiremos  \texttt{LaTeX}. \LaTeXe{}  se  pronuncia
``Lei-tegh tu  íi'' ---aunque muchos  nos empeñamos en  decir ``Lategh
dos e''--- y se puede escribir \texttt{LaTeX2e}.

\paragraph{Conceptos básicos}
 
\paragraph{Autor, diseñador y cajista.}

Normalmente, para una  publicación el autor le entrega  a la editorial
un escrito  a máquina. El diseñador  de libros de la  editorial decide
entonces el formato del documento  (longitud de los renglones, tipo de
letra, espacios antes y después de cada capítulo, etc.)\ y le da estas
instrucciones al cajista para producir este formato.

La persona que  diseña estos libros intenta  averiguar las intenciones
del autor mientras  ha realizado el escrito. Entonces  decide sobre el
modo de presentar los títulos de capítulos, citas, ejemplos, fórmulas,
etc.,\  basándose  en su  saber  profesional  y  en el  contenido  del
escrito.

En un  entorno de \LaTeX, \LaTeX{}  realiza el papel del  diseñador de
libros y emplea a \TeX{} como cajista. Pero \LaTeX{} \emph{sólo} es un
programa  y, por  tanto, necesita  más ayuda  para sus  decisiones que
un  diseñador humano  de  libros. El  autor  tiene que  proporcionarle
información  adicional que  describa la  estructura lógica  del texto.
Esta  información  se  indica  dentro   del  texto  a  través  de  las
\emph{instrucciones} u \emph{órdenes} de \LaTeX.

Esto  es bastante  diferente del  enfoque \wi{WYSIWYG}\footnote{Siglas
que significan \emph{What you see is what you get,} lo que se ve es lo
que  se  obtiene.}  de  la  mayoría  de  los  procesadores  de  textos
tales  como  \emph{Microsoft  Word}  o  \emph{FrameMaker}.  Con  estas
aplicaciones, el autor  establece el formato del texto  con la entrada
interactiva al introducirlo en el ordenador. En cada momento, el autor
ve  en pantalla  el  aspecto que  tendrá el  trabajo  final cuando  lo
imprima.

Por regla general, al emplear \LaTeX{}  el autor no ve, al redactar el
texto, cómo va  a resultar la composición final.  Sin embargo, existen
herramientas que permiten mostrar en  pantalla lo que finalmente se va
a  obtener tras  procesar los  ficheros con  \LaTeX. Con  éllas puedes
siempre realizar correcciones antes de  enviar el documento final a la
impresora.

\paragraph{Diseño del formato.}

El  diseño tipográfico  es una  artesanía  que se  debe aprender.  Los
autores inexpertos  cometen con  frecuencia graves errores  de diseño.
Muchos profanos creen erróneamente que  el diseño tipográfico es, ante
todo,  una cuestión  de estética:  si  el documento  presenta un  buen
aspecto desde  el punto  de vista artístico,  entonces creen  que está
bien ``diseñado''. Sin embargo, ya que  los documentos se van a leer y
no  a colgarse  en un  museo, es  más importante  presentar una  mayor
legibilidad y una mejor comprensión que un aspecto más agradable.

Por  ejemplo, algunas  reglas básicas  que deberían  seguirse son  las
siguientes:

\begin{itemize}

\item Se debe  elegir el tamaño de  las letras y la  numeración de los
títulos de modo que la estructura de los capítulos y las secciones sea
fácilmente reconocible.

\item Se debe elegir la longitud de los renglones de modo que se evite
el cansancio por  el movimiento de los  ojos del lector y  no para que
rellenen,  a ser  posible, las  páginas con  un aspecto  estéticamente
bueno.

\end{itemize}

Con  los  sistemas  \wi{WYSIWYG}  los autores  producen,  en  general,
documentos estéticamente bonitos pero con  una estructura muy escasa o
inconsistente. \LaTeX{}  evita estos  errores de  formato, ya  que con
\LaTeX{} el autor está obligado  a indicar la estructura \emph{lógica}
del texto. Entonces \LaTeX{} elige el formato más apropiado para éste.

\paragraph{Ventajas e inconvenientes.}

Una  cuestión que  se  discute  a menudo  cuando  la  gente del  mundo
\mbox{\wi{WYSIWYG}} se encuentra con la  gente que utiliza \LaTeX{} es
sobre  ``las \wi{ventajas  de \LaTeX}  sobre un  procesador de  textos
normal'' o al revés. Cuando uno se ve ante una discusión como ésta, lo
mejor que se  puede hacer es mantener una postura  de asentimiento, ya
que las  cosas suelen  salirse de  control. Pero a  veces no  se puede
huir\ldots

\medskip\noindent En estas circunstancias  hay que recordar algunas de
las principales  ventajas de \LaTeX\  sobre los procesadores  de texto
normales, que son las siguientes:

\begin{itemize}

\item  Existe  mayor cantidad  de  diseños  de texto  profesionales  a
disposición, con los que realmente  se pueden crear documentos como si
fueran ``de imprenta''.

\item Se facilita la composición de fórmulas con un cuidado especial.

\item El  usuario sólo necesita introducir  instrucciones sencillas de
entender con las que se indica la estructura del documento. Casi nunca
hace falta  preocuparse por los  detalles de creación con  técnicas de
impresión.

\item También  las estructuras complejas  como notas a pie  de página,
bibliografía, índices,  tablas y muchas  otras se pueden  producir sin
gran esfuerzo.

\item Existen paquetes adicionales sin coste alguno para muchas tareas
tipográficas que no se facilitan  directamente por el \LaTeX{} básico.
Por  ejemplo,  existen  paquetes  para  incluir  gráficos  en  formato
\mbox{\textsc{PostScript}}  o  para  componer  bibliografías  conforme
a  determinadas  normas. Muchos  de  estos  paquetes se  describen  en
\companion.

\item \LaTeX{}  hace que  los autores tiendan  a escribir  textos bien
estructurados porque así  es como trabaja \LaTeX, o  sea, indicando su
estructura.

\item  \TeX,  la  máquina  de composición  de  \LaTeXe,  es  altamente
transportable y gratis. Por esto, el sistema funciona prácticamente en
cualquier en cualquier plataforma.

\end{itemize}

\LaTeX\ tiene, naturalmente, también inconvenientes:

\begin{itemize}

\item  Para hacer  funcionar un  sistema de  \LaTeX, se  necesitan más
recursos (memoria,  espacio de  disco y  potencia de  procesamiento, y
espacio de  almacenamiento) que  para un  procesador de  texto simple.
Aunque  no  es  menos  cierto  que  las  cosas  van  siendo  cada  vez
mejores, y  las distintas  versiones que cada  vez van  apareciendo de
\emph{Microsoft Word}  necesita cada vez  más espacio de disco  que un
sistema de \LaTeX{}  normal. Cuando analizamos el  uso del procesador,
podemos  ver que  \LaTeX{}  supera en  prestaciones cualquier  sistema
\wi{WYSIWYG} ya  que necesita  mucha cantidad  de CPU  pero únicamente
cuando el documento se procesa, mientras que los paquetes \wi{WYSIWYG}
tienen ocupada la CPU continuamente.

\item Si  bien se pueden  ajustar algunos  parámetros de un  diseño de
documento predefinido,  la creación de  un diseño entero es  difícil y
lleva mucho tiempo\footnote{Los  rumores dicen que este es  uno de los
puntos claves  sobre el  que se  hará hincapié  en el  próximo sistema
LaTeX 3.}.\index{\LaTeX!LaTeX3@\LaTeX 3}

\end{itemize}

\subsection{Ficheros de entrada de \LaTeX}

La  entrada  para   \LaTeX{}  es  un  fichero  de   texto  en  formato
\texttt{ASCII}.  Se  puede  crear  con  cualquier  editor  de  textos.
Contiene   tanto   el   texto   que  se   debe   imprimir   como   las
``instrucciones'', con las  cuales se le puede indicar  a \LaTeX\ cómo
debe disponer el texto.

\paragraph{Signos de espacio.}

Los caracteres ``invisibles'', como el espacio en blanco, el tabulador
y  el  final  de  línea,  son tratados  por  \LaTeX\  como  signos  de
\wi{espacio}  propiamente dichos.  \emph{Varios} espacios  seguidos se
tratan  como \emph{un}  \wi{único  espacio  en blanco}.  Generalmente,
un  espacio  en  blanco  al  comienzo   de  una  línea  se  ignora,  y
\emph{varios} renglones en blanco se tratan como un renglón en blanco.
\index{\LaTeX!espacio en blanco!al comienzo de una línea}

Un renglón en blanco entre dos líneas  de texto definen el final de un
párrafo. \emph{Varias} líneas en blanco se tratan como \emph{una sola}
línea en blanco. El texto que  mostramos a continuación es un ejemplo.
A la  derecha se  encuentra el  texto del  fichero de  entrada y  a la
izquierda la salida formateada.

\begin{example}
No importa si introduces
varios        espacios tras
una palabra.

Con una línea vacía se
empieza un párrafo nuevo.
\end{example}
 
\paragraph{Caracteres especiales.}

Los símbolos  siguientes son \wi{caracteres reservados}  que tienen un
significado especial para \LaTeX{} o que no están disponibles en todos
los  tipos.  Si los  introduces  en  su  fichero directamente  es  muy
probable que no se impriman o que fuercen a \LaTeX{} a hacer cosas que
probablemente te parecerán extrañas.
%
\begin{code}
\verb.$ & % # _ { }  ~  ^  \ . 
\end{code}

Como podemos ver, estos caracteres se pueden incluir en sus documentos
anteponiendo   el  carácter   \verb|\|  (\emph{barra   invertida}):  %
\begin{example} \$ \& \% \# \_ \{ \} \end{example}

Los  restantes  símbolos  y  otros  muchos  caracteres  especiales  se
pueden  imprimir en  fórmulas matemáticas  o como  tildes con  órdenes
específicas.

\paragraph{Las órdenes de \LaTeX{}.}

En las órdenes\index{\LaTeX!ordenes@órdenes} de \LaTeX{} se distinguen
las letras mayúsculas y las minúsculas.  Toman uno de los dos formatos
siguientes:

\begin{itemize}

\item  Comienzan  con  una  \emph{barra  invertida}\index{\LaTeX!barra
invertida}  \verb|\|\cih{\bs} y  tienen un  nombre compuesto  sólo por
letras. Los  nombres de las órdenes  acaban con uno o  más espacios en
blanco, un carácter especial o una cifra.

\item Se compone de una \emph{barra invertida} y un carácter especial.

\end{itemize}

\LaTeX{} ignora  los espacios en blanco  que van tras las  órdenes. Si
deseas introducir \index{\LaTeX!espacio en blanco!tras instrucción} un
espacio en blanco tras una instrucción, se debe poner o bien \verb|{}|
y un espacio, o bien una instrucción de espaciado después de la orden.
Con \verb|{}|  se fuerza  a \LaTeX{}  a dejar de  ignorar el  resto de
espacios que se encuentren después de la instrucción.

\begin{example}
He leído que Knuth distingue a la
gente que trabaja con \TeX{} en
\TeX{}nicos y \TeX pertos.\\
Hoy es \today.
\end{example}

Algunas instrucciones  necesitan un  \wi{parámetro} que se  debe poner
entre \wi{llaves} \verb|{ }| tras la instrucción. Otras órdenes pueden
llevar \wi{parámetros opcionales} que se añaden a la instrucción entre
\wi{corchetes}~\verb|[  ]|  o no.  El  siguiente  ejemplo usa  algunas
órdenes de \LaTeX{} que te explicaremos más adelante.

\begin{example}
¡Te puedes \textsl{apoyar} en mí!
\end{example}
\begin{example}
¡Por favor, comienza una nueva
línea justamente aquí!%
\linebreak[3] Gracias.
\end{example}

\paragraph{Comentarios.}
\index{\LaTeX!comentarios}

Cuando  \LaTeX{} encuentra  un carácter  \verb|%| mientras  procesa un
fichero de entrada,  ignora el resto de la línea.  Esto suele ser útil
para introducir notas en el fichero  de entrada que no se mostrarán en
la versión impresa.

\begin{example}
Esto es un % tonto
% Mejor: instructivo <----
ejemplo.
\end{example}

Esto a  veces puede resultar  útil cuando no quieras  líneas demasiado
largas en  el fichero  fuente. Si no  quisieras introducir  un espacio
entre dos palabras, y prefieres tener dos renglones, entonces el signo
\verb|%| debe  ir justo  al final  del renglón  pero pegado  al último
carácter. De este  modo se comenta el carácter de  ``salto de línea'',
que de otro modo se hubiese tratado como un espacio en blanco.

\begin{example}
Este es otro ejem% y
% ahora el resto
plo.
\end{example}

\subsection{Estructura de un fichero de entrada}

Cuando \LaTeXe{}  procesa un  fichero de entrada,  espera de  éste que
siga  una determinada  \wi{estructura}. Todo  fichero de  entrada debe
comenzar con la orden

\begin{code}
\verb|\documentclass{...}|
\end{code}

Esto indica qué tipo de documento se desea crear. Tras esto, se pueden
incluir órdenes que  influirán sobre el estilo del  documento entero o
puedes cargar \wi{paquete}s  con las que añadir  nuevas propiedades al
sistema de \LaTeX. Para cargar uno de estos paquetes puedes emplear la
instrucción \begin{code} \verb|\usepackage{...}| \end{code}

Cuando     todo      el     trabajo     de      configuración     esté
realizado\footnote{El   área   entre  \texttt{\bs   documentclass}   y
\texttt{\bs    begin$\mathtt{\{}$document$\mathtt{\}}$}    se    llama
\emph{\wi{preámbulo}}.} entonces  comienza el cuerpo del  texto con la
instrucción

\begin{code}
\verb|\begin{document}|
\end{code}

A  partir de  entonces puedes  introducir  el texto  mezclado con  las
instrucciones que  consideremos necesarios de \LaTeX.  Al finalizar el
documento debes poner la orden

\begin{code}
\verb|\end{document}|
\end{code}

LaTeX{} ignorará cualquier cosa que pongas tras esta instrucción.

La  figura~\ref{mini} muestra  el contenido  mínimo de  un fichero  de
\LaTeXe.  En  la figura~\ref{document}  se  expone  un \wi{fichero  de
entrada} algo más complejo.

\begin{figure}[!bp]
\begin{lined}{6cm}
\begin{verbatim}
\documentclass{article}
\begin{document}
Lo pequeño es bello.
\end{document}
\end{verbatim}
\end{lined}
\caption{Un fichero mínimo de \LaTeX} \label{mini}
\end{figure}
 
\begin{figure}[!bp]
\begin{lined}{10cm}
\begin{verbatim}
\documentclass[a4paper,11pt]{article}
\usepackage{latexsym}
\usepackage[activeacute,spanish]{babel}
\author{T.~Bautista}
\title{Minimizando}
\frenchspacing
\begin{document}
\maketitle
\tableofcontents
\subsection{Inicio}
Bien\ldots{} y aquí comienza mi artículo tan
estupendo.
\subsection{Fin}
\ldots{} y aquí acaba.
\end{document}
\end{verbatim}
\end{lined}
\caption{Ejemplo para un artículo científico (en) español.} \label{document}
\end{figure}
 
\subsection{El formato del documento}
 
\paragraph{Clases de documentos.}\label{sec:documentclass}

Cuando procesa  un fichero de  entrada, lo primero que  necesita saber
\LaTeX{} es el tipo de documento  que quieres crear. Esto se lo puedes
indicar con la instrucción \ci{documentclass}.

\begin{command}
\ci{documentclass}\verb|[|\emph{opciones}\verb|]{|\emph{clase}\verb|}|
\end{command}

\noindent En  este caso, la  \emph{clase} indica el tipo  de documento
que  se  creará. En  la  tabla~\ref{documentclasses}  se muestran  las
clases  de  documento  que   explicaremos  aquí.  La  distribución  de
\LaTeXe{} proporciona más clases para  otros documentos, como cartas y
transparencias.  El parámetro  de \emph{\wi{opciones}}  personaliza el
comportamiento de la clase de documento elegida. Las opciones se deben
separar con comas.  En la tabla~\ref{options} se  indican las opciones
más comunes para las clases de documento estándares.

\begin{table}[!bp]
\caption{Clases de documentos} \label{documentclasses}
\begin{lined}{12cm}
\begin{description}

\item   [\normalfont\texttt{article}]  para   artículos  de   revistas
especializadas,  ponencias,   trabajos  de  prácticas   de  formación,
trabajos    de    seminarios,    informes    pequeños,    solicitudes,
dictámenes,    descripciones    de     programas,    invitaciones    y
muchos   otros.\index{\LaTeX!articulo@artículo}%   
\index{\LaTeX!clase\texttt{article}@clase article}

\item   [\normalfont\texttt{report}]   para   informes   mayores   que
constan    de    más    de    un   capítulo,    proyectos    fin    de
carrera,   tesis    doctorales,   libros    pequeños,   disertaciones,
guiones      y     similares.\index{\LaTeX!informe}\index{\LaTeX!clase
\texttt{report}@clase report}

\item       [\normalfont\texttt{book}]       para      libros       de
verdad\index{\LaTeX!libro}\index{\LaTeX!clase      \texttt{book}@clase
book}

\item    [\normalfont\texttt{slide}]    para   transparencias.    Esta
clase      emplea     tipos      grandes     \textsf{sans      serif}.
\index{\LaTeX!transparencias}\index{\LaTeX!clase  \texttt{slide}@clase
slide}

\end{description}

\end{lined}
\end{table}

\begin{table}[!bp]
\caption{Opciones de clases de documento} \label{options}
\begin{lined}{12cm}
\begin{flushleft}
\begin{description}

\item[\normalfont\texttt{10pt},  \texttt{11pt},  \texttt{12pt}]  \quad
Establecen  el  tamaño  (cuerpo) para  los  tipos\footnote{El  término
adecuado para designar lo que normalmente se conoce como \emph{fuente}
es el  de \emph{tipo}}. Si  no se  especifica ninguna opción,  se toma
\texttt{10pt}.\index{\LaTeX!tamaño de los tipos!del documento}

\item[\normalfont\texttt{a4paper},    \texttt{letterpaper},    \ldots]
\quad  Definen  el  tamaño  del  papel.  Si  no  se  indica  nada,  se
toma   \texttt{letterpaper}.  Aparte   de   éstos   se  puede   elegir
\texttt{a5paper},    \texttt{b5paper},    \texttt{executivepaper}    y
\texttt{legalpaper}.  \index{\LaTeX!papel legal}  \index{\LaTeX!tamaño
del    papel}\index{\LaTeX!papel   DIN-A4}    \index{\LaTeX!papel   de
carta}   \index{\LaTeX!papel   DIN-A5}\  \index{\LaTeX!papel   DIN-B5}
\index{\LaTeX!papel ejecutivo}

\item[\normalfont\texttt{fleqn}] \quad Con  esta opción las ecuaciones
se disponen hacia la izquierda en vez de centradas.

\item[\normalfont\texttt{leqno}]  \quad   Coloca  el  número   de  las
ecuaciones a la izquierda en vez de a la derecha.

\item[\normalfont\texttt{titlepage},    \texttt{notitlepage}]    \quad
Indica si  se debe comenzar  una página  nueva tras el  \wi{título del
documento} o no. Si no se  indica otra cosa, la clase \texttt{article}
no  comienza   una  página  nueva,  mientras   que  \texttt{report}  y
\texttt{book} sí.\index{\LaTeX!titlepage@\texttt{titlepage}}

\item[\normalfont\texttt{twocolumn}]  \quad  Le  dice a  \LaTeX{}  que
componga el documento en \wi{dos columnas}.

\item[\normalfont\texttt{twoside,  oneside}]  \quad Especifica  si  se
debe generar el documento a una o a dos caras. En caso de no indicarse
otra cosa,  las clases  \texttt{article} y  \texttt{report} son  a una
cara y la clase \texttt{book} es a dos.

\item[\normalfont\texttt{openright,  openany}]  \quad   Hace  que  los
capítulos comiencen  o bien sólo  en páginas a  la derecha, o  bien en
la  próxima  que  esté  disponible.  Esto no  funciona  con  la  clase
\texttt{article}, ya que  en esta clase no existen  capítulos. De modo
predeterminado, la clase \texttt{report}  comienza los capítulos en la
próxima página disponible  y la clase \texttt{book} las  inicia en las
páginas a la derecha.

\end{description}
\end{flushleft}
\end{lined}
\end{table}

Por ejemplo:  un fichero de entrada  para un documento de  \LaTeX{} se
podría comenzarlo con

\begin{code}
\ci{documentclass}\verb|[11pt,twoside,a4paper]{article}|
\end{code}

Con esto  se le indica  a \LaTeX{} que  componga el documento  como un
\emph{artículo}  utilizando tipos  del  cuerpo 11  ---en términos  más
coloquiales, con fuente de 11 puntos--- y que produzca un formato para
impresión a \emph{doble cara} en \emph{papel DIN-A4}.

\pagebreak[2]  \paragraph{Paquetes} \index{\LaTeX!paquete}  Mientras e
escribe el  documento, probablemente  nos encuentremos  en situaciones
donde el  \LaTeX{} básico no basta  para realizar lo que  buscamos. Si
deseamos incluir \wi{gráficos}, \wi{texto en color} o el código fuente
de un  fichero, se necesita  mejorar las capacidades de  \LaTeX. Tales
mejoras se  realizan con  ayuda de  los llamados  \emph{paquetes.} Los
paquetes se activan con la orden

\begin{command}
\ci{usepackage}\verb|[|\emph{opciones}\verb|]{|\emph{paquete}\verb|}|
\end{command}

\noindent   donde  \emph{paquete}   es   el  nombre   del  paquete   y
\emph{opciones} es una  lista de palabras clave  que activan funciones
especiales del paquete, a las que  \LaTeX{} les añade las opciones que
previamente hayamos podido indicar  en la orden \verb|\documentclass|.
Algunos paquetes vienen con la distribución básica de \LaTeXe{} (véase
la tabla~\ref{packages}).  Otros se  proporcionan por separado.  En la
\guia{}  (si existe)  se  puede encontrar  más  información sobre  los
paquetes disponibles en  la instalación a la que se  tenga acceso. Una
buena  fuente de  información sobre  \LaTeX{} es  \companion. Contiene
descripciones de cientos de paquetes,  así como información sobre cómo
escribir sus propias extensiones a \LaTeXe.

\begin{table}[!hbp]
\caption{Algunos paquetes distribuidos con \LaTeX} \label{packages}
\begin{lined}{11cm}
\begin{description}

\item[\normalfont\pai{doc}]  Permite la  documentación  de paquetes  y
otros  ficheros de  \LaTeX.\\  Se describe  en  \texttt{doc.dtx} y  en
\companion.

\item[\normalfont\pai{exscale}]   Proporciona    versiones   escaladas
de   los   tipos   adicionales   para   matemáticas.\\   Descrito   en
\texttt{ltexscale.dtx}.

\item[\normalfont\pai{fontenc}]  Especifica  qué  \wi{codificación  de
tipo} debe usar \LaTeX.\\ Descrito en \texttt{ltoutenc.dtx}.

\item[\normalfont\pai{ifthen}]   Proporciona   instrucciones   de   la
forma\\  `si\ldots{}  entonces\ldots{}   si  no\ldots'\\  Descrito  en
\texttt{ifthen.dtx} y en \companion.

\item[\normalfont\pai{latexsym}] Para  que \LaTeX{} acceda al  tipo de
símbolos,  se debe  usar el  paquete \texttt{latexsym}.\\  Descrito en
\texttt{latexsym.dtx} y en \companion.

\item[\normalfont\pai{makeidx}]    Proporciona   instrucciones    para
producir índices de materias.\\ Descrito en el \companion.

\item[\normalfont\pai{syntonly}]    Procesa     un    documento    sin
componerlo.\\ Se describe en \texttt{syntonly.dtx} y en \companion. Es
útil para la verificación rápida de errores.

\item[\normalfont\pai{inputenc}]  Permite  la  especificación  de  una
codificación de  entrada como ASCII  (con la opción  \pai{ascii}), ISO
Latin-1  (con la  opción  \pai{latin1}), ISO  Latin-2  (con la  opción
\pai{latin2}),  páginas de  código de  437/850 IBM  (con las  opciones
\pai{cp437} y  \pai{cp580}, respectivamente), Apple Macintosh  (con la
opción \pai{applemac}), Next (con  la opción \pai{next}), ANSI-Windows
(con  la opción  \pai{ansinew}) o  una definida  por el  usuario. Está
descrito en \texttt{inputenc.dtx}.

\end{description}
\end{lined}
\end{table}

\clearpage
%
% Puntero a información de los paquetes
%

\paragraph{Estilo de página}

Con \LaTeX{} existen tres combinaciones predefinidas de \wi{cabeceras}
y   \wi{pies  de   página},   a   las  que   se   llaman  estilos   de
página.\index{\LaTeX!estilo de  pagina@estilo de página}  El parámetro
\emph{estilo} de la instrucción

\index{\LaTeX!estilo de pagina@estilo de página!plain@\texttt{plain}}%
\index{\LaTeX!plain@\texttt{plain}}%
\index{\LaTeX!estilo de pagina@estilo de página!headings@\texttt{headings}}%
\index{\LaTeX!headings@\texttt{headings}}%
\index{\LaTeX!estilo de pagina@estilo de página!empty@\texttt{empty}}%
\index{\LaTeX!empty@\texttt{empty}}%

\begin{command}
\ci{pagestyle}\verb|{|\emph{estilo}\verb|}|
\end{command}

\noindent   define    cuál   queremos    que   \LaTeX\    emplee.   La
tabla~\ref{pagestyle} muestra los estilos de página predefinidos.

\begin{table}[!hbp]
\caption{Estilos de página predefinidos en \LaTeX} \label{pagestyle}
\begin{lined}{12cm}
\begin{description}

\item[\normalfont\texttt{plain}] imprime  los números de página  en el
centro del pie de las páginas. Este es el estilo de página que se toma
si no se indica otro.

\item[\normalfont\texttt{headings}]  en  la  cabecera de  cada  página
imprime el  capítulo que  se está  procesando y  el número  de página,
mientras que el pie está vacío.

\item[\normalfont\texttt{empty}] deja tanto la cabecera como el pie de
las páginas vacíos.

\end{description}
\end{lined}
\end{table}

Es posible  cambiar el  estilo de  página de la  página actual  con la
instrucción

\begin{command}
\ci{thispagestyle}\verb|{|\emph{estilo}\verb|}|
\end{command}

En \companion{} existe  una descripción de cómo  crear nuetras propias
cabeceras y pies de página\footnote{Uno de los paquetes más socorridos
para  este tipo  de cosas  es \texttt{fancyhdr}.  Lo más  recomendable
sería mirar  su documentación,  fundamentalmente la  que existe  en el
fichero \texttt{fancyhdr.dtx}.}.

%
% Puntero a la descripción del paquete fancyhdr
%
% Información sobre la numeración de páginas, ... \pagenumbering

\subsection{Proyectos grandes}

Cuando trabajes  con documentos  grandes, podrías, si  lo consideramos
oportuno, dividir  el fichero  de entrada  en varias  partes. \LaTeX{}
tiene dos instrucciones que te ayudan a realizar esto.

\begin{command}
\ci{include}\verb|{|\emph{fichero}\verb|}|
\end{command}

\noindent se puede utilizar en el cuerpo del documento para introducir
el contenido  de otro  fichero. En este  caso, \LaTeX{}  comenzará una
página nueva antes de procesar el texto del \emph{fichero}.

La segunda instrucción sólo se  puede emplear en el preámbulo. Permite
indicarle a \LaTeX{}  que sólo tome la entrada de  algunos ficheros de
los indicados con \verb|\include|.

\begin{command}
\ci{includeonly}\verb|{|\emph{fichero}\verb|,|\emph{fichero}%
\verb|,|\ldots\verb|}|
\end{command}

Una vez se  encuentre esta instrucción en el  preámbulo del documento,
sólo  se procesarán  las instrucciones  \ci{include} con  los ficheros
indicados en el argumento de  la orden \ci{includeonly}. Vigila que no
hayan espacios entre los nombres de los ficheros y las comas.



\section{Composición del texto}

\begin{intro}

Tras leer este tema ya deberíamos conocer los elementos básicos de los
que se  compone un  documento de \LaTeXe.  En este  tema completaremos
la  estructura  sobre la  que  normalmente  se trabaja  para  componer
documentos reales.

\end{intro}

\subsection{Salto de línea y de página}
 
\subsubsection{Párrafos justificados}

Normalmente los libros se suelen  componer con todos los renglones del
mismo  tamaño. \LaTeX{}  inserta los  saltos de  línea y  los espacios
entre las palabras  optimizando el contenido de  los párrafos enteros.
Si es  necesario, también  introduce guiones, dividiendo  las palabras
que no encajen bien al final de los renglones. El modo de componer los
párrafos depende  de la clase  de documento. Normalmente  se introduce
una  sangría  horizontal en  la  primera  línea  de  un párrafo  y  no
se  introduce espacio  adicional  entre cada  dos  párrafos. Para  más
información véase el apartado~\ref{parsp}.

En casos especiales  se le puede ordenar a \LaTeX{}  que introduzca un
salto de línea.

\begin{command}
\ci{\bs} o \ci{newline} 
\end{command}

\noindent comienza una línea nueva sin comenzar un párrafo nuevo.

\begin{command}
\ci{\bs*}
\end{command}

\noindent además  prohíbe que se produzca  un salto de página  tras el
salto de línea.

\begin{command}
\ci{newpage}
\end{command}

\noindent comienza una página nueva. 
\pagebreak

\begin{command}
\ci{linebreak}\verb|[|\emph{n}\verb|]|,
\ci{nolinebreak}\verb|[|\emph{n}\verb|]|, 
\ci{pagebreak}\verb|[|\emph{n}\verb|]| and
\ci{nopagebreak}\verb|[|\emph{n}\verb|]|
\end{command}

\noindent hacen  lo que  indican sus nombres:  salto de  línea, ningún
salto de línea,  salto de página y ningún salto  de página. Además nos
permite influir  sobre las  acciones a  través del  argumento opcional
\emph{n}. Se puede establecer a un valor entre cero y cuatro. Al poner
\emph{n} menor de 4 se le deja a \LaTeX{} la posibilidad de ignorar la
orden si el resultado resulta muy malo.

\LaTeX{}  siempre  intenta  realizar  los saltos  de  línea  lo  mejor
posible. Si no puede  encontrar ninguna posibilidad satisfactoria para
producir los bordes de los  párrafos totalmente rectos, cumpliendo con
las reglas impuestas,  entonces dejará un renglón  demasiado largo. En
este caso \LaTeX{} producirá el correspondiente mensaje de advertencia
(``\wni{overfull  box}'')  mientras  procesa el  fichero  de  entrada.
Esto  sucede  en  especial  si  no se  encuentra  un  lugar  apropiado
para  introducir  un  guión  de  divisón  entre  las  sílabas.  Si  se
introduce  la orden  \ci{sloppy}, \LaTeX{}  será menos  severo en  sus
exigencias y evita tales  renglones con longitudes mayores, aumentando
la  separación entre  las palabras  ---si bien  el resultado  final no
es  de lo  mejor---.  En  este caso  se  dan  mensajes de  advertencia
(``\wni{underfull  hbox}'').  El  resultado  suele  ser  aceptable  la
mayoría de las veces. La  orden \ci{fussy} actúa en sentido contrario.
Esto podría hacerlo en caso que desee ver a \LaTeX{} quejarse en todos
los sitios.

\subsubsection{Silabeo} \label{hyph}

\LaTeX{}  silabea  las  palabras   cuando  resulta  necesario.  Si  el
algoritmo de silabeo no produce  los resultados correctos, entonces se
puede remediar esta  situación con órdenes como las  que presentamos a
continuación.  Esto  suele  ser especialmente  necesario  en  palabras
compuestas o de idiomas extranjeros.

La instrucción
\begin{command}
\ci{hyphenation}\verb|{|\emph{lista de palabras}\verb|}|
\end{command}

\noindent da  lugar a que las  palabras mencionadas en ella  se puedan
dividir en cualquier  momento únicamente en los  lugares indicados con
``\verb|-|''\@.  Esta  orden  debería  aparecer en  el  preámbulo  del
fichero de  entrada y debería contener  solamente palabras construidas
sin caracteres especiales.

%%% Alguna idea de la definición de ``letras normales''???
%%% Es lo que aparece en el documento en inglés...

No se hacen  distinciones entre las letras mayúsculas  y minúsculas de
las palabras  a las que  se refiera  esta orden. El  ejemplo siguiente
permitirá localizar las sílabas de ``fichero'' y ``Fichero'' del mismo
modo,  e  impedirá que  en  las  palabras ``FORTRAN'',  ``Fortran''  y
``fortran''  se introduzcan  guiones.  No se  permiten caracteres  con
acentos o símbolos en el argumento.

%%% Pues podrían planteárselo por lo menos... :-)

Ejemplo:
\begin{code}
\verb|\hyphenation{FORTRAN fi-che-ro}|
\end{code}

Dentro de una palabra, la  instrucción \ci{-} establece un sitio donde
colocar un  guión si fuese  necesario. Además, éstos se  convierten en
los únicos  lugares donde  se permite introducir  los guiones  en esta
palabra. Esta instrucción es especialmente  útil para las palabras que
contienten  caracteres especiales  (como, p.ej.,\  los caracteres  con
acento ortográfico), ya que \LaTeX{} no silabea de modo automático las
palabras que contienen estos caracteres.

%\footnote{A no ser que esté usando los nuevos
%\wi{tipos DC}}.

\begin{example}
Me parece que esto es: su\-per\-%
ca\-li\-fra\-gi\-lis\-ti\-co\-%
ex\-pia\-li\-do\-so
\end{example}

También se pueden mantener varias palabras  en el mismo renglón con la
orden

\begin{command}
\ci{mbox}\verb|{|\emph{texto}\verb|}|
\end{command}

\noindent  Hace  que  su  argumento se  mantenga  siempre  unido  bajo
cualquier circunstancia, es decir, que no se puede dividir.

\begin{example}
Dentro de poco tendré otro teléfono.
Será el \mbox{(040) 3783-225}.

El parámetro \mbox{\emph{nombre
de fichero}} debe contener el nombre
del fichero.
\end{example}

\subsection{Caracteres especiales y símbolos}
 
\subsubsection{Comillas}

Para las  \wi{comillas} no  se debe utilizar  el carácter  de comillas
\index{\LaTeX!""@\texttt{""}} que se usa  en las máquinas de escribir.
Para las publicaciones se suelen utilizar caracteres especiales, tanto
para abrir como para cerrar comillas. En \LaTeX{} se usan dos~\verb|`|
para abrir comillas y dos~\verb|'| para cerrarlas.

\begin{example}
``Por favor, pulse la tecla `x.'\,''
\end{example}
% También podríamos haber escrito
% ... la tecla `x'.''
% El dejar una comilla delante o detrás de un signo de
% puntuación parece que es una cuestión de gusto.
% Sin embargo, una vez se planteó esta cuestión en
% Spanglish@EUnet.es y se había concluido que en esta situación se
% debía emplear la segunda opción. Como estamos con un ejemplo...

\subsubsection{Guiones y rayas}

\LaTeX{} reconoce  cuatro tipos de  \wi{guiones}. Para tener  acceso a
tres de éstos se pone  una cantidad diferente de guiones consecutivos.
El cuarto tipo es el signo matemático `menos':

\index{\LaTeX!-}%
\index{\LaTeX!--}%
\index{\LaTeX!---}%
\index{\LaTeX!-@$-$}%
\index{\LaTeX!matemático!menos}%

\begin{example}
psico-terapéutico \\
10--18~horas \\
Madrid -- Barcelona \\
?`Sí? ---dijo ella--- \\
0, 1 y $-1$
\end{example}

% En inglés, los nombres de estos guiones son:
% \texttt{-} \wi{hyphen}, \texttt{--} \wi{en-dash},
% \texttt{---} \wi{em-dash} y \verb|$-$| \wi{signo menos}.

\subsubsection{Puntos suspensivos (`\ldots')}

En  una máquina  de escribir,  tanto para  la \wi{coma}  como para  el
\wi{punto} se les da el mismo espaciado que a cualquier otro carácter.
En la  impresión de  libros, estos caracteres  sólo ocupan  un pequeño
espacio y  se colocan muy  próximos al  carácter que les  precede. Por
eso,  los ``puntos  suspensivos''  no se  pueden  introducir con  tres
puntos normales, ya que no  tendrían el espaciado correcto. Para estos
puntos existe una instrucción especial llamada

\begin{command}
\ci{ldots}
\end{command}
\index{\LaTeX!...@\ldots}
\begin{example}
No así ... sino así:\\
New York, Tokyo, Budapest\ldots
\end{example}
 
\subsubsection{Ligaduras}

Algunas  combinaciones de  letras  no se  componen  con las  distintas
letras  que  la  forman,  sino  que, de  hecho,  se  emplean  símbolos
especiales.

\begin{code}
{\large ff fi fl ffi\ldots}\quad
en lugar de\quad {\large f{}f f{}i f{}l f{}f{}i \ldots}
\end{code}
Estas \wi{ligaduras} se pueden evitar intercalando \ci{mbox}\verb|{}|
entre el par de letras en cuestión.

\subsubsection{Tildes y caracteres especiales}

\LaTeX{} permite  el uso  de \wi{tildes} y  \wi{caracteres especiales}
de  numerosos  idiomas. En  la  tabla~\ref{accents}  puedes ver  todos
los  tipos de  tildes  que  se pueden  aplicar  a  la letra  \emph{o}.
Naturalmente, también funciona con otras letras.

Para  colocar la  tilde  sobre una  \emph{i} o  una  \emph{j} se  debe
eliminar el puntito superior de estas letras. Esto se consigue con las
instrucciones \verb|\i| y \verb|\j|.

\begin{example}
H\^otel, na\"\i ve, \`el\`eve,\\ 
sm\o rrebr\o d, ¡Se\~norita!,\\
Sch\"onbrunner Schlo\ss{} 
Stra\ss e
\end{example}

\begin{table}[!hbp]
\caption{Tildes y caracteres especiales} \label{accents}
\begin{lined}{10cm}
\begin{tabular}{*4{cl}}
\A{\`o} & \A{\'o} & \A{\^o} & \A{\~o} \\
\A{\=o} & \A{\.o} & \A{\"o} \\[6pt]
\B{\u}{o} & \B{\v}{o} & \B{\H}{o} & \B{\c}{o} \\
\B{\d}{o} & \B{\b}{o} & \B{\t}{oo} \\[6pt]
\A{\oe}  &  \A{\OE} & \A{\ae} & \A{\AE} \\
\A{\aa} & \A{\aa} & \A{\AA} \\[6pt]
\A{\o}  & \A{\O} & \A{\l} & \A{\L} \\
\A{\i}  & \A{\j} & ¡ & \verb|¡| & ?` & \verb|?`| 
\end{tabular}

\index{\LaTeX!i y j sin puntito@\i{} y \j{} sin puntito}%
\index{\LaTeX!Letras escandinavas}%
\index{\LaTeX!ae@\ae}%
\index{\LaTeX!umlaut@\emph{umlaut}}%
\index{\LaTeX!dieresis@diéresis}%
\index{\LaTeX!grave@\emph{grave}}%
\index{\LaTeX!acute@\emph{acute}}%
\index{\LaTeX!acento!ortográfico}%
\index{\LaTeX!oe@\oe}

\medskip
\end{lined}
\end{table}
 
\subsection{Facilidades para lenguajes internacionales}
\index{\LaTeX!internacional}

Si necesitamos escribir documentos en otros \wi{idiomas} distintos del
inglés,  \LaTeX{}  debe utilizar  otras  \wi{reglas  de silabeo}  para
producir un resultado correcto.

Para muchos idiomas, estos cambios  se pueden llevar a cabo utilizando
el  paquete \pai{babel}  de  Johannes L.\  Braams.  Para emplear  este
paquete, el sistema \LaTeX{} debe  estar configurado de modo especial.
En  la  \guia{}  deberíamos   encontrar  más  información  sobre  este
particular.

Si tu sistema se configura  de modo apropiado, entonces podrás activar
el paquete \pai{babel} con la instrucción

\begin{command}
\ci{usepackage}\verb|[|\emph{idioma}\verb|]{babel}| 
\end{command}

\noindent tras  la orden \verb|\documentclass|. En  la \guia{} también
debería  aparecer  un listado  de  los  \emph{idiomas} que  acepta  tu
sistema.

Para   algunos   idiomas,   \textsf{babel}   también   define   nuevas
instrucciones  con las  que  se simplifica  la  entrada de  caracteres
especiales.  En el  idioma español\index{\LaTeX!espanol@español},  por
ejemplo, se utilizan letras con acento ortográfico. Con \textsf{babel}
y  el  estilo  \textsf{spanish},  se  puede  introducir  \emph{í}  con
\verb|'i| en vez de~\verb|\'{\i}|\footnote{En  este caso particular de
los  acentos ortográficos,  al  paquete \textsf{babel}  \emph{también}
debe pasársele la opción \textsf{activeacute}.}.

Además,  con  \textsf{babel} se  vuelven  a  definir los  títulos  que
producen  algunas  instrucciones de  \LaTeX,  que  normalmente son  en
inglés.  Por  ejemplo,  si introduces  la  orden  \ci{tableofcontents}
aparecerá  en  el  resultado  final   el  índice  del  documento.  Sin
embargo, el  título de este  índice dependerá del  idioma seleccionado
(`\emph{Table  of contents}'\  si  es inglés,  `\emph{Índice}'\ si  es
español, `\emph{In\-halt\-ver\-zeich\-nis}'\ si es alemán, etc.)

Con \textsf{babel} también se modifica la definición de la instrucción
\ci{today} para que introduzca la fecha del día en el idioma elegido.

% Aquí podríamos introducir muchas más cosas para el idioma
% español. Muchas de ellas ya están recogidas en otros apartados.

Algunos  sistemas   de  ordenadores  permiten   introducir  caracteres
especiales directamente desde el  teclado. \LaTeX{} puede manejar esos
caracteres. Desde la versión básica de \LaTeXe{} de diciembre de 1994,
se da la  posibilidad de la utilización de  diversos codificaciones de
entrada.  Para  esta facilidad  véase  el  paquete \pai{inputenc}.  Si
usamos este paquete deberíamos considerar  que otra gente puede no ser
capaz de  ver sus ficheros  en su  ordenador de forma  correcta porque
utilizan una  codificación diferente.  Por ejemplo, el  símbolo alemán
\emph{\"a} tiene en un PC el código 132 y en algunos sistemas Unix que
emplean ISO-LATIN~1 tiene  el código 228. Por lo  tanto, se recomienda
utilizar esta facilidad con sumo cuidado.

\subsection{Distancias entre palabras}

Para conseguir un margen derecho  recto en la salida, \LaTeX{} inserta
cantidades variables de  espacios entre las palabras. Al  final de una
oración,  introduce  unos  espacios  algo  mayores  que  favorecen  la
legibilidad del  texto. \LaTeX{} presupone  que las frases  acaban con
puntos, signos de interrogación y de  admiración. Si hay un punto tras
una  letra mayúscula,  entonces esto  no se  considera el  fin de  una
oración ya  que los puntos  tras las letras mayúsculas  normalmente se
utilizan para abreviaturas.

Cualquier   excepción  a   estas  reglas   deberemos  indicarla.   Una
\emph{barra  invertida}  \verb|\|  antes   de  un  espacio  en  blanco
produce  un  espacio en  blanco  que  no  se ensanchará.  Un  carácter
de  tilde~`\verb|~|'\ genera  un  espacio que  no  se puede  ensanchar
pero  en  el  que  además  no  se  puede  producir  ningún  cambio  de
renglón.  Si  antes de  un  punto  aparece la  instrucción  \verb|\@|,
significa  que  este punto  acaba  una  oración, aunque  se  encuentre
tras   una   letra   mayúscula.   \cih{"@}   \index{\LaTeX!~@\verb.~.}
\index{\LaTeX!tilde@tilde (\verb.~.)} \index{\LaTeX!.! espacio tras}

\begin{example}
En la fig.\ 1 del cap.\ 1\dots \\
El Dr.~López se encuentra \\
con Dña.~Pérez. \\
\dots\ 5~m de ancho. \\
Necesito vitamina~C\@. ?`Y tú?
\end{example}

Este tratamiento especial para los  espacios al final de las oraciones
se puede evitar con la instrucción

\begin{command}
\ci{frenchspacing}
\end{command}

\noindent que le indica a  \LaTeX{} que \emph{no} inserte más espacios
tras un punto  que tras cualquier otro carácter. Esto  es muy común en
diversos  idiomas, como  es  el  caso del  español.  En  este caso  la
instrucción \verb|\@| no es necesaria.

    
\subsection{Títulos, capítulos y apartados}

Para ayudar al  lector a seguir cómodamente el tema  de su trabajo, se
debería dividirlo en capítulos,  apartados y subapartados. \LaTeX{} lo
facilita  con  instrucciones especiales  que  toman  el título  de  la
sección como su argumento. De  nosotros depende emplearlos en el orden
correcto.

Para la clase \texttt{article} existen las siguientes órdenes de
seccionado:
\nopagebreak
\begin{code}
\ci{section}\verb|{...}           |\ci{paragraph}\verb|{...}|\\
\ci{subsection}\verb|{...}        |\ci{subparagraph}\verb|{...} |\\
\ci{paragraph}\verb|{...}     |\ci{appendix}
\end{code}

Con  las clases  \texttt{report} y  \texttt{book} puedes  utilizar dos
instrucciones de seccionado adicionales:

\begin{code}
\ci{part}\verb|{...}              |\ci{chapter}\verb|{...}|
\end{code}

Ya que  la clase  \texttt{article} no sabe  de capítulos,  es bastante
sencillo añadir  los artículos  como capítulos  de un  libro. \LaTeX{}
coloca automáticamente  el espaciado entre secciones,  la numeración y
los tipos de los títulos.

Dos de las instrucciones de seccionado son un poco especiales:

\begin{itemize}

\item La orden  \ci{part} no influye en la secuencia  de numeración de
los capítulos.

\item  La orden  \ci{appendix} no  toma ningún  argumento. Simplemente
cambia la modo de numeración  de los capítulos\footnote{Para el estilo
de artículo  lo que cambia  es la forma  de numerar los  apartados.} a
letras.

\end{itemize}

\LaTeX{}  crea  un  índice  tomando las  cabeceras  de  las  distintas
secciones y los  números de página del último  tratamiento del fichero
de entrada. La instrucción

\begin{command}
\ci{tableofcontents}
\end{command}

\noindent  introduce este  índice en  el lugar  donde coloquemos  esta
orden. Un documento  nuevo se debe procesar dos veces  para obtener un
índice\index{\LaTeX!indice@índice}  correcto. En  algunos casos  puede
llegar a ser necesario compilar el documento una tercera vez. \LaTeX{}
nos lo indicará en caso necesario.

De  todas  las órdenes  de  seccionado  que  se han  indicado  también
existen  versiones  modificadas,  que se  construyen  añadiéndoles  un
asterisco \verb|*|  al nombre de la  instrucción. Producen encabezados
de  sección  que  no  aparecen  en  el índice  y  no  se  numeran.  La
instrucción \verb|\subsection{Ayuda}| podría, por ejemplo, convertirse
en \verb|\subsection*{Ayuda}|.

Normalmente los  encabezados de  las secciones  aparecen en  el índice
exactamente tal  como los  introducimos en  el texto.  En determinadas
ocasiones esto no es posible porque  el título es demasiado largo para
caber en el  índice. Entonces se puede especificar la  entrada para el
índice con un argumento opcional antes del encabezado real.

\begin{code}
\verb|\chapter[¡Léelo! Te gustará]{Esto es un título largo|\\
\verb|                 y que puede aburrir a mucha gente}|
\end{code}

El \wi{título} de todo el documento se genera con la instrucción

\begin{command}
\ci{maketitle}
\end{command}

\noindent El contenido del título se debe definir con las órdenes

\begin{command}
\ci{title}\verb|{...}|, \ci{author}\verb|{...}| 
y opcionalmente \ci{date}\verb|{...}| 
\end{command}

\noindent  antes de  llamar a  \verb|\maketitle|. En  el argumento  de
\ci{author}  se pueden  proporcionar varios  nombres separados  con la
orden \ci{and}.

Un ejemplo  de algunas  de las  instrucciones mencionadas  las podemos
encontrar en la fig.~\ref{document} de la página~\pageref{document}.

Además  de  las  instrucciones  de seccionado  que  se  han  indicado,
\LaTeXe{} inserta 3 instrucciones adicionales para su uso con la clase
\verb|book|:

\begin{command}
\ci{frontmatter}, \ci{mainmatter} y \ci{backmatter}
\end{command}

\noindent Son útiles para  dividir tu publicación. Estas instrucciones
cambian  los encabezados  de  los  capítulos y  la  numeración de  las
páginas del mismo modo que en un libro normal.

\subsection{Referencias cruzadas}

En los libros, informes y artículos existen, a menudo, \wi{referencias
cruzadas} a  figuras, tablas  y segmentos especiales  de texto  que se
hallan  en  otros  lugares  del documento.  \LaTeX{}  proporciona  las
siguientes instrucciones para producir referencias cruzadas:

\begin{command}
  \ci{label}\verb|{|\emph{marcador}\verb|}|,
  \ci{ref}\verb|{|\emph{marcador}\verb|}| y
  \ci{pageref}\verb|{|\emph{marcador}\verb|}|
\end{command}

\noindent donde  \emph{marcador} es un identificador  que podemos usar
en  cualquier momento  para hacer  referencia  al elemento  al que  lo
asociemos. \LaTeX{} reemplaza \verb|\ref|  por el número del apartado,
subapartado, figura, tabla  o teorema donde se  insertó la instrucción
\verb|\label|  correspondiente. La  orden  \verb|\pageref| imprime  el
número de  página donde  se produce la  orden \verb|\label|  con igual
argumento.  Aquí también  se  utilizan los  números del  procesamiento
anterior.

\begin{example}
Una referencia a este subapartado
\label{sec:este} aparecería como:

``vea el apartado~\ref{sec:este} en
la página~\pageref{sec:este}.''
\end{example}
 
\subsection{Notas a pie de página}

Con la instrucción

\begin{command}
\ci{footnote}\verb|{|\emph{texto de la nota al pie}\verb|}|
\end{command}

\noindent se imprime una nota en el pie de la página actual.

\begin{example}
Las notas a pie de página%
\footnote{Esta es una nota a pie
de página} son utilizadas con
frecuencia por la gente que usa
\LaTeX.
\end{example}
 
También existe una variante de esta instrucción, que es

\begin{command}
\ci{footnote}\verb|[|\emph{número}\verb|]{|\emph{texto de la nota al
    pie}\verb|}|
\end{command}

De esta forma para la nota  al pie correspondiente se empleará para el
marcador el  \emph{número} que  se ha  indicado en  vez del  valor del
contador de notas  al pie. Esta variante \emph{sólo}  se puede emplear
dentro de los párrafos.

\subsection{Palabras resaltadas}

En los  escritos a  máquina, para  resaltar determinados  segmentos de
texto  normalmente se  $\underline{\mathrm{subrayan}}$. En  los libros
impresos estas  palabras se  \emph{resaltan} o se  \emph{destacan}. La
orden para cambiar a un tipo de letra \emph{resaltado} es

\begin{command}
\ci{emph}\verb|{|\emph{texto}\verb|}|
\end{command}
\noindent Su argumento es el texto que se debe \wi{resaltar}.

\begin{example}
\emph{Si está empleando
\emph{resalte} en un texto
ya resaltado, entonces \LaTeX{}
utiliza \emph{redonda} para volver
a resaltar texto.}
\end{example}
 

\subsection{Entornos} \label{env}

Para componer textos con un  propósito especial \LaTeX{} define muchos
tipos de \wi{entornos} para toda clase de diseños:

\begin{command}
\ci{begin}\verb|{|\emph{nombre}\verb|}|\quad
   \emph{texto}\quad
\ci{end}\verb|{|\emph{nombre}\verb|}|
\end{command}

\noindent donde \emph{nombre}  es el nombre del  entorno. Los entornos
son ``grupos'' o ``agrupaciones''. También  podemos cambiar a un nuevo
entorno dentro de otro, en cuyo caso hay que prestar especial atención
a la secuencia:

\begin{code}
\verb|\begin{aaa}...\begin{bbb}...\end{bbb}...\end{aaa}|
\end{code}

En   los  apartados   siguientes  explicaremos   todos  los   entornos
importantes.

\subsubsection{Listas y descripciones (\texttt{itemize},
  \texttt{enumerate}, \texttt{description})}

El  entorno  \ei{itemize}  es  adecuado  para  las  listas  sencillas,
el  entorno  \ei{enumerate} para  relaciones  numeradas  y el  entorno
\ei{description} para descripciones.\cih{item}

\begin{example}
\begin{enumerate}
\item Puedes mezclar los entornos
de listas a tu gusto:
\begin{itemize}
\item Pero podría comenzar a
parecer incómodo.
\item Si abusa de ellas.
\end{itemize}
\item Por lo tanto, recuerda:
\begin{description}
\item[Lo innecesario] no va a
resultar adecuado porque
lo coloques en una lista.
\item[Lo adecuado,] sin embargo,
se puede presentar agradablemente
en una lista.
\end{description}
\end{enumerate}
\end{example}
 
\subsubsection{Justificaciones y centrado (\texttt{flushleft},
            \texttt{flushright}, \texttt{center})}

Los  entornos  \ei{flushleft}   y  \ei{flushright}  producen  párrafos
justificados  a  la  izquierda  y  a la  derecha  (sin  nivelación  de
bordes).%

\index{\LaTeX!justificado  a la  izquierda}\index{\LaTeX!justificado a
la derecha}  El entorno  \ei{center} genera texto  centrado. Si  no se
introduce \ci{\bs}  para dividir  los renglones, entonces  \LaTeX{} lo
hará automáticamente.

\begin{example}
\begin{flushleft}
Este texto está\\ justificado a
la izquierda. \LaTeX{} no intenta
forzar que todas las líneas
tengan longitud.
\end{flushleft}
\end{example}

\begin{example}
\begin{flushright}
Este texto está\\ justificado a
la derecha. \LaTeX{} no intenta
forzar que todas las líneas
tengan igual longitud.
\end{flushright}
\end{example}

\begin{example}
\begin{center}
En el centro\\de la tierra
\end{center}
\end{example}

\subsubsection{Citas        (\texttt{quote},       \texttt{quotation},
\texttt{verse})}

El  entorno \ei{quote}  sirve  para citas  pequeñas,  ejemplos y  para
resaltar oraciones.

\begin{example}
Una regla de oro en tipografía
para la longitud de los renglones
dice:
\begin{quote}
Ningún renglón debe contener
más de 66~letras.
\end{quote}
Por esto se suelen utilizar varias
columnas en los periódicos.
\end{example}
% Curioso lo que hace el ``66~letras'' en el primer renglón. Hay un
% modo de indicar preferencias entre la división de ``debe'' y
% ``66~letras''?


Hay dos entornos muy parecidos: el entorno \ei{quotation} y el entorno
\ei{verse}.  El  entorno  \texttt{quotation} es  adecuado  para  citas
mayores que consten  de varios párrafos. El  entorno \texttt{verse} es
apropiado para  poemas en los  que la  separación de los  renglones es
esencial. Los  versos (los  renglones) se dividen  con \ci{\bs}  y las
estrofas con renglones en blanco.



%Este verso se sale. Me gustaría saber qué pensarán los peninsulares
%y demás gente no canaria sobre el término ``gofio''\dots
\begin{example}
\begin{flushleft}
\begin{verse}
Soberano gofio en polvo,\\
sustento de mi barriga,\\
el día que no te como\\
para mí no hay alegría.
\end{verse}
\end{flushleft}
\end{example}

\subsubsection{Edición directa (\texttt{verbatim}, \texttt{verb})}

El texto  que se  coloque entre  \verb|\begin{|\ei{verbatim}\verb|}| y
\verb|\end{verbatim}| aparecerá  tal como  se ha escrito,  redactado a
partir de una máquina de escribir,  con todos los espacios en blanco y
cambios de línea y sin interpretación de las instrucciones de \LaTeX.

Dentro de un párrafo se puede lograr el mismo efecto con

\begin{command}
\ci{verb}\verb|+|\emph{text}\verb|+|
\end{command}

\noindent  El \verb|+|  sólo es  un ejemplo  de carácter  delimitador.
Podemos  emplear cualquier  carácter  excepto las  letras, \verb|*|  o
caracteres en blanco.

\begin{example}
La instrucción \verb|\ldots|%
\ldots

\begin{verbatim}
10 PRINT "HELLO WORLD ";
20 GOTO 10
\end{verbatim}
\end{example}

\begin{example}
\begin{verbatim*}
La version con estrella del
entorno          verbatim
destaca los espacios     en
el  texto con un símbolo
  especial.
\end{verbatim*}
\end{example}

La instrucción \ci{verb} se puede usar exactamente del mismo modo, con
un asterisco:

\begin{example}
%\verb*|de esta   manera :-) |
\end{example}

El entorno  \texttt{verbatim} y la instrucción  \verb|\verb| no pueden
utilizarse como parámetros de otras instrucciones.

\subsubsection{Estadillos (\texttt{tabular})}

El entorno  \ei{tabular} sirve para crear  \wi{estadillos}, con líneas
horizontales y  verticales según cómo deseemos.  \LaTeX{} determina la
anchura de las columnas de modo automático.

El argumento \emph{especificaciones del estadillo} de la instrucción

\begin{command}
\verb|\begin{tabular}{|\emph{especificaciones del estadillo}\verb|}|
\end{command} 

\noindent define  el diseño del estadillo. Se puede utilizar \texttt{l}
para una columna con texto justificado a la izquierda, \texttt{r} para
justificar  el texto  a la  derecha, \texttt{c}  para texto  centrado,
\verb|p{|\emph{ancho}\verb|}| para una columna  que contenga texto con
saltos de línea y \verb.|. para una línea vertical.


Dentro de  un entorno  \texttt{tabular}, \verb|&|  salta a  la próxima
columna,  \ci{\bs} separa  los  renglones y  \ci{hline} introduce  una
línea horizontal.

\index{\LaTeX!"|@ \verb."|.}
\begin{example}
\begin{tabular}{|r|l|}
\hline
Cantidad & Base \\
\hline
\hline
7C0 & hexadecimal \\
3700 & octal \\
11111000000 & binario \\
\hline \hline
1984 & decimal \\
\hline
\end{tabular}
\end{example}

\begin{example}
\begin{tabular}{|p{4.7cm}|}
\hline
Bienvenido al párrafo del Sr.\
Cajón. Esperamos que disfrute
del espectáculo.\\
\hline
\end{tabular}

\end{example}

Con la construcción \verb|@{...}| se puede especificar el separador de
columnas. Esta  construcción elimina  el espacio  entre columnas  y lo
reemplaza con lo que hayamos  introducido entre los paréntesis. Un uso
muy  frecuente de  esta construcción  se explica  más adelante  con el
problema de la alineación de la coma decimal. Otra posible utilización
es para eliminar el espacio que  antecede y precede a los renglones de
una tabla con \verb|@{}|.

\begin{example}
\begin{tabular}{@{} l @{}}
\hline
ningún espacio a la izquierda
ni derecha\\\hline
\end{tabular}
\end{example}
\begin{example}
\begin{tabular}{l}
\hline
espacios a la izquierda
y a la derecha\\
\hline
\end{tabular}
\end{example}

\index{\LaTeX!alineación  decimal}% Ya  que  no  hay ningún  mecanismo
incorporado  para alinear  columnas  numéricas sobre  la coma  decimal
\footnote{Si se  halla instalado el  conjunto `tools'\ en  su sistema,
sería  recomendable echar  un vistazo  al paquete  \pai{dcolumn}.}, se
podría  ``imitarlo'' utilizando  dos  columnas: un  entero alineado  a
la  derecha y  luego  los  decimales a  la  izquierda. La  instrucción
\verb|@{,}|  en el  argumento de  \verb|\begin{tabular}| reemplaza  el
espacio  normal entre  columnas  con una  ``,'',  dando la  apariencia
de  una   única  columna  justificada   por  la  coma   decimal.  ¡No
debemos  olvidarnos  de reemplazar  la  coma  decimal en  sus  números
con  un  separador  de  columna   (\verb|&|)!  Se  puede  colocar  una
etiqueta sobre  nuestra ``columna'' numérica empleando  la instrucción
\ci{multicolumn}.

\begin{example}  
\begin{tabular}{cr@{,}l}
Expresión  en  pi  & \multicolumn{2}{c}{Valor} \\  
\hline 
$\pi$  & 3&1416 \\  
$\pi^{\pi}$ &
36&46 \\ 
$(\pi^{\pi})^{\pi}$ & 80662&7 
\\ \end{tabular} 
\end{example}

\subsection{Elementos flotantes}

Hoy  en  día,  la  mayoría   de  las  publicaciones  contienen  muchas
ilustraciones  y  tablas.  Estos elementos  necesitan  un  tratamiento
especial porque  no se pueden  cortar entre páginas. Un  método podría
ser comenzando  una página nueva  cada vez  que una ilustración  o una
tabla sea demasiado larga para caber en la página actual. Este enfoque
deja páginas parcialmente vacías, lo que resulta poco estético.

La solución a este problema es hacer que cualquier ilustración o tabla
que no quepa  en la página actual `flote'\ hasta  una página posterior
mientras se rellena la página actual con el texto del documento.

\LaTeX{}  ofrece  dos  entornos  para  los  \wi{elementos  flotantes}.
Uno   para  las   tablas   y  otro   para   las  ilustraciones.   Para
aprovechar  completamente estos  dos entornos  es importante  entender
aproximadamente   cómo  maneja   \LaTeX{}   estos  objetos   flotantes
internamente. Si no, los objetos  flotantes se pueden convertir en una
fuente de frustaciones  porque \LaTeX{} nunca los pone  dónde se desea
situar.

%\bigskip

Primeramente,  echemos un  vistazo  a las  instrucciones que  \LaTeX{}
proporciona para objetos flotantes.

Cualquier cosa que  se incluya en un entorno  \ei{figure} o \ei{table}
se  tratará   como  un  objeto  flotante.   Ambos  entornos  flotantes
proporcionan un parámetro opcional

\begin{command}
\verb|\begin{figure}[|\emph{designador de colocado}\verb|]| o\\
\verb|\begin{table}[|\emph{designador de colocado}\verb|]|
\end{command}

\noindent llamado el \emph{designador  de colocado}. Este parámetro se
emplea para indicarle  a \LaTeX{} los lugares donde  le permitimos que
vaya colocado el objeto flotante.  Un \emph{designador de colocado} se
construye con  una cadena  de \emph{permisos de  colocación flotante}.
Los podemos encontrar en la tabla~\ref{tab:permiss}.

\begin{table}[!bp]
\caption{Permisos de colocación flotante}\label{tab:permiss}
\noindent \begin{minipage}{\textwidth}
\medskip
\begin{center}
\begin{tabular}{@{}cp{10cm}@{}}
  Designador&Permiso para colocar el objeto flotante\ldots\\ \hline
  \rule{0pt}{1.05em}
\texttt{h} & aquí (\emph{here}), muy próximo al
  lugar en el texto donde se ha introducido. Es útil, principalmente,
  para objetos flotantes pequeños.\\[0.3ex]
\texttt{t} & en la parte superior de una página (\emph{top}).\\[0.3ex]
\texttt{b} & en la parte inferior de una página
  (\emph{bottom}).\\[0.3ex]
\texttt{p} & en una \emph{página} especial que sólo contenga
  elementos flotantes.\\[0.3ex]
\texttt{!} & sin considerar la mayoría de los parámetros
  internos\footnote{Como el número máximo de elementos flotantes en
  una página.} que impedirían a este objeto flotante que se colocase.
\end{tabular}
\end{center}
\end{minipage}
\end{table}

\pagebreak[3]

Una tabla se podría comenzar con, por ejemplo, la siguiente línea:

\begin{code}
\verb|\begin{table}[!hbp]|
\end{code}

\noindent El  \wi{designador de  colocado} \verb|[!hbp]| le  permite a
\LaTeX{}  colocar la  tabla justamente  aquí (\texttt{h})  o al  final
(\texttt{b})  de  alguna  página  o en  alguna  página  especial  para
elementos  flotantes,  y en  cualquier  parte  si  no queda  tan  bien
(\texttt{!}). Si no indicamos  ningún designador de colocado, entonces
se sobreentienden las clases normalizadas \verb|[tbp]|.

\LaTeX{} colocará todos los objetos  flotantes que encuentre según los
de\-sig\-na\-do\-res de  colocado que  hayamos indicado. Si  un objeto
flotante no se puede colocar en la página actual entonces se aplaza su
colocación, para lo cual se introduce en una cola\footnote{Son de tipo
\emph{fifo}: lo que se introdujo  primero es lo primero en extraerse.}
de \emph{tablas} o \emph{figuras}  (ilustraciones). Cuando se comienza
una nueva  página, lo  primero que  \LaTeX{} hace  es confirmar  si se
puede construir una  página especial con los objetos  flotantes que se
hallan en  las colas. Si  no es posible,  entonces se trata  el primer
objeto  que  se encuentra  en  las  colas  como  si lo  acabásemos  de
introducir.  Entonces \LaTeX{}  vuelve  a intentar  colocar el  objeto
según  sus designadores  de  colocado  (eso sí,  sin  tener en  cuenta
la  opción  `\verb|h|',\  que  ya no  es  posible).  Cualquier  objeto
flotante  nuevo que  aparezca  en el  texto se  introduce  en la  cola
correspondiente. \LaTeX{} mantiene estrictamente  el orden original de
apariciones de cada tipo de objeto flotante.

Ésta es la razón por la que una ilustración que no se puede colocar ya
que desplaza  al resto de las  figuras al final del  documento. Por lo
tanto:

\begin{quote}
Si  \LaTeX{} no  coloca  los objetos  flotantes  como esperaba,  suele
deberse únicamente a un objeto flotante  que está atascando una de las
dos colas de objetos flotantes.
\end{quote}

%\bigskip

\noindent Además, existen algunas cosas más que se deben indicar sobre
los entornos \ei{table} y \ei{figure}. Con la instrucción

\begin{command}
\ci{caption}\verb|{|\emph{texto de título}\verb|}|
\end{command}

\noindent se puede definir un título para el objeto flotante. \LaTeX{}
le añadirá la cadena ``Figura'' o ``Tabla'' y un número de secuencia.

Las dos instrucciones
\begin{command}
\ci{listoffigures} y \ci{listoftables} 
\end{command}

\noindent    funcionan     de    modo     análogo    a     la    orden
\verb|\tableofcontents|, imprimiendo un índice  de figuras o de tablas
respectivamente. En  estas listas se repetirán  los títulos completos.
Debeimos tener presente que si  tendemos a utilizar títulos largos, es
recomendable  tener  una versión  de  estos  títulos más  cortos  para
utilizarlos  para estos  índices. Esto  se consigue  dando la  versión
corta entre corchetes tras la orden \verb|\caption|.

\begin{code}
\verb|\caption[Corto]{LLLLLaaaaaaaaarrrrrrrrgggggooooooo}| 
\end{code}

Con  \verb|\label| y  \verb|\ref|  se pueden  crear  referencias a  un
objeto flotante dentro del texto.

El siguiente ejemplo dibuja un cuadrado  y lo inserta en el documento.
Podría utilizar esto si desea reservar espacios para imágenes que vaya
a pegar en el documento acabado.

\begin{code}
\begin{verbatim}
La ilustración~\ref{blanco} es un ejemplo del Pop-Art.
\begin{figure}[!hbp]
\makebox[\textwidth]{\framebox[5cm]{\rule{0pt}{5cm}}}
\caption{$5\times 5$ centímetros} \label{blanco}
\end{figure}
\end{verbatim}
\end{code}

\noindent   En   el   ejemplo  anterior\footnote{suponiendo   que   la
cola   de   figuras   esté  vacía.}   \LaTeX{}   intentará   \emph{por
todos  los  medios}~(\texttt{!})  colocar la  ilustración  exactamente
\emph{aquí}~(\texttt{h}).  Si  no  puede, intentará  colocarla  en  la
\emph{parte  inferior}~(\texttt{b})  de  la  página.  Si  no  consigue
colocar esta figura en la página actual, determina si es posible crear
una  página  (\verb|p|)  con elementos  flotantes  exclusivamente  que
contenga esta  ilustración y algunas  tablas que pudieran haber  en la
cola de tablas. Si no hay material suficiente para una página especial
de objetos  flotantes, entonces \LaTeX{}  comienza una página  nueva y
otra vez trata la figura como si acabase de aparecer en el texto.

Bajo determinadas condiciones podría ser necesario emplear la orden
\begin{command}
\ci{clearpage}
\end{command}

\noindent Le ordena a \LaTeX{} que coloque \emph{inmediatamente} todos
los  objetos  flotantes que  se  encuentren  en  las colas  y  después
comenzar una página nueva.

%Más adelante veremos cómo incluir imágenes en formato PostScript en
%tus documentos de \LaTeXe.


\subsection{Añadiendo instrucciones y entornos nuevos}

En  el primer  apartado explicamos  que \LaTeX{}  necesita información
sobre la estructura lógica del  texto para elegir el formato adecuado.
Éste es  un concepto  muy bien  cuidado. Pero  en la  práctica solemos
chocar  con las  limitaciones que  esto  nos impone,  ya que  \LaTeX{}
simplemente no tiene el entorno  especializado o la orden que deseamos
para un propósito específico.

Una solución  es emplear varias  órdenes de \LaTeX{} para  producir el
diseño que se  tiene en mente. Si  tenemos que hacer esto  una vez, no
hay ningún problema. Pero si esto sucede repetidamente, entonces lleva
mucho tiempo. Si  alguna vez deseáramos cambiar  el formato tendríamos
que revisar el fichero de entrada  entero y editar todos los elementos
en cuestión.

Para  resolver este  problema, \LaTeX{}  nos permite  definir nuestras
propias instrucciones y entornos.

\subsubsection{Instrucciones nuevas}

Para añadir nuestras propias instrucciones utilizaremos la orden

\begin{command}
\ci{newcommand}\verb|{|%
   \emph{nombre}\verb|}[|\emph{num}\verb|]{|\emph{definición}\verb|}|
\end{command}

\noindent  Básicamente,   la  instrucción  necesita   dos  argumentos:
el  \emph{nombre}   de  la  instrucción   que  queremos  crear   y  la
\emph{definición}  de la  instrucción.  El  argumento entre  corchetes
\emph{num} es opcional.  Podemos usarlo para crear  órdenes nuevas que
tomen hasta 9 argumentos.

Los  dos ejemplos  siguientes deberían  ayudar  a captar  la idea.  El
primer ejemplo define una  instrucción nueva llamada \verb|\udl|. Ésta
es una forma  abreviada de introducir ``Una  Descripción de \LaTeXe''.
Una  orden como  ésta sería  muy útil  si tuviéramos  que escribir  el
título de este tema una y otra vez.

\begin{example}
\newcommand{\udl}
    {Una Descripción de \LaTeXe}
% en el cuerpo del documento :
``\udl'' \ldots{} ``\udl''
\end{example}

El ejemplo  siguiente ilustra  cómo usar  el argumento  \emph{num}. La
secuencia  \verb|#1|  encuentra  un  sustituto  en  el  argumento  que
especifiquemos.  Si  quisiéramos  más  de  un  argumento,  emplearemos
\verb|#2| y así sucesivamente.

\begin{example}
\newcommand{\txsit}[1]
    {Una Descripción \emph{#1}
     Pequeña de \LaTeXe}
% en el cuerpo del documento: 
\begin{itemize}
\item \txsit{no tan}
\item \txsit{muy}
\end{itemize}
\end{example}

\LaTeX{}  no  nos  permitirá  crear   una  instrucción  nueva  con  un
nombre  que ya  existe.  Si  queremos ignorar  de  modo explícito  una
instrucción  existente  debemos   utilizar  \ci{renewcommand}.  Aparte
de  su   nombre,  utiliza  la   misma  sintaxis  que   la  instrucción
\verb|\newcommand|.  En  determinados   casos  podríamos  utilizar  la
instrucción \ci{providecommand}.  Funciona como  \ci{newcommand}, pero
si ya hay una instrucción definida con este nombre, entonces \LaTeXe{}
simplemente ignora esta otra definición que acabáramos de indicar.

\subsubsection{Entornos nuevos}

De  modo  análogo  a  la  instrucción  \verb|\newcommand|  existe  una
orden para  crear tus  propios entornos. Cuando  estábamos escribiendo
este  tema,  hemos creado  entornos  especiales  para estructuras  que
se  empleaban  repetidamente  en toda  la  descripción:  ``ejemplos'',
``segmentos de código'' y ``cajas de definición de instrucciones''. La
instrucción \ci{newenvironment} utiliza la siguiente sintaxis:

\begin{command}
\ci{newenvironment}\verb|{|%
       \emph{nombre}\verb|}[|\emph{num}\verb|]{|%
       \emph{antes}\verb|}{|\emph{después}\verb|}|
\end{command}

Al  igual  que  la   instrucción  \verb|\newcommand|,  se  puede  usar
\ci{newenvironment}   con   o   sin   argumento   opcional.   Lo   que
especifiquemos  en  el argumento  \emph{antes}  se  procesa antes  que
el  texto  dentro del  entorno.  Lo  que  indiquemos en  el  argumento
\emph{después}  se   procesa  cuando   se  encuentra   la  instrucción
\verb|\end{|\emph{nombre}\verb|}|.

El   siguiente   ejemplo   ilustra    el   uso   de   la   instrucción
\ci{newenvironment}.

\begin{example}
\newenvironment{king}
    {\begin{quote}}{\end{quote}}
% use esto en el cuerpo
\begin{king}
Mis humildes vasallos\ldots
\end{king}
\end{example}

El  argumento   \emph{num}  se   utiliza  igual  que   la  instrucción
\verb|\newcommand|. \LaTeX{} se asegura de que no definamos un entorno
que ya existía. Si alguna  vez deseamos cambiar una entorno existente,
entonces podemos utilizar  la instrucción \ci{renewenvironment}. Tiene
la misma sintaxis que la instrucción \ci{newenvironment}.


\section{Composición de fórmulas matemáticas}

\begin{intro}
¡Ahora  a  admirarse con  este  apartado!  Vamos  a abordar  el  punto
fuerte  de  \TeX:  la  composición  matemática.  Pero  advertimos  que
este  apartado  sólo  muestra  la superficie.  Mientras  lo  que  aquí
explicamos es  suficiente para mucha  gente, no hay  que desesperarase
si  no  se  puede  encontrar   una  solución  a  nuestras  necesidades
de  composición  de  expresiones  matemáticas.  Es  muy  probable  que
en  esos  casos  nuestro  problema ya  esté  abordado  en  AMS-\LaTeXe
\footnote{\texttt{CTAN:/tex-archive/macros/latex/packages/amslatex}} o
en algún otro paquete.
\end{intro}

\subsection{Generalidades}

\LaTeX{}   posee   un   modo  especial   para   componer   expresiones
\wi{matemáticas}.  En un  párrafo,  el texto  matemático se  introduce
entre   \ci{(}    y   \ci{)},    \index{\LaTeX!\$@\texttt{\$}}   entre
\texttt{\$}  y \texttt{\$}  o entre  \verb|\begin{|\ei{math}\verb|}| y
\verb|\end{math}|. \index{\LaTeX!formulas@fórmulas}

\begin{example}
Siendo $a$ y $b$ los catetos
y $c$ la hipótenusa
de un triángulo rectángulo,
entonces $c^{2}=a^{2}+b^{2}$
(Teorema de Pitágoras).
\end{example}

\begin{example}
\TeX{} se pronuncia como
 $\tau\epsilon\chi$.\\[6pt]
100~m$^{2}$ de área útil \\[6pt]
De mi $\heartsuit$.
\end{example}

Las  fórmulas  matemáticas  mayores  o las  ecuaciones  suelen  quedar
mejor  en renglones  separados del  texto.  Para ello  se ponen  entre
\ci{[}  y  \ci{]}  o  entre  \verb|\begin{|\ei{displaymath}\verb|}|  y
\verb|\end{displaymath}|.   Esto  produce   fórmulas  sin   número  de
ecuación. Si quieres que \LaTeX{} las enumere, entonces puedes emplear
el entorno \ei{equation}.

\begin{example}
Siendo $a$ y $b$ los catetos
y $c$ la hipótenusa
de un triángulo rectángulo,
entonces
\begin{displaymath}
c = \sqrt{  a^{2}+b^{2}  }
\end{displaymath}
(Teorema de Pitágoras).
\end{example}

Con \ci{label} y \ci{ref} se puede hacer referencia a una ecuación del
documento.

\begin{example}
\begin{equation} \label{eq:eps}
\epsilon > 0
\end{equation}
De (\ref{eq:eps}) se deduce\ldots
\end{example}

Fíjate  que las  expresiones se  componen con  un estilo  diferente al
disponerlas en párrafos separados del texto:

\begin{example}
$\lim_{n \to \infty}
\sum_{k=1}^n \frac{1}{k^2}
= \frac{\pi^2}{6}$
\end{example}
\begin{example}
\begin{displaymath}
\lim_{n \to \infty}
\sum_{k=1}^n \frac{1}{k^2}
= \frac{\pi^2}{6}
\end{displaymath}
\end{example}


Existen diferencias  entre el  \emph{modo matemático} y  el \emph{modo
texto}. Por ejemplo, en el \emph{modo matemático}:

\begin{enumerate}

\item Los espacios en blanco y los cambios de línea no tienen ningún
  significado. Todos los espacios se determinarán a partir de la
  lógica de la expresión matemática o se deben indicar con
  instrucciones especiales como \ci{,}, \ci{quad}, \ci{qquad}, \ci{:},
  \ci{;}, \verb|\ | y \verb|\!|\cih{"!}.%
\index{\LaTeX!instrucciones!\quad@\verb".\ ".}%
\index{\LaTeX!\quad@\hspace*{-1.2ex}\verb".\ ".}%

\begin{example}
\begin{equation}
\forall x \in \mathbf{R}:
\qquad x^{2} \geq 0
\end{equation}
\end{example}
 
\item Los renglones en blanco están prohibidos. Sólo puede haber un
  párrafo por fórmula.
\item Cada letra en particular se tendrá en cuenta como el nombre de
  una variable y se pondrá como tal (cursiva con espacios
  adicionales). Para introducir texto normal dentro de un texto
  matemático (con escritura en redondilla y con espacios entre
  palabras) debes incluirlo dentro de la orden \verb|\textrm{...}|.

\begin{example}
\begin{equation}
x^{2} \geq 0\qquad
\textrm{para todo }x\in\mathbf{R}
\end{equation}
\end{example}
 
\end{enumerate}

%
% Añada el paquete AMSSYB ... Blackboard bold .... R para números
% reales
%

Los   matemáticos  pueden   ser   muy  exigentes   con  los   símbolos
que   se  emplean:   aquí   sería   más  \emph{convencional}   emplear
`\emph{blackboard bold}' \index{\LaTeX!blackboard bold@\emph{blackboad
bold}} \index{\LaTeX!simbolos  en negrita@símbolos en negrita}  que se
obtienen con  \ci{mathbb} del paquete \pai{amsfonts}  o \pai{amssymb}.
\ifx\mathbb\undefined\else El último ejemplo se convierte en

\begin{example}
\begin{displaymath}
x^{2} \geq 0\qquad
\textrm{para todo }x\in\mathbb{R}
\end{displaymath}
\end{example}
\fi

\subsection{Agrupaciones en modo matemático}

En  modo  matemático  la  mayoría de  las  instrucciones  sólo  afecta
al  carácter  siguiente.  Si  se desea  que  una  instrucción  influya
sobre varios  caracteres, entonces  debes agruparlos  empleando llaves
(\verb|{...}|).

\begin{example}
\begin{equation}
a^x+y \neq a^{x+y}
\end{equation}
\end{example}
 
\subsection{Elementos de las fórmulas matemáticas}

A continuación  presentamos las  instrucciones más importantes  que se
utilizan en las fórmulas matemáticas.

\textbf{Las    \wi{letras   griegas}    minúsculas}   se    introducen
como   \verb|\alpha|,   \verb|\beta|,   \verb|\gamma|\ldots,   y   las
mayúsculas\footnote{No   hay  definida   ninguna  Alfa   mayúscula  en
\LaTeXe{} porque tiene  el mismo aspecto que la redondilla  A. Una vez
que  se haga  la  nueva codificación  matemática,  esto cambiará.}  se
introducen como \verb|\Gamma|, \verb|\Delta|\ldots

\begin{example}
$\lambda,\xi,\pi,\mu,\Phi,\Omega$
\end{example}

\index{\LaTeX!exponente}\index{\LaTeX!subíndice}%
\textbf{Los exponentes y los subíndices} se pueden indicar empleando
el carácter \verb|^|\index{\LaTeX!^@\verb"|^"|} y el carácter
\verb|_|\index{\LaTeX!_@\verb"|_"|}.

\begin{example}
$a_{1}$ \qquad $x^{2}$ \qquad
$e^{-\alpha t}$ \qquad
$a^{3}_{ij}$\\
$e^{x^2} \neq {e^x}^2$
\end{example}

El \textbf{\wi{signo de raíz cuadrada}}  se introduce con \ci{sqrt}, y
la raíz  \mbox{$n$-ésima} con \verb|\sqrt[|$n$\verb|]|.  \LaTeX\ elige
automáticamente el  tamaño del signo  de raíz. Si sólo  necesitamos el
signo de la raíz utiliza \verb|\surd|.

\begin{example}
$\sqrt{x}$ \qquad 
$\sqrt{ x^{2}+\sqrt{y} }$ 
\qquad $\sqrt[3]{2}$\\[3pt]
$\surd[x^2 + y^2]$
\end{example}

Las    instrucciones   \ci{overline}    y   \ci{underline}    producen
\textbf{líneas  horizontales}  directamente  encima o  debajo  de  una
expresión.

\index{\LaTeX!linea@línea!horizontal}
\begin{example}
$\overline{m+n}$
\end{example}

Las  órdenes  \ci{overbrace}  y \ci{underbrace}  crean  \textbf{llaves
horizontales} largas encima o bien debajo de una expresión.

\index{\LaTeX!llave!horizontal}
\begin{example}
$\underbrace{ a+b+\cdots+z }_{26}$
\end{example}

\index{\LaTeX!acentos!matemáticos}Para poner acentos matemáticos, como
pequeñas flechas o \wi{tilde}s a las variables, se pueden utilizar las
órdenes  que aparecen  en la  tabla~\ref{mathacc}. Los  ángulos y  las
tildes que abarcan varios caracteres  se obtienen con \ci{widetilde} y
\ci{widehat}.  Con el  símbolo \verb|'|\index{\LaTeX!'@\verb"|'"|}  se
introduce el signo de \wi{prima}. % un guión es --

\begin{example}
\begin{displaymath}
y=x^{2}\qquad y'=2x\qquad y''=2
\end{displaymath}
\end{example}

\begin{table}[!h]
\caption{Acentos en modo matemático}  \label{mathacc}
\begin{symbols}{*4{cl}}
\W{\hat}{a}     & \W{\check}{a} & \W{\tilde}{a} & \W{\acute}{a} \\
\W{\grave}{a} & \W{\dot}{a} & \W{\ddot}{a}    & \W{\breve}{a} \\
\W{\bar}{a} &\W{\vec}{a} &\W{\widehat}{A}&\W{\widetilde}{A}\\  
\end{symbols}
\end{table}

Con  frecuencia  los  \textbf{\wi{vectores}} se  indican  añadiéndoles
\wi{símbolos  de  flecha} pequeños  encima  de  la variable.  Esto  se
realiza con  la orden \ci{vec}. Para  designar al vector que  va desde
$A$ hasta $B$ resultan adecuadas las instrucciones \ci{overrightarrow}
y \ci{overleftarrow}.

\begin{example}
\begin{displaymath}
\vec a\quad\overrightarrow{AB}
\end{displaymath}
\end{example}

Existen    funciones    matemáticas     (seno,    coseno,    tangente,
logaritmos\ldots)  que  se  presentan con  redondilla  y  \emph{nunca}
en   itálica.  Para   éstas,  \LaTeX{}   proporciona  las   siguientes
instrucciones: \index{\LaTeX!funciones!matemáticas}

\begin{verbatim}
\arccos   \cos    \csc   \exp   \ker     \limsup  \min   \sinh
\arcsin   \cosh   \deg   \gcd   \lg      \ln      \Pr    \sup
\arctan   \cot    \det   \hom   \lim     \log     \sec   \tan
\arg      \coth   \dim   \inf   \liminf  \max     \sin   \tanh
\end{verbatim}

\begin{example}
\[\lim_{n \rightarrow 0}
\frac{\sin x}{x}=1\]
\end{example}

Para la  \wi{función módulo} existen dos  órdenes distintas: \ci{bmod}
para el  operador binario, como en  ``$a \bmod b$'', y  \ci{pmod} para
expresiones como ``$x\equiv a \pmod{b}$''.

Un                       \textbf{\wi{quebrado}}                      o
\textbf{fracción}\index{\LaTeX!fraccion@fracción}   se  pone   con  la
orden  \ci{frac}\verb|{...}{...}|.  Para  los  quebrados  sencillos  a
veces   suele   ser   preferible  utilizar   el   operador   \verb|/|,
%\index{\LaTeX!/@\verb|/|} como en $1/2$.

\begin{example}
$1\frac{1}{2}$~horas
\begin{displaymath}
\frac{ x^2 }{ k+1 }\qquad
x^{ \frac{2}{k+1} }\qquad
x^{ 1/2 }
\end{displaymath}
\end{example}

Los \textbf{\wi{coeficientes de los binomios}} y estructuras similares
se pueden componer con  la instrucción \verb|{... |\ci{choose}% \verb|
{...}| o  \verb|{... |\ci{atop}\verb| ...}|.  Con la segunda  orden se
consigue lo mismo pero sin paréntesis.

\begin{example}
\begin{displaymath}
{n \choose k}\qquad {x \atop y+2}
\end{displaymath}
\end{example}
 
\medskip

El  \textbf{\wi{signo  de integral}}  se  obtiene  con \ci{int}  y  el
\textbf{\wi{signo de sumatorio}} con  \ci{sum}. Los límites superior e
inferior se indican con~\verb|^| y~\verb|_|, tal como se hace para los
superíndices y subíndices.

\begin{example}
\begin{displaymath}
\sum_{i=1}^{n} \qquad
\int_{0}^{\frac{\pi}{2}} \qquad
\end{displaymath}
\end{example}

Para    las    \textbf{\wi{llaves}}   y    otros    \wi{delimitadores}
tenemos     todos    los     tipos    de     símbolos    de     \TeX{}
(p.~ej.~$[\;\langle\;\|\;\updownarrow$).   Los    paréntesis   y   los
corchetes se  introducen con  las teclas correspondientes,  las llaves
con \verb|\{|  y \verb|\}|,  y el  resto con  instrucciones especiales
(p.~ej.~\verb|\updownarrow|).

\begin{example}
\begin{displaymath}
{a,b,c}\neq\{a,b,c\}
\end{displaymath}
\end{example}

Para que \LaTeX\ elija de modo  automático el tamaño apropiado se pone
la orden  \ci{left} delante del  delimitador de apertura  y \ci{right}
delante del  que cierra.  Debemos tener en  cuenta que  debemos cerrar
cada \ci{left} con el \ci{right} correspondiente. Si no deseas nada en
la derecha, entonces emplea `\ci{right.}'.


\begin{example}
\begin{displaymath}
1 + \left( \frac{1}{ 1-x^{2} }
    \right) ^3
\end{displaymath}
\end{example}


En     algunos     casos     es    necesario     fijar     de     modo
explícito      el      tamaño       correcto      del      delimitador
matemático\index{\LaTeX!delimitador!matemático}.  Para esto  se pueden
utilizar   las   instrucciones   \ci{big},   \ci{Big},   \ci{bigg}   y
\ci{Bigg}   como  prefijos   de   la  mayoría   de   las  órdenes   de
delimitadores\footnote{Estas  instrucciones  pueden no  funcionar  del
modo  deseado  si  hemos  utilizado  una  instrucción  de  cambio  del
tamaño del  tipo, o si  se ha  especificado la opción  \texttt{11pt} o
\texttt{12pt}. Los  paquetes \pai{exscale} o \pai{amstex}  sirven para
corregir esta anomalía.}.

\begin{example}
$\Big( (x+1) (x-1) \Big) ^{2}$\\
$\big(\Big(\bigg(\Bigg($\quad
$\big\}\Big\}\bigg\}\Bigg\}$\quad
$\big\|\Big\|\bigg\|\Bigg\|$
\end{example}

Para  poner  los  \textbf{\wi{puntos  suspensivos}}  en  una  ecuación
existen varias órdenes. \ci{ldots} coloca  los puntos en la línea base
y \ci{cdots} los  pone en la zona media del  renglón. Ademas de éstos,
también existen las instrucciones  \ci{vdots} para puntos verticales y
\ci{ddots} para puntos en diagonal.

\index{\LaTeX!puntos suspensivos!verticales}%
\index{\LaTeX!puntos suspensivos!en diagonal}%
\index{\LaTeX!puntos suspensivos!horizontales}%
En el apartado \ref{sec:vert} existe otro ejemplo.

\begin{example}
\begin{displaymath}
x_{1},\ldots,x_{n} \qquad
x_{1}+\cdots+x_{n}
\end{displaymath}
\end{example}
 
\subsection{Espaciado en modo matemático}

\index{\LaTeX!espaciado   en   modo   matematico@espaciado   en   modo
matemático} Si  no estamos satisfechos  con los espaciados  que \TeX{}
elige dentro de una fórmula, éstos se pueden alterar con instrucciones
especiales.  Las  más  importantes  son \ci{,}  para  un  espacio  muy
pequeño,  %  \verb*.\ .  para  una  mediana  (\verb*. .  significa  un
carácter en blanco), \ci{quad} y  \ci{qquad} para espaciados grandes y
\verb|\!|\cih{"!} para la disminución de una separación.

\begin{example}
\newcommand{\rd}{\mathrm{d}}
\begin{displaymath}
\int\!\!\!\int_{D} g(x,y)
  \, \rd x\, \rd y
\end{displaymath}
en lugar de
\begin{displaymath}
\int\int_{D} g(x,y)\rd x \rd y
\end{displaymath}
\end{example}

Observemos  que  la  `d'{}  en  la  diferencial  se  compone  de  modo
convencional  en   redondilla\footnote{En  este  ejemplo  la   `d'  en
redondilla se ha introducido a través de la orden \texttt{\bs rd}, que
previamente se  ha definido con \texttt{\bs  newcommand\{\bs rd\}\{\bs
mathrm\{d\}\}}.  De  esta  forma   se  evita  estar  introduciendo  la
secuencia \texttt{\bs mathrm\{d\}} repetidamente.}.


\subsection{Colocación de signos encima de otros}
\label{sec:vert}

Para  componer  \textbf{matrices}  y  similares se  tiene  el  entorno
\ei{array}. Éste funciona de modo similar al entorno \texttt{tabular}.
Para dividir los renglones se utiliza la instrucción \verb|\\|.

\begin{example}
\begin{displaymath}
\mathbf{X} =
\left( \begin{array}{ccc}
x_{11} & x_{12} & \ldots \\
x_{21} & x_{22} & \ldots \\
\vdots & \vdots & \ddots
\end{array} \right)
\end{displaymath}
\end{example}

También se puede usar el  entorno \ei{array} para componer expresiones
de  funciones  que  tienen  ``\verb|.|''  como  delimitador  invisible
derecho, es decir, \ci{right}\verb|.|.

\begin{example}
\begin{displaymath}
y = \left\{ \begin{array}{ll}
 a & \textrm{si $d>c$}\\
 b+x & \textrm{por la mañana}\\
 l & \textrm{el resto del día}
  \end{array} \right.
\end{displaymath}
\end{example}

Para las ecuaciones que ocupen varios renglones o para los sistemas de
ecuaciones \index{\LaTeX!sistema de ecuaciones}  se pueden emplear los
entornos \ei{eqnarray}  y \verb|eqnarray*|. En  \texttt{eqnarray} cada
renglón contiene  un número  de ecuación.  Con \verb|eqnarray*|  no se
produce ninguna numeración.

Los entornos  \texttt{eqnarray} y \verb|eqnarray*| funcionan  como una
tabla de 3 columnas con  la disposición \verb|{rcl}|, donde la columna
central se utiliza para el  signo de igualdad, desigualdad o cualquier
otro signo que deba ir. La instrucción \verb|\\| divide los renglones.

\begin{example}
\begin{eqnarray}
f(x) & = & \cos x       \\
f'(x) & = & -\sin x     \\
\int_{0}^{x} f(y) \mathrm{d}y &
 = & \sin x
\end{eqnarray}
\end{example}

\noindent Observemos  que existe demasiado  espacio a cada lado  de la
columna central,  donde se encuentran  los signos. Para  reducir estas
separaciones se puede  emplear \verb|\setlength\arraycolsep{2pt}| como
en el ejemplo siguiente.

\index{\LaTeX!ecuaciones largas} Las  \textbf{ecuaciones largas} no se
dividen automáticamente.  Somos nosotros  los que determinamos  en qué
lugares se deben fraccionar y cuánto  se debe sangrar. Los dos métodos
siguientes son las variantes más utilizadas para esto.

\begin{example}
{\setlength\arraycolsep{2pt}
\begin{eqnarray}
\sin x & = & x -\frac{x^{3}}{3!}
     +\frac{x^{5}}{5!}-{}
                    \nonumber\\
 & & {}-\frac{x^{7}}{7!}+{}\cdots
\end{eqnarray}}
\end{example}
\pagebreak[1]

\begin{example}
\begin{eqnarray}
\lefteqn{ \cos x = 1
     -\frac{x^{2}}{2!} +{} }
                    \nonumber\\
 & & {}+\frac{x^{4}}{4!}
     -\frac{x^{6}}{6!}+{}\cdots
\end{eqnarray}
\end{example}

\enlargethispage{\baselineskip}  La  instrucción \ci{nonumber}  impide
que  \LaTeX{} coloque  un  número  para la  ecuación  en  la que  está
colocada la orden.


\subsection{Tamaño del tipo para ecuaciones}

\index{\LaTeX!tamano del  tipo@tamaño del tipo!para ecuaciones}  En el
modo  matemático  \TeX{}  selecciona  el  tamaño  del  tipo  según  el
contexto.  Los superíndices,  por ejemplo,  se  ponen en  un tipo  más
pequeño. Si quiere introducir un texto en redondilla en una ecuación y
utilizamos la  instrucción \verb|\textrm|, el mecanismo  de cambio del
tamaño del tipo  no funcionará, ya que \verb|\textrm|  conmuta de modo
temporal al  modo de  texto. Entonces  se debe  emplear \verb|\mathrm|
para que  se mantenga activo  el mecanismo  de cambio de  tamaño. Pero
hemos de  tener cuidado, ya  que \ci{mathrm} sólo funcionará  bien con
cosas pequeñas. Los  espacios no son aún activos y  los caracteres con
acentos  no funcionan\footnote{El  paquete  AMS-\LaTeX{}  hace que  la
orden \ci{textrm} funcione bien con el cambio de tamaños.}.

\begin{example}
\begin{equation}
2^\textrm{o} \quad 
2^\mathrm{o}
\end{equation}
\end{example}

Sin  embargo,   a  veces   resulta  necesario  indicarle   a  \LaTeX{}
el    tamaño   del    tipo   correcto.    En   modo    matemático   el
tamaño   del   tipo   se    fija   con   las   cuatro   instrucciones:
\begin{flushleft}       \ci{displaystyle}~($\displaystyle       123$),
\ci{textstyle}~($\textstyle   123$),   \ci{scriptstyle}~($\scriptstyle
123$)     y     \ci{scriptscriptstyle}~($\scriptscriptstyle     123$).
\end{flushleft}

El cambio de estilo también afecta al modo de presentar los límites.


\begin{example}
\begin{displaymath}
\mathrm{corr}(X,Y)= 
 \frac{\displaystyle 
   \sum_{i=1}^n(x_i-\bar x)
   (y_i-\bar y)} 
  {\displaystyle\sqrt{
 \sum_{i=1}^n(x_i-\bar x)^2
\sum_{i=1}^n(y_i-\bar y)^2}}
\end{displaymath}    
\end{example}
% Esto no es un acento matemático, y ningún libro de Matemáticas lo
% compondría de este modo.
% mathop produce el espaciado correcto.
 
 
\noindent Éste es uno de los ejemplos en los que se necesitan
corchetes mayores que los normalizados que proporciona %
\verb|\left[| y \verb|\right]|.


\subsection{Descripción de variables}

Para algunas  de sus  ecuaciones podríamos  querer añadir  una sección
donde  se describan  las variables  utilizadas. Para  esto nos  podría
servir de ayuda el siguiente ejemplo:

\begin{example}
\begin{displaymath}
a^2+b^2=c^2
\end{displaymath}
{\settowidth{\parindent}
   {donde:\ }

\makebox[0pt][r]
 {donde:\ }$a$, $b$ son  
los adjuntos del ángulo recto
de un triángulo rectángulo.

$c$ es la hipotenusa
del triángulo}
\end{example}

Si   necesitamos   componer  a   menudo   segmentos   de  texto   como
éste,    ahora   es    el   momento    idóneo   para    practicar   la
instrucción   \verb|\newenvironment|.    Es   recomendable   emplearla
para    crear     un    entorno    especializado     para    describir
variables.\index{\LaTeX!descripción    de    variables}   Revisa    la
descripción al final del tema anterior.

\subsection{Teoremas, leyes\ldots}

Cuando se  escriben documentos matemáticos,  probablemente necesitemos
un  modo  para  componer ``lemas'',  ``definiciones'',  ``axiomas''  y
estructuras similares. \LaTeX{} facilita esto con la orden

\begin{command}
\ci{newtheorem}\verb|{|\emph{nombre}\verb|}[|\emph{contador}\verb|]{|%
         \emph{texto}\verb|}[|\emph{sección}\verb|]|
\end{command}

El argumento \emph{nombre}  es una palabra clave corta  que se utiliza
para  identificar el  ``teorema''.  Con el  argumento \emph{texto}  se
define el nombre del ``teorema'' que aparecerá en el documento final.

Los argumentos entre  corchetes son opcionales. Ambos  se emplean para
especificar  la  numeración  utilizada  para el  ``teorema''.  Con  el
argumento \emph{contador} se puede  especificar el \emph{nombre} de un
``teorema'' declarado  previamente. El  nuevo ``teorema''  se numerará
con  la  misma  secuencia.  El argumento  \emph{sección}  nos  permite
indicar la  unidad de sección con  la que te gustaría  numerar nuestro
``teorema''.

Tras  ejecutar  la instrucción  \ci{newtheorem}  en  el preámbulo  del
documento, dentro del texto se puede usar la instrucción siguiente:

\begin{code}
\verb|\begin{|\emph{nombre}\verb|}[|\emph{texto}\verb|]|\\
Este es un teorema interesante\\
\verb|\end{|\emph{nombre}\verb|}|     
\end{code}

He aquí otro ejemplo de las posibilidades de este entorno:

\begin{example}
% Definiciones para el documento.
% Preámbulo
\newtheorem{ley}{Ley}
\newtheorem{jurado}[ley]{Jurado}
% En el documento
\begin{ley} \label{law:box}
No se esconda en la caja testigo
\end{ley}
\begin{jurado}[Los doce]
Podría ser Vd. Por tanto, tenga
cuidado y vea la ley
\ref{law:box}\end{jurado}
\begin{ley}No, No, No\end{ley}
\end{example}

El teorema ``Jurado'' emplea el mismo contador que el teorema ``Ley''.
Por  ello,  toma  un  número  que está  en  secuencia  con  las  otras
``Leyes''.  El argumento  que  está entre  corchetes  se utiliza  para
especificar un título o algo parecido para el teorema.

\begin{example}
\newtheorem{mur}{Ley de Murphy}[section]
\begin{mur} Si algo puede ir mal,
irá mal.
\end{mur}
\end{example}

El teorema ``Ley de Murphy'' obtiene  un número que está ligado con el
apartado actual.  También se  podría utilizar  otra unidad,  como, por
ejemplo, un capítulo o un subapartado.

\subsection{Símbolos en negrita}
\index{\LaTeX!simbolos en negrita@símbolos en negrita}

Es  bastante   difícil  obtener  símbolos  en   negrita  en  \LaTeX\@.
Probablemente esto sea  intencionado ya que los  compositores de texto
aficionados tienden  a abusar  de ellos.  La orden  de cambio  de tipo
\verb|\mathbf| produce  letras en negrita, pero  éstas son redondillas
mientras que  los símbolos  matemáticos normalmente van  en versalita.
Existe una orden \ci{boldmath}, pero \emph{sólo se puede emplear fuera
del modo matemático}. También funciona con los símbolos.

\begin{example}
\begin{displaymath}
\mu, M \qquad \mathbf{M} \qquad
\mbox{\boldmath $\mu, M$}
\end{displaymath}
\end{example}

\noindent Observa que  la coma también está en negrita,  lo cual puede
que a veces no los creas adecuado.

El paquete  \pai{amsbsy} (incluido por \pai{amsmath})  hace esto mucho
más  fácil. Incluye  una  orden \ci{boldsymbol}  y  una ``negrita  del
hombre  pobre'' \ci{pmb}  (``\emph{poor man's  bold}''), que  opera de
forma análoga a  las máquinas de escribir, que para  poner un texto en
negrita se escribe encima del texto ya escrito.

\ifx\boldsymbol\undefined\else
\begin{example}
\begin{displaymath}
\mu, M \qquad
\boldsymbol{\mu}, \boldsymbol{M}
\qquad \pmb{\mu}, \pmb{M}
\end{displaymath}
\end{example}
\fi



\section{Especialidades}
\begin{intro}
Llegados a este punto, si  yanos sentimos lo sucifientemente seguro de
nosotros  mismos,  entonces  ahora  ya  podemos  comenzar  a  escribir
nuestros documentos  en \LaTeX.  El propósito de  este tema  es añadir
algunas  `especias'\  a  nuestros   conocimientos  de  \LaTeX.  En  el
{\normalfont\manual{}} y  {\normalfont \companion}  podremos encontrar
una descripción más  completa de las especialidades y  de las posibles
mejoras que podemos realizar con \LaTeX.
\end{intro}

\subsection{Tipos y tamaños}

\index{\LaTeX!tipo}\index{\LaTeX!tamaño  del tipo}  \LaTeX{} elige  el
tipo y  el tamaño de los  tipos basándose en la  estructura lógica del
documento (apartados, notas al  pie\ldots). En algunos casos podríamos
querer cambiar  directamente los  tipos y  los tamaños.  Para realizar
esto  se  pueden  usar  las instrucciones  de  las  tablas~\ref{fonts}
y~\ref{sizes}. El  tamaño real de  cada tipo  es cuestión de  diseño y
depende de la clase de documento y de sus opciones.

\begin{example}
{\small Los pequeños y
\textbf{gordos} romanos dominaron}
{\Large toda la grande
\textit{Italia}.}
\end{example}

Una característica importante de \LaTeXe{} es que los atributos de los
tipos  son  independientes.  Esto  significa que  se  puede  llamar  a
instrucciones  de cambio  de tamaño  o incluso  de tipo  y aún  así se
mantienen los  atributos de negrita  o inclinado que  se establecieron
previamente.

En el \emph{modo matemático} se pueden emplear instrucciones de cambio
de  tipos  para  salir  temporalmente  del  \emph{modo  matemático}  e
introducir texto normal. Si para componer las ecuaciones prefiriésemos
utilizar otro tipo  existe un conjunto especial  de instrucciones para
ello. Consulta para esto la tabla~\ref{mathfonts}.

\begin{table}[!bp]
\caption{Tipos} \label{fonts}
\begin{lined}{12cm}
%
\begin{tabular}{@{}rl@{\qquad}rl@{}}
\ci{textrm}\verb|{...}|        &  \textrm{\wi{redonda}}&
\ci{textsf}\verb|{...}|        &  \textsf{\wi{sin línea de pie}}\\
\ci{texttt}\verb|{...}|        &  \texttt{de máquina}\\
                               &  \texttt{de escribir}&
                               &                         \\[6pt]
\ci{textmd}\verb|{...}|        &  \textmd{media}&
\ci{textbf}\verb|{...}|        &  \textbf{\wi{negrita}}\\[6pt]
\ci{textup}\verb|{...}|        &  \textup{\wi{vertical}}&
\ci{textit}\verb|{...}|        &  \textit{\wi{itálica}}\\
\ci{textsl}\verb|{...}|        &  \textsl{\wi{inclinada}}&
\ci{textsc}\verb|{...}|        &  \textsc{\wi{versalita}}\\[6pt]
\ci{emph}\verb|{...}|          &  \emph{resaltada} &
\ci{textnormal}\verb|{...}|    &  tipo del\\
 & & & \textnormal{documento}
\end{tabular}

\bigskip
\end{lined}
\end{table}


\begin{table}[!bp]
\index{\LaTeX!tamaños del tipo}
\caption{Tamaños de los tipos} \label{sizes}
\begin{lined}{12cm}
\begin{tabular}{@{}ll}
\ci{tiny}      & \tiny            letra diminuta \\
\ci{scriptsize}   & \scriptsize   letra muy pequeña\\
\ci{footnotesize} & \footnotesize letra bastante pequeña \\
\ci{small}        &  \small       letra pequeña \\
\ci{normalsize}   &  \normalsize  letra normal \\
\ci{large}        &  \large       letra grande
\end{tabular}%
\qquad\begin{tabular}{ll@{}}
\ci{Large}        &  \Large       letra mayor \\[5pt]
\ci{LARGE}        &  \LARGE       muy grande \\[5pt]
\ci{huge}         &  \huge        enorme \\[5pt]
\ci{Huge}         &  \Huge        la mayor
\end{tabular}

\bigskip
\end{lined}
\end{table}

\begin{table}[!bp]
\caption{Tipos matemáticos} \label{mathfonts}
\begin{lined}{12.2cm}
\begin{tabular}{@{}lll@{}}
\textit{Orden}&\textit{Ejemplo}&    \textit{Resultado}\\[6pt]
\ci{mathcal}\verb|{...}|&    \verb|$\mathcal{B}=c$|&     $\mathcal{B}=c$\\
\ci{mathrm}\verb|{...}|&     \verb|$\mathrm{K}_2$|&      $\mathrm{K}_2$\\
\ci{mathbf}\verb|{...}|&     \verb|$\sum x=\mathbf{v}$|& $\sum x=\mathbf{v}$\\
\ci{mathsf}\verb|{...}|&     \verb|$\mathsf{G\times R}$|&        $\mathsf{G\times R}$\\
\ci{mathtt}\verb|{...}|&     \verb|$\mathtt{L}(b,c)$|&   $\mathtt{L}(b,c)$\\
\ci{mathnormal}\verb|{...}|& \verb|$\mathnormal{R_1}=R_1$|&      $\mathnormal{R_1}=R_1$\\
\ci{mathit}\verb|{...}|&     \verb|$eficaz\neq\mathit{eficaz}$|& $eficaz\neq\mathit{eficaz}$
\end{tabular}

\bigskip
\end{lined}
\end{table}

Conjuntamente con las  instrucciones de los tamaños de  los tipos, las
\wi{llaves} juegan un papel  significativo. Se utilizan para construir
agrupaciones o \emph{grupos}. Los \wi{grupo}s  limitan el ámbito de la
mayoría de las instrucciones de \LaTeX.

\begin{example}
A él le gustan las {\LARGE
letras grandes y las letras
{\small pequeñas}}.
\end{example}

Las  instrucciones de  tamaño del  tipo también  alteran el  espaciado
entre renglones, pero sólo si el  párrafo termina dentro del ámbito de
la orden de tamaño del tipo. Por  ello, la llave de cierre \verb|}| no
debería aparecer  antes de  lo indicado. Fijémonos  la posición  de la
instrucción \verb|\par| en los dos ejemplos siguientes.

\begin{example}
{\Large ¡No leas esto! No es
cierto. ¡Créeme!\par}
\end{example}
\begin{example}
{\Large Esto no es cierto.
Pero recuerda que digo
mentiras.}\par
\end{example}

Para concluir este viaje al mundo de los tipos y los tamaños de tipos,
debemos tener presente un pequeño consejo:

\nopagebreak
\begin{quote}
  \underline{\textbf{¡Recuerda\Huge!}} que \textit{Cuántos}
  \textsf{M\textbf{\LARGE Á} \textsl{S}} tipos \Huge utilices 
  \footnotesize \textbf{en} un \small \texttt{documento}\Huge,
  \large \textit{más} \normalsize \textsc{legible} y
  \textsl{\textsf{agradable} resul\large t\Large a\LARGE r\huge
    á}.\footnote{¡Ojo!, que se trata de una pequeña sátira.
    ¡Espero que te des cuenta!}
\end{quote}

\subsection{Separaciones}
 
\subsubsection{Separaciones entre renglones}

\index{\LaTeX!separaciones  entre   renglones}  Si   queremos  emplear
mayores  separaciones  entre  renglones,   podemos  cambiar  su  valor
poniendo la orden

\begin{command}
\ci{linespread}\verb|{|\emph{factor}\verb|}|
\end{command}

\noindent    en    el    preámbulo   de    su    documento.    Utiliza
\verb|\linespread{1.3}|   para   textos   a    espacio   y   medio   y
\verb|\linespread{1.6}| para  textos a doble espacio.  Normalmente los
renglones  no  se  separan  tanto,  por  lo  que,  a  no  ser  que  se
indique  otra cosa,  el factor  de separación  entre renglones  es~1.%
\index{\LaTeX!doble espacio}

\subsubsection{Diseño de los párrafos}\label{parsp}

En \LaTeX{} existen  dos parámetros que influyen en el  formato de los
párrafos. Si se pone una definición como

\begin{code}
\ci{setlength}\verb|{|\ci{parindent}\verb|}{0pt}| \\
\verb|\setlength{|\ci{parskip}\verb|}{1ex plus 0.5ex minus 0.2ex}|
\end{code}

\noindent en    el     preámbulo    del    fichero     de    entrada\footnote{El
preámbulo     es     la     zona    del     fichero     de     entrada
entre     las     instrucciones     $\backslash$\texttt{documentclass}
y   $\backslash$\texttt{begin$\mathtt{\{}$document$\mathtt{\}}$}.}  se
puede  cambiar el  aspecto de  los párrafos.  Estas dos  líneas pueden
aumentar el espacio entre dos párrafos  y dejarlos sin sangrías. En la
Europa continental, a menudo se separan los párrafos con algún espacio
y no  se le  pone sangría.  Pero hay  que tener  cuidado, ya  que esto
también  tiene efecto  en  el índice  general,  haciendo que  nuestras
líneas queden más separadas.

Si  quisioérmos sangrar  un  párrafo que  no  tiene sangría,  entonces
utilizaremos

\begin{command}
\ci{indent}
\end{command}

\noindent  al comienzo  del  párrafo\footnote{Para  sangrar el  primer
párrafo  después  de cada  cabecera  de  apartado, emplea  el  paquete
\pai{indentfirst} del conjunto `tools'.}.  Esto sólo funcionará cuando
\verb|\parindent| no esté puesto a cero.

Para crear un párrafo sin sangría se  emplea

\begin{command}
\ci{noindent}
\end{command}

\noindent como  primera orden del  párrafo. Esto podría  resultar útil
cuando comencemos un documento con  texto y sin ninguna instrucción de
seccionado.

\subsubsection{Separaciones horizontales}

\LaTeX{} determina  automáticamente las separaciones entre  palabras y
oraciones. Para producir otras separaciones horizontales se utiliza:

\index{\LaTeX!espacio!horizontal}

\begin{command}
\ci{hspace}\verb|{|\emph{longitud}\verb|}|
\end{command}

Cuando  se debe  producir una  separación  como ésta,  incluso si  cae
al  final  o  al  comienzo   de  un  renglón,  emplea  \verb|\hspace*|
en  vez  de \verb|\hspace|.  La  indicación  de la  distancia  consta,
en  el   caso  más   simple,  de   un  número   más  una   unidad.  En
la  tabla~\ref{units}   encontrarás  las  unidades   más  importantes.
\index{\LaTeX!unidades}\index{\LaTeX!dimensiones}

\begin{example}
Este\hspace{1.5cm}es un espacio
de 1.5 cm.
\end{example}
\suppressfloats
\begin{table}[tbp]
\caption{Unidades de \TeX} \label{units}\index{\LaTeX!unidades}
\begin{lined}{9.5cm} 
\begin{tabular}{@{}ll@{}}
\texttt{mm} &  milímetro $\approx 1/25$~pulgada \quad \demowidth{1mm} \\
\texttt{cm} & centímetro = 10~mm  \quad \demowidth{1cm}                    \\
\texttt{in} & pulgada $\approx$ 25~mm \quad \demowidth{1in}                 \\
\texttt{pt} & punto $\approx 1/72$~pulgada $\approx \frac{1}{3}$~mm  \quad\demowidth{1pt}\\
\texttt{em} & aprox.{} el ancho de una \texttt{m} en el tipo actual \quad \demowidth{1em}\\
\texttt{ex} & aprox.{} la altura de una \texttt{x} en el tipo actual \quad \demowidth{1ex}
\end{tabular}

\bigskip
\end{lined}
\end{table}
 
La instrucción
\begin{command}
\ci{stretch}\verb|{|\emph{n}\verb|}|
\end{command} 

\noindent produce  una separación  especial elástica. Se  alarga hasta
que el espacio que resta en  un renglón se llena. Si dos instrucciones
\verb|\hspace{\stretch{|\emph{n}\verb|}}|   aparecen   en   el   mismo
renglón, los espaciados crecen según sus `factores de alargamiento'.

\begin{example}
x\hspace{\stretch{1}}
x\hspace{\stretch{3}}x
\end{example}

\subsubsection{Separaciones verticales especiales}

\LaTeX{}  determina  de modo  automático  las  separaciones entre  dos
párrafos, apartados, subapartados\ldots\ En casos especiales se pueden
forzar separaciones adicionales \emph{entre dos párrafos} con la orden

\begin{command}
\ci{vspace}\verb|{|\emph{longitud}\verb|}|
\end{command}

Esta  orden se  debería indicar  siempre entre  dos renglones  vacíos.
Cuando esta se\-pa\-ra\-ción  se debe introducir aunque vaya  al principio o
al final  de una  página, entonces  en vez  de \verb|\vspace|  se debe
utilizar \verb|\vspace*|. \index{\LaTeX!separación vertical}

Se  puede   utilizar  la   orden  \verb|\stretch|   conjuntamente  con
\verb|\pagebreak| para llevar texto al  borde inferior de una página o
para centrarlo verticalmente.

\begin{code}
\begin{verbatim}
Algo de texto \ldots

\vspace{\stretch{1}}
Esto va en el último renglón de la página.\pagebreak
\end{verbatim}
\end{code}

Las  separaciones  adicionales  entre dos  renglones  \emph{del  mismo
párrafo} o dentro de una tabla se consiguen con la orden

\begin{command}
\ci{\bs}\verb|[|\emph{longitud}\verb|]|
\end{command}

\subsection{Inclusión de gráficos EPS}

Con los entornos \texttt{figure} y \texttt{table} \LaTeX{} proporciona
las facilidades básicas para trabajar con objetos flotantes, entre los
que se incluyen las imágenes y los gráficos.

Existen varias alternativas para generar \wi{gráficos} con el \LaTeX{}
básico o un  paquete de extensiones de \LaTeX{}  mediante órdenes. Por
desgracia,  la mayoría  de los  usuarios los  encuentran difíciles  de
entender. Por  esto, no  las vamos  a explicar  aquí. Si  realmente se
desea buscar  más información sobre  este particular siempre  se puede
echar mano de  textos más específicos y extensos,  como \companion{} y
el \manual.

Un   modo   más  sencillo   de   poner   gráficos  en   un   documento
es    produciéndolos    con     un    paquete    de    \emph{software}
especializado\footnote{Tales como XFig, CorelDraw!, Freehand, Gnuplot,
Tgif, Paint  Shop Pro, Gimp\ldots}  e incluir los gráficos  dentro del
documento. En este punto, también hay paquetes de \LaTeX{} que ofrecen
muchas alternativas.  En esta descripción  sólo mostraremos el  uso de
gráficos en \wi{PostScript Encapsulado} (EPS), ya que es un método muy
sencillo  y utilizado  ampliamente. Para  utilizar dibujos  en formato
EPS,  debemos  disponer  una  impresora  \wi{PostScript}\footnote{Otra
posibilidad   para  imprimir   PostScript  es   con  el   programa  de
GNU  \textsc{\wi{GhostScript}},   que  puede  encontrar,   p.~ej.,  en
\texttt{CTAN:/tex-archive/support/ghostscript}.} para imprimir.

Un    buen    conjunto   de    órdenes    para    la   inclusión    de
gráficos    se    proporciona    con   el    paquete    \pai{graphicx}
de    D.~P.~Carlisle.    Forma    parte   de    todo    un    conjunto
de    paquetes    que    se   llama    el    conjunto    ``graphics''%
\footnote{\texttt{CTAN:/tex-archive/macros/latex/packages/graphics}.}.

Suponiendo que dispongamos de una impresora PostScript para imprimir y
utilicemos el paquete \textsf{graphicx},  se puede seguir la siguiente
lista de pasos para incluir un dibujo dentro de nuestros documentos:

\begin{enumerate}
\item Exportar el dibujo desde tu programa de gráficos en formato EPS.
\item Cargar el paquete \textsf{graphicx} en el preámbulo del fichero
  de entrada con
\begin{command}
\verb|\usepackage[|\emph{driver}\verb|]{graphicx}|
\end{command}

\emph{driver}  es  el  nombre  de   su  conversor  ``de  \emph{dvi}  a
PostScript''\footnote{El  programa más  utilizado para  esto se  llama
\texttt{dvips}.}.  El  paquete  necesita esta  información  porque  la
inclusión de los gráficos la realiza el \emph{driver} de la impresora.
Una vez que se conozca  el \emph{driver}, el paquete \textsf{graphicx}
inserta las órdenes correctas en el fichero~\texttt{.dvi} para incluir
el gráfico que se desea con el \emph{driver} de impresora.


\item Utilizar la orden

\begin{command}
\ci{includegraphics}\verb|[|\emph{clave}=\emph{valor},
 \ldots\verb|]{|\emph{fichero}\verb|}|
\end{command}
para incluir \emph{fichero} en tus documentos. El parámetro opcional
acepta una lista de \emph{claves} separadas por comas y sus
\emph{valores} asociados. Las \emph{claves} se pueden emplear para
modificar el ancho, la altura y el giro del gráfico incluido. En la
tabla~\ref{keyvals} puedes encontrar las claves más importantes.
\end{enumerate}

\begin{table}[htb]
\caption{Nombres de las claves para el paquete \textsf{graphicx}}
\label{keyvals}
\begin{lined}{10.1cm}
\begin{tabular}{@{}ll}
\texttt{width}& escalado gráfico al ancho indicado\\
\texttt{height}& escalado gráfico a la altura indicada\\
\texttt{angle}& giro del gráfico en el sentido de las agujas del reloj
\end{tabular}

\bigskip
\end{lined}
\end{table}

El siguiente ejemplo podrá ayudar a aclarar algunas de estas ideas:

\begin{code}
\begin{verbatim}
\begin{figure}
\begin{center}
\includegraphics[angle=90, width=10cm]{test.eps}
\end{center}
\end{figure}
\end{verbatim}
\end{code}

Este  código  inserta  el  gráfico  que se  encuentra  en  el  fichero
\texttt{test.eps}.  El gráfico  se  gira  \emph{primero} 90$^\circ$  y
\emph{después} se escala hasta lograr los 10\,cm de ancho. La relación
de aspecto es de 1.0 porque no se ha indicado ninguna altura especial.

Para buscar  más información  sobre este  paquete y  sus posibilidades
siempre podemos acudir a su documentación en~\cite{graphics}.

\section{Herramientas de \LaTeX}

Copiamos esta fórmula en algún  fichero, por ejemplo {\tt ejemplo.tex}
y veamos los pasos necesarios  para {\em compilarlo}. Para compilar el
fuente  lo  único que  hay  que  hacer  es ejecutar  {\LaTeX}  seguido
del  nombre  del fichero  fuente  {\LaTeX}.  En nuestro  ejemplo,  nos
colocaremos  en el  directorio donde  tenemos el  {\tt ejemplo.tex}  y
teclearemos:

\begin{verbatim}
$ latex ejemplo.tex
\end{verbatim}

Esto generará varios ficheros, pero el que realmente interesa en estos
momentos  es {\tt  ejemplo.dvi}. Este  fichero es  el resultado  de la
compilación y lo podemos ver usando por ejemplo {\tt xdvi}. Para verlo
teclearemos en el mismo directorio en donde está el fichero

\begin{verbatim}
$ xdvi ejemplo.dvi
\end{verbatim}

Dado  que  el formato  {\tt  DVI}  (DeVice  Independent) no  está  muy
extendido se  han generado conversores  de formato, de tal  manera que
podemos  pasar  desde dvi  a  formato  {\tt  PostScript} y  a  formato
{\tt  PDF}, ambos  mucho  más extendidos  que el  dvi.  Para pasar  un
fichero generado  por \LaTeX~ en formato  {\tt DVI} a un  fichero {\tt
PostScript} usaremos el comando {\tt dvips} de la siguiente manera:

\begin{verbatim}
$ dvips ejemplo.dvi -o ejemplo.ps
\end{verbatim}

Ahora  podemos  ver  el  resultado con  cualquier  visor  de  ficheros
PostScript, por ejemplo {\tt gv}:

\begin{verbatim}
$ gv ejemplo.ps
\end{verbatim}

Una  vez comprobado  que todo  va  bien intentemos  pasar este  último
fichero en  formato {\tt  PostScript} a formato  {\tt PDF}.  Para ello
usaremos el comando {\tt ps2pdf} como se explica a continuación:

\begin{verbatim}
$ ps2pdf ejemplo.ps ejemplo.pdf
\end{verbatim}

Y una vez hecho esto ya podemos usar el visor de PDF para leer el {\tt
ejemplo.pdf}.

Existe también  la posibilidad  de convertir el  código \LaTeX~  de un
documento  al estándar  HTML, lo  que resulta  muy útil  a la  hora de
publicar en internet.  El programa apropiado para esto es  el {\tt GNU
LaTeX2HTML}, y su uso básico es sumamente sencillo:

\begin{verbatim}
$ latex2html ejemplo.tex
\end{verbatim}

Al terminar  latex2html encontramos  un nuevo directorio  llamado {\tt
ejemplo}  que  contiene  la  conversión  a  HTML  de  nuestro  ejemplo
de  \LaTeX, incluidos  los  gráficos correspondientes  a las  fórmulas
matemáticas.  El resultado  obtenido no  está nada  mal. Además,  esta
conversión  puede  personalizarse en  gran  medida  mediante hojas  de
estilo.

%\begin{verbatim}
%$ ls ejemplo
%ejemplo.css   images.tex  img14.png  img3.png  img8.png    WARNINGS
%ejemplo.html  img10.png   img15.png  img4.png  img9.png
%images.aux    img11.png   img16.png  img5.png  index.html
%images.log    img12.png   img1.png   img6.png  labels.pl
%images.pl     img13.png   img2.png   img7.png  node1.html
%\end{verbatim}

\section{Tablas de símbolos matemáticos}

En las tablas siguientes se indican todos los s'imbolos que
normalmente se pueden utilizar en el \emph{modo matem'atico}.

%
% Texto v'alido en el caso de tener instalados los tipos de la AMS
%
\ifx\noAMS\relax Para usar los s'imbolos de las
tablas~\ref{AMSD}--\ref{AMSNBR}\footnote{Estas tablas provienen de
  \texttt{symbols.tex} y luego se hicieron muchas modificaciones
  seg'un las sugerencias de Josef~Tkadlec}, se debe cargar el paquete
\pai{amssymb} en el pre'ambulo del documento y adem'as deber'an
encontrarse en el sistema los tipos matem'aticos de la \emph{American
  Mathematical Society} (AMS).  Si no est'an instalados el paquete y
los tipos de la AMS, entonces eche un vistazo a\\ 
\centerline{\texttt{CTAN:/tex-archive/macros/latex/packages/amslatex}}\fi


\begin{table}[!h]
\caption{Acentos en modo matem'atico}  \label{mathacc}
\begin{symbols}{*4{cl}}
\W{\hat}{a}     & \W{\check}{a} & \W{\tilde}{a} & \W{\acute}{a} \\
\W{\grave}{a} & \W{\dot}{a} & \W{\ddot}{a}    & \W{\breve}{a} \\
\W{\bar}{a} &\W{\vec}{a} &\W{\widehat}{A}&\W{\widetilde}{A}\\  
\end{symbols}
\end{table}

\begin{table}[!h]
\caption{Letras griegas may'usculas}
\begin{symbols}{*4{cl}}
 \X{\Gamma}     & \X{\Lambda}    & \X{\Sigma}     & \X{\Psi}      \\
 \X{\Delta}     & \X{\Xi}        & \X{\Upsilon}   & \X{\Omega}    \\
 \X{\Theta}     & \X{\Pi}        & \X{\Phi} 
\end{symbols}
\end{table}
\clearpage 
 
\begin{table}[!h]
\caption{Letras griegas min'usculas}
\begin{symbols}{*4{cl}}
 \X{\alpha}     & \X{\theta}     & \X{o}          & \X{\upsilon}  \\
 \X{\beta}      & \X{\vartheta}  & \X{\pi}        & \X{\phi}      \\
 \X{\gamma}     & \X{\iota}      & \X{\varpi}     & \X{\varphi}   \\
 \X{\delta}     & \X{\kappa}     & \X{\rho}       & \X{\chi}      \\
 \X{\epsilon}   & \X{\lambda}    & \X{\varrho}    & \X{\psi}      \\
 \X{\varepsilon}& \X{\mu}        & \X{\sigma}     & \X{\omega}    \\
 \X{\zeta}      & \X{\nu}        & \X{\varsigma}  & &             \\
 \X{\eta}       & \X{\xi}        & \X{\tau} 
\end{symbols}
\end{table}

\begin{table}[!htbp]
\caption{Relaciones}
\bigskip Puede realizar las negaciones correspondientes a estos
s'imbolos a~nadi'endoles una orden \verb|\not| como prefijo a las
instrucciones siguientes.
\begin{symbols}{*3{cl}}
 \X{<}           & \X{>}           & \X{=}          \\
 \X{\leq}o \verb|\le|   & \X{\geq}o \verb|\ge|   & \X{\equiv}     \\
 \X{\ll}         & \X{\gg}         & \X{\doteq}     \\
 \X{\prec}       & \X{\succ}       & \X{\sim}       \\
 \X{\preceq}     & \X{\succeq}     & \X{\simeq}     \\
 \X{\subset}     & \X{\supset}     & \X{\approx}    \\
 \X{\subseteq}   & \X{\supseteq}   & \X{\cong}      \\
 \X{\sqsubset}$^a$ & \X{\sqsupset}$^a$ & \X{\Join}$^a$    \\
 \X{\sqsubseteq} & \X{\sqsupseteq} & \X{\bowtie}    \\
 \X{\in}         & \X{\ni}, \verb|\owns|  & \X{\propto}    \\
 \X{\vdash}      & \X{\dashv}      & \X{\models}    \\
 \X{\mid}        & \X{\parallel}   & \X{\perp}      \\
 \X{\smile}      & \X{\frown}      & \X{\asymp}     \\
 \X{:}           & \X{\notin}      & \X{\neq}o \verb|\ne|
\end{symbols}
\centerline{\footnotesize $^a$Para obtener este s'imbolo emplee el
  paquete \textsf{latexsym}}
\end{table}

\begin{table}[!htbp]
\caption{Operadores binarios}
\begin{symbols}{*3{cl}}
 \X{+}              & \X{-}              & &                 \\
 \X{\pm}            & \X{\mp}            & \X{\triangleleft} \\
 \X{\cdot}          & \X{\div}           & \X{\triangleright}\\
 \X{\times}         & \X{\setminus}      & \X{\star}         \\
 \X{\cup}           & \X{\cap}           & \X{\ast}          \\
 \X{\sqcup}         & \X{\sqcap}         & \X{\circ}         \\
 \X{\vee}, \verb|\lor|     & \X{\wedge}, \verb|\land|  & \X{\bullet}       \\
 \X{\oplus}         & \X{\ominus}        & \X{\diamond}      \\
 \X{\odot}          & \X{\oslash}        & \X{\uplus}        \\
 \X{\otimes}        & \X{\bigcirc}       & \X{\amalg}        \\
 \X{\bigtriangleup} &\X{\bigtriangledown}& \X{\dagger}       \\
 \X{\lhd}$^a$         & \X{\rhd}$^a$         & \X{\ddagger}      \\
 \X{\unlhd}$^a$       & \X{\unrhd}$^a$       & \X{\wr}
\end{symbols}
\centerline{\footnotesize $^a$Para obtener este s'imbolo emplee el
  paquete \textsf{latexsym}}
\end{table}

\begin{table}[!htbp]
\caption{Operadores ``grandes''}
\begin{symbols}{*4{cl}}
 \X{\sum}      & \X{\bigcup}   & \X{\bigvee}   & \X{\bigoplus}\\
 \X{\prod}     & \X{\bigcap}   & \X{\bigwedge} &\X{\bigotimes}\\
 \X{\coprod}   & \X{\bigsqcup} & &             & \X{\bigodot} \\
 \X{\int}      & \X{\oint}     & &             & \X{\biguplus}
\end{symbols}
 
\end{table}


\begin{table}[!htbp]
\caption{Flechas}
\begin{symbols}{*3{cl}}
 \X{\leftarrow}o \verb|\gets|& \X{\longleftarrow}     & \X{\uparrow}          \\
 \X{\rightarrow}o \verb|\to|& \X{\longrightarrow}    & \X{\downarrow}        \\
 \X{\leftrightarrow}    & \X{\longleftrightarrow}& \X{\updownarrow}      \\
 \X{\Leftarrow}         & \X{\Longleftarrow}     & \X{\Uparrow}          \\
 \X{\Rightarrow}        & \X{\Longrightarrow}    & \X{\Downarrow}        \\
 \X{\Leftrightarrow}    & \X{\Longleftrightarrow}& \X{\Updownarrow}      \\
 \X{\mapsto}            & \X{\longmapsto}        & \X{\nearrow}          \\
 \X{\hookleftarrow}     & \X{\hookrightarrow}    & \X{\searrow}          \\
 \X{\leftharpoonup}     & \X{\rightharpoonup}    & \X{\swarrow}          \\
 \X{\leftharpoondown}   & \X{\rightharpoondown}  & \X{\nwarrow}          \\
 \X{\rightleftharpoons} & \X{\iff}(espacios mayores)& \X{\leadsto}$^a$

\end{symbols}
\centerline{\footnotesize $^a$Para obtener este s'imbolo emplee el
  paquete \textsf{latexsym}}
\end{table}

\begin{table}[!htbp]
\caption{Delimitadores}\label{tab:delimiters}
\begin{symbols}{*4{cl}}
 \X{(}            & \X{)}            & \X{\uparrow} & \X{\Uparrow}    \\
 \X{[}o \verb|\lbrack|   & \X{]}o \verb|\rbrack|  & \X{\downarrow}   & \X{\Downarrow}  \\
 \X{\{}o \verb|\lbrace|  & \X{\}}o \verb|\rbrace|  & \X{\updownarrow} & \X{\Updownarrow}\\
 \X{\langle}      & \X{\rangle}  & \X{|}o \verb|\vert| &\X{\|}o \verb|\Vert|\\
 \X{\lfloor}      & \X{\rfloor}      & \X{\lceil}       & \X{\rceil}      \\
 \X{/}            & \X{\backslash}   & &. (vac'io dual)
\end{symbols}
\end{table}

\begin{table}[!htbp]
\caption{Delimitadores grandes}
\begin{symbols}{*4{cl}}
 \Y{\lgroup}      & \Y{\rgroup}      & \Y{\lmoustache}  & \Y{\rmoustache} \\
 \Y{\arrowvert}   & \Y{\Arrowvert}   & \Y{\bracevert} 
\end{symbols}
\end{table}


\begin{table}[!htbp]
\caption{S'imbolos diversos}
\begin{symbols}{*4{cl}}
 \X{\dots}       & \X{\cdots}      & \X{\vdots}      & \X{\ddots}     \\
 \X{\hbar}       & \X{\imath}      & \X{\jmath}      & \X{\ell}       \\
 \X{\Re}         & \X{\Im}         & \X{\aleph}      & \X{\wp}        \\
 \X{\forall}     & \X{\exists}     & \X{\mho}$^a$      & \X{\partial}   \\
 \X{'}           & \X{\prime}      & \X{\emptyset}   & \X{\infty}     \\
 \X{\nabla}      & \X{\triangle}   & \X{\Box}$^a$     & \X{\Diamond}$^a$ \\
 \X{\bot}        & \X{\top}        & \X{\angle}      & \X{\surd}      \\
\X{\diamondsuit} & \X{\heartsuit}  & \X{\clubsuit}   & \X{\spadesuit} \\
 \X{\neg}o \verb|\lnot| & \X{\flat}       & \X{\natural}    & \X{\sharp}

\end{symbols}
\centerline{\footnotesize $^a$Para obtener este s'imbolo emplee el
  paquete \textsf{latexsym}}
\end{table}

\begin{table}[!htbp]
\caption{S'imbolos no matem'aticos}
\bigskip
Los siguientes s'imbolos tambi'en se pueden utilizar en modo texto.
\begin{symbols}{*3{cl}}
\SC{\dag} & \SC{\S} & \SC{\copyright}  \\
\SC{\ddag} & \SC{\P} & \SC{\pounds}  \\
\end{symbols}
\end{table}

%
% Si no disponemos del material de AMS, entonces nos saltamos el resto
% :-)
%
\noAMS

\begin{table}[!htbp]
\caption{Delimitadores de la AMS}\label{AMSD}
\bigskip
\begin{symbols}{*4{cl}}
\X{\ulcorner}&\X{\urcorner}&\X{\llcorner}&\X{\lrcorner}
\end{symbols}
\end{table}

\begin{table}[!htbp]
\caption{S'imbolos griegos y hebreos de la AMS}
\begin{symbols}{*5{cl}}
\X{\digamma}     &\X{\varkappa} & \X{\beth}& \X{\daleth}     &\X{\gimel}
\end{symbols}
\end{table}

\begin{table}[!htbp]
\caption{Relaciones binarias de la AMS}
\begin{symbols}{*3{cl}}
 \X{\lessdot}           & \X{\gtrdot}            & \X{\doteqdot}o \verb|\Doteq| \\
 \X{\leqslant}          & \X{\geqslant}          & \X{\risingdotseq}     \\
 \X{\eqslantless}       & \X{\eqslantgtr}        & \X{\fallingdotseq}    \\
 \X{\leqq}              & \X{\geqq}              & \X{\eqcirc}           \\
 \X{\lll}o \verb|\llless|      & \X{\ggg}o \verb|\gggtr| & \X{\circeq}  \\
 \X{\lesssim}           & \X{\gtrsim}            & \X{\triangleq}        \\
 \X{\lessapprox}        & \X{\gtrapprox}         & \X{\bumpeq}           \\
 \X{\lessgtr}           & \X{\gtrless}           & \X{\Bumpeq}           \\
 \X{\lesseqgtr}         & \X{\gtreqless}         & \X{\thicksim}         \\
 \X{\lesseqqgtr}        & \X{\gtreqqless}        & \X{\thickapprox}      \\
 \X{\preccurlyeq}       & \X{\succcurlyeq}       & \X{\approxeq}         \\
 \X{\curlyeqprec}       & \X{\curlyeqsucc}       & \X{\backsim}          \\
 \X{\precsim}           & \X{\succsim}           & \X{\backsimeq}        \\
 \X{\precapprox}        & \X{\succapprox}        & \X{\vDash}            \\
 \X{\subseteqq}         & \X{\supseteqq}         & \X{\Vdash}            \\
 \X{\Subset}            & \X{\Supset}            & \X{\Vvdash}           \\
 \X{\sqsubset}          & \X{\sqsupset}          & \X{\backepsilon}      \\
 \X{\therefore}         & \X{\because}           & \X{\varpropto}        \\
 \X{\shortmid}          & \X{\shortparallel}     & \X{\between}          \\
 \X{\smallsmile}        & \X{\smallfrown}        & \X{\pitchfork}        \\
 \X{\vartriangleleft}   & \X{\vartriangleright}  & \X{\blacktriangleleft}\\
 \X{\trianglelefteq}    & \X{\trianglerighteq}   &\X{\blacktriangleright}
\end{symbols}
\end{table}

\begin{table}[!htbp]
\caption{Flechas de la AMS}
\begin{symbols}{*3{cl}}
 \X{\dashleftarrow}      & \X{\dashrightarrow}     & \X{\multimap}          \\
 \X{\leftleftarrows}     & \X{\rightrightarrows}   & \X{\upuparrows}        \\
 \X{\leftrightarrows}    & \X{\rightleftarrows}    & \X{\downdownarrows}    \\
 \X{\Lleftarrow}         & \X{\Rrightarrow}        & \X{\upharpoonleft}     \\
 \X{\twoheadleftarrow}   & \X{\twoheadrightarrow}  & \X{\upharpoonright}    \\
 \X{\leftarrowtail}      & \X{\rightarrowtail}     & \X{\downharpoonleft}   \\
 \X{\leftrightharpoons}  & \X{\rightleftharpoons}  & \X{\downharpoonright}  \\
 \X{\Lsh}                & \X{\Rsh}                & \X{\rightsquigarrow}   \\
 \X{\looparrowleft}      & \X{\looparrowright}     &\X{\leftrightsquigarrow}\\
 \X{\curvearrowleft}     & \X{\curvearrowright}    & &                      \\
 \X{\circlearrowleft}    & \X{\circlearrowright}   & &
\end{symbols}
\end{table}

\begin{table}[!htbp]
\caption{Relaciones binarias y flechas negadas de la AMS}\label{AMSNBR}
\begin{symbols}{*3{cl}}
 \X{\nless}           & \X{\ngtr}            & \X{\varsubsetneqq}  \\
 \X{\lneq}            & \X{\gneq}            & \X{\varsupsetneqq}  \\
 \X{\nleq}            & \X{\ngeq}            & \X{\nsubseteqq}     \\
 \X{\nleqslant}       & \X{\ngeqslant}       & \X{\nsupseteqq}     \\
 \X{\lneqq}           & \X{\gneqq}           & \X{\nmid}           \\
 \X{\lvertneqq}       & \X{\gvertneqq}       & \X{\nparallel}      \\
 \X{\nleqq}           & \X{\ngeqq}           & \X{\nshortmid}      \\
 \X{\lnsim}           & \X{\gnsim}           & \X{\nshortparallel} \\
 \X{\lnapprox}        & \X{\gnapprox}        & \X{\nsim}           \\
 \X{\nprec}           & \X{\nsucc}           & \X{\ncong}          \\
 \X{\npreceq}         & \X{\nsucceq}         & \X{\nvdash}         \\
 \X{\precneqq}        & \X{\succneqq}        & \X{\nvDash}         \\
 \X{\precnsim}        & \X{\succnsim}        & \X{\nVdash}         \\
 \X{\precnapprox}     & \X{\succnapprox}     & \X{\nVDash}         \\
 \X{\subsetneq}       & \X{\supsetneq}       & \X{\ntriangleleft}  \\
 \X{\varsubsetneq}    & \X{\varsupsetneq}    & \X{\ntriangleright} \\
 \X{\nsubseteq}       & \X{\nsupseteq}       & \X{\ntrianglelefteq}\\
 \X{\subsetneqq}      & \X{\supsetneqq}      &\X{\ntrianglerighteq}\\[0.5ex]
 \X{\nleftarrow}      & \X{\nrightarrow}     & \X{\nleftrightarrow}\\
 \X{\nLeftarrow}      & \X{\nRightarrow}     & \X{\nLeftrightarrow}

\end{symbols}
\end{table}

\begin{table}[!htbp]
\caption{Operadores binarios de la AMS}
\begin{symbols}{*3{cl}}
 \X{\dotplus}        & \X{\centerdot}      & \X{\intercal}      \\
 \X{\ltimes}         & \X{\rtimes}         & \X{\divideontimes} \\
 \X{\Cup}o \verb|\doublecup|& \X{\Cap}o \verb|\doublecap|& \X{\smallsetminus} \\
 \X{\veebar}         & \X{\barwedge}       & \X{\doublebarwedge}\\
 \X{\boxplus}        & \X{\boxminus}       & \X{\circleddash}   \\
 \X{\boxtimes}       & \X{\boxdot}         & \X{\circledcirc}   \\
 \X{\leftthreetimes} & \X{\rightthreetimes}& \X{\circledast}    \\
 \X{\curlyvee}       & \X{\curlywedge}  
\end{symbols}
\end{table}

\begin{table}[!htbp]
\caption{S'imbolos diversos de la AMS}
\begin{symbols}{*3{cl}}
 \X{\hbar}             & \X{\hslash}           & \X{\Bbbk}            \\
 \X{\square}           & \X{\blacksquare}      & \X{\circledS}        \\
 \X{\vartriangle}      & \X{\blacktriangle}    & \X{\complement}      \\
 \X{\triangledown}     &\X{\blacktriangledown} & \X{\Game}            \\
 \X{\lozenge}          & \X{\blacklozenge}     & \X{\bigstar}         \\
 \X{\angle}            & \X{\measuredangle}    & \X{\sphericalangle}  \\
 \X{\diagup}           & \X{\diagdown}         & \X{\backprime}       \\
 \X{\nexists}          & \X{\Finv}             & \X{\varnothing}      \\
 \X{\eth}              & \X{\mho}       
\end{symbols}
\end{table}



\begin{table}[!htbp]
\caption{Alfabetos matem'aticos}
\begin{symbols}{@{}*3l@{}}
Ejemplo& Instrucci'on & Paquete necesario\\
\hline
\rule{0pt}{1.05em}$\mathrm{ABCdef}$
        & \verb|\mathrm{ABCdef}|
        &       \\
$\mathit{ABCdef}$
        & \verb|\mathit{ABCdef}|
        &       \\
$\mathnormal{ABCdef}$
        & \verb|\mathnormal{ABCdef}|
        &       \\
$\mathcal{ABC}$
        & \verb|\mathcal{ABC}|
        &       \\
% Para cuando no d'e problemas con el estilo spanish de babel... :-(
% \ifx\EuScript\undefined\else
% $\EuScript{ABC}$
        & \verb|\mathcal{ABC}|
        &\textsf{euscript} con opci'on \textsf{mathcal}   \\
        & \verb|\mathscr{ABC}|
        &\textsf{euscript} con opci'on \textsf{mathscr}\\
$\mathfrak{ABCdef}$
        & \verb|\mathfrak{ABCdef}|
        &\textsf{eufrak}                \\
%\fi
$\mathbb{ABC}$
        & \verb|\mathbb{ABC}|
        &\textsf{amsfonts} o \textsf{amssymb}        \\
\end{symbols}
\end{table}

