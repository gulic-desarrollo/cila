%Autor: miguev & lasarux
%lasarux: 2
%miguev: n-2

\newcommand{\gimp}{{\sf The GIMP}~{}}
\newcommand{\bir}{botón izquierdo del ratón~{}}
\newcommand{\bcr}{botón central del ratón~{}}
\newcommand{\bdr}{botón derecho del ratón~{}}

\chapter{The GIMP}
\label{gimp.tex}
\index{\gimp}

\gimp es el Programa de Manipulación de Imágenes de GNU (The GNU Image
Manipulation Program).  Es un  programa libremente  distribuible, útil
para trabajos  como retoques de fotografía,  composición y publicación
de imágenes. \gimp ha sido escrito por Peter Mattis y Spencer Kimball,
y  liberado  bajo la  Licencia  General  Pública (GNU  General  Public
License). En octubre de 2005 \gimp 2.2 se encuentra disponible para varias
plataformas UNIX (incluyendo GNU/Linux), Mac OS X y Microsoft Windows.


\gimp es un programa bastante común en las distribuciones de
GNU/Linux,  y muy  popular  entre los  usuarios medios/avazanzados  de
Linux. Para saber  más sobre \gimp o descargarlo visita  su página web
en  {\tt http://www.gimp.org}.  También  puede  ser  de  utilidad  la
traducción al castellano de la Referencia de \gimp\cite{gimpref}.

\section{Conceptos fundamentales}

% Mapas de bits, modelos de colores, transparencia

Antes de entrar  a conocer \gimp debes tener  claros ciertos conceptos
y  términos  que encontrarás  con  bastante  frecuencia en  documentos
como  este y  en los  diálogos  del propio  \gimp. Vamos  a darte  una
introducción  a  estos  conceptos  sin entrar  en  demasiado  lujo  de
detalles.

\subsection{Mapas de bits}
\index{\gimp!mapas de bits}

Existen varias formas de almacenar  imágenes en un fichero, pero \gimp
sólo maneja una de estas formas:  la denominada ``mapa de bits''. Esta
forma de  almacenar imágenes en ficheros  parte de la idea  de que una
imagen (rectangular) es una tabla de puntos, cada punto tiene un color
exacto, que es uniforme dentro de  él, y es indivisible. Esta tabla de
puntos es lo que se denomina  un ``mapa'' de ``bits''. Estos puntos se
denominan ``píxeles''.

Las imágenes de mapa de bits tienen tamaños enteros; i.e. puedes crear
una imagen  de $2000 \times  2500$ píxeles,  pero no de  $20.25 \times
25.75$ píxeles.  Insisto en que  los píxeles son  indivisibles, porque
son las células o los átomos  que componen la imagen. Dependiendo del
modelo de color que se utilice en una imagen, el color del píxel se
representa mediante un valor entero de 0 a 255 (escala de grises),
una terna de valores enteros de 0 a 255 (RGB),
o bien un valor entero menor que una cierta cantidad que limita el
número de colores diferentes que pueden aparecer en la imagen (indexado).

\subsection{Modelos de colores}
\index{\gimp!modelos de colores}

Existen varios modos de representar el color de un píxel, 
denominados  ``modelos de  colores''.  Veamos los más usuales:

\begin{description}

\item[RGB] son las  iniciales de {\bf Red, Green, Blue}  (rojo verde y
azul). Este modelo de  color se basa en la suma de  colores de la luz,
que no  es la misma  que en  la pintura. En  la luz los  colores rojo,
verde y azul ``suman'' blanco; i.e. si iluminaras un objeto blanco con
un foco de color  rojo puro, otro de color verde puro  y otro de color
azul puro,  verías el objeto de  color blanco. Si apagaras  el foco de
color azul lo verías de color amarillo, porque en la luz (que no en la
pintura) los colores rojo y verde ``suman'' amarillo.

A partir de esta propiedad de la luz, se puede descomponer el color de
cada píxel  en tres  componentes que expresan  la proporción  de rojo,
verde y azul que tiene ese color. Estas proporciones suelen expresarse
con números {\bf  enteros} desde el  0 ($0\%$) hasta el  255 ($100\%$),
así un píxel rojo está representado por la terna $(255,0,0)$, el color
amarillo  por  la  terna  $(255,255,0)$  y  el  blanco  por  la  terna
$(255,255,255)$.

Este modelo  de colores  es ideal  para los gráficos  de mapa  de bits
generados por ordenador, y en el campo  de la imagen digital es el más
extendido.

\item[HSV]  son   las  iniciales  de  {\bf   Hue,  Saturation,  Value}
(tonalidad, saturación e  intensidad). Este modelo es  similar al RGB
pero en lugar de descomponer el color de cada píxel en sus componentes
rojo verde y azul lo descompone  de una forma más descriptiva para los
humanos.

La  tonalidad (Hue)  representa el  color tal  como lo  entendemos los
humanos  mediante  un  ángulo de  $0\textdegree$  a  $360\textdegree$.
La  tonalidad comienza  por  el rojo  puro  ($0\textdegree$) y  avanza
en  sentido   horario     hacia  el  amarillo   ($60\textdegree$),  el
verde   ($120\textdegree$),  el   cian  ($180\textdegree$),   el  azul
($240\textdegree$), el magenta ($300\textdegree$) y finalmente regresa
al rojo ($360\textdegree$).

La saturación  (Saturation) representa la  pureza del color.  Un color
sin saturación no  es color, es sólo  un tono de gris,  mientras que un
color con saturación máxima es un color puro.

La  intensidad  (Value)  representa   la  luminosidad del color.
Un  color  sin intensidad es un negro absoluto  mientras que un color
con luminosidad máxima  es  un  blanco  puro,  sin importar  ni  la
tonalidad  ni  la saturación.

Aunque la tonalidad  es un ángulo y la saturación  y la intensidad son
porcentajes, en  los ficheros estos  valores se representan mediante
números enteros entre  0 y 255.

\item[HLS] se  refiere a {\bf Hue,  Lightness, Saturation} (tonalidad,
luminosidad y saturación)  y es análogo al HSV  intercambiando las dos
últimas componentes; es decir, la luminosidad es la intensidad.

\end{description}

En los  tres  modelos anteriores el color cada píxel se representa mediante
tres valores, cada uno con distintos rangos. Sin embargo, los ficheros de mapas
de bits utilizan un byte (número entero entre 0 y 255) para almacenar cada uno
de estos valores, por lo que todos ellos quedan en el rango entre 0 y 255.
Dado que el color se determina por tres valores enteros entre 0 y 255, se
dispone de hasta $256^3 = 16777216$ colores distintos, que son más de los
que el ojo humano puede distinguir.

\begin{description}

\item[CMYK] hace referencia a {\bf Cyan, Magenta, Yellow, inK}, que es
la  forma de  descomponer los  colores en  las imprentas: los tres colores
básicos de la pintura (cyan, magenta y amarillo) y una cuarta componente
de color negro (tinta). Los modelos anteriores descomponen los colores
basándose  en la luz  porque están diseñamos para las pantallas de ordenador y
televisión, pero el modelo CMYK descompone los colores basándose en las
mezclas de tintas empleadas en la imprenta.

Debido a que \gimp ha sido diseñado para gráficos generados por ordenador
y no para trabajos  de preedición  e impresión, no  soporta este  modelo de
colores.

\item[Escala de grises]  Este modelo es similar  al modelo HSV
pero cada píxel  sólo tiene una componente que es  la intensidad. Esta
forma  de representar  el color  sólo permite  representar escalas  de
grises, por lo que cada píxel tiene un único valor que va de 0 (negro)
a 255 (blanco).

\item[Indexado] Por último  vamos a recordar un modelo  de colores que
no suele  usarse mucho. Algunas  veces la forma de  almacenar imágenes
está muy limitada  y hay que utilizar un número  máximo de colores muy
reducido (por ejemplo 16 ó 256). En estos casos el valor de  cada píxel
es un índice que  hace referencia a  un color que  está en una  tabla de
colores  previamente fijada  para  la  imagen, denominada ``paleta de colores''.

\end{description}

\subsection{Transparencias}
\index{\gimp!transparencias}

En  ocasiones  hay   que  trabajar  con  imágenes   que  tienen  zonas
transparentes, donde los píxeles no tienen ningún color (simplemente
no están). Sin embargo esa  propiedad de ``no estar'' no se
puede representar en los modelos de color que \gimp puede manejar. Por
eso  se añade  un bit  (o byte)  adicional a  cada píxel  que representa  la
transparencia  y esta  información adicional  pasa a  formar un  nuevo
canal  denominado ``canal  alpha''.
%  Más sobre  canales  en la  página \pageref{gimp.canales}



% OBSOLETO
% Antes  de  continuar  permíteme  una  aclaración.  Encontrarás  en  el
% libro ejemplos  duplicados con dos  versiones de \gimp  con interfaces
% diferentes. Cuando esto  suceda el primer ejemplo (o  de la izquierda)
% corresponde a la versión 1.2.3 (estable) y el segundo ejemplo (o el de
% la derecha) corresponde a la versión 1.3.15 (inestable).

\section{Primeros pasos}

La interfaz de usuario de \gimp resulta resulta extraña y poco intuitiva
a simple vista, con un  característico estilo que
la  diferencia  de  las  interfaces de  programas  similares  como  el
Photoshop  de Adobe.  La mayoría  de aplicaciones  de manipulación  de
gráficos tienen una  ventana que ocupa toda la pantalla,  dentro de la
cual se crean  nuevas ventanas. Esto es lo que  se denomina un entorno
MDI. \gimp por el contrario tiene una pequeña ventana que apenas ocupa
espacio  en  la  pantalla,  con  un escueto  menú  y  los  botones  de
herramientas y diálogos imprescindibles.

Esta interfaz  produce la sensación  a primera  vista de que  \gimp no
tiene apenas herramientas ni opciones de  menú, pero no es verdad. Tan
pronto  como crees  o abras  una imagen  verás que  al clickar  con el
\bdr aparecerá  un menú  lleno de  posibilidades que  exploraremos más
adelante.

Versiones anteriores de \gimp
mostraban un diálogo de  instalación de usuario al ejecutarlo por primera vez.
Este diálogo, que se refiere realmente a la personalización del programa,
preguntaba acerca de la resolución de la pantalla (suele ser 72 dpi),
los  directorios donde  guardar  ciertos objetos (brochas,  patrones,
gradientes, etc.) y  el directorio de intercambio. Dado  que el diálogo
estaba en español resultaba muy fácil de seguir. Aceptar las opciones
por defecto era suficiente.

En las versiones recientes de \gimp este dialogo se ha obviado para dar una
mejor bienvenida al usuario ahorrándole todas esas preguntas. Conviene recordar,
sin embargo, una precaución necesaria a la hora de trabajar en entornos
con sistemas de ficheros remotos como NFS o Samba, en los que el directorio
personal del usuario se encuentra en una máquina distinta de la que ejecuta
\gimp. En estos entornos se conveniente modificar la opción {\sf Carpeta
de intercambio} para que utilice {\tt /tmp}, como indica el texto de ayuda
del diálogo de configuración correspondiente del \gimp
mostrado en la figura \ref{gimp-2.2_swap}.

\begin{figura}{gimp-2.2_swap}{0.99}
\caption{Carpeta de intercambio en las preferencias de \gimp 2.2}
\end{figura}




Cuando finalmente \gimp se presenta ante nosotros vemos la ventana del
``consejo diario'',  donde podemos  leer trucos  y consejos  que \gimp
tendrá la amabilidad de enseñarnos. Es una buena idea leer estos consejos
al menos una vez antes de utilizar \gimp por primera vez, puesto que
proporciona información muy útil y a en ocasiones poco evidente.
%
Puedes regresar a la ventana del ``consejo diario'' para leer
los consejos de \gimp en cualquier momento a través del menú
\textsf{A\underbar{y}uda} \lyxarrow{} \textsf{\underbar{C}onsejo del día} 
en la ventana principal.

La interfaz inical de \gimp (figura \ref{gimp-2.2_first})
dedica una ventana separada para los diálogos
de capas, canales y caminos que veremos más adelante. También proporciona,
en la ventana principal, un espacio opcional con las opciones actuales
de la herramienta seleccionada. Si bien todo esto es muy útil cuando se lo
necesita, para empezar a manejar \gimp es prescindible. 

Dado que la interfaz de \gimp es bastante flexible, dejaremos en manos
del lector la configuración de la misma, pudieron cerrar diálogos que
no utilice o por el contrario añadiendo los que considere necesarios.
La figura \ref{gimp-2.2_minimal} muestra una configuración minimalista
de la interfaz.

\begin{figura}{gimp-2.2_first}{0.88}
\caption{Aspecto de \gimp 2.2 al ejecutarlo por primera vez}
\end{figura}

\begin{figura}{gimp-2.2_minimal}{0.18}
\caption{Configuración minimalista de la interfaz de \gimp 2.2}
\end{figura}



\section{Maniobras básicas}

Es importante que  tengas claros los conceptos  del anterior apartado.
Si te  queda alguna duda  sobre los mismos vuélvelo  a leer. Si  no es
suficiente busca información en Internet,  pregunta a tu experto local
o consulta algún libro introductorio al manejo de imágenes digitales.

Si  ya los  tienes claros,  pasemos a  explorar las  posibilidades que
\gimp te ofrece.

\subsection{Crear una imagen nueva}

Para  empezar a  familiarizarte con  el  entorno de  trabajo de  \gimp
empieza por  crear una imagen  nueva. Para  ello elige la  opción {\sf
Nuevo\dots} del menú  de fichero ({\sf Fich.} o {\sf  Archivo}) o bien
pulsa la combinación  de teclas {\tt C-N}. Verás un  cuadro de diálogo
como los siguientes:

\begin{figura}{gimp_new_image}{0.8}
\caption{Cuadros de diálogo para crear una nueva imagen}
\end{figura}

En la  parte superior del  diálogo puedes  introducir la anchura  y la
altura de la  imagen en píxeles. Si prefieres generar  el tamaño de tu
imagen a partir del tamaño real que quieres obtener en la impresora (u
otro  medio) la  parte central  del diálogo  te permite  establecer el
tamaño de tu imagen en  milímetros (mm), centímetros (cm), puntos (pt)
y picas (pc)  así como la resolución que deseas  en píxeles por unidad
de tamaño (que puede ser mm, cm, pt o pc).

Finalmente en la  parte inferior del diálogo puedes elegir  el tipo de
imagen y el tipo de relleno. El tipo de imagen debe ser {\sf RGB} para
imágenes en color o bien {\sf Tonos  de gris} o {\sf Escala de grises}
para imágenes  en blanco y  negro. El tipo  de relleno puede  ser {\sf
Frente} para  rellenar la imagen  con el  color de primer  plano, {\sf
Fondo}  para  rellenar  la  imagen  con el  color  de  segundo  plano,
{\sf  Blando} para  rellenar la  imagen de  color blanco  o bien  {\sf
Transparente}  para  no rellenar  la  imagen  con ningún  color,  sino
dejarla transparente.

El botón {\sf Restaurar} o {\sf Reiniciar} reestablece los valores del
diálogo a sus valores por defecto.

\subsection{Abrir un fichero de imagen}

Probablemente ya  tendrás algún  fichero de  imagen que  puedas abrir.
Pues es tan  fácil como eligir la opción {\sf  Abrir\dots} del menú de
fichero ({\sf Fich.}  o {\sf Archivo}) o bien pulsa  la combinación de
teclas {\tt  C-O}. El  cuadro de  diálogo es casi  igual al  cuadro de
diálogo de  abrir fichero de GTK+,  excepto por la posibilidad  de ver
una vista previa en miniatura de las imágenes antes de abrirlas.

Este cuadro de diálogo también tiene un desplegable para determinar el
tipo de fichero que quieres abrir.  Por defecto \gimp reconoce el tipo
de los ficheros automáticamente por el nombre y la cabecera del mismo.
Si fallara tienes en ese desplegable  la opción de especificar el tipo
de fichero que estás intentando abrir.

\subsection{Guardar una imagen en un fichero}

Cuando quieras guardar una imagen en un fichero puede que te sorprenda
que en el menú {\sf Fich.} ó  {\sf Archivo} no hay ninguna opción para
ello. Esto  es así porque  quieres guardar {\bf  una} imagen y  por lo
tanto es una operación que afecta a {\bf una} imagen en concreto.

Para  acceder a  esta operación  debes  utilizar el  menú que  aparece
cuando clicas con el  \bdr en la ventana de la  imagen. Si has abierto
un fichero de  imagen y sólo quieres grabar los  cambios puedes pulsar
la combinación de teclas {\tt C-S} y será suficiente.

Si has  creado una imagen nueva  y quieres guardarla por  primera vez,
puedes usar la  misma combinación de teclas, pero  aparecerá un cuadro
de  diálogo para  preguntarte  por el  nombre del  fichero  en el  que
la  quieres guardar  y  algunos  detalles más.  Lo  mismo sucederá  si
en  cualquier  imagen  eliges {\sf  Archivo}\lyxarrow{\sf  Guardar
como...} en el menú de la imagen.

\begin{figura}{gimp_save_as}{0.95}
\caption{Cuadros de diálogo para guardar una imagen a fichero}
\end{figura}

Este  cuadro de  diálogo  es casi  idéntido al  cuadro  de diálogo  de
guardar  fichero de  GTK+, salvo  por  el desplegable  que te  permite
elegir el formato de fichero con  el que quieres guardar la imagen. Si
este desplegable eliges {\sf Por extensión} \gimp adivinará el formato
de fichero que deseas usar según  la extensión que le pongas al nombre
del fichero.

Tanto si usas la  opción {\sf Por extensión} como si  no verás que hay
algunas opciones  que no puedes  usar. Para  comprender a qué  se debe
esto demos un repaso por los distintos formatos de ficheros soportados
por \gimp

\subsubsection{Los formatos de fichero}
% by Lasarux, mods by miguev
\index{\gimp!formatos de fichero}

Del  formato  del  fichero  que utilices  para  guardar  tus  imágenes
dependerá  la calidad  de  las  mismas (las  fotos  almacenadas en  un
formato que permita bastante compresión normalmente lo hace a costa de
su calidad)  y el espacio  que ocupe. Algunos además  pueden almacenar
información relativa al  tamaño de la foto para la  impresión (TIFF) e
información sobre su transparencia (GIF, PNG).

\begin{description}

\item[AA] es la  abreviatura de Ascii Art. Este  formato ({\tt .ansi})
permite guardar una imagen en forma de texto Ascii Art, con las mismas
varientes que permite el programa {\sf aview}.

\item [BMP]  fué desarrollado  e impulsado por  Microsoft, propietario
del mismo. BMP  es una abreviatura de Windows BitMaP  (Mapa de Bits de
Windows). Este formato permite guadar imágenes en RGB o indexadas, sin
comprimir o con la compresión RLE (sin pérdidas).

\item [CEL] es  originario del Animator Studio. Es  muy utilizado para
guardar {\em sprites}, o mejor dicho imágenes pequeñas para juegos.

\item [FITS] es el formato standard en Astronomía.

\item [FLI] fue  originalmete creado por Autodesk  para la realizacion
de animaciones virtuales con ordenador.

\item [Fax G3] se usa para poder procesar faxes.

\item [GBR] es para las Brochas de \gimp (Gimp brush).

\item [GIF]  (Formato de  Intercambio de  Gráficos) fue  inventado por
CompuServe y es uno de los estándares de las imágenes en la World Wide
Web. Sin embargo, las patentes de Unisys e IBM que cubren el algoritmo
de compresión LZW que es utilizado  para crear los archivos GIF, hacen
imposible tener software libre que genere GIFs adecuados.

\item  [GIH]  lo usa  \gimp  para  guardar  las brochas  animadas  que
aparecen  en las  herramientas. GIH  es el  acróninimo de  (GIMP Image
Hose).

\item [GIcon] es  el formato nativo de los iconos  \gimp. Este formato
sólo permite escala de grises.

\item [HRZ] es siempre a 256x240 pixels y es (o mejor, era) usado para
edición de imágenes de TV. No tiene compresión.

\item [HTML] genera  una tabla HTML que emula cada  píxel de la imagen
utilizando una celda vacía con el color de fondo igual al del píxel.

\item [MNG] (Multiple-image Network Graphics)  es un formato libre que
substituirá a los GIFs animados.

\item [Jpeg]  es acrónimo de  Joint Photographic Experts  Group (Unión
de  Grupo  de  Expertos  en  Fotografía)  y  funciona  con  todas  las
profundidades de color. La compresión  de la imagen es ajustable, pero
altas compresiones dañarán la calidad final  de la foto, ya que es una
compresión con pérdidas.

\item [MPEG]  es acrónimo  de Motion Picture  Experts Group  (Grupo de
Expertos  en  Animación). Es  un  bien  conocido  de los  formatos  de
animación.

\item [PAT] es el formato nativo de Patrones (Patterns) de \gimp

\item [PCX] este formato gráfico fue creado por ZSoft y difundido por 
la familia de programas de dibujo Paintbrush.                         

\item [PIX] este  es el formato usado por  el programa Alias/Wavefront
en estaciones SGI (Silicon Graphics). Sólo permite imágenes a color de
24-bits e imágenes en escala de grises de 8-bits.

\item [PNG] El  formato PNG (Portable Network Graphics)  es un formato
gráfico que usa compresión sin  pérdidas (loseless compression). Es el
formato actualmente  recomendado por  la organización W3C  (World Wide
Web Consortium)  para imágenes sin  pérdida de calidad.  Permite canal
alpha de 8 bits.

\item [PNM] Acrónimo de Portable aNyMap. PNM permite paleta de colores
indexada, escala de grises e imágenes a todo color.

\item [PSD] formato usado por  el Adobe Photoshop (\gimp mantendrá las
capas existentes).

\item [PSP]  formato usado por  el PaintShop Pro (\gimp  mantendrá las
capas existentes).

\item  [PostScript  (PS)]  PostScript  fue creado  por  Adobe.  Es  un
lenguaje  para  describir  páginas,  y  es  usado  principalmente  por
impresoras y otros dispositivos de  impresión. Es una manera estupenda
de distribuir documentos. También  podemos leer ficheros PDF (Acrobat)
con esta opción.

\item [SGI]  es el  formato originalmente  usado por  las aplicaciones
gráficas de SGI.

\item  [SUNRAS] acrónimo  de  SUN RASterfile.  Este  formato es  usado
principalmente por las diferentes  aplicaciones de Sun. Permite escala
de grises, color indexado y todo color.

\item [TGA]  este formato permite  compresión a 8,  16, 24, 32  bits de
profundidad.

\item [TIFF] es acrónimo de Tagged Image File Format. Este formato fué
diseñado para ser un estandar. Este es un formato de alta calidad y es
perfecto  cuando quieras  importar  imágenes de  otros programas  como
FrameWork o Corel Draw.

% \item  [URL] acrónimo  de Uniform  Resource Locator  (Localizador de
% Recursos  Uniforme).  Podrás  descargar una  imágen  desde  internet
% diréctamente al  {\sf GIMP}.  El formato del  nombre del  fichero es
% {\tt ftp://dirección/archivo} o {\tt http://dirección/archivo}.

% \item [WMF] acrónimo de Windows Meta File (Meta Fichero de Windows).
% Es un formato que permite guardar tanto gráficos vectoriales como en
% mapas de bits.

\item [XCF] es el  formato nativo de \gimp y es  el que debes utilizar
siempre  que  puedas para  guardar  tus  imágenes mientras  las  estés
manipulando. Cuando termines de  manipularlas puedes guardar una copia
en otro formato más exportable,  pero para mantener la calidad durante
el proceso de manipulación es  conveniente usar este formato. No tiene
ninguna compresión.

\item [XWD] (X Window Dump) es  el formato de las capturas tomadas por
X-Window. Sólo se utiliza de forma temporal.

\item  [XPM]  (X  PixMap)  es   un  formato  para  imágenes  pequeñas,
comúnmente usado  para los  iconos de  las aplicaciones  de X-Windows.
Permite canal alpha de 1 bit.

\item [bzip2] comprime la imagen con  el compresor {\sf bzip2} (el más
eficaz existente en Linux)

\item [gzip] comprime  la imagen con el compresor {\sf  bzip2} (no tan
eficaz como bzip2 pero sí más rápido)

\item [XJT]  es un formato  que permite guardar  todo al igual  que el
formato  XCF, pero  de una  forma sencilla  que permite  comprimir las
imágenes y  recuperarlas sin necesidad  de utilizar \gimp.  Un fichero
con este formato es realmente un  fichero tar que contiene las capas y
los caminos en ficheros JPEG y un fichero de texto con las propiedades
de las capas.

\end{description}

\subsubsection{Elegir el formato adecuado}

Como   ves    \gimp   puede    guardar   las   imágenes    en   muchos
formatos  diferentes,  pero  la   elección  del  formato  es  problema
tuyo\footnote{Recuerda: ``El problema es la elección''}. Para ayudarte
en esta  elección vamos a considerar  la utilidad de los  formatos más
usuales:

\begin{description}

\item [XCF] es  el formato adecuado para guardar un  trabajo a medias.
Si  estás retocando  una  foto, haciendo  una  composición compleja  o
cualquier trabajo en  el que tengas capas,  selecciones, caminos, etc.
el formato XCF es el más apropiado  puesto que es el formato nativo de
\gimp y  te permite conservar  todos los  atributos de la  imagen. Sin
embargo este  formato no  tiene compresión alguna  y además  no puedes
leerlo con otros programas.

\item [PSD]  es el formato  del {\sf Adobe  PhotoShop} y en  teoría te
permite  consevar todos  los atributos  de la  imagen, con  la ventaja
añadida  de que  podrás  abrirlo con  {\sf  PhotoShop}. Tampoco  tiene
compresión.

\item  [gzip y  bzip2] te  permiten comprimir  tus ficheros  si tienes
escasez de espacio y necesitas guardar tu imagen en formato XCF o PSD.
El  efecto es  el  mismo  que si  comprimieras  los  ficheros con  los
comandos  {\tt gzip}  o {\tt  bzip2},  pero \gimp  te permite  hacerlo
directamente sin necesidad de utilizar los comandos.

\item [JPEG] es el formato de  fichero más popular en Internet, debido
a  que  permite reducir  mucho  el  tamaño  de  un fichero  de  imagen
aplicando  una  compresión  {\bf  con} pérdidas.  Esta  compresión  es
adecuada  para  las  fotografías,  pero no  tanto  para  los  gráficos
generados por ordenador. No permite transparencias ni capas.

\item [PNG] es el formato más adecuado para los gráficos generados por
ordenador,  ya que  aplica una  compresión {\bf  sin} pérdidas  que se
ajusta mejor a éstos que la compresión del formato JPEG.

\item  [PostScript] (usualmente  encapsulado) es  la opción  necesaria
para incluir tus imágenes en un  documento escrito en \LaTeX (ver tema
\ref{latex.tex} o \LyX (ver tema \ref{lyx.tex})

\item [XPM] es el formato  usual para guardar iconos para aplicaciones
X-Windows.  Sin  embargo  también  puedes  usar  el  formato  PNG  que
además te  permite tener  256 niveles de  transparencia en  cada píxel
(canal  alpha de  8 bits)  mientras que  XPM sólo  permite 2  (opaco o
transparente, canal alpha de 1 bit).

\item  [TIFF]  es  un  formato   de  alta  calidad  útil  para  enviar
fotografías y diseños  por ordenador a las imprentas o  a personas que
no trabajan con  \gimp. Permite guardar canal alpha de  8 bits y tiene
compresión sin pérdidas (LZW) o  con pérdidas (JPEG). También se puede
utilizar sin  compresión. Sin embargo  el algoritmo de  compresión LZW
está patentado y  no viene con la distribución libre  de \gimp, tienes
que añadir un plug-in de distribución no libre.

\item  [GIF] es  un  formato  propietario y  limitado  por la  patente
del  algoritmo LZW,  por  lo  que cada  vez  más gente  (especialmente
en  el  entorno  del  Software  Libre) está  dejando  de  usarlo.  Sin
embargo, si instalas  el plug-in de distribución  no libre mencionando
anteriormente,  te  permite guardar  imágenes  en  este formato.  Este
formato no  soporta color RGB, sólo  indexado, por lo que  tendrás que
transformar la  imagen primero. Por  otra parte sí te  permite guardar
varias  capas  (más sobre  capas  en  la página  \pageref{gimp.capas})
y   construir  así   animaciones  (más   sobre  esto   en  la   página
\pageref{gimp.animaciones.gif}).

\end{description}

\subsection{Exportar una imagen}

Cuando hayas  elegido el formato  de fichero que  más se ajuste  a tus
necesidades y te dispongas a guardar la imagen puede que te encuentres
con el cuadro de diálogo de exportar  imagen. Esto se debe a que estás
intentando guardar  una imagen  en un formato  que no  permite guardar
todas las  propiedades de esa imagen.  Un caso típico es  intentar una
imagen con  más de  una capa  o con trasparencia  en formato  JPEG. La
figura \ref{gimp_export} muestra los cuatros de diálogo de este caso.

\begin{figura}{gimp_export}{1}
\caption{Cuadros de diálogo para exportar una imagen antes de guardarla}
\end{figura}

\subsection{Adquirir una imagen}

\gimp puede  también leer  imágenes de  otros dispositivos  de entrada
además  de  los  ficheros.  Dos  casos típicos  son  las  capturas  de
pantallas (screenshots) y los escáners.

\subsubsection{Capturas de pantalla}

Tomar una captura de pantalla con \gimp no es tan sencillo como pulsar
la  tecla {\sf  Impre Pant},  pero es  más flexible.  En el  menú {\sf
Fich.} o  {\sf Archivo} elige  la opción {\sf Captura  de pantalla...}
dentro del submenú {\sf Adquirir} o  {\sf Acquire}. Verás un cuadro de
diálogo como los de la figura \ref{gimp_acquire}

\begin{figura}{gimp_acquire}{1}
\caption{Cuadros de diálogo para tomar una captura de pantalla}
\end{figura}

\subsubsection{Escanear una imagen}

La forma  más usual de  utilizar un scanner  en GNU/Linux es  a través
de  SANE\footnote{Scanner Access  Now  Easy, más  información en  {\tt
http://www.mostang.com/sane/}} y su interfaz  {\sf xsane}. El programa
{\sf xsane}  permite escanear  imágenes con  las opciones  básicas que
proporciona el  scanner. 

Sin  embargo las  imágenes  escaneadas suelen  necesitar retoques,  lo
que  nos lleva  a querer  que  las imágenes  escaneadas sean  enviadas
directamente a \gimp,  cosa que {\sf xsane} no hace  por sí solo. Para
esto busca el  submenú {\sf Fich.}\lyxarrow{\sf Adquirir}\lyxarrow{\sf
XSane} o {\sf Archivo}\lyxarrow{\sf Acquire}\lyxarrow{\sf XSane}.

Si no  te aparece este menú  haz un enlace simbólico  en tu directorio
personal  {\tt  .gimp-1.2/plug-ins/}\footnote{Si  usas  \gimp  1.3  el
directorio es  {\tt .gimp-1.3/plug-ins/}} que apunte  al ejecutable de
{\sf xsane}. Esto se traduce en ejecutar el siguiente comando:

\begin{verbatim}
$ ln -s /usr/bin/xsane ~/.gimp-1.2/plug-ins/
\end{verbatim}

Debes reiniciar \gimp  para que este cambio tenga efecto.  Una vez que
tengas el  submenú {\sf XSane}  verás que  al menos aparece  la opción
{\sf Device  dialog...}. Si  aparece una o  más entradas  diferentes a
esta, se trata  de los scanners que tengas soportados  por Sane. Si no
aparece ninguno elige  la opción {\sf Device dialog...}  y {\sf xsane}
buscará la lista de scanners soportados en tu sistema.

Cuando finalmente XSane  funcione para tí, verás un par  de ventanas y
tal vez algunas más. Las más importantes son la ventana principal y la
de vista previa, que puedes ver en la figura \ref{xsane_main}.

\begin{figura}{xsane_main}{0.99}
\caption{Ventanas principales de {\sf XSane}}
\label{xsane_main}
\end{figura}

La ventana principal contiene un par de menús desplegables que siempre
estarán ahí,  otros que  dependen de  las opciones  que eligas  en los
primeros.

% El  menú desplegable  {\sf Modo  de  XSane} te  dará a  elegir lo  que
% quieres hacer con las imágenes escaneadas:
% 
% \begin{description}
% 
% \item[Visor interno] para verlas directamente en XSane.
% 
% \item[Guardar  imagen]  para  guardarlas directamente  en  fichero  en
% formato JPEG, PNM, PNG, \mbox{PostScript} o TIFF.
% 
% \item[Copiar a impresora] para enviarlas directamente a la impresora, 
% como su fuera una fotocopiadora.
% 
% \item[Enviar por FAX] para enviarlo por FAX.
% 
% \item[Enviar  por   correo  electrónico]  para  enviarlo   por  correo
% electrónico.
% 
% \end{description}

Hay un menú  desplegable importante que define el modelo  de color que
quieres  utilizar  para las  imágenes  escaneadas.  Por defecto  suele
aparecer  selecionado {\sf  Lineart}, pero  como  verás no  es el  más
adecuado en la mayoría de los casos. Las posibilidades son:

\begin{description}

\item[Lineart] utiliza el modelo de color indexado con sólo 2 colores

\item[Grayscale] utiliza el modelo de escala de grises

\item[Color] utiliza el modelo RGB

\end{description}

Dependiendo del modelo  de scanner y del soporte  de los controladores
de Sane para  ese modelo verás otro menú desplegable  con las opciones
propias  de ese  modelo,  como puede  ser escanear  algo  plano o  una
película (negativa o de diapositiva).

Una  observación importante  si  utilizas los  modelos  de color  {\sf
grayscale} o {\sf color} verás que  aparece un menú desplegable con la
opción {\sf  Rango de color completo}.  Si tienes un scanner  capaz de
escanear películas (negativos o diapositivas) aquí tienes la opción de
utilizar el  rango de colores  corresponiente a la película  que estés
usando. Si no estás escaneando  películas puedes dejarlo en {\sf Rango
de color completo}.

Debajo de  los menús desplegables  hay cuatro opciones que  se regulan
mediante controles numéricos o deslizadores, que son:

\begin{description}

\item[La  resolución], i.e.  la cantidad  de píxeles  por pulgada.  La
resolución mínima es de 12 dpi (ridícula) y la máxima puede rondar los
1200 dpi (depende del modelo del scanner). Normalmente para retocar la
imagen y más  adelante imprimirla suele ser  suficiente una resolución
de 300 dpi para  calidad media o 600 dpi para alta  calidad. Si lo que
quieres  es incluir  tu imagen  en una  página web  es mejor  bajar la
resolución para no obtener un fichero  demasiado grande, 100 ó 150 dpi
puede estar bien.

\item[El factor Gamma]

\item[El Brillo] regula la luminosidad de la imagen escaneada.

\item[El Contraste] puede acentuar o  atenuar las diferencias de color
entre los píxeles de la imagen escaneada respecto del original.

\end{description}

Debajo de los controles deslizantes hay seis botones con las siguiente
funcionalidades:

\begin{itemize}

\item Desglosar  los controles por canales.  Normalmente regularás los
controles gamma, brillo y contraste para la imagen afectando por igual
a los tres canales: rojo, verde  y azul. Si estás manipulando imágenes
con  desequilibrios en  los  colores\footnote{Por ejemplo  fotografías
tomadas con  película para luz  de día  en lugares iluminados  con luz
artificial}  puedes necesitar  diferentes  ajuste de  gamma, brillo  o
contraste para cada canal.

\item Invertir los  colores de la imagen. Al hacer  esto el control de
gamma  se  invierte multiplicativamente  y  el  control de  brillo  se
invierte aditivamente.

\item Autoajustar los controles gamma, brillo y contraste. {\sf XSane}
examina la imagen de la vista previa y trata de determinar los mejores
valores  para estos  ajustes. Pueden  no ser  los ajustes  que más  te
gusten, pero pueden ser una buena aproximación inicial.

\item Restablece los valores neutros  de los controles gamma, brillo y
contraste. Estos valores son respectivamente $1$, $0$, $0$.

\item Recupera  los últimos ajustes  de los controles gamma,  brillo y
contraste previamente memorizados.

\item Memoriza los ajustes de los controles gamma, brillo y contraste.

\end{itemize}

Por último  en la parte  de abajo de  la ventana principal  tienes dos
barras  de estado  que te  informan del  tamaño que  tendrá la  imagen
escaneada finalmente.  Junto a  estas barras de  estado está  el botón
{\sf Escanear} y {\sf Cancelar}.

Pero antes de  escanear una imagen pulsa el botón  {\sf Adquirir vista
previa}. Al cabo de un tiempo verás  en la ventana una vista previa de
lo que hayas  puesto en el scanner. Si tu  scanner es antiguo, utiliza
el puerto paralelo o está en otra máquina, esto puede tardar un poco.

\begin{figura}{xsane_preview}{0.99}
\caption{Vista previa de una revista en un scanner plano}
\end{figura}

Sobre esta vista  previa selecciona la región que  quieres escanear En
caso de  duda sé generoso  y selecciona un poco  más de imagen,  ya la
recortarás  luego si  te  sobra. Lo  que {\bf  nunca}  debes hacer  es
seleccionar  más allá  del límite  físico del  scanner, ya  que muchos
scanners no tienen un sensor que  indique que han llegado al final del
recorrido e intentan seguir más  allá. Si esto te sucede probablemente
oirás un sonido horrible y  entonces si no desconectas la alimentación
del  scanner puede  que  se rompa.  Aunque son  raros  los casos,  ten
cuidado con esto.

Una vez que  tengas seleccionada la región que  quieras escanear pulsa
el  botón  {\sf Escanear}  y  espera  pacientemente  a que  la  imagen
escaneada aparezca en \gimp.

\begin{figura}{xsane_final}{0.99}
\caption{La imagen escaneada adquirida desde \gimp}
\end{figura}

\section{Herramientas}
\index{\gimp!herramientas}

En  la ventana  principal de  \gimp tienes  una serie  de botones  con
iconos que  son las herramientas,  además de  accesos a los  diálos de
selección de colores, brochas, patrones y gradientes. Cada herramienta
tiene sus propias  opciones en cuyos detalles  procuraremos no entrar,
pero  que debes  saber  que  están ahí  y  echarles  un vistazo.  Para
seleccionar  una herramienta  clica  sobre ella,  para  acceder a  sus
opciones haz doble click sobre ella. Veamos qué son estas herramientas
y diálogos y cuál es su utilidad.

\subsection{Herramientas de selección}

Las  primeras   seis  herramientas  son  para   hacer  selecciones  en
la  imagen.  Una  selección  te  permite  limitar  el  efecto  de  las
operaciones que  hagas en la imagen,  así como cortar, copiar  y pegar
regiones. Hay  una herramienta de  selección para cada uso.  La figura
\ref{gimp_tools_selection}  muestra  los  botones  correspondientes  a
estas herramientas, que pasamos a describir a continuación.

\begin{figura}{gimp_tools_selection}{0.8}
\caption{Herramientas de selección}
\end{figura}

\subsubsection{Seleccionar regiones rectangulares (1)}
\index{\gimp!herramientas!seleccionar regiones rectangulares}

Esta  es la  herramienta  de  selección más  básica  y usual,  permite
seleccionar regiones  rectangulares o  cuadradas. Para  seleccionar un
rectángulo marcando  dos esquinas opuestas  mantén pulsado el  \bir en
una esquina y arrastra el ratón hasta la esquina opuesta, al soltar el
\bir la región quedará seleccionada. Si quieres seleccionar una región
cuadrada mantén pulsada la tecla {\tt  Shift} {\bf antes de soltar} el
\bir. Si  quieres que el punto  inicial sea el centro  de la selección
mantén pulsada la  tecla {\tt Control} {\bf antes de  soltar} el \bir.
Ambos efectos pueden sumarse si mantén pulsadas las teclas {\tt Shift}
y {\tt Control} {\bf antes de soltar} el \bir.

Además  también  puedes  sumar,  restar  e  intersectar  las  regiones
seleccionadas. Para  sumar una nueva  selección a la  existente mantén
pulsada la  tecla {\tt  Shift} {\bf  antes de  pulsar} el  \bir (luego
puedes soltarla antes de soltar el  \bir). Si lo que quieres es restar
una  nueva selección  a  la  existente mantén  pulsada  la tecla  {\tt
Control} {\bf  antes de  pulsar} el \bir.  Para intersectar  una nueva
selección con  la existente mantén  pulsadas las teclas {\tt  Shift} y
{\tt Control} {\bf antes de pulsar} el \bir.

Las  operaciones de  suma,  resta e  intersección  de selecciones  son
válidas para las siete herramientas de selección que describimos aquí.
Si bien en algunas herramientas el  momento en el que debes pulsar las
teclas {\tt Shift} o {\tt Control} pueden variar ligeramente, ya te lo
indicaremos más adelante.  Si no te lo indicamos es  que se hace igual
que con la selección de regiones rectangulares.

\subsubsection{Seleccionar regiones elípticas (2)}
\index{\gimp!herramientas!seleccionar regiones }

Su  manejo  es  exactamente  igual  que  el  de  la  herramienta  para
seleccionar  regiones  rectangulares,  pero en  lugar  de  seleccionar
regiones  rectangulares (o  cuadradas) las  regiones son  elípticas (o
circulares).

\subsubsection{Seleccionar regiones dibujadas a mano (3)}
\index{\gimp!herramientas!seleccionar regiones dibujadas a mano}

Esta  forma  de seleccionar  regiones  es  muy útil  para  seleccionar
regiones   con  precisión   cuando  \gimp   no  encuentre   una  forma
suficientemente precisa  de hacerlo. No  es una herramienta  que debas
utilizar  para  seleccionar  regiones  desde cero,  sino  para  afinar
las  selecciones  que aproximes  mediante  las  otras herramientas  de
selección.

\subsubsection{Seleccionar regiones contínuas (4)}
\index{\gimp!herramientas!seleccionar regiones contínuas }

Sirve para seleccionar una región  de píxeles con colores muy cercanos
al color del píxel sobre el  que se aplica la herramienta. Por ejemplo
en una fotografía de paisaje clicando con esta herramienta en un cielo
despejado quedaría seleccionado casi todo el cielo. Una característica
importante de esta herramienta es  que las selecciones resultantes son
contínuas, o mejor dicho conexas.

\subsubsection{Seleccionar regiones utilizando curvas de Bèzier (5)}
\index{\gimp!herramientas!seleccionar regiones utilizando curvas de Bèzier}

Las  curvas de  bezier están  determinadas  por dos  puntos (origen  y
destino)  que determinan  los  extremos  de la  curva  y dos  vectores
(atractores) que  determinan ``cómo'' llega  la curva a  sus extremos.
Esta  herramienta  permite  seleccionar  regiones  delimitándolas  con
curvas  de  Bèzier  cerradas.  Los  polígonos  cerrados  son  un  caso
particular de estas curvas. 

El manejo de esta herramienta no es del todo trivial:

\begin{itemize}

\item clica en  puntos de la imagen que pertenezcan  al contorno de la
región  (o en  los  vértices del  polígono)  que quieras  seleccionar,
terminando  en el  mismo punto  que empezaste.  Con esto  consigues un
polígono cerrado  que es  tu primera aproximación  a la  curva cerrada
final  que quieres  obtener\footnote{En algunos  contextos encontrarás
que estas curvas cerradas se denominan ``beziérgonos''}.

\item Cuando tengas el polígono cerrado clica dentro del polígono y se
convertirá  en  una  selección,  pero mantendrá  marcados  los  puntos
originales. Si ya  tenías una selección existente este  era el momento
de  mantener pulsadas  las teclas  {\tt  Shift} o  {\tt Control}  para
sumar, restar o intersectar ambas regiones.

\item Si quieres mejorar la aproximación  de la curva de Bèzier tienes
que modificar los  atractores de cada punto. Para ello  clica y mantén
pulsado el \bir  sobre el punto cuyos atractores  quieras modificar, y
ahora arrastra  el cuadradito  que aparecerá en  ambos extremos  de un
segmento  centrado en  el  punto cuyos  atractores estás  modificando.
Repite con todos los puntos que quieras cuantas veces quieras.

\item Cuando tengas la curva modificada a tu gusto, tendrás ocasión de
substituir la selección que ya tienes por una nueva con la forma de la
curva que  has obtenido; o bien  sumar, restar o intersectar  la nueva
curva  con  la selección  previa.  Para  substituir simplemente  clica
dentro de la  curva; para sumar, restar o  intersectar mantén pulsadas
las teclas correspondientes mientras clicas dentro de la curva.

\end{itemize}

La figura \ref{gimp_bezier} muestra  un ejemplo de selección realizada
con curvas de Bèzier.

\begin{figura}{gimp_bezier}{0.6}
\caption{Selección realizada con curvas de Bèzier}
\end{figura}

\subsubsection{Seleccionar formas de la imagen (6)}
\index{\gimp!herramientas!seleccionar formas de la imagen}

Esta herramienta, también llamada  ``tijeras inteligentes'', se maneja
igual que la selección de regiones mediantes curvas de Bèzier, pero su
forma de construir la curva que  determina los límites de la región es
más inteligente. Entre cada dos  puntos \gimp traza un camino buscando
bordes en  la imagen, eligiendo  la trayectoria  que deje a  sus lados
píxeles con  mayores diferencias.  Esta herramienta es  sumamente útil
para  aproximar  selecciones de  objectos  en  una imagen  plana  (sin
capas),  por ejemplo  para seleccionar  un  coche en  concreto en  una
fotografía llena  de coches. La figura  \ref{gimp_scissors} muestra un
ejemplo de selección realizada con tijeras inteligentes.

\begin{figura}{gimp_scissors}{0.6}
\caption{Selección realizada con tijeras inteligentes}
\end{figura}

\subsubsection{Seleccionar regiones por colores (7)}
\index{\gimp!herramientas!seleccionar regiones por colores}

Sirve para seleccionar {\bf todas} las regiones de píxeles con colores
muy cercanos al color del píxel sobre el que se aplica la herramienta.
Su manejo es idéntico al de la selección de regiones contínuas, pero a
diferencia de ésta las selecciones resultantes {\bf no} son conexas.

% \begin{figura}{gimp_porcolores}{0.6}
% \caption{Selección por colores sobre una región negra}
% \end{figura}

\subsection{Herramientas de transformaciones geométricas}

Las  primeras   seis  herramientas  son  para   hacer  selecciones  en
la  imagen.  Una  selección  te  permite  limitar  el  efecto  de  las
operaciones que  hagas en la imagen,  así como cortar, copiar  y pegar
regiones. Hay  una herramienta de  selección para cada uso.  La figura
\ref{gimp_tools_transformation} muestra los botones correspondientes a
estas herramientas, que pasamos a describir a continuación.

\begin{figura}{gimp_tools_transformation}{0.99}
\caption{Herramientas de transformaciones geométricas}
\end{figura}


\subsubsection{Mover capas y selecciones (8)}

Esta herramienta tiene dos modos  de funcionamiento, dependiendo de si
existe  una selección  o no.  Si  hay una  selección esta  herramienta
desplaza  sus límites,  pero  no  su contenido.  Si  no hay  selección
desplaza  la capa  sobre la  que  se aplique  el desplazamiento.  Esto
quiere decir que si una imagen tiene varias capas e intentas mover con
esta herramienta una de ellas debes aplicar la herramienta en un píxel
que pertenezca  a la capa  que quieres  mover, teniendo cuidado  de no
confundirte de capa por las transparencias.

\subsubsection{Aumento y disminución (lupa) (9)}

El comportamiento de  esta herramienta es aumentar  el acercamiento de
la vista  sobre la centrada en  el píxel donde se  aplica. Manteniendo
pulsada la tecla {\tt Control} se invierte el comportamiento.

\subsubsection{Recortar o redimensionar una imagen (10)}

Cuando  haces  una  captura  de pantalla  o  escaneas  una  fotografía
normalmente tienes  que quedarte sólo  con una  región de la  imagen y
desechar el  resto. Esta herramienta  se ajusta perfectamente  a estet
comentido.  Selecciona  con  ella  la región  que  quieres  conservar,
puedes reajustar la selección  cuantas veces necesites arrastrando las
esquinas de  la región.  Cuando finalmente  tengas la  región definida
pulsa en {\sf Recortar} y obtendrás el resultado deseado.

% \begin{figura}{gimp-1.2_recortar}{0.5}
% \caption{Cuadro de díalogo para recortar imagen en \gimp 1.2}
% \end{figura}

\begin{figura}{gimp-1.3_recortar}{0.99}
\caption{Cuadro de díalogo para recortar imagen en \gimp 1.3}
\end{figura}


\subsubsection{Rotación, escalado, cizalladura y perspectiva (11)}

Estas  herramientas están  reunidas  en  una sola  en  \gimp 1.2,  por
lo  que  para  elegir  la  operación que  quieras  hacer  tendrás  que
hacer  doble click  sobre  la  herramienta y  elegir  las opciones  en
el  cuádro  de diálogo  que  puedes  ver en  el  centro  de la  figura
\ref{gimp_tools_transformation}.  En  \gimp  1.3  cada  una  de  estas
herramientas tiene su propio botón en la caja de herramientas.

Cuando se aplica una de estas herramientas sobre una selección pasamos
a manejar esta  selección mediante la malla rectangular  mínima que la
contenga.

Cada  una  de  estas   herramientas  tiene  diferentes  efectos  sobre
la  imagen  o  selección  sobre  la que  se  apliquen.  En  la  figura
\ref{gimp_transformations} puedes  ver los  efectos que  describimos a
continuación:

\begin{figura}{gimp_transformations}{0.99}
\caption{Rotación, escalado, cizalladura y perspectiva en \gimp 1.3}
\end{figura}

\begin{description}

\item[Rotación] transforma la selección  rotándola sobre un punto, que
inicialmente es  el centro de  la malla rectangular.  Puedes desplazar
este punto  arrastrándolo con el  ratón para  cambiar el centro  de la
rotación, y rotar la malla arrastrando una de sus esquinas.

\item[Escalado] altera  la anchura y  altura de la  malla rectangular,
manteniéndola centrada  sober el  punto centrar de  referencia. Puedes
desplazar este punto  arrastrándolo con el ratón y  modificar la malla
arrastrando una de sus esquinas o cualquier otro punto del interior de
la malla.

\item[Cizalladura] es un caso  particular de la siguiente herramienta.
Sólo  permite  desplazar  las  esquinas   de  la  malla  horizontal  o
verticalmente para  desplazar un lado de  la malla en una  dirección y
sentido  y  el  lado  opuesto  en  la  misma  dirección  pero  sentido
contrario.

\item[Perspectiva] es la transformación  más flexible, pues te permite
desplazar  cada esquina  de  la  malla a  tu  antojo  para obtener  la
deformación  que quieras.  También puedes  desplazar el  punto central
para trasladar la malla.

\end{description}

\subsubsection{Invertir simétricamente (12)}

Esta herramienta  invierte simétricamente la selección,  capa o imagen
donde se aplique.  Por defecto produce una  inversión horizontal, para
cambiar haz  doble click en  el botón de  la herremienta y  cámbialo a
{\sf vertical}.

\subsection{Herramientas de dibujo}

\gimp tiene muchas  herramientas de dibujo, pero en este  tema vamos a
tratar sólo las más básicas y usuales.

\begin{figura}{gimp_tools_drawing}{0.8}
\caption{Herramientas básicas de dibujo}
\end{figura}

\subsubsection{Recoger colores de la imagen (13)}

Esta herramienta, comunmente denominada ``pipeta'', permite cambiar el
color de frente tomándolo de una imagen abierta. Basta con hacer click
en un píxel para que su  color quede seleccionado como color de frente
y nos  muestra un  cuadro con la  información del  color seleccionado:
valores  RGB  ({\sf  Rojo},  {\sf  Verde}  y  {\sf  Azul}),  valor  de
transparencia ({\sf Alpha}) y terna hexadecimal RGB (útil para edición
de páginas web).

\begin{figura}{gimp_pipeta}{0.8}
\caption{Cuadros de información de color seleccionado con la pipeta}
\end{figura}

% \subsubsection{Añadir texto (14)}

% \subsubsection{Rellenar con un color o patrón (15)}

% \subsubsection{Dibujar trazos afilados de lápiz (16)}

% \subsubsection{Borrar al color de fondo o transparente (17)}

% \subsubsection{Dibujar trazos borrosos de brocha (18)}

% \subsubsection{Rellenar con un gradiente de colores (19)}

% \subsubsection{Aerógrafo de presión variable (20)}

% \subsubsection{Pintar usando patrones o regiones de la imagen (20)}

% \subsubsection{Desenfocar o enfocar (21)}

% \subsubsection{Dibujar con tinta (22)}

% \subsubsection{Blanquear o ennegrecer (23)}

% \subsubsection{Borronear (24)}

% \section{Diálogos}
% \index{\gimp!diálogos}

% \subsection{Selección de color}
% \index{\gimp!selección de color}

% \subsection{Selección de brocha}
% \index{\gimp!selección de brocha}

% \subsection{Selección de patrón}
% \index{\gimp!selección de patrón}

% \subsection{Selección de gradiente}
% \index{\gimp!selección de gradiente}

% \subsection{Capas, canales y caminos}
% \label{gimp.capas}
% \label{gimp.canales}
% \label{gimp.caminos}
% \index{\gimp!capas}
% \index{\gimp!canales}
% \index{\gimp!caminos}

% \subsection{Paletas}
% \label{gimp.paletas}
% \index{\gimp!paletas}

% \section{Filtros}

% \section{Scripts-Fu}

% \section{Técnicas y trucos}

% \subsection{Animaciones GIF}
% \label{gimp.animaciones.gif}
% \index{\gimp!animaciones GIF}


