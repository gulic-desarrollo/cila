%Autor: faraox & _ese
%faraox: 6 (confirmar)
%_ese: 6 (confirmar)


\chapter{Administración básica}
\label{administracion.tex}
\index{administracion}
\section{Paquetes de instalación}

Un paquete es un archivo que  contiene varios ficheros que permiten la
instalación de un programa. En linux existen varios tipos de paquetes:

\begin{description}

\item[RPM:]  \index{Administracion!RPM,   paquetes}  es  el   tipo  de
paquetes más popular. Creada por Red  Hat y que utilizan la mayoría de
las las distribuciones.

\item[DEB:] \index{Administracion!DEB,  paquetes} creada por  el grupo
Debian.

\item[TARBALLS:]  \index{Administracion!TARBALLS, paquetes}  éstos son
ficheros comprimidos  que contienen  el código  fuente del  programa y
pueden instalarse en cualquier distribución.

\end{description}

Una parte importante y muy delicada  de un sistema de paquetes son las
dependencias.  Cada  paquete  necesita  que  estén  instalados  en  el
sistemas otros paquetes que son necesarios para su funcionamiento, los
paquetes  necesarios para  su instalación  se llaman  dependencias. Si
esto no  funciona, no se  puese asegurar  que el sistema  funcione con
estabilidad y fiabilidad al sistema GNU/Linux.

% Continuar con lo de dependencias.


%Esto es de Félix
\subsection{Sistema de paquetes RPM}
\index{RPM}
Este sistema  de paquetes fue creado  por la compañía \index{Red Hat} Red  Hat. De ahí
viene el  nombre RPM,  {\bf Red  Hat Package  Manager}. Con  el tiempo
otras distribuciones han ido adoptando  este sistema de empaquetado de
ficheros. Las principales  distribuciones que lo adotan  son: Red Hat,
SuSE \index{SuSE} y Mandrake. \index{Mandrake}

Un paquete de software construido con RPM es un conjunto de ficheros e
información  asociada  a ello,  como  un  nombre,  una versión  y  una
descripción. que terminan con la extensión {\tt .rpm}. A diferencia de
el  sistema  de  Debian,  cuando queremos  instalar  un  paquete  rpm,
hemos  de bajárnoslo  nosotros mismos  y efectuar  las operaciones  de
instalación.  La aplicación  que  se utiliza  para  la instalar  estos
paquetes es {\tt rpm}. Ésta es su forma de utilización:

\subsubsection{Instalación de paquetes RPM}
\index{RPM!instalación}
RPM se utiliza  principalmente para instalar de  paquetes software. La
sintaxis general de una orden de instalación {\tt rpm} es:

\begin{verbatim}
rpm [opciones] [paquetes]
\end{verbatim}

\noindent donde {\tt opciones} puede ser alguna de las siguientes:
\begin{description}

\item [{\tt -v}:] Muestra por  pantalla el trabajo que está realizando
RPM.

\item [{\tt  -h o --hash}:] Imprime  en pantalla el carácter  {\bf \#}
muchas veces seguidas mientras se está instalando el paquete para marcar el proceso de instalación.

\item [{\tt  --percent}:] Muestra en pantalla  el porcentaje realizado
mientras se extraen los ficheros de un paquete.

\item [{\tt --test}:] Realiza el proceso de instalación de un paquete,
pero no instala nada. Se utiliza  más bien para detectar problemas que
pudieran surgir en la instalación.

\item [{\tt --nodeps}:] No  se realizan verificaciones de dependencias
antes de la instalación de un paquete.

\item  [{\tt --force}:]  Fuerza la  instalación de  un paquete  aunque
surjan problemas de conflictos.

\end{description}

Cuando se le pasan opciones  a RPM, independientemente del modo, todas
las opciones indicadas  por una sola letra se pueden  unir formando un
único bloque. Así, si se hace:

\begin{verbatim}
rpm -i -v -h vim-4.5-2.i386.rpm
\end{verbatim}

\noindent sería equivalente a

\begin{verbatim}
rpm -ivh vim-4.5-2.i386.rpm
\end{verbatim}

Observemos que este paquete sigue el convenio de nomenclatura:

\begin{verbatim}
nombre-versión-emisión.arquitectura.rpm
\end{verbatim}

Puede ocurrir que se intenta instalar el paquete ``dosemu''

\begin{verbatim}
# rpm -ivh dosemu-0.66.2-1.i386.rpm

failed  dependences: kernel  >=  2.2.16 is  needed by  dosemu-0.66.2-1
dosemu = 0.64.1 is needed by xdosemu-0.64.1-1 
\end{verbatim}

Estos mensajes indican dos cosas: Es necesario actualizar a la versión
2.2.16 del  kernel y, si se  instala, una versión más  moderna de {\tt
dosemu}, también  habrá queinstalar  una versión  más moderna  de {\tt
xdosemu}. Aunque no suele ser una buena idea ignorar estos errores, en
caso de que se quiera instalar  sin atender a estos mensajes, sólo hay
que usar la opción {\tt --nodeps}.

\subsubsection{Actualización de paquetes RPM}
\index{RPM!actualización}
El modo de actualización de  RPM proporcina un método sencillo renovar
paquetes ya instalados a versiones más  modernas. Su uso es similar al
de la instalación:

\begin{verbatim}
rpm -U [opciones] [paquetes]
\end{verbatim}

\subsubsection{Desinstalación de paquetes RPM}
\index{RPM!desinstalación}
El modo  de desinstalación  de RPM proporciona  un método  limpio para
eliminar los archivos que pertenecen a un determinado paquete y que se
encuentran en diferentes lugares.

Muchos paquetes  instalan ficheros  en {\tt /etc},  {\tt /usr}  y {\tt
/lib}, por lo  que puede ser complicado eliminarlos, pero  con RPM, se
puede desinstalar un paquete completo con la orden:

\begin{verbatim}
rpm -e [opciones] [paquetes]
\end{verbatim}

Hay que tener  en cuenta que para desinstalar sólo  hay que indicar el
nombre de la aplicación, no el del paquete, como se hace en el proceso
de instalación.

\subsubsection{Pidiendo ayuda a  RPM}

Como es obvio,  RPM tiene muchísimas opciones y  también puede ocurrir
que no nos acordemos de alguna. Entonces podemos pedirle ayuda a RPM:

\begin{verbatim}
rpm -h
\end{verbatim}

Con esto tenemos  para saber manejar el RPM de  forma básica. El resto
son horas y ganas.


%Esto es de faraox
\subsection{Sistema de paquetes Deb}
\index{DEB}
Todos los paquetes deb tienen el siguiente formato:
{\tt nombre-del-paquete\_version(1.3.34-5).deb}  

La distribución  Debian tiene diversas utilidades  para la instalación
de  paquetes,  entre  ellas,  APT,   que  permite  la  instalación  de
paquetes de  forma fácil y  rápida, advirtiendo de las  dependencias y
recomendando  paquetes. El  sistema APT  engloba varios  comandos como
apt-get, apt-cache, apt-cdrom,...

\subsubsection{El fichero /etc/apt/sources.list}
\index{DEB!sources.list}
Este fichero  es imprescindible  para la  instalación de  paquetes con
APT. En  él se guardan  las direcciones de  donde APT se  descarga los
paquetes. Los medios por los que  se pueden descargar los paquetes son
varios: file(podemos elegir un directorio albitrario de donde bajarnos
los paquetes, esto es útil para mirrors locales o carpetas NTFS),de un
cdrom,de un servidor web(http), de un ftp, por rsh/ssh.

Vamos a ver un ejemplo:

\begin{verbatim}
deb http://http.us.debian.org/debian woody main contrib non-free
deb http://non-us.debian.org/debian-non-US woody/non-US main contrib non-free
deb-src http://http.us.debian.org/debian woody main contrib non-free
\end{verbatim}

La  diferencia entre  {\tt  deb} y  {\tt deb-src}  es  que el  primero
indica la  descarga de  paquetes .deb, que  son ficheros  binarios, es
decir,  preparados  para  ejecutarse,  mientras  que  con  el  segundo
podemos descargarnos  el código  fuente del paquete(usando  el comando
apt-get  source).

La  siguiente  parte   de  la  línea  es  el  {\tt   URI},  es  decir,
el  tipo  de   sistema  para  la  descarga,   recordemos  que  existen
varios(file,cdrom,ftp,http,rsh/ssh). En  este caso  es de  un servidor
web.

Seguidamente escribimos la {\tt  localización} del mirror de paquetes,
este caso  tenemos varias líneas  con diferente localización,  esto se
debe  a que  en los  Estados Unidoas  es ilegal  utilizar aplicaciones
de  encriptación,   así  que   para  bajar  esos   programas,  existen
líneas  especiales que  contienen  la palabra  non-US.  Después de  la
localización, separado por  un espacio, se escribe la  {\tt versión de
debian}, es válido tanto el alias de  la version como en qué estado se
encuentra (stable, unstable,testing).

Por último  se escriben las  {\tt secciones de software}  que usaremos
(main,  contrib, non-free).  Debian  organiza los  paquetes en  varias
carpetas segun su licencia.

\begin{itemize}

\item La sección {\tt main} agrupa los paquetes en los que su licencia
cumple con  los criterios  de la DGFS(``Guías  de Debian  del Software
libre'').

\item La sección {\tt contrib}  agrupa paquetes que tiene una licencia
libre pero  que sin embargo dependen  de otros paquetes que  no cumple
con las normas del DGFS.

\item Y por  último, la sección {\tt non-free}  contienen paquetes que
son  de  libre  distribución  pero  que sin  embargo  no  cumplen  las
directrices  de  la  DGFS (no  distribuye  el  código,  no  se  permite
redistribuir el código,etc).

\end{itemize}

\subsubsection{Apt-get}
\index{DEB!apt-get}
El comando {\bf  apt-get} se utiliza para la  manipulación de paquetes
deb. Permite la instalación de paquetes, borrado, \dots

%probar a hacer tabla 

\begin{description}

\item[{\tt apt-get install paquete1 paquete2 ...}] Instala paquetes.

\item[{\tt apt-get remove paquete1 paquete2 ...}] Borra paquetes.

\item[{\tt apt-get  source paquete1 paquete2 ...}]  Descarga el código
fuente de los paquetes.

\item[{\tt  apt-get   update  }]   Actualiza  la  lista   de  paquetes
disponibles para instalar.

\item[{\tt  apt-get upgrade  }] Instala  las nuevas  versiones de  los
diferentes paquetes disponibles.

\item[{\tt  apt-get  dist-upgrade}]  Función adicional  de  la  opción
upgrade que modifica  las dependencias por la de  las nuevas versiones
de los paquete.

\item[{\tt  apt-get  build-dep  paquete1 paquete2  ...}]  Instala  los
paquetes  necesarios para  la  compilación del  código  fuente de  los
paquetes.

\item[{\tt  apt-get clean}]  Elimina  los ficheros  que se  encuentran
en /var/cache/apt/archives  y /var/cache/apt/archives/partial.  Ahí se
encuentran los paquetes que hemos descargado para instalar.

\end{description}

\begin{description}

\item[{\tt  -d,--download-only}]  Sólo  descarga  el  paquete,  no  lo
instala.

\item[{\tt  -f,--fix-broken}]  Esta   opción  es  importante,  intenta
arreglar problemas de dependencias que tengamos en el sistema.

\item[{\tt   -s,--simulate}]  Nos   muestra  los   resultados  de   la
instalación de un paquete.

\item[{\tt  -b,--build}]  Compila  el  paquete de  código  fuente  que
hayamos bajado.

\end{description}

\subsubsection{Apt-cache}
\index{DEB!apt-cache}
El  comando apt-cache  trabaja  con  la caché  de  los paquetes.  Este
comando no  manipula el  estado del  sistema, así  que lo  pueden usar
usuarios normales. Es de gran  utilidad ya que nos muestra información
valiosa sobre los paquetes.

Algunas opciones más importantes:

\begin{description}

\item [apt-cache show  paquete1:] Este comando muestra  la cabecera de
los paquetes.  Muestra el  desarrollador, las dependencias,  una breve
descripción  del mismo,  su tamaño,  el  nombre del  fichero donde  se
encuentra, entre otros.

\item [apt-cache search texto:] Muestra una lista de todos los paquete y
una breve descripción relacionado con el texto que hemos buscado.

\item  [apt-cache depends  paquete:] Muestra  las dependencias  de dicho
paquete.

\item [apt-cache stats:] Muestra la estadística de el cache.

\end{description}

\subsubsection{El fichero {\tt /etc/apt/apt.conf}}
\index{DEB!apt.conf}
El fichero apt.conf sirve para la configuración por defecto de APT. En
el fichero podemos, por ejemplo, darle  las órdenes al APT para el uso
de  un  proxy.  Podemos  encontrar  un ejemplo  del  fichero  en  {\tt

/usr/share/doc/apt/examples/configure-index.gz}

\subsubsection{Apt-cdrom}
\index{DEB!apt-cdrom}
El comando  apt-cdrom permite añadir nuevos  CD-ROM's al sources.list.
Para añadir un cdrom la orden es {\tt apt-cdrom add}


\section{La gestión de los procesos.}
\index{Procesos}
Un proceso es  cualquier instrucción o programa que en  ese momento se
está ejecutando en nuestro sistema. Todo proceso tiene un PID (Process
IDentifier), es decir, un número que  le identifica y le diferencia de
todos los demás. Una característa importante es que todo proceso tiene
un estado: corriendo, durmiendo, zombie o parado.

\subsection{El comando kill}
\index{Procesos!kill}
El comando kill nos permite interactuar con cualquier proceso mandando
señales (signal). Cuando  ejecutamos {\tt kill pid} lo  que hacemos es
mandar la señal de TERM(terminar) con  lo cual se termina ese proceso.
Podemos usar cualquier  otro tipo de señal, para  ello utilizamos {\tt
kill signal pid}.  Podemos conseguir una lista de  señales usando {\tt
kill -l}. Una señal útil para alunas ocasiones es {\tt -9}, esta señal
fuerza a  terminar cualquier proceso.  Como su nombre  indica, estamos
matando el proceso.

También podemos utilizar el comando  {\tt killall} con el que podemos
mandar señales a un proceso utilizando el nombre, en vez del PID.

Entre  los procesos  diferenciamos  los que  se  están ejecuntando  en
$1^{er}$ o 2º  plano. Los que se  ejecutan en primer plano  son los que
interactúan con el  usuario en ese momento, mientras  que los procesos
en segundo plano se ejecutan pero están ocultos, y muy posiblemente el
usuario no tenga constancia de que se esté ejecutando.

Sólo puede haber un proceso en  primer plano por consola. Eso nos deja
las  manos atadas  si no  estamos en  el entorno  gráfico. Para  poder
ejecutar  varios  comandos,  lo  que podemos  hacer  es  ejecutar  los
comandos en segundo plano. Para  ello solo tenemos que añadir \verb.&.
al final del comando. Vamos a poner un ejemplo:

\begin{verbatim}
$ls -R / > /dev/null &
\end{verbatim}

En  el anterior  ejemplo  listamos  todos los  ficheros  de todos  los
directorios del  sistema. Enviamos la  salida a /dev/null para  que su
salida no nos moleste. El carácter \verb.&. manda el proceso a segundo
plano.

El comando {\tt jobs} nos muestra los procesos que se están ejecutando
en segundo plano:

\begin{verbatim}
$ls
[1]+  Running                 ls --color -R / >/dev/null &
\end{verbatim}

Aquí estamos ejecutando el comando anterior. El elemento {\tt [1]} nos
indica  el número  del  proceso  que se  están  ejecutando en  segundo
plano  y cuál  es su  estado. En  este caso  {\tt Running}(corriendo).
Seguidamente nos muestra  cuál es el proceso  Podemos utilizar también
el  comando {\tt  fg} para  mandar  un proceso  al primer  plano y  el
comando {\tt bg} para mandar el proceso al segundo plano.

\begin{verbatim}
$fg
ls --color -R / >/dev/null
\end{verbatim}

{\tt fg}  manda el proceso al  primer plano y nos  muestra el programa
que ha mandado.  Si tenemos varios procesos en  segundo plano añadimos
el número del proceso.

El comando {\tt  bg} se utiliza cuando tenemos,  por ejemplo, procesos
suspendidos. Estos  procesos son  programas que están  ``parados'', es
decir, no consumen ni CPU ni memoria,  y que podemos volver a poner en
archa en cualquier  momento. Para suspender un  proceso utilizamos la
combinación de teclas {\tt C-z}, al igual que para interrumpir un
proceso utilizamos {\tt C-c}.

\begin{verbatim}
$jobs
[1]+  Stopped                 ls --color -R / >/dev/null
\end{verbatim}

Esta tarea está parada(Stopped).

\subsection{El comando ps}

El  comando {\tt  ps} permite  mostrar  todos los  procesos que  están
corriendo  en nuestro  sistema. Veamos  una  parte de  una salida  del
comando ps:

\begin{verbatim}
$ps -aux
faraox@menut:~/doc/glup_0.6-1.1-html-1.1$ ps xau
USER       PID %CPU %MEM   VSZ  RSS TTY      STAT START   TIME COMMAND
root         1  0.0  0.2  1272  436 ?        S    16:00   0:04 init [2] 
root         2  0.0  0.0     0    0 ?        SW   16:00   0:01 [keventd]
root         3  0.0  0.0     0    0 ?        SW   16:00   0:00 [kapmd]
faraox    1363  0.0  0.8  2740 1564 pts/2    S    18:57   0:00 -bash
\end{verbatim}

Los parámetros  xau nos permiten ver  todos los procesos que  se están
ejecutando. El parámetro {\tt a} muestra  lo que se está ejecutando en
las tty conocidas,  el parámetro {\tt x} añade los  procesos que no se
conece la  tty en  la que se  están ejecutando y  {\tt u}  muestra los
usuarios que están ejecutando esos procesos.

Algunas partes  de la salida  le serán conocidas. La  columna ``USER''
nos dice que  usuario está ejecutando el proceso,``PID''  es su número
de proceso, ``\%CPU''  es el porcentaje de CPU que  está utilizando al
igual que  ``\%MEM'' es el  porcentaje de memoria. También  incluye la
cantidad de  memoria en kylobytes  que ha utilizado dicho  proceso, se
muestra en  la columan ``RSS''.La  columna ``TTY'' muestra  la consola
desde la  que se está ejecutando.  ``STAT'' nos muestra el  estado del
proceso:S(drmiendo), R(corriendo), T(parado),  Z(zombie). Las opciones
``W''  y ``N''  son especiales  para procesos  del kernel.  La columna
``START'' muestra  la hora a  la que empezó  el proceso, y  la columna
``TIME'' muestra el tiempo de CPU que ha usado el proceso desde que se
inició  y  ``COMMAND'' muestra  el  nombre  del  comando que  se  está
ejecutando.

\subsection{El comando top}
\index{Procesos!top}
El comando  top es una utilidad  que permite la monitorización  de los
procesos de  la CPU. También muestra  el estado de la  memoria. Es una
mezcla del comando uptime, free y ps.

\begin{verbatim}
20:07:54 up  4:07,  5 users,  load average: 0.07, 0.05, 0.05
60 processes: 58 sleeping, 1 running, 0 zombie, 1 stopped
CPU states:   0.4% user,   0.6% system,   0.0% nice,  99.0% idle
Mem:    182900K total,   172404K used,    10496K free,    35064K buffers
Swap:    96352K total,    14284K used,    82068K free,    43228K cached

PID USER     PRI  NI  SIZE  RSS SHARE STAT %CPU %MEM   TIME COMMAND
1565 faraox    14   0  1040 1040   820 R     0.5  0.5   0:00 top
300 root       9 -10 24736 9.9M  1524 S <   0.1  5.5   2:47 XFree86
1541 faraox    10   0  3148 3148  2184 S     0.1  1.7   0:00 Eterm
1 root       8   0   480  436   416 S     0.0  0.2   0:04 init
\end{verbatim}
\begin{table}[htbp]
\centering
\begin{tabular}{|c|p{0.7\textwidth}|}
\hline
{\tt espacio} & Actualiza la pantalla \\
{\tt C-l} & Borra y reescribe la pantalla \\
{\tt k} & Mata un proceso(kill) \\
{\tt r} & Cambia la prioridad de cualquier proceso \\
{\tt s} & Cambia el intervalo de refresco(por defecto es cada 5 segundos) \\
{\tt o} & Cambia el orden de los elementos \\
{\tt N} & Ordenar procesos por PID \\
{\tt P} & Mostrar procesos por el uso de la CPU(por defecto)\\
{\tt M} & Mostrar procesos por el uso de memoria \\
{\tt T} & Mostrar procesos por tiempo \\
{\tt W} & Guarda la configuración  en ~/.toprc\\
\hline
\end{tabular}
\caption{Teclas para la interactuación con el top}
\end{table}

\subsection{{\bf Nice}: prioridad en procesos}
\index{Procesos!nice}
Una CPU tiene que compartir su  tiempo de cálculo con varios procesos.
{\tt  Nice} es  un programa  que permite  cambiar la  prioridad de  un
proceso en  nuestro sistema. La prioridad  tiene un rango desde  -20 a
20.  Un usuario  normal  tiene prioridad  0  cuando ejecuta  cualquier
comando desde la consola. Con el comando nice, ese usuario normal sólo
puede  bajar la  prioridad del  proceso, nunca  subirlo, sólo  root es
capaz de  ello. El  esquema de utilización  sería: {nice  -n prioridad
comando}  o  también podemos  utilizar  el  comando {\tt  rnice}  para
cambiar la prioridad de un proceso: {\tt renice prioridad PID}

\section{Los servicios en Debian}
\index{Administracion!servicios}
Los servicios o demonios son  procesos que se ejecutan automáticamente
al  arrancar  el  sistema  o  al llamarlos  y  que  esperan  cualquier
petición.  Es el  caso por  ejemplo de  Apache, Qmail,  SSH, etc.  Los
servicios normalmente se  inician al iniciar el  sistema, pero podemos
iniciarlo(start),  pararlo(stop)  o  reiniciándolo,  sería  así:  {\tt
/etc/init.d/servicio opción}.

Para  que el  servicio  no  se inicie  cuando  arrancamos el  sistema,
podemos  usar el  comando  update-rc.d. Este  comando  crea enlaces  a
los  diferentes  directorios  de  runlevels. Para  que  no  se  inicie
automáticamente: {\tt update-rc.d -f  servicio remove}. Para añadir un
servicio, simplemente creamos el script  de nuestro servicio en Bourne
Shell Script y lo copiamos al directorio /etc/init.d/. Si queremos que
se  inicie automáticamente  al arrancar  escribimos: {\tt  update-rc.d
servicio defaults}

\section{Los archivos de registro}
\index{Administracion!registro, ficheros}
Los ficheros de registro (logging files) se almacenan los resultados e
información útil de algunos programas.  Éstos son muy importantes para
conocer el  estado de  nuestra máquina. Los  ficheros de  registro más
importantes se encuentran en  la carpeta {\tt /var/log/}. Normalmente,
estos ficheros se  pueden visualizar con cualquier editor  de texto ya
que están guardados en un fichero ASCII.

\begin{description}

\item [{\tt syslog}:] El fichero syslog es  el resultado de los mensajes del
demonio Syslogd.  Este demonio  se encarga de  la organización  de los
mensajes del  kernel, así  que en el  fichero syslog  podemos enontrar
toda la información sobre lo que ocurre en nuestra máquina.

\item  [{\tt messages}:]  Este fichero  es  igual  al syslog,  pero  muestra
información  más simple.  En  el podemos  controlar,  por ejemplo,  la
información del demonio pppd.

\item [{\tt \~{}/.bash\_history}:] Estos  ficheros muestran los comandos
que han sido ejecutados por los usuarios desde la consola.

\item [{\tt wtmp}:]  En el hay un listado de  todas las conexiones que
ha tenido la máquina mientras  ha permanecido encendida. Se puede leer
con el comando {\tt last}.

\end{description}

Normalmente cada  demonio tiene  un fichero de  registro al  igual que
otros  muchos programas.  Sólo  hay que  mirar  la documentación.  Los
ficheros  de registro  de  nuestro servidor  de  correo se  encuentran
en  /var/log/mail.log,  así  como  los del  Apache  se  encuentran  en
/var/log/apache,etc.

\section{Otras utilidades para la administración}

\subsection*{uptime}
\index{Administracion!uptime}
El comando {\tt uptime} nos indica  el tiempo que ha estado corriendo
la máquina.

\begin{verbatim}
$ uptime
17:56:33 up  1:55,  8 users,  load average: 0.05, 0.01, 0.01
\end{verbatim}

El primer elemento  es la hora actual. El  siguiente elemento, seguido
de  la palabra  ''up'' es  el tiempo  que la  máquina está  encendida.
Seguidamente  nombra el  número de  usuarios que  se encuentran  en el
sistema. Por  último muestra  la carga  media de  la máquina,  en tres
tiempos, 1,5 y 15 minutos.

\subsection*{w}
\index{Administracion!w}
Muestra los usuarios que se encuentran en el sistema. Veamos un ejemplo:

\begin{verbatim}
$w
18:00:59 up  2:00,  8 users,  load average: 0.01, 0.02, 0.00
USER     TTY      FROM              LOGIN@   IDLE   JCPU   PCPU  WHAT
faraox   tty1     -                16:01    1:58m  2.69s  0.00s  w
\end{verbatim}

La primera  línea que muestra  el comando {\tt w}  es la salida  de el
comando  {\tt uptime}.  Por orden,  la información  que muestra  es el
nombre de usuario,  la consola desde donde ha entrado,  desde donde se
conecta, la  hora de  entrada, el tiempo  que ha  permanecido inactivo
(idle), el tiempo usado por todos los procesos de esa consola(tty), el
tiempo  usado por  los procesos  actuales y  lo que  está haciendo  el
usuario.

\subsection*{free}
\index{Administracion!free}
El  comando {\tt  free}  muestra  información sobre  el  estado de  la
memoria del  sistema. Muestran  tanto el estado  de la  memoria física
como de  la swap. También muestra  el búffer utilizado por  el kernel.
Una salida del comando {\tt free}

\begin{verbatim}
$free
	     total       used       free     shared    buffers     cached
Mem:     182900     173300      9600          0      11796      66588
-/+ buffers/cache:      94916      87984
Swap:    96352        448      95904
\end{verbatim}

Este comando lee la información del fichero /proc/meninfo.

\subsection*{dmesg}
\index{Administracion!dmesg}
Este comando  muestra los  mensajes del kernel  durante el  inicio del
sistema.

%%Esta parte es de -ese

\section{Grupos y usuarios}

Como  es  sabido,  Linux  es  un sistema  multiusuario.  Eso  no  sólo
significa que  cada uno puede  tener su cuenta  y trabajar en  él sino
además  puede  hacerlo al  mismo  tiempo  que  otro usuario  que  esté
trabajando en ese instante.

Una de  las competencias  del administrador es  gestionar la  parte de
usuarios y englobarlos dentro de grupos de trabajo.

\subsection{Gestión de usuarios}
\index{Administracion!usuarios, gestión}
Para  la   gestión  de  usuarios   hay  dos  operaciones:   Adición  y
eliminación.

\subsubsection{Añadiendo nuevos usuarios}

En Debian, para añadir un usuario, se utiliza el comando {\tt adduser
nombre}. En ese momento, no sólo se creará la cuenta del usuario sino
también su directorio de trabajo, un nuevo grupo de trabajo que se
llamará igual que el usuario y añadirá una serie de ficheros de
configuración al directorio de trabajo del nuevo usuario:

\begin{verbatim}
root@cila:/home# adduser pepe
Adding user pepe...
Adding new group pepe (1000).
Adding new user pepe (1000) with group pepe.
Creating home directory /home/pepe.
Copying files from /etc/skel
Enter new UNIX password:
Retype new UNIX password:
passwd: password updated successfully
Changing the user information for pepe
Enter the new value, or press return for the default
        Full Name []:
	        Room Number []:
		        Work Phone []:
			        Home Phone []:
				        Other []:
					Is the information correct?
					[y/n] y
\end{verbatim}

En ese momento, el usuario ya puede trabajar en el sistema.

\subsubsection{Eliminando usuarios}

Para  eliminar  la  cuenta  de  un  usuario,  emplearemos  el  comando
{\tt  deluser nombre}.  La  pega de  este comando  es  que no  elimina
automáticamente el directorio de trabajo del usuario.

\begin{verbatim}
root@cila:/home# deluser pepe
Removing user pepe...
done.
\end{verbatim}

Una vez  realizado este proceso, es  responsabilidad del administrador
decidir si elimina el directorio de trabajo del antiguo usuario.

Si tratáramos  de eliminar  un usuario  que no  existe en  el sistema,
recibiremos el siguiente aviso:

\begin{verbatim}
root@cila:/home# deluser pepe
/usr/sbin/deluser: `pepe' does not exist.
\end{verbatim}


\subsection{Gestión de grupos}
\index{Administracion!grupos, gestión}
Cuando creamos un  usuario, siempre lo vamos a incluir  en algún grupo
de trabajo, ya sea el suyo propio o bien, en uno común.

\subsubsection{Añadiendo nuevos grupos}

La forma de hacerlo es bien fácil:

\begin{verbatim}
root@cila:/home# addgroup usuarios
Adding group usuarios (105)...
Done.
\end{verbatim}

El número ``105'' nos indica que  ése el identificador numérico que se
le asigna al nuevo grupo en el momento de su creación.

\subsubsection{Eliminando grupos}
De forma similar, la eliminación de un grupo se hace de esta forma:

\begin{verbatim}
root@cila:/home# delgroup usuarios
Removing group usuarios...
done.
\end{verbatim}

¿Qué  puede pasar  si tratamos  de eliminar  un grupo  inexistente? El
sistema nos avisará con el siguiente mensaje:

\begin{verbatim}
root@cila:/home# delgroup usuarios
/usr/sbin/delgroup: `usuarios' does not exist.
\end{verbatim}

\subsubsection{Añadiendo y eliminando usuarios de los grupos}

Para añadir un usuario {\bf pepe} a un grupo {\bf usuarios} haremos:

\begin{verbatim}
root@cila:/home# adduser pepe usuarios
Adding user pepe to group usuarios...
Done.
\end{verbatim}

Y para eliminarlo de ese grupo:

\begin{verbatim}
root@cila:/home# deluser pepe usuarios
Removing user pepe from group usuarios...
done.
\end{verbatim}

Con esto queda por terminada  esta introducción a la administración en
Linux.
