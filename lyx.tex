%Autor: miguev

\chapter{\LyX}

\section{�Qu� es \LyX?}

\LyX~ es  un procesador de  {\TeX}tos avanzado, basado en  \LaTeX, que
permite  crear documentos  de calidad  de forma  m�s sencilla  que con
\LaTeX.  Aunque \LyX~  utiliza \LaTeX,  es  mucho m�s  que una  simple
interface para LaTeX,  tambi�n puede generar otros  formatos. Una cosa
muy interesante de \LyX~ es que  permite insertar c�digo de \LaTeX, de
forma que los conocimientos sobre  \LaTeX~ pueden aplicarse tambi�n al
uso de \LyX~ aunque no sean necesarios.

\LyX, proporciona  una sencilla  introducci�n y  un breve  tutorial en
castellano que son ayuda suficiente  para comenzar a escribir apuntes,
trabajos, cartas  y otros  documentos sin  necesidad de  estudiar nada
complicado. Para leer la introducci�n y el tutorial de \LyX~ basta con
elegir {\tt Introducci�n} o {\tt  Tutorial} respectivamente en el men�
{\tt Ayuda}. En el mismo men� se encuentran los restantes cap�tulos de
la  documentaci�n de  \LyX~ en  ingl�s: {\tt  Gu�a del  usuario}, {\tt
Caracter�sticas  extendidas}, {\tt  Personalizaci�n},  {\tt Manual  de
referencia} y {\tt Preguntas de  uso frecuente}. 

No  hay  mucho  que  explicar  de  \LyX~ que  no  est�  en  su  propia
documentaci�n, as� que para no engrosar este libro m�s de lo necesario
ni duplicar esfuerzos,  te remitimos a la  excelente documentaci�n que
incorpora \LyX.

La mayor�a de los procesadores de textos est�n basados en la filosof�a
WYSIWYG (what you see is what you get, lo que ves es lo que tienes) en
la que el usuario no s�lo  tiene que escribir sino tambi�n preocuparse
de d�nde aparacer�  cada elemento en el documento final.  Esto lleva a
conservar pr�cticas antiguas provenientes  de las m�quinas de escribir
mec�nicas. Entre  estas pr�cticas  todos conocen los  tabuladores, los
guiones para partir palabras al final de l�nea, dejar varias l�neas en
blanco para rellenar, etc. Todo eso es definitivamente prediluviano.

\LyX~, a diferencia  de los dem�s procesadores de texto,  es lo que se
denomina un  programa WYSIWYM (what you  see is what you  mean, lo que
ves es lo que quieres decir).  Esta filosof�a de hacer un documento se
basa  en que  el usuario  no debe  preocuparse de  la composici�n  del
texto, sino �nicamente del contenido. Es el programa quien se ocupa de
encajar todo en el papel para que quede bien.


\section{Introducci�n a \LyX}


\subsection{Navegando por la documentaci�n}

�Bienvenido a \LyX{}!

Para responder m�s f�cilmente a tus preguntas y describir todas las
caracter�sticas de \LyX{}, la documentaci�n ha sido dividida en varios
ficheros. Cada uno tiene un prop�sito concreto, como se ver� m�s adelante.
Sin embargo, antes de que te topes con alguno de estos ficheros, deber�as
leer �ste en primer lugar, ya que contiene mucha informaci�n y comentarios
�tiles que te pueden ahorrar tiempo.

\bigskip{}
{\centering �����ATENCI�N!!!!! �����ATENCI�N!!!!! �����ATENCI�N!!!!\par}

{\centering �Peligro, Will Robinson, Peligro!%
\footnote{Referencia a una antigua teleserie norteamericana, {}``Perdidos en
el espacio'', donde un robot avisaba siempre a su due�o de esta manera
en situaciones de peligro. (N. del T.)
}\par}
\bigskip{}

Aunque \LyX{} est� ya en su versi�n 1.0 (o mayor), parte de la documentaci�n
puede estar incompleta o desfasada. Como el resto de \LyX{}, los manuales
son trabajo de un grupo de voluntarios que tienen {}``verdaderos
trabajos'', familias, platos que fregar, cestos del gato que limpiar,
etc, etc, etc. Hacemos el mayor esfuerzo posible por mantener los
manuales en buena forma, pero no siempre podemos lograrlo. (Si quieres
ayudar, aseg�rate de leer la secci�n \ref{sec:Contrib}, adem�s del
resto del documento).

Tambi�n puedes hacernos otro favor: si cualquer cosa de estos manuales
te parece confusa, poco clara o err�nea, �no dudes en hac�rnoslo saber!
Puedes ponerte en contacto con los actuales encargados de mantener
la documentaci�n escribiendo a \texttt{lyx-docs@lists.lyx.org}.


\subsection{El formato de los manuales}

Algunos de vosotros habreis impreso el manual (o manuales). Otros
podeis estar ley�ndolos {}``en pantalla'', desde \LyX{}. Existen
algunas diferencias entre la versi�n impresa y el fichero \LyX{}.
En primer lugar, el t�tulo est� simplemente al principio del documento
y no separado del resto (en la primera p�gina) como en algunas de
las versiones impresas. Ni las notas a pie de p�gina ni el �ndice
se encuentran visibles. Existe una opci�n de men� para poder ver el
�ndice. Para ello elige \textsf{�ndice general} en el men� \textsf{Edici�n}.
Para acceder a una nota a pie de p�gina, que aparece de esta forma,%
\footnote{�Hola!
} pulsa el bot�n izquierdo del rat�n estando sobre ella.

Ahora que hemos aclarado algunas de las diferencias entre las versiones
impresas y en pantalla de este fichero, podemos empezar con el formato
de este documento. De vez en cuando encontrar�s cosas en diferentes
fuentes de letra:

\begin{itemize}
\item \emph{El estilo resaltado} se usa para �nfasis en general, razonamientos
gen�ricos, t�tulos de libros, secciones de otros manuales, y notas
de los autores.
\item \texttt{La fuente m�quina de escribir} se usa para programas y nombres
de fichero, c�digo \LaTeX{}, y c�digo y funciones de \LyX{}.
\item \textsf{La fuente Sans Serif} se usa para men�s, botones, o nombres
de men�s emergentes, as� como nombres y combinaciones de teclas.
\item \textsc{El Estilo Nombre} se usa para nombres propios de personas.
\end{itemize}
Respecto a las teclas y sus combinaciones, probablemente ser�s enviado
a la secci�n \emph{Combinaciones de teclas asociadas} del {}``\emph{Manual
de referencia}'' (fichero \texttt{Reference.lyx}). Cuando necesitemos
hacer referencia a asociaciones de teclas, usaremos la siguiente convenci�n:

\begin{itemize}
\item {}``\textsf{C-}{}`` indica la tecla \textsf{Control}.
\item {}``\textsf{S-}{}`` indica la tecla \textsf{Shift}.
\item {}``\textsf{M-}{}`` indica la tecla \textsf{Meta}, que en la mayor�a
de los teclados ser� la tecla \textsf{Alt}.
\item {}``\textsf{F1}'' \ldots{} {}``\textsf{F12}'' ser�n las teclas
de funci�n.
\item {}``\textsf{Esc}'' ser� la tecla de escape.
\item {}``\textsf{Izquierda}'' {}``\textsf{Derecha}'' {}``\textsf{Arriba}''
{}``\textsf{Abajo}'': autoexplicativas (tambi�n llamadas teclas
de direcci�n o cursores).
\item \textsf{{}``Insert}ar'' {}``\textsf{Suprimir}'' {}``\textsf{Inicio}''
{}``\textsf{Fin}'' {}``\textsf{P�gina Arriba}'' {}``\textsf{P�gina
Abajo}'': ser�n las 6 teclas que aparecen sobre los cursores en la
mayor�a de los teclados de PC. {}``\textsf{P�gina Arriba}'' y {}``\textsf{P�gina
Abajo}'' son llamadas {}``\textsf{Prior}'' y {}``\textsf{Next}''
en algunos teclados.
\item \textsf{Retorno de carro} e \textsf{Intro} se refieren ambas a la
misma tecla. Algunos teclados la etiquetan como {}``Return,'' otros
como {}``Enter,'' incluso algunos tienen dos teclas. \LyX{} trata
a todas como la misma tecla, as� que usaremos \textsf{Retorno de carro}
e \textsf{Intro} indistintamente.
\end{itemize}
Tambi�n te encontrar�s de tanto en tanto con {}``(Ver \emph{{}`Manual
de referencia}')''. Hemos listado las asociaciones de teclas por
defecto posibles para cada funci�n en su correspondiente entrada en
el \emph{{}``Manual de referencia''}, as� que busca tambi�n all�.


\subsection{Los manuales}

La siguiente lista describe el contenido de cada uno de los ficheros
de la documentaci�n:

\begin{description}
\item [\emph{Introducci�n}]\emph{~}\\
Este fichero.
\item [\emph{Tutorial}]\emph{~}\\
Si eres nuevo usando \LyX{}, y nunca antes has usado ni o�do hablar
de \LaTeX{}, entonces debes empezar aqu�. Si ya \emph{has} usado \LaTeX{},
deber�as leer a�n as� la secci�n {}``\LyX{} para usuarios de \LaTeX{}''
(y hojear el resto del documento no te har�a da�o).
\item [\emph{Gu�a~del~Usuario}]\emph{~}\\
La documentaci�n principal. Intentaremos cubrir aqu� la mayor parte
de las opciones y caracter�sticas b�sicas de \LyX{}. El manual principal
asume que ya tienes algunos conocimientos de \LaTeX{}, o que has le�do
el \emph{Tutorial}.
\item [\emph{Caracter�sticas~Extendidas}]\emph{~}\\
Ampliaci�n de la Gu�a del Usuario. Documenta los formatos adicionales
y caracter�sticas de edici�n de prop�sito espec�fico, incluyendo algunos
trucos de expertos en \LaTeX{}.
\item [\emph{Personalizaci�n}]\emph{~}\\
Una descripci�n de caracter�sticas avanzadas de \LyX{}, entre las
que se incluyen c�mo personalizar el comportamiento global del programa:
cosas tales como asociaciones de teclas, internacionalizaci�n y ficheros
de configuraci�n.
\item [\emph{Manual~de~Referencia}]\emph{~}\\
Contiene una descripci�n detallada y exhaustiva de las caracter�sticas
de \LyX{}, pero sin los �tiles trucos y descripciones de uso de la
\emph{Gu�a del Usuario} u otros manuales.
\item [\emph{Errores~conocidos}]\emph{~}\\
Este fichero contiene una lista de errores detectados en \LyX{}, as�
como informaci�n sobre lo que deber�as hacer si encuentras un error
en el programa.
\item [\emph{Configuraci�n~de~\LaTeX{}}]\emph{~}\\
\LyX{} explora tu sistema durante la instalaci�n. Este fichero contiene
la informaci�n que \LyX{} ha aprendido de tu instalaci�n. Exam�nalo
si echas en falta algo que deber�as tener.
\end{description}
Estos ficheros har�n referencia unos a otros cuando sea necesario.
Por ejemplo, la \emph{Gu�a del Usuario} habla \emph{algo} sobre la
instalaci�n y personalizaci�n, pero env�a al lector al \emph{Manual
de Personalizaci�n} para m�s informaci�n.

De nuevo, destacamos un punto importante:

\vspace{0.51cm}
{\centering Si nunca antes has usado u o�do hablar de \LaTeX{}, lee
el \emph{Tutorial}. Ahora\@.\par}
\vspace{0.51cm}

Si no, \LyX{} te parecer� bizantino e imposible de utilizar.


\subsection{\label{sec:Contrib}Contribuir al proyecto de documentaci�n}


\subsubsection{Informar de errores en los manuales}

�sta es una tarea bastante f�cil de realizar. Puedes mandar un mensaje
bien a la Lista de Correo de Desarrolladores, bien a la Lista de Correo
de Usuarios. No obstante, el Equipo de Documentaci�n preferir�a que
lo mandases a \texttt{lyx-docs@lists.lyx.org}. Si los arreglos son
peque�os, alguien los har�. Si son grandes, sugerimos que te unas
a la lista de correo y hables con nosotros antes de hacer nada.

\vspace{1.06cm}
{\centering \emph{�Nunca realices cambios en ninguno de los manuales
sin el consentimiento previo del Equipo de Documentaci�n!}\par}
\vspace{1.06cm}

Cualquiera de estos cambios arbitrarios no se ver�n reflejados finalmente
en la documentaci�n, ya que el Equipo de Documentaci�n mantiene las
copias maestras a trav�s del �rbol de versiones CVS.


\subsubsection{Unirse al Equipo de Documentaci�n}

El Equipo de Documentaci�n de \LyX{}, como todo lo dem�s en el projecto
de \LyX{}, puede aprovechar tu ayuda siempre. Si est�s interesado
en contribuir al Projecto de Documentaci�n, \emph{primero} necesitas
hacer lo siguiente:

\begin{enumerate}
\item Consigue el �ltimo c�digo fuente de \LyX{}. Descompr�melo. Encontrar�s
un directorio en el �rbol principal llamado \texttt{development}.
Dentro de ese directorio hay un fichero llamado \texttt{DocStyle.lyx}.
L�elo; se trata de las condiciones de estilo de la documentaci�n.
\item T�mate un descanso de un d�a o dos, y vuelve a leerlo.
\item Despu�s, lee la \emph{Gu�a} \emph{del} \emph{Usuario} y el \emph{Tutorial}.


El objetivo de este ejercicio es darte ideas. El \emph{Tutorial} y
la \emph{Gu�a} \emph{del} \emph{Usuario} ser�n siempre lo m�s actualizado
de la documentaci�n. Deber�as ser capaz de hacerte una idea de c�mo
queremos que sean los manuales.

\item Lee \texttt{DocStyle.lyx} una vez m�s.
\item Por �ltimo, contacta con el equipo: 


\texttt{lyx-docs@lists.lyx.org}

\end{enumerate}
Queremos estar seguros de que entiendes las condiciones de estilo
\emph{antes} de que empieces a escribir. De lo contrario, te haremos
reescribir tu trabajo.

\emph{No} se te ocurra \emph{}empezar a escribir nada hasta que hablemos
v�a correo electr�nico. Puedes revisar el progreso de cualquier actividad
relacionada con los documentos en la p�gina web del Equipo de Documentaci�n
(una vez que la tengamos) o a trav�s de los archivos de la lista de
correo en \texttt{www.lyx.org}.


\section{Tutorial de \LyX}



\subsection{�Bienvenido a \LyX{}!}

Este fichero ha sido dise�ado para todos aquellos que nunca han o�do
hablar de \LaTeX{}, o no lo conocen muy bien. No tengas miedo, no
tendr�s que aprender \LaTeX{} para poder usar \LyX{}. Ese es, al fin
y al cabo, el punto fuerte de \LyX{}: proporcionar un interfaz casi
WYSIWIG (\emph{What You See Is What You Get}) para \LaTeX{}. Sin embargo,
hay algunas cosas que necesitar�s aprender para usar \LyX{} de forma
eficiente.

Probablemente has acabado consultando este documento porque has intentado
poner dos espacios despu�s de un punto, o tres l�neas en blanco entre
dos p�rrafos. Tras mucha frustraci�n, has comprobado que no se puede.
De hecho, descubrir�s que la mayor�a de los peque�os trucos que estabas
acostumbrado a usar con otros procesadores de texto no funcionan en
\LyX{}. La raz�n es que la mayor�a de los procesadores de texto que
has usado hasta ahora necesitaban que el usuario hiciera a mano todo
el espaciado, los cambios de tipo de letra, etc. As�, no s�lo se escrib�a
el documento, sino que adem�s se acababa realizando todo el trabajo
de formato y composici�n. \LyX{} realiza este trabajo por t� de forma
consistente, dejando que te centres en lo m�s importante: el contenido
del documento.

As� pues, ten paciencia con nosotros y sigue leyendo. Merece la pena
que leas este tutorial.


\subsection{Qu� \emph{es} este tutorial y qu� \emph{no} \emph{es}}

Antes de que empecemos con esta secci�n, queremos hacer un peque�o
apunte. El \emph{Tutorial} usa la notaci�n se�alada en la \emph{Introducci�n}.
Si has llegado a este manual primero, ve y lee la \emph{Introducci�n}.
S�, ahora.

Una vez que ya sabes qu� significa cada tipo de letra, vamos a hablar
un poco sobre la finalidad de este \emph{Tutorial}.


\subsubsection{Para aprovechar al m�ximo el tutorial}

Este \emph{Tutorial} se compone de ejemplos y ejercicios. Para obtener
el m�ximo provecho de este documento, deber�as leerlo todo, tecleando
todos los peque�os detalles que te vayamos explicando (por simples
que sean) e intentando hacer todos los ejercicios para comprobar que
lo entiendes. Por comodidad, puede interesarte imprimir la versi�n
PostScript� de este documento.

Si est�s familiarizado con \LaTeX{}, probablemente podr�s leer el
\emph{Tutorial} m�s deprisa, ya que muchas de las ideas de \LyX{}
son realmente ideas de \LaTeX{} disfrazadas. No obstante, \LyX{} no
tiene peculiaridades%
\footnote{o dicho de manera m�s optimista, {}``caracter�sticas''
} que tengas que aprender. Incluso aunque no te apetezca leer el resto
del \emph{Tutorial}, har�as bien en mirar la secci�n \ref{sec:latexusers},
que ha sido escrita espec�ficamente para usuarios experimentados de
\LaTeX{}.

La secci�n \ref{sec:what-is-lyx} ha quedado sin actualizar de una
versi�n anterior del \emph{Tutorial}, y es un poco general. A�n as�,
es una buena introducci�n {}``panor�mica'' a \LyX{}, as� que deber�as
echarle un vistazo para ir haci�ndote una idea sobre �l.


\subsubsection{Qu� \emph{no} vas a encontrar}

\begin{itemize}
\item Las lecciones masticadas y dadas de comer con cuchara.


La tendencia actual%
\footnote{Nota de \textsc{John} Weiss: \ldots{}bueno, al menos en Estados Unidos,
donde reducimos todo al m�nimo com�n denominador\ldots{}
} en la literatura autodidacta inform�tica parece ser: {}``Se asume
que el usuario tiene el coeficiente intelectual de un mosquito''.
Nosotros no vamos a hacer eso. 

Por otro lado, somos conscientes de que la mayor�a de los usuarios
acuden a un manual, especialmente a un tutorial, cuando se encuentran
perdidos. As� pues, asumiremos que t�, el usuario, \emph{no} eres
est�pido, pero entenderemos que puedas estar un poco despistado o
confuso.

\item Instrucciones de c�mo usar el rat�n o el teclado.


Si todav�a no has aprendido a usar tu ordenador no podemos ayudarte,
est� m�s all� del alcance de estos manuales.%
\footnote{Adem�s de que, si est�s usando \LyX{} para empezar, seguramente tienes
m�s de medio cerebro en la cabeza.
}

\item Explicaci�n detallada de todas las caracter�sticas de \LyX{}.


�Qu�? �quieres la \emph{Gu�a del Usuario} dos veces?

Hablando en serio, nuestro objetivo con este tutorial es prepararte
para que s�lo necesites la \emph{Gu�a del Usuario}. Tratar de duplicar
toda la informaci�n sobre las caracter�sticas de \LyX{} aqu� ser�a
redundante, demasiado largo y estar�a siempre obsoleto. Todo lo que
pretendemos es introducir las cosas; como puedes imaginar, hay un
{}``ver \emph{Gu�a del Usuario}'' al final de cada secci�n.

\item Explicaci�n detallada de \LaTeX{}.


Innecesario. Si de verdad tienes curiosidad por aprender algunos de
los trucos que puedes hacer con \LaTeX{}, siempre puedes conseguir
un libro espec�fico. Hay varios libros buenos sobre el tema en el
mercado. Despu�s de todo, no hay necesidad de volver a inventar la
rueda\ldots{}

\end{itemize}
As� pues, esp�ritu intr�pido, es hora de seguir adelante. Puedes hacer
una breve excursi�n por la siguiente secci�n, o puedes continuar con
la secci�n \ref{sec:first-doc-ex}.


\subsection{�Qu� es \LyX{}?\label{sec:what-is-lyx}}


\subsubsection{Visi�n general}

Parte del reto inicial de usar \LyX{} surge del cambio en la manera
de pensar que t�, el usuario, debes hacer. En su momento, todo lo
que ten�amos para crear documentos eran m�quinas de escribir, as�
que aprendimos verdaderas artima�as para evitar sus limitaciones.
Subrayar, que es poco m�s que sobreescribir con el car�cter {}``\_'',
se convirti� en un forma de resaltar texto. Para crear una tabla,
establec�as a mano el ancho de cada columna y pon�as las tabulaciones
necesarias. Lo mismo se aplicaba para cartas y otros textos sangrados
a la derecha. Adem�s, la ruptura de palabras al final de l�nea requer�a
ser muy cuidadoso y previsor.

En otras palabras, todos hemos sido entrenados para preocuparnos por
los peque�os detalles de {}``qu� caracter va en qu� lugar''.

Como consecuencia, casi todos los procesadores de texto se basan en
esta mentalidad. Todav�a usan tabuladores para a�adir espacios en
blanco. Todav�a te tienes que preocupar de en qu� parte exacta de
la p�gina saldr� cada cosa. Resaltar texto significa cambiar el tipo
de letra, similar a cambiar la rueda de una m�quina de escribir.

Aqu� es donde \LyX{} difiere de un procesador de texto corriente.
No te tienes que preocupar de que una letra vaya en un sitio determinado.
Le dices a \LyX{} \emph{lo que est�s haciendo} y �l se preocupa de
todo lo dem�s, siguiendo un conjunto de reglas llamado estilo. Veamos
un peque�o ejemplo.

Sup�n que est�s realizando un informe. Quieres que comience con una
secci�n llamada {}``Introducci�n''. As� pues, te diriges a cualquiera
que sea el men� de tu procesador de texto que cambia el tama�o de
fuente y eliges un nuevo tama�o. Despu�s cambias tambi�n a negrita.
Seguidamente escribes: {}``1.\_Introducci�n''. Por supuesto, si
m�s tarde decides que esta secci�n pertenece a alguna otra parte del
documento, o bien insertas una nueva secci�n anterior a �sta, tienes
que cambiarle la numeraci�n a ella y a todas las posteriores, adem�s
de las correspondientes entradas en el �ndice.

En \LyX{}, te diriges a la lista situada a la derecha de todos los
botones y eliges \textsf{Secci�n}, y escribes {}``Introducci�n''.

Eso es todo. Si cortas y pegas la secci�n en otra parte, todo es renumerado
autom�ticamente. Se puede hacer incluso que \LyX{} actualice cualquier
referencia a la secci�n que est� dentro del fichero.

Con el procesador de texto tradicional hay problemas de consistencia.
Cinco d�as m�s tarde, abres tu informe y comienzas la secci�n 4. Sin
embargo, has olvidado que estabas usando la letra en negrita de 18
puntos, y usas la de 16, as� que acabas escribiendo el encabezado
de la secci�n 4 con un tipo de letra distinto al que usaste para la
secci�n 1. Este problema ni siquiera existe en \LyX{}. El ordenador
se encarga de todo el tedioso trabajo de llevar la cuenta de tama�os
y fuentes, no t�. Al fin y al cabo, para eso est� hecho.

Otro ejemplo. Sup�n que est�s haciendo una lista. En otros procesadores
de texto una lista es s�lo una mera secuencia de tabuladores y saltos
de l�nea. Necesitas pensar d�nde poner la etiqueta de cada elemento
de la lista, qu� debe ser esa etiqueta, cu�ntas l�neas en blanco hay
que poner entre cada elemento, etc. Con \LyX{}, s�lo tienes dos preocupaciones:
qu� clase de lista es, y qu� vas a poner en ella. Eso es todo.

As� pues, la idea esencial detr�s de \LyX{} es especificar lo que
se est� haciendo, no c�mo hacerlo. En lugar de un procesador {}``lo
que ves es lo que obtienes'' (WYSIWIG, What You See Is What You Get),
el modelo de \LyX{} es {}``lo que ves es lo que quieres decir''
(WYSIWIM, What You See Is What You Mean).


\subsubsection{Diferencias entre \LyX{} y los procesadores de texto usuales}

He aqu� una lista de cosas que no encontrar�s en \LyX{}:

\begin{itemize}
\item La regla (para medir m�rgenes)
\item Tabuladores
\item Espacios en blanco adicionales (i.e. pulsar \textsf{Intro} o \textsf{Espacio}
dos o m�s veces)
\end{itemize}
Los tabuladores, as� como la regla (que te muestra la posici�n de
cada elemento en la p�gina), son in�tiles en \LyX{}. El programa se
preocupa de d�nde tiene que ir cada cosa, no t�. Con los espacios
en blanco adicionales ocurre lo mismo; \LyX{} los a�ade conforme son
necesarios, seg�n el contexto. Al principio puede resultar molesto
no poder escribir dos l�neas en blanco seguidas, pero cobra mucho
m�s sentido una vez que empiezas a pensar en t�rminos WYSIWYM.

Y aqu� tienes algunas cosas que presenta \LyX{} pero que no se usan
como podr�as pensar:

\begin{itemize}
\item Controles de sangrado
\item Saltos de p�gina
\item Espaciado entre l�neas (i.e. espaciado simple, doble, etc.)
\item Espacio en blanco, horizontal y vertical
\item Tipos de letra y tama�o
\item Estilo de letra (negrita, cursiva, subrayado, etc.)
\end{itemize}
Aunque aparecen en \LyX{}, no se necesitan normalmente. El programa
se preocupa de estas cosas por t�, actuando en consecuencia seg�n
lo que est�s haciendo. Diferentes partes del documento son autom�ticamente
puestas en diferente tama�o y estilo. El sangrado de cada p�rrafo
es dependiente del contexto; cada tipo de p�rrafo se sangra de manera
diferente. Los saltos de p�gina se manejan tambi�n de forma autom�tica.
En general, el espacio entre l�neas, entre palabras y entre p�rrafos
es variable, elegido por \LyX{}.%
\footnote{Se pueden ajustar todas estas caracter�sticas (s�lo el ajuste de unas
pocas requiere conocimientos de \LaTeX{}), tanto para todo el documento
como para una parte concreta. Ver Gu�a del Usuario para m�s detalles.
}

Por �ltimo, �stas son las �reas en las que \LyX{} (y \LaTeX{}) sobrepasan
a muchos procesadores de texto:

\begin{itemize}
\item Separaci�n de palabras a final de l�nea
\item Listas de cualquier tipo
\item Matem�ticas
\item Tablas
\item Referencias cruzadas
\end{itemize}
Por supuesto, muchos procesadores de texto modernos manejan s�mbolos
matem�ticos, tablas, separaci�n de palabras a final de l�nea, e incluso
comienzan a aproximarse a las definiciones de estilo y el concepto
WYSIWYM. Sin embargo, acaban de empezar a incluir estas caracter�sticas,
mientras que \LyX{} est� construido sobre el sistema de proceso de
documentos \LaTeX{}. �ste lleva m�s de 10 a�os con ellas, y \emph{funciona}.
Todos los errores han sido subsanados hace tiempo.%
\footnote{De acuerdo, nada es perfecto, pero \LaTeX{} es lo m�s cercano a un
programa libre de errores que se puede conseguir.
}


\subsubsection{�Qu� diablos es \LaTeX{}?}

\LaTeX{} es un sistema de preparaci�n de documentos dise�ado por Leslie
Lamport en 1985.%
\footnote{La informaci�n de esta secci�n ha sido extra�da de {}``A Guide to
\LaTeXe{}'', de Helmut Kopka y Patrick Daly, documento incluido en
la bibliograf�a de la \emph{Gu�a del Usuario}.
} Fue construido gradualmente sobre un lenguaje de composici�n de documentos
llamado \TeX{}, creado por Donald Knuth en 1984. {}``\TeX{}'' se
pronuncia como {}``blech'' en ingl�s%
\footnote{El ruido que se hace cuando se come algo con mal sabor, especialmente
los ni�os peque�os cuando no les gusta la comida. (N. del T.)
}. Sin embargo, muchos no comprenden qu� es exactamente. \TeX{} toma
una secuencia de �rdenes de composici�n escritos en un fichero ASCII,
y los ejecuta. Es m�s complicado que una m�quina de escribir, pero
sin llegar a la especializaci�n y complejidad de una aut�ntica imprenta.
En cualquier caso, produce como salida un fichero de formato llamado
{}``independiente de dispositivo'', o \texttt{dvi} para abreviar.
El fichero \texttt{dvi} puede ser le�do depu�s por otro programa que
acepte este formato, o convertido a otros formatos, como PostScript�.

Si no tuviera m�s caracter�stica que �sta, ser�a un mero motor de
composici�n. Sin embargo, \TeX{} tambi�n permite definir macros. Aqu�
es donde comienza la acci�n.

La mayor�a de la gente que usa \TeX{} est� usando realmente un conjunto
de macros que Knuth cre� para ocultar muchos de los detalles de composici�n.
Esto es en lo que piensa la gente cuando habla de \TeX{}. Los usuarios
normales no trabajan con \TeX{} puro, un esqueleto desnudo formado
�nicamente por comandos de composici�n. S�lo aqu�llos que crean conjuntos
de macros lo hacen. Y aqu� es donde Leslie Lamport entra en nuestra
historia. �l quer�a macros que fueran m�s orientadas al usuario y
menos a la composici�n, un conjunto de comandos que sirvieran para
componer secciones, tablas o f�rmulas matem�ticas de una forma consistente
y uniforme, sin demasiadas complicaciones. As� naci� \LaTeX{}.

Ahora, de manera simult�nea al desarrollo y crecimiento de \LaTeX{},
otras personas est�n creando sus propios paquetes personalizados de
macros para \TeX{}, algunos para realizar trabajos en publicaciones
matem�ticas y cosas as�. Unos usaron \TeX{} directamente, otros comenzaron
a modificar \LaTeX{}. Para tratar de unificar este l�o, un equipo
de expertos en \LaTeX{} (incluyendo a Lamport, por supuesto), empezaron
a trabajar en \LaTeXe{}, la versi�n actual del programa, a finales
de los ochenta. Esta nueva versi�n posee comandos que proporcionan
una interfaz m�s f�cil para la creaci�n de macros, ayuda para usar
las nuevas fuentes, y m�s mejoras. De hecho, \LaTeX{} es en s� mismo
un vasto lenguaje por derecho propio. Usuarios de todo el mundo han
estado creando sus propios a�adidos para \LaTeX{}, adem�s de los est�ndar.

Existen dos formas de extender \LaTeX{}: las clases y los estilos.
Una \emph{clase} es un conjunto de macros de \LaTeX{} (y de \TeX{})
que describen un nuevo tipo de documento, como un libro o un art�culo.
Hay clases para transparencias, para publicaciones de f�sica y matem�ticas\ldots{}
�algunas universidades tienen incluso una clase para su propio formato
de tesis! Un \emph{estilo,} a diferencia de una clase, no define un
nuevo tipo de documento, sino un nuevo tipo de \emph{comportamiento},
que puede ser utilizado por cualquier documento. Por ejemplo, \LyX{}
controla los m�rgenes de p�gina y el espaciado entre l�neas usando
dos ficheros de estilo diferentes de \LaTeX{}, dise�ados para este
fin. Hay ficheros de estilo para gran cantidad de cosas: imprimir
etiquetas o sobres, cambiar el sangrado normal del texto, a�adir nuevos
tipos de letra, manipular gr�ficos, dise�ar elaborados encabezados
de p�gina, personalizar bibliograf�as, alterar la posici�n y apariencia
de notas a pie de p�gina, tablas y figuras, personalizar listas, etc,
etc.

Aqu� tienes un resumen:

\begin{lyxlist}{00.00.0000}
\item [\TeX{}:]Lenguaje de composici�n con capacidad para uso y creaci�n
de macros.
\item [\LaTeX{}:]Paquete de macros construido sobre \TeX{}\@.
\item [clases:]Descripciones de un tipo de documento, usando \LaTeX{}\@.
\item [estilos:]Alteran alg�n aspecto del comportamiento normal de \LaTeX{}.
\item [\LyX{}:]Procesador de texto visual, WYSIWYM, que usa toda la potencia
de \LaTeX{} para realizar el trabajo de composici�n.
\end{lyxlist}
La idea de esta secci�n ha sido tratar de explicarte \emph{por qu�}
\LyX{} funciona de manera diferente a otros procesadores de texto.
La raz�n es simple: usa \LaTeX{} como motor de composici�n. Como este
�ltimo, se centra en el contexto de tu escritura \emph{(lo que} est�s
escribiendo). El ordenador se encarga entonces de c�mo debe aparecer.

Ah, una cosa m�s. \LaTeX{} se pronuncia como \TeX{}. Rima con {}``hey
blech''.%
\footnote{Estos detalles de pronunciaci�n no son relevantes para hispanohablantes.
\TeX{}, \LaTeX{} y \LyX{} tienen pronunciaci�n directa en espa�ol
(N. del T.)
} Lamport, sin embargo, dice en su libro que {}``\emph{lay}-tecks
tambi�n es posible''. Por otro lado, {}``\LyX{}'' se pronuncia
como {}``licks'', {}``lucks'' o {}``looks'', dependiendo de
la pronunciaci�n de cada pa�s\ldots{}


\section{Empezando con \LyX{}}


\subsection{Tu primer documento \LyX{}}

\label{sec:first-doc-ex} Muy bien\@. Ya est�s listo para empezar
a escribir. Sin embargo, antes de que lo hagas hay un par de cosas
que debemos decir, y que esperamos har�n el \emph{Tutorial} m�s instructivo,
�til y divertido.

Como hay mucha informaci�n que no te vamos a dar aqu�, lo primero
que tienes que hacer es encontrar los otros ficheros de ayuda. Afortunadamente,
esto es muy simple. Arranca \LyX{}. Elige la \emph{Gu�a del Usuario}
en el men� \textsf{A}\textsf{\underbar{y}}\textsf{uda}. Tambi�n puedes
cargar el \emph{Tutorial} (si es que no lo est�s leyendo ya desde
ah�). De esta forma, puedes leerlos mientras escribes tu propio fichero%
\footnote{Tambi�n pueden servir como buenos ejemplos de uso de las caracter�sticas
de \LyX{}.
}. Ten en cuenta que, una vez que tienes m�s de un documento abierto,
puedes usar el men� \textsf{\underbar{D}}\textsf{ocumentos} para alternar
entre ellos. El \emph{Tutorial} no va a cubrir en detalle aquellos
temas que sean tratados en otros manuales de \LyX{}. Esto puede complicarte
la vida al principio, pero evita que el \emph{Tutorial} se haga muy
extenso. Te acostumbrar� tambi�n a usar los dem�s manuales, lo cual
---a largo plazo--- te ahorar� mucho tiempo.

En este \emph{Tutorial}, vamos a asumir que tienes instalada una versi�n
de \LyX{} funcionando perfectamente, as� como \LaTeX{}, \texttt{xdvi}
o cualquier otro visor de ficheros dvi, \texttt{dvips} o alguna otra
forma de convertir documentos \texttt{dvi} a PostScript�, y una impresora.
Esto es asumir mucho. Si alguna de estas condiciones no se cumple,
t� mismo deber�s configurar aquello que falte (o bien tu administrador
del sistema). Encontrar�s informaci�n sobre configuraci�n en otros
manuales.

Finalmente, hemos preparado un fichero para que practiques tus habilidades
con \LyX{} en �l. Se llama \texttt{es\_ejemplo\_sin\_lyx.lyx}. Imagina
que fue escrito por alguien que no conoce ninguna de las magn�ficas
caracter�sticas de \LyX{}. Conforme vayas aprendiendo nuevas funciones
te iremos sugiriendo que corrijas las partes correspondientes del
fichero \texttt{es\_ejemplo\_sin\_lyx.lyx}. Adem�s, contiene {}``sutiles''
trucos sobre c�mo arreglar las cosas%
\footnote{Los trucos se encuentran en {}``notas'' amarillas. Puedes leerlas
pulsado con el rat�n sobre ellas.
}. Si quieres hacer trampa (o comprobar lo que has hecho), hay tambi�n
un fichero llamado \texttt{es\_ejemplo\_con\_lyx.lyx} que contiene
el mismo texto, pero escrito por un experto en \LyX{}.

Los ficheros de ejemplo se pueden encontrar en el directorio \texttt{examples/},
que puedes conseguir seleccionando \textsf{\underbar{A}}\textsf{rchivo\lyxarrow{}}\textsf{\underbar{A}}\textsf{brir}
y pulsando el bot�n \textsf{Ejemplos}. Abre el documento sin procesar
(\texttt{es\_ejemplo\_sin\_lyx.lyx}), y usa el men� \textsf{\underbar{A}}\textsf{rchivo\lyxarrow{}Guardar}
\textsf{\underbar{C}}\textsf{omo} para guardar una copia en tu propio
directorio para que puedas trabajar con �l. Conforme vayas arreglando
el documento, comprueba c�mo los cambios afectan a la salida dvi (\textsf{\underbar{A}}\textsf{rchivo\lyxarrow{}Ver
dvi}).

Por cierto, el directorio \texttt{examples/} contiene muchos otros
ficheros de ejemplo que te ense�ar�n c�mo hacer con \LyX{} algunas
cosas bastante elaboradas. Son especialmente �tiles para mostrar aquello
que no cabr�a en la documentaci�n (por su extensi�n u otras razones).
Despu�s de leer el \emph{Tutorial}, o cuando est�s confundido a la
hora de hacer algo complicado con \LyX{}, echa un ojeada a estos ficheros.


\subsubsection{Escribiendo, viendo e imprimiendo}

\begin{itemize}
\item Abre un fichero nuevo con \textsf{\underbar{A}}\textsf{rchivo\lyxarrow{}}\textsf{\underbar{N}}\textsf{uevo}
\item Escribe una frase: \texttt{�Este es mi primer documento LyX!}%
\footnote{De acuerdo. Realmente puedes escribir lo que quieras. No importa.
Nos disculpamos por la estupidez de esta frase, y de todas las que
te vamos a pedir que escribas de aqu� en adelante.
}
\item Guarda el documento con \textsf{\underbar{A}}\textsf{rchivo\lyxarrow{}Guardar}
\textsf{\underbar{C}}\textsf{omo\@.}
\item Ejecuta \LaTeX{} para crear un fichero \texttt{dvi}, con \textsf{\underbar{A}}\textsf{rchivo\lyxarrow{}Ver
dvi}\@. Puedes ver que se imprimen algunas l�neas en la ventana desde
la que has ejecutado el comando lyx. Se trata de mensajes de \LaTeX{},
que por ahora puedes ignorar. \LyX{} lanzar� el programa \texttt{xdvi}
(o alg�n otro visualizador de ficheros \texttt{dvi}), que abrir� una
nueva ventana mostr�ndote el aspecto de tu documento cuando est� impreso.%
\footnote{Puedes ahorrar tiempo dejando que \texttt{xdvi} se ejecute en segundo
plano. De esta forma, puedes usar \textsf{\underbar{A}}\textsf{rchivo->Actualizar
dvi} y simplemente pulsar sobre la ventana de \texttt{xdvi} (o restaurarla)
cuando \LaTeX{} termine de ejecutarse.
}
\item Imprime con \textsf{}\textsf{\underbar{A}}\textsf{rchivo\lyxarrow{}Im}\textsf{\underbar{p}}\textsf{rimir}
y pulsa \textsf{OK\@.}
\end{itemize}
�Enhorabuena! Has escrito e impreso tu primer documento \LyX{}. Todo
lo dem�s son detalles, a cubrir por el resto del \emph{Tutorial},
la \emph{Gu�a del Usuario} y el \emph{Manual de Referencia}.


\subsubsection{Operaciones sencillas}

Por supuesto, \LyX{} puede realizar la mayor�a de las cosas a las
que est�s acostumbrado con tu procesador de texto. Separar� las palabras
y justificar� los p�rrafos autom�ticamente. Basta que accedas a un
par de men�s%
\footnote{Si eres como tantos usuarios de \textsc{unix}, lo habr�s hecho ya
mucho antes de empezar a leer el \emph{Tutorial}.
} para que veas c�mo la mayor parte de los comandos simples (i.e.,
\textsf{\underbar{A}}\textsf{rchivo\lyxarrow{}Salir,} \textsf{\underbar{E}}\textsf{dici�n\lyxarrow{}P}\textsf{\underbar{e}}\textsf{gar,}
\textsf{\underbar{A}}\textsf{rchivo\lyxarrow{}Im}\textsf{\underbar{p}}\textsf{rimir)}
tienen los nombres que esperas que tengan, est�n en el men� donde
esperas que est�n, y funcionan tal y como esperas que funcionen. A
continuaci�n tienes una descripci�n r�pida de c�mo realizar algunas
acciones sencillas.

\begin{description}
\item [Deshacer]\LyX{} tiene capacidad para {}``deshacer infinitas veces'',
lo que significa que puedes deshacer todo lo que lo que hayas hecho
desde que empezaste la sesi�n actual, aplicando una y otra vez \textsf{\underbar{E}}\textsf{dici�n\lyxarrow{}Deshacer}.
Si deshaces demasiado, elige simplemente \textsf{\underbar{E}}\textsf{dici�n\lyxarrow{}}\textsf{\underbar{R}}\textsf{ehacer}
para recuperar los cambios. Actualmente, el comando deshacer est�
\emph{limitado a 100 pasos.} Tampoco funciona \emph{para todo} (por
ejemplo, en los cambios de formato de documento).
\item [Cortar/Pegar/Copiar]Utiliza \textsf{\underbar{E}}\textsf{dici�n\lyxarrow{}}\textsf{\underbar{C}}\textsf{ortar},
\textsf{\underbar{E}}\textsf{dici�n\lyxarrow{}}\textsf{\underbar{P}}\textsf{egar},
y \textsf{\underbar{E}}\textsf{dici�n\lyxarrow{}C}\textsf{\underbar{o}}\textsf{piar}
para cortar, pegar y copiar. O pega autom�ticamente el texto seleccionado
con el bot�n central del rat�n.
\item [Buscar/Reemplazar]Utiliza \textsf{\underbar{E}}\textsf{dici�n\lyxarrow{}}\textsf{\underbar{B}}\textsf{uscar~y~Reemplazar}
para realizar una b�squeda sensible a las may�sculas. En el men� que
se despliega a tal efecto, puedes desplazarte hacia delante y hacia
atr�s en la b�squeda mediante las flechas, y reemplazar aquellas palabras
que hayas encontrado con el bot�n \textsf{\underbar{R}}\textsf{eemplazar}.%
\footnote{Cierra la ventana cuando hayas acabado, o d�jala abierta si lo encuentras
m�s conveniente. La mayor�a de los men�s contextuales de \LyX{} (incluyendo
los de \textsf{\underbar{B}}\textsf{uscar~y~Reemplazar, �ndice~Gener}\textsf{\underbar{a}}\textsf{l}
y \textsf{\underbar{F}}ormato, as� como los de matem�ticas) son ventanas
que pueden ser apartadas, en vez de cerradas. Unos pocos men�s como
\textsf{\underbar{A}}\textsf{rchivo\lyxarrow{}}\textsf{\underbar{A}}\textsf{brir},
no te dejar�n escribir nada en la ventana principal hasta que los
cierres. Aseg�rate de que el foco est� en la ventana correcta cuando
est�s tratando escribir en la ventana principal de \LyX{} o introduciendo
un comando en alguna ventana de di�logo.
}
\item [Formato~de~caracteres]Puedes \emph{resaltar} texto (lo que normalmente
significa poner los caracteres en cursiva), ponerlo en \textbf{negrita},
o en \textsc{Estilo Nombre} (habitualmente en min�sculas, para nombres
propios de personas) desde los botones interruptor en el men� \textsf{\underbar{F}}\textsf{ormato}.
\item [Barra~de~herramientas]Sus botones (justo debajo de los men�s) te
permiten realizar las funciones m�s usuales, como \textsf{Pegar} e
\textsf{Imprimir}. Si mantienes el cursor del rat�n sobre alguno de
los botones de la barra, una peque�a nota amarilla te informar� sobre
la funci�n concreta del bot�n.
\item [\emph{Minibuffer}]La franja gris en la parte de abajo de la ventana
de \LyX{} recibe el nombre de \emph{minibuffer}. Se encarga de mostrarte
toda clase de informaci�n �til. Por ejemplo, cuando guardas, te dice
el nombre del fichero que acabas de guardar. Tambi�n muestra algunos
mensajes de error. Observa que puedes escribir en �l. Esto te ofrece
una gran funcionalidad, incluyendo la posibilidad de estropear el
documento. En otras palabras, no escribas en el \emph{minibuffer}
a menos que sepas lo que est�s haciendo.
\end{description}
Por supuesto, todav�a no has escrito suficiente para encontrar �tiles
todas estas funciones. Conforme vayas escribiendo m�s, prueba a deshacer,
copiar, pegar, etc.


\subsubsection{WYSIWYM: el espacio en blanco en \LyX{}}

\label{sec:whitespace}Una de las cosas m�s dif�ciles para los nuevos
usuarios es acostumbrarse a la forma en que \LyX{} maneja el espacio
en blanco. Por mucho que pulses \textsf{Retorno de carro}, s�lo conseguir�s
una �nica l�nea en blanco. Por mucho que pulses la \textsf{Barra espaciadora},
s�lo conseguir�s un �nico espacio en blanco. En una l�nea vac�a \LyX{}
no te dejar� poner ni siquiera un espacio. El \textsf{Tabulador} no
te adelantar� ning�n espacio, �de hecho no hay tabulaci�n! Tampoco
hay ninguna regla en la parte superior de la p�gina que te permita
definir tabulaciones o m�rgenes.

Muchos procesadores de texto comerciales est�n basados en el principio
WYSIWYG: {}``lo que ves es lo que obtienes''. \LyX{}, por el contrario,
est� basado en el principio {}``lo que ves es lo que quieres decir''.
Escribes lo que quieres decir, y \LyX{} se preocupar� de la composici�n
por t� para que el resultado final quede bien. Un \textsf{Retorno
de carro} gramaticalmente separa p�rrafos, y de la misma forma un
\textsf{espacio} separa palabras, as� que no hay ninguna raz�n para
poner varios seguidos; un \textsf{Tabulador} no tiene funci�n gramatical
alguna, as� que \LyX{} no los usa. Con \LyX{} emplear�s m�s tiempo
en el contenido del documento, y menos en la forma. Ver Secci�n \ref{sec:what-is-lyx}
para m�s informaci�n sobre el concepto WYSIWYM.

\LyX{} tiene (muchas) formas de ajustar al detalle el formato del
documento. Despu�s de todo, podr�a no imprimirse exactamente lo que
quer�as decir. La \emph{Gu�a del Usuario} contiene informaci�n sobre
eso. Incluye espaciado vertical y horizontal ---mucho m�s potentes
y vers�tiles que m�ltiples espacios o l�neas en blanco--- y formas
de cambiar tama�o y estilo de letra y alineaci�n de p�rrafos a mano.
La idea es que puedas escribir todo el documento concentr�ndote en
el contenido, y solamente te preocupes de ajustar los detalles al
final. Con los procesadores de texto convencionales, el formato te
distrae continuamente.


\subsection{Entornos}

Las diferentes partes de un documento tienen prop�sitos diferentes;
llamaremos a estas partes \emph{entornos}. La mayor parte del documento
est� formada por texto normal. Los t�tulos de secci�n (cap�tulos,
subsecciones, etc.) permiten al lector saber que se va a tratar un
nuevo concepto o idea. Ciertos tipos de documentos tienen entornos
especiales. Un art�tculo de peri�dico tendr� un resumen y un t�tulo.
Una carta no tendr� nada de eso, pero probablemente contendr� un entorno
para la direcci�n del remitente.

Los entornos son una parte importante en la filosof�a {}``lo que
ves es lo que quieres decir'' de \LyX{}. Un entorno dado puede requerir
un cierto estilo o tama�o de letra, sangrado, espaciado y otras caracter�sticas.
Este problema se agrava cuando el formato exacto de un entorno puede
cambiar: un peri�dico puede usar letra en negrita, de 18 puntos con
p�rrafos centrados para los t�tulos, mientras que otros pueden usar
p�rrafos justificados con letra cursiva de 15 puntos; idiomas distintos
pueden tener diferentes convenios para el sangrado; y los formatos
de bibliograf�a pueden variar ampliamente. \LyX{} te evita tener que
aprender todos los diferentes estilos de formato.

La caja de \textsf{Entorno} se sit�a al final a la izquierda en la
barra de herramientas (justo debajo del men� \textsf{\underbar{A}}\textsf{rchivo}).
Indica qu� entorno est�s usando en cada momento. Mientras escrib�as
tu primer documento, dec�a {}``Standard'' (normal), que es el entorno
por defecto para texto. Ahora vas a usar varios entornos en el nuevo
documento para que puedas ver c�mo funcionan. Lo har�s con el men�
\textsf{Entorno,} que puedes abrir pulsando sobre el icono con {}``la
flecha hacia abajo'' justo a la derecha de la caja de \textsf{Entorno.}


\subsubsection{Secciones y subsecciones}

Escribe la palabra \texttt{Introducci�n} en la primera l�nea de tu
fichero \LyX{}, y elige \textsf{Section} (secci�n) en el men� de entorno%
\footnote{No tienes que seleccionar la l�nea. Si no hay nada seleccionado, \LyX{}
cambia el p�rrafo en el que est�s escribiendo ahora al entorno elegido.
Alternativamente, puedes cambiar varios p�rrafos seleccion�ndolos
antes de elegir el nuevo entorno.
}. \LyX{} numera la secci�n como {}``1'' y escribe el encabezado
(t�tulo ) en un tipo de letra mayor (por supuesto, el encabezado aparecer�
de esta forma en el fichero \texttt{dvi} y en el documento impreso).
Ahora pulsa \textsf{Retorno de carro.} Observa que la caja de entorno
cambia de {}``Section'' a {}``Standard''. Se asume que los t�tulos
de secci�n, como muchos entornos, terminan cuando introduces un \textsf{Retorno
de carro}%
\footnote{Ver \emph{Gu�a del Usuario} para poder escribir t�tulos de m�s de
una l�nea. Desde luego, el entorno \textsf{Standard} puede continuar
a lo largo de varios p�rrafos. Los entornos de listas (ver m�s adelante)
tampoco terminan con \textsf{Retorno de carro.} Siempre puedes saber
el entorno en el que est�s mirando la caja de \textsf{Entorno.}
}\textsf{.} Escribe la introducci�n del documento:

\begin{lyxcode}
Esto~es~una~introducci�n~a~mi~primer~documento~LyX.
\end{lyxcode}
Pulsa \textsf{Retorno de carro} otra vez, y elige de nuevo \textsf{Section}
en el men� entorno. \LyX{} escribe un {}``2'' y espera que intoduzcas
un t�tulo. Escribe \texttt{M�s cosas}, y ver�s que de nuevo lo establece
como t�tulo de secci�n.

La cosa mejora. Ve al final de la secci�n 1 otra vez (tras {}``mi
primer documento \LyX{}''), pulsa \textsf{Retorno de carro}, y selecciona
\textsf{Section} en el men� \textsf{Entorno}. Una vez m�s, \LyX{}
escribe {}``2'' y espera a que introduzcas un t�tulo. Escribe \texttt{Acerca
de este documento}. La secci�n {}``M�s cosas'', que antes era la
secci�n 2, �ha sido autom�ticamente renumerada a secci�n 3! De una
forma verdaderamente WYSIWYM, s�lo necesitas identificar los t�tulos
de las distintas partes de que se compone el texto, y \LyX{} se encarga
de la numeraci�n de secciones y su formato.

Pulsa \textsf{Retorno de carro} para volver al entorno \textsf{Standard}
y escribe las siguientes cinco l�neas:

\begin{lyxcode}
Secciones~y~subsecciones~se~describen~m�s~adelante.

Descripci�n~de~secci�n

Las~secciones~son~mayores~que~las~subsecciones.

Descripci�n~de~subsecci�n

Las~subsecciones~son~menores~que~las~secciones.
\end{lyxcode}
Col�cate en la segunda l�nea y elige \textsf{Subsection} (subsecci�n)
en el men� \textsf{Entorno}. \LyX{} numera la secci�n como {}``2.1'',
y la escribe con un tama�o de letra mayor que el de texto regular
pero menor que el de un t�tulo de secci�n. Cambia tambi�n el entorno
de la cuarta l�nea a \textsf{Subsection}. Como probablemente esperabas,
\LyX{} la numera autom�ticamente como secci�n {}``2.2''. Si pones
una secci�n m�s antes de la secci�n 2, �sta ser� renumerada como secci�n
3, y sus subsecciones corespondientes como {}``3.1'' y {}``3.2''.

Niveles m�s profundos de secci�n son la subsubsecci�n (\textsf{Subsubsection}),
el p�rrafo (\textsf{Paragraph}), y el subp�rrafo (\textsf{Subparagraph}).
Dejaremos que juegues t� mismo con ellos. Notar�s que los t�tulos
de p�rrafo y de subp�rrafo no est�n numerados por defecto, y que los
subp�rrafos est�n sangrados; ver \emph{Gu�a del Usuario} para cambiar
esto. Los encabezados de cap�tulo (\textsf{Chapter}) son realmente
el nivel m�s alto de la jerarqu�a, por encima de las secciones, pero
solamente se pueden utilizar en ciertos tipos (clases) de documentos
(ver Secci�n \ref{sec:textclasses}). 

Finalmente, puede que quieras usar secciones y subsecciones sin numerar.
Existen entornos para esto tambi�n. Si cambias uno de los encabezados
de secci�n al entorno \textsf{Section{*}} (puede que tengas que bajar
en el men� \textsf{Entorno} para encontralo), \LyX{} usar� el mismo
tama�o de letra que en las secciones normales, pero no la numerar�.
Tambi�n est�n los entornos {}``no numerados'' correspondientes a
\textsf{Subsection} y \textsf{Subsubsection}. Intenta cambiar alguna
de tus secciones o subsecciones a entornos no numerados, y comprueba
c�mo los dem�s n�meros de secci�n se actualizan.

\textbf{Ejercicio}: Arregla los encabezados de secci�n y subsecci�n
del fichero \texttt{es\_ejemplo\_sin\_lyx.lyx}. 


\subsubsection{Listas y sublistas}

\LyX{} tiene diferentes entornos para componer listas. Los variados
entornos de listas te evitan tener que pulsar el \textsf{Tabulador}
un mill�n de veces cuando est�s escribiendo un esquema, o de renumerar
toda la lista cuando quieres a�adir un nuevo punto en mitad de ella.
As� puedes concentrarte en el contenido de la lista%
\footnote{S�, estamos recalcando una y otra vez este punto a lo largo de todo
el \emph{Tutorial}. Pero \emph{es} la principal filosof�a de \LyX{},
as� que por favor, disc�lpanos.
}. Distintos tipos de documentos requieren, l�gicamente, entornos de
lista diferentes:

\begin{itemize}
\item Una exposici�n de diapositivas podr�a usar las listas simples (etiquetadas
con bolos) del entorno \textsf{Itemize} para describir los diferentes
puntos.
\item Un esquema usar�a las listas numeradas (y sublistas etiquetadas con
letras) del entorno \textsf{Enumerate} (enumeraci�n). 
\item Un documento que describa varios paquetes de software usar�a el entorno
\textsf{Description} (descripci�n), en el que cada elemento de la
lista comienza con una palabra en negrita. 
\item El entorno \textsf{List} (que no existe en \LaTeX{}) es ligeramente
diferente al entorno \textsf{Description}.
\end{itemize}
Vamos a escribir una lista de razones por las que \LyX{} es mejor
que otros procesadores de texto. En cualquier parte de tu documento
escribe:

\begin{lyxcode}
LyX~es~mejor~que~otros~procesadores~de~texto~porque:~
\end{lyxcode}
y pulsa \textsf{Retorno de carro}. Ahora elige \textsf{Itemize} en
el men� de \textsf{Entorno}. \LyX{} pondr� un bolo (realmente un asterisco,
que se convertir� en un c�rculo en la salida final) en la l�nea. Introduce
tus razones:

\begin{lyxcode}
LyX~realiza~la~composici�n~por~t�.

Las~matem�ticas~son~WYSIWYG

�Las~listas~son~muy~f�ciles~de~crear!
\end{lyxcode}
Los entornos de listas, al contrario que los encabezados, no terminan
cuando introduces un \textsf{Retorno de carro}. En vez de eso, \LyX{}
asume que vas a introducir el siguiente elemento de la lista. La anterior
resultar� ser una lista de tres elementos. Si deseas m�s de un p�rrafo
en un solo \emph{elemento} de la lista, una forma de conseguirlo es
mediante un \textsf{Retorno de carro protegido}, pulsando \textsf{C-Retorno
de carro}. Para salir de la lista tienes que volver a seleccionar
el entorno \textsf{Standard} (o usar la combinaci�n de teclas \textsf{M-p~s}).

Has conseguido una bonita lista simple. Puedes ejecutar \LaTeX{} para
ver c�mo aparece en la salida impresa. Pero, �y si quer�as numerar
las razones? Bien, simplemente selecciona toda la lista%
\footnote{\LyX{} no te dejar� seleccionar el primer bolo a menos que selecciones
el p�rrafo anterior a la lista, que probablemente no es lo que quieres
hacer. De forma similar, no puedes seleccionar el n�mero en el t�tulo
de una secci�n. No te preocupes de eso.
} y elige el entorno \textsf{Enumerate} en el men�. �Incre�ble! Como
ya dijimos, si a�ades o borras un elemento de la lista, \LyX{} arregla
la numeraci�n.

Mientras la lista est� selecionada, puedes cambiar a los otros dos
entornos, \textsf{Description} y \textsf{List}, para ver c�mo son.
Para ambos, cada elemento de la lista est� compuesto por un t�rmino,
que es la primera palabra del elemento, seguido de una definici�n,
que es el resto del p�rrafo (hasta que pulses \textsf{Retorno de carro}).
El t�rmino se escribe en negrita (\textsf{Description}) o separado
por un {}``Tabulador''%
\footnote{Pero un tabulador de composici�n, que cambiar� para ajustarse al tama�o
del mayor t�rmino de la lista, no un r�gido, inmutable y pat�tico
\textsf{Tabulador} de m�quina de escribir.
}(\textsf{List}) del resto del p�rrafo. Si quieres m�s de una palabra
en el t�rmino, separa las palabras con \textsf{Espacios~protegidos},
que se obtienen al pulsar \textsf{C-Espacio} y aparecen como peque�as
{}``ues'' rosas.

\textbf{Ejercicio}: Escribe correctamente la lista en el fichero \texttt{es\_ejemplo\_sin\_lyx.lyx}

Puedes anidar listas unas dentro de otras en toda clase de formas
interesantes. Un ejemplo ser�a escribir esquemas. Las listas simples
y numeradas tendr�n diferentes tipos de bolos y diferente numeraci�n
en las sublistas. Ver \emph{Gu�a del Usuario} para los detalles de
los distintos tipos de listas, as� como ejemplos de c�mo usar \emph{mucho}
anidamiento.


\subsubsection{M�s entornos: estrofas, citas y otros}

Hay dos entornos para separar las citas del texto que las rodea: \textsf{Quote}
para citas cortas y \textsf{Quotation} para las m�s largas. El c�digo
de ordenador (el entorno \textsf{\LyX{}-Code}, usado tambi�n en este
\emph{Tutorial} para ejemplos largos) se escribe en letra de \texttt{m�quna
de escribir}; este entorno es el �nico sitio en \LyX{} donde se permite
usar varios espacios seguidos para permitir el sangrado del c�digo.
Puedes incluso escribir poes�a%
\footnote{\ldots{}suponiendo que seas lo suficientemente creativo para comenzar
escribiendo poes�a.
} mediante el entorno \textsf{Verse} (estrofa), usando \textsf{Retornos
de carro} para separar los versos, y \textsf{C-Retorno de carro} para
separar l�neas dentro de un verso. Ver Gu�a del Usuario para una descripci�n
m�s completa de todos los entornos disponibles en \LyX{}.

\textbf{Ejercicio}: Usa los entornos \textsf{Quote, \LyX{}-Code,}
y \textsf{Verse} donde corresponda en el fichero \texttt{es\_ejemplo\_sin\_lyx.lyx}


\section{Escribiendo documentos}

Esperamos que el cap�tulo anterior te haya servido para acostumbrarte
a esribir con \LyX{}. Te hemos presentado las operaciones b�sicas
de edici�n, as� como el potente m�todo de escribir con entornos. Sin
embargo, la mayor�a de la gente que usa \LyX{} querr� escribir documentos:
peri�dicos, art�culos, libros, manuales o cartas. Este cap�tulo pretende
que pases de escribir simple texto a escribir un documento completo.
Te presentar� las clases de texto, que te permiten crear distintos
tipos de documentos, y te describir� muchas caracter�sticas nuevas
que convierten texto en un documento, como t�tulos, notas a pie de
p�gina, referencias cruzadas, bibliograf�as e �ndices.


\subsection{Clases de texto y modelos}

\label{sec:textclasses}Diferentes tipos de documentos deben componerse
de forma diferente. Por ejemplo, normalmente los libros se imprimen
a doble cara, mientras que los art�culos se imprimen a simple. Adem�s,
muchos documentos contienen entornos especiales: las cartas tienen
entornos (como la direcci�n del remitente o la firma) que no tienen
sentido en un libro o un art�culo. Las \emph{clases de texto} de \LyX{}%
\footnote{Para usuarios de \LaTeX{}: equivalente a las clases de documento de
\LaTeX{} (\emph{documentclass}).
} se encargan de estas grandes diferencias entre cada tipo de documento.
Este Tutorial, por ejemplo, se ha escrito con la clase de texto \textsf{Book}
(libro). Estas clases son otro de los grandes pilares de la filosof�a
WYSIWYM; le dicen a \LyX{} c�mo tiene que componer el documento, as�
que t� no necesitas saberlo.

Tu documento se est� escribiendo seguramente con la clase \textsf{Article}
(art�culo)%
\footnote{Normalmente el art�culo es la clase de texto por defecto, aunque puedes
establecerla en tu fichero de configuraci�n \texttt{lyxrc}.
}. Prueba a cambiar a otras clases (usa el men� \textsf{\underbar{C}}\textsf{lase}
dentro de \textsf{\underbar{F}}\textsf{ormato\lyxarrow{}}\textsf{\underbar{D}}\textsf{ocumento})
para ver c�mo se compone cada una de ellas. Si la cambias a \textsf{Book}
y miras en el men� \textsf{Entorno}, ver�s que la mayor�a de entornos
permitidos son los mismos. Sin embargo, ahora puedes utilizar el entorno
\textsf{Chapter} (cap�tulo). Si alguna vez no est�s seguro de cu�les
tienes disponibles en una clase de texto dada, s�lo tienes que consultar
el men� \textsf{Entorno}.

El tama�o de letra, la impresi�n a una o dos columnas, o los encabezados
de p�gina son s�lo algunas de las cosas en las que difiere el formato
de composici�n de los distintos peri�dicos. Conforme la Era Digital
ha ido madurando, �stos han empezado a aceptar presentaciones electr�nicas,
creando {}``ficheros de estilo'' \LaTeX{} para que los autores puedan
enviar sus art�culos correctamente maquetados. \LyX{} tambi�n est�
preparado para esto. As� por ejemplo, ofrece soporte para composici�n
(y entornos adicionales) para los peri�dicos de la Sociedad Americana
de Matem�ticas mediante la clase de texto \textsf{Article~(AMS).}

A continuaci�n te damos una breve referencia r�pida de algunas clases
de texto. Como siempre, dir�gete a la \emph{Gu�a del Usuario} para
m�s detalles.

\vspace{0.3cm}
{\centering \begin{tabular}{|c|c|}
\hline 
Nombre&
Comentarios\\
\hline
\hline 
article&
art�culo --- simple cara, sin cap�tulos\\
\hline 
article (AMS)&
formato y entornos de la Sociedad Americana de Matem�ticas\\
\hline 
report&
informe --- m�s extenso que el art�culo, doble cara\\
\hline 
book&
libro --- informe + portada y contraportada\\
\hline
slides&
transparencias (incluyendo Foil\TeX{})\\
\hline
letter&
carta --- entornos adicionales para la direcci�n, la firma\ldots{}\\
\hline
\end{tabular}\par}
\vspace{0.3cm}


\subsection{Modelos: escribir una carta}

Una de las clases de texto m�s populares es la carta. Una forma de
escribir una carta ser�a abrir un \textsf{\underbar{N}}\textsf{uevo}
fichero, y elegir \textsf{Letter} en el men� \textsf{\underbar{C}}\textsf{lase}
dentro de \textsf{\underbar{F}}\textsf{ormato\lyxarrow{}}\textsf{\underbar{D}}\textsf{ocumento}.
Aunque esta es la manera m�s obvia de hacerlo, supone trabajo de m�s.
Cada vez que escribes una carta de negocios pones tu direcci�n, la
del destinatario, el cuerpo, la firma, etc. Por tanto, \LyX{} ofrece
un \emph{modelo} para cartas, que contiene un ejemplo de carta; una
vez que tienes el modelo, s�lo tienes que sustituir un par de cosas
cada vez que quieras escribir una nueva carta.

Abre un archivo nuevo con \textsf{\underbar{A}}\textsf{rchivo\lyxarrow{}Nuevo~basado~en~Modelo}.
Tras decidir un nombre para el nuevo fichero, elige \texttt{latex\_letter.lyx}
en el men� \textsf{Seleccionar~Modelo}. Guarda e imprime el fichero
para ver c�mo se componen los distintos entornos.

Si te fijas en el men� \textsf{Entorno}, ver�s algunos entornos, como
\textsf{My~Address} (direcci�n del remitente), que no est�n disponibles
en otras clases. Otros, como \textsf{Quote} y \textsf{Description},
son familiares. Puedes jugar con ellos para ver c�mo funcionan. Comprobar�s,
por ejemplo, que en el entorno \textsf{Signature} (firma) la palabra
{}``Signature:{}`` en rojo antecede al texto de la firma. Esta palabra
no se muestra en la verdadera carta, como podr�s ver si imprimes el
fichero. S�lo est� ah� para que sepas d�nde va la firma. Ten en cuenta
tambi�n que no importa d�nde est� situada la l�nea \textsf{Signature}.
Recuerda que \LyX{} es WYSIWYM, as� que puedes poner el entorno \textsf{Signature}
en el lugar que quieras, �l sabe que en la salida impresa la firma
debe ir al final.

Un modelo es simplemente un fichero de \LyX{}. Esto quiere decir que
puedes completarlo con tu direcci�n y tu firma y guardarlo como un
nuevo modelo. A partir de ahora, siempre que quieras escribir una
carta ahorrar�s tiempo usando tu nueva plantilla. Probablemente no
necesitamos sugerirte que hagas un verdadero {}``ejercicio'', �escribe
una carta a alguien!%
\footnote{Una advertencia si est�s escribiendo a partir de un modelo. Si borras
todo el texto de un entorno ---por ejemplo, si borras todo el campo
\textsf{My~Address} para poder poner tu propia direcci�n--- y entonces
mueves el cursor sin escribir nada, el entorno podr�a desaparecer.
Esto se debe a que muchos entornos no pueden existir sin contener
texto alguno. Para restituirlo, solamente tienes que volver a seleccionarlo
en el men� \textsf{Entorno}.
}

Los modelos pueden ahorrar much�simo tiempo, as� que te instamos a
que los uses siempre que puedas. Adem�s, te pueden ayudar a usar algunas
de las clases de texto m�s elaboradas y complejas. Finalmente, pueden
ser �tiles para alguien que quiera configurar \LyX{} para un grupo
de usuarios poco experimentados con los ordenadores. A la hora de
empezar a trabajar con \LyX{}, resulta mucho menos intimidatorio tener,
por ejemplo, un modelo de carta personalizado para tu empresa.


\subsection{T�tulo del documento}

\LyX{} (al igual que \LaTeX{}) considera el t�tulo ---que puede incluir
el t�tulo propiamente dicho, el autor, la fecha e incluso el resumen
del documento--- como una parte independiente.

Vuelve a tu documento \texttt{nuevo-archivo.lyx} y aseg�rate de que
est� usando la clase de texto \textsf{Article}.%
\footnote{No debes usar la clase carta, ya que �sta no permite t�tulos.
} Escribe un t�tulo en la primera l�nea, y c�mbiala al entorno \textsf{Title}
(t�tulo). En la siguiente l�nea escribe tu nombre y c�mbiala a entorno
\textsf{Author.} En la siguiente escribe la fecha con el entorno \textsf{Date}
(fecha). Escribe un p�rrafo o dos resumiendo el contenido de tu documento
y usa el entorno \textsf{Abstract} (resumen). Ahora mira c�mo queda
una vez impreso.

\textbf{Ejercicio}: Arregla el t�tulo, la fecha y el autor en \texttt{es\_ejemplo\_sin\_lyx.lyx}


\subsection{Etiquetas y referencias cruzadas}

\label{sec:labels}Puedes etiquetar una secci�n de tu documento (o
una subsecci�n, o incluso, con menos frecuencia, un fragmento de texto
cualquiera). Una vez que lo hagas puedes hacer referencia a esta secci�n
desde otras partes del documento mediante referencias cruzadas. Puedes
referirte al n�mero de secci�n o bien a la p�gina donde aparece. Como
suced�a con las secciones y las notas a pie de p�gina, el propio \LyX{}
se encarga tambi�n de las referencias. La gesti�n autom�tica de etiquetas
y referencias cruzadas es una de las mayores ventajas de \LyX{} (y
\LaTeX{}) sobre los procesadores de texto convencionales.


\subsubsection*{Tu primera etiqueta}

Vamos a marcar nuestra segunda secci�n, cuyo t�tulo es {}``Acerca
de este documento''. Pincha al final de la l�nea del t�tulo, y selecciona
en el men� \textsf{\underbar{I}}\textsf{nsertar\lyxarrow{}Etiqueta}.
Se deplegar� una ventana de di�logo pregunt�ndote el nombre de la
etiqueta. Escribe \texttt{sec:acercadeldocumento}, que parece una
etiqueta suficientemente descriptiva para evitar confusiones con otras
que podamos a�adir m�s adelante%
\footnote{Escribimos {}``sec:{}`` porque tambi�n podemos etiquetar ecuaciones,
tablas y figuras.
}. Cuando pulses el bot�n \textsf{OK}, el nombre de la etiqueta se
situar� en un recuadro cerca del t�tulo de la secci�n.

Por cierto, tambi�n puedes colocar la etiqueta en cualquier lugar
de la secci�n; las referencias a secci�n se referir�n a la �ltima
cuyo encabezado vaya antes que la etiqueta. Sin embargo, ponerla en
la misma l�nea que el t�tulo (o bien en la primera l�nea de texto)
asegura que las referencias a p�gina se refieran al comienzo de la
secci�n.

Hasta ahora no hemos hecho nada (el fichero \texttt{dvi} tiene el
mismo aspecto, ya que las etiquetas no se muestran en el documento
impreso). Pero has a�adido una, y ahora puedes hacer referencia a
ella. Vamos a hacerlo a continuaci�n.


\subsubsection*{Tu primera referencia}

Sit�a el cursor en alg�n lugar de la secci�n 2 de tu documento. Escribe

\begin{lyxcode}
Si~quieres~saber~m�s~acerca~de~este~documento,~ve~a~mirar~\\
la~secci�n~,~que~se~puede~encontrar~en~la~p�gina~.
\end{lyxcode}
Ahora ---con el cursor tras la palabra {}``secci�n''--- elige \textsf{\underbar{I}}\textsf{nsertar\lyxarrow{}Refe}\textsf{\underbar{r}}\textsf{encia~Cruzada}.
Aparecer� la ventana \textsf{Insertar~referencia}. �sta muestra una
lista de etiquetas posibles a las que puedes hacer referencia. Por
el momento s�lo hay una, {}``sec:acercadeldocumento''. Selecci�nala
(estar� seleccionada por defecto) y pulsa el bot�n \textsf{Insertar~refe}\textsf{\underbar{r}}\textsf{encia}.
Ahora sit�a el cursor tras la palabra {}``p�gina'', y pulsa el bot�n
\textsf{Insertar n�mero~de~}\textsf{\underbar{p}}\textsf{�gina}
en la misma ventana.

\LyX{} coloca las referencias dentro de un recuadro en el lugar donde
estaba el cursor. En el documento impreso, cada marcador de referencia
ser� reemplazado por el n�mero de p�gina o de secci�n (seg�n lo que
hayas elegido en el men� \textsf{Insertar~referencia}). De manera
muy pr�ctica, las referencias act�an como enlaces hipertexto cuando
est�s editando el documento con \LyX{}; pinchar sobre una de ellas
mover� el cursor hasta la etiqueta referenciada. Utiliza \textsf{\underbar{A}}\textsf{rchivo\lyxarrow{}Actualizar~dvi},
y ver�s como en la �ltima p�gina hacemos referencia a la {}``secci�n
2'' y la {}``p�gina 1'' (o cualquiera que sea la p�gina en donde
aparezca el t�tulo de la secci�n 2).


\subsubsection*{M�s diversi�n con etiquetas}

Te hemos dicho que \LyX{} se encarga �l solo de la numeraci�n de las
referencias cruzadas; ahora podr�s comprobarlo. A�ade una nueva secci�n
antes de la secci�n 2. Ahora vuelve a ejecutar \LaTeX{}, y ---voil�!---
la referencia ha cambiado a la secci�n 3. Convierte {}``Acerca de
este documento'' en una subsecci�n, y la referencia indicar� ahora
la subsecci�n 2.1 en lugar de la secci�n 3. Por supuesto, la referencia
de p�gina no cambiar� a menos que a�adas una p�gina entera antes de
la etiqueta.

\begin{sloppypar}

Si quieres practicar algo m�s con las etiquetas, prueba a poner otra,
{}``sec:miprimeraetiqueta'', donde estaba la primera referencia
cruzada, y haz referencia a aqu�lla desde cualquier parte del documento.
Si vas a usar las referencias a menudo (si, por ejemplo, est�s escribiendo
un art�culo de prensa), puede ser conveniente que dejes la ventana
\textsf{Insertar~referencia} abierta.

\end{sloppypar}

Si quieres asegurarte de que las referencias a p�gina se mantienen
correctas incluso en documentos extensos, usa \textsf{C}\textsf{\underbar{o}}\textsf{piar}
para llevar un par de p�ginas de la Gu�a del Usuario al portapapeles,
y usa \textsf{P}\textsf{\underbar{e}}\textsf{gar} para inserrtar el
texto robado en tu documento%
\footnote{Por cierto, copiar un cap�tulo causar� un error de \LyX{}, ya que
los cap�tulos no se permiten en la clase art�culo. Si te sucede, borra
simplemente el t�tulo de cap�tulo. Si quieres saber por qu� ocurre
esto, ve a la secci�n \ref{sec:textclasses}.
}.

\textbf{Ejercicio}: Arregla las referencias en el fichero \texttt{es\_ejemplo\_sin\_lyx.lyx}


\subsection{Notas a pie de p�gina y notas al margen}

Las notas a pie de p�gina se pueden a�adir usando el bot�n \textsf{Insertar
Nota~a~pie} en la barra de herramientas%
\footnote{El bot�n muestra una flecha se�alando texto en rojo, justo debajo
de texto en negro.
} o bien accediendo en el men� a \textsf{\underbar{I}}\textsf{nsertar\lyxarrow{}Nota~al~pie}\@.
Pincha al final de la palabra {}``\LyX{}'' en cualquier parte de
tu documento y pulsa el bot�n \textsf{Insertar Nota~a~pie}. Una
l�nea de pie de p�gina se abrir� debajo de la l�nea en la que estabas
escribiendo. En el extremo izquierdo ver�s la palabra {}``\emph{foot''}
(pie) escrita en rojo sobre fondo gris. El resto de la l�nea est�
enmarcada en rojo; aqu� es donde escribir�s la nota. \LyX{} sit�a
el cursor al principio de la l�nea. Escribe:

\begin{lyxcode}
LyX~es~un~procesador~de~texto~de~composici�n.
\end{lyxcode}
Ahora pulsa en la palabra {}``\emph{foot''}. La l�nea de la nota
desaparece, dejando {}``\emph{foot'',} subrayada en rojo, mostrando
el sitio donde aparecer� el marcador de la nota en el texto impreso.
A esto se le denomina {}``plegar'' la nota. Puedes desplegar la
nota en cualquier momento ---y volver a editar el texto si quieres---
pulsando de nuevo en el marcador rojo.

Te preguntar�s por qu� el marcador de nota es una palabra en vez de
un n�mero. La respuesta es que \LyX{} se encarga tambi�n de la numeraci�n
de las notas en el texto impreso. Puedes comprobarlo t� mismo mirando
la salida \texttt{dvi} o impresa. Si a�ades m�s notas, \LyX{} las
renumera. Como \LyX{} (bueno, realmente \LaTeX{}) se preocupa de esto,
no hay necesidad de poner los n�meros en el fichero \LyX{}.

Una nota al pie puede ser cortada y pegada como texto normal. Adelante,
�int�ntalo! Todo lo que necesitas es seleccionar el marcador%
\footnote{Puede serte m�s f�cil seleccionarla con el teclado, ya que puedes
abrir sin querer la nota si est�s intentando seleccionar el marcador
con el rat�n.
}, \textsf{\underbar{C}}\textsf{ortar}lo y \textsf{P}\textsf{\underbar{e}}\textsf{gar}lo.
Adem�s, puedes convertir texto normal en una nota, basta que lo selecciones
y pulses el bot�n \textsf{Insertar Nota~a~pie}; convierte una nota
en texto normal pulsando el mismo bot�n cuando el cursor est� dentro
de la nota.

Las notas al margen se pueden a�adir mediante el bot�n \textsf{Insertar
Nota~al~margen}%
\footnote{El bot�n muestra una flecha apuntando a texto en rojo, al lado de
(al margen de) texto en negro, y se encuentra cerca del bot�n \textsf{Insertar
Nota~a~pie}; en la barra de herramientas.
} o bien el men� \textsf{\underbar{I}}\textsf{nsertar\lyxarrow{}Nota~al~}\textsf{\underbar{M}}\textsf{argen}\@.
Son como las notas a pie de p�gina, salvo que:

\begin{itemize}
\item los marcadores en pantalla dicen {}``\emph{margin''} (margen) en
vez de {}``\emph{foot''}
\item las notas se sit�an en el margen de la p�gina, en vez de bajo el texto.
\item no se numeran
\item cuando una nota es plegada, se sit�a un signo de admiraci�n en el
margen, que no se ver� en el texto impreso.
\end{itemize}
Convierte ahora tu nota al pie en texto, selecci�nala y convi�rtela
en una nota al margen. Ejecuta \LaTeX{} de nuevo para ver su aspecto.

\textbf{Ejercicio}: Arregla la nota a pie de p�gina en \texttt{es\_ejemplo\_sin\_lyx.lyx}


\subsection{Bibliograf�a}

\label{sec:bibliographies}Las bibliograf�as funcionan de manera similar
a las referencias cruzadas. La bibliograf�a contiene una lista de
referencias al final del documento que pueden ser referenciadas desde
cualquier parte del texto. Al igual que los t�tulos de secci�n, \LyX{}
y \LaTeX{} hacen tu trabajo m�s f�cil numerando autom�ticamente los
elementos de la bibliograf�a y modificando las referencias cuando
la numeraci�n cambia.

Ve al final del documento y activa el entorno \textsf{Bibliography}.
Ahora, cada p�rrafo que escribas ser� una referencia. Escribe \texttt{El
Tutorial de Lyx, por el equipo documentaci�n de LyX} como primera
referencia. Observa que \LyX{} pone autom�ticamente un n�mero encerrado
en un recuadro antes de cada referencia. Pincha con el rat�n en el
recuadro, y se abrir� una ventana de di�logo \textsf{Elemento~de~Bibliograf�a}.
El primer campo, la clave, te sirve para referirte a esta entrada
desde el documento \LyX{}. Por defecto es un n�mero. Cambia la clave
a {}``tutorialdelyx'' para que sea f�cil de recordar.

Ahora escoge un lugar cualquiera del documento donde querr�as insertar
una referencia. Hazlo con \textsf{\underbar{I}}\textsf{nsertar\lyxarrow{}Referenc}\textsf{\underbar{i}}\textsf{a~a~Cita\@.}
El programa dibuja un recuadro gris con tres signos de interrogaci�n
entre corchetes, y aparece una ventana \textsf{Cita}. El primer campo
tambi�n se llama \textsf{Clave}, y te permite elegir la entrada bibliogr�fica
que quieres citar%
\footnote{Esta es la raz�n por la que es una buena idea dar a las claves nombres
�nicos y l�gicos, en lugar de dejar el n�mero por defecto.
}. Con ayuda de la flecha hacia abajo a la derecha del campo \textsf{Clave},
selecciona {}``tutorialdelyx'' (en este momento es el �nico elemento
de la bibliograf�a). Ahora ejecuta \LaTeX{}, y ver�s que la cita aparece
entre corchetes en el texto, referenciando a la bibliograf�a al final
del documento.

�C�mo se usan los dem�s campos? El campo \textsf{Comenta}\textsf{\underbar{r}}\textsf{io}
en la ventana de \textsf{Cita} pondr� un comentario (como una referencia
a una p�gina o cap�tulo del libro o art�culo) entre corchetes tras
la referencia. Si quieres que las referencias tengan etiquetas en
vez de n�meros en la salida impresa (por ejemplo, algunos peri�dicos
usar�an {}``{[}Smi95{]}'' para hacer referencia a un art�culo escrito
por Smith en 1995), utiliza el campo \textsf{Label} (etiqueta) en
la ventana \textsf{Elemento~de~Bibliograf�a}. Como siempre, puedes
obtener m�s informaci�n en la \emph{Gu�a del Usuario}.

\textbf{Ejercicio:} Arregla la bibliograf�a y las citas en \texttt{es\_ejemplo\_sin\_lyx.lyx}


\subsection{�ndice general}

Puede que quieras poner un �ndice al principio de tu documento. \LyX{}
hace que sea algo muy f�cil. Simplemente pulsa \textsf{Intro} despu�s
del t�tulo del documento y antes del t�tulo de la primera secci�n%
\footnote{No te desesperes tratando de pinchar o borrar un n�mero de secci�n.
No funcionar�. No se permite editar el n�mero de secci�n de ninguna
forma, ya que \LyX{} controla el numerado de secciones.
}, y elige en el men� \textsf{\underbar{I}}\textsf{nsertar\lyxarrow{}Listas~e~�ndice~gral.\lyxarrow{}�ndice~general}.
Aparecer� una caja (tambi�n conocida como recuadro) con las palabras
{}``�ndice general'' en la primera l�nea del documento.

Esto puede no parecer muy �til. Sin embargo, si observas el fichero
\texttt{dvi}, ver�s que se ha generado un �ndice con todas las secciones
y subsecciones de tu documento. Una vez m�s, si reordenas las secciones
o a�ades alguna, estos cambios se ver�n reflejados en el fichero \texttt{dvi}
cuando lo actualices.

El �ndice no se imprime en la versi�n en pantalla del documento porque
no puedes editarlo de ninguna manera. Sin embargo, puedes mostrarlo
en una ventana separada pinchado con el rat�n en el recuadro del �ndice
o bien mediante \textsf{\underbar{E}}\textsf{ditar\lyxarrow{}}\textsf{\underbar{I}}\textsf{ndice~gener}\textsf{\underbar{a}}\textsf{l}%
\footnote{El comando del men� funcionar� incluso si no tienes recuadro de �ndice
en tu documento.
}. Esta ventana es una herramienta muy pr�ctica. Puedes usarla para
moverte a trav�s de tu documento. Pulsando en una (sub)secci�n del
�ndice se resaltar� esa l�nea y el cursor se mover� a ese lugar del
documento en la ventana de edici�n de \LyX{}. Tambi�n puedes usar
los cursores para moverte arriba y abajo en el �ndice. Puede que te
resulte conveniente dejar esta ventana abierta a lo largo de las sesiones
de edici�n.

Para deshacerte del �ndice, puedes borrar su marcador como cualquier
otro car�cter.

\textbf{Ejercicio}: Arregla el �ndice en \texttt{es\_ejemplo\_sin\_lyx.lyx}


\section{Usando las matem�ticas}

\LaTeX{} es utilizado por muchos cient�ficos porque ofrece una gran
calidad en el aspecto de las ecuaciones, evitando los caracteres de
control usados por otros procesadores de texto y sus editores de ecuaciones.
Sin embargo, muchos de estos cient�ficos se sienten frustrados porque
escribir ecuaciones con \LaTeX{} se parece m�s a programar que a escribir.
Afortunadamente, \LyX{} tiene soporte WYSIWYM para las ecuaciones.
Si est�s acostumbrado a \LaTeX{}, ver�s que sus comandos matem�ticos
usuales se pueden introducir normalmente, aunque se muestran de forma
WYSIWYM. Por el contrario, si nunca has usado \LaTeX{}, el \textsf{Panel~de~F�rmulas}
te permitir� escribir matem�ticas de apariencia profesional de una
forma r�pida y f�cil%
\footnote{Lyx no puede comprobar si los resultados matem�ticos que escribes
son realmente \emph{correctos}. Lo sentimos.
}.


\subsection{El modo matem�tico}

Escribe en cualquier parte de tu documento: 

\begin{lyxcode}
Me~gusta~lo~que~dijo~Einstein,~E=mc\textasciicircum{}2,~porque~es~tan~simple.~
\end{lyxcode}
Ahora la ecuaci�n no tiene muy buen aspecto, incluso en el fichero
\texttt{dvi}; no hay ning�n espacio entre las letras y el signo igual,
y te gustar�a escribir el {}``2'' como un verdadero exponente. Esta
composici�n tan mala se debe a que no le hemos dicho a \LyX{} que
estamos escribiendo una expresi�n matem�tica, as� que la compone como
texto normal.

Las expresiones matem�ticas se escriben en modo matem�tico o de f�rmulas.
Para entrar en dicho modo, s�lo tienes que pinchar en el bot�n de
la barra de herramientas con un \( \frac{a+b}{c} \) escrito en azul.
\LyX{} abrir� un peque�o cuadro azul, con un rect�ngulo magenta a
su alrededor. El cuadrado azul es el punto de inserci�n, que te indica
que est� esperando a que insertes algo, y el rectangulo indica que
est�s en el modo matem�tico. \LyX{} ha situado el cursor en el cuadro
azul, as� que introduce de nuevo \texttt{E=mc\textasciicircum{}2}.
La expresi�n se escribe en azul, y el cuadro azul desaparece tan pronto
como el punto de inserci�n deja de estar vac�o. Ahora pulsa \textsf{Esc}
para dejar el modo matem�tico (nota: pinchar en el bot�n modo de f�rmulas
otra vez no te servir�). El rect�ngulo magenta desaparece, dejando
el cursor a la derecha de la expresi�n y ahora si escribes algo ser�
texto normal.

Ejecuta \LaTeX{} y mira el fichero \texttt{dvi}. Observa que la expresi�n
se ha impreso bien, con espacios entre las letras y el signo igual,
y el {}``2'' como super�ndice. Se asume que en modo matem�tico las
letras son variables, y por tanto vienen en cursiva. Los n�meros son
s�lo n�meros y no van en cursiva.

Este nuevo modo es otro ejemplo de la filosof�a WYSIWYM. En \LaTeX{},
escribes una expresi�n matem�tica mediante texto y comandos como \texttt{\textbackslash{}sqrt};
esto puede ser desesperante, ya que no puedes ver c�mo queda la expresi�n
hasta que ejecutas \LaTeX{}, y puedes perder mucho tiempo buscando
alg�n par�ntesis que falte u otros fallos. Por el contrario, \LyX{}
no pretende que la expresi�n aparezca perfecta en la pantalla (WYSIWYG),
pero te da una buena idea de cu�l ser� su aspecto impreso. \LaTeX{}
se encarga de la composici�n profesional. El 99\% del tiempo no tendr�s
que preocuparte de cambiar tama�os de letra o espaciado alguno. De
esta forma (sentimos ser tan repetitivos) puedes concentrarte en el
\emph{contenido} de la expresi�n matem�tica, no en su formato.


\subsection{Navegando por una ecuaci�n}

Ahora vamos a cambiar \( E=mc^{2} \) a \( E=1+mc^{2} \). Usa los
cursores para introducirte en la expresi�n. Observa que cuando entras
en la expresi�n el rect�ngulo magenta aparece para informarte de que
est�s de nuevo en modo de f�rmulas. Ahora puedes utilizar las teclas
\textsf{Izquierda} y \textsf{Derecha} para mover el cursor tras el
signo de igualdad, y escribir {}``1+''. De nuevo, utiliza los cursores
o \textsf{Esc} para salir de la expresi�n, desapareciendo as� el rect�ngulo
magenta. Mucha gente encuentra apropiadas las teclas de direcci�n,
pero tambi�n puedes pichar con el rat�n en cualquier parte de la expresi�n
para situar el cursor ah� y empezar el modo matem�tico.

Salvo para las teclas especiales descritas m�s adelante, escribir
en el modo matem�tico es como editar texto normal. Usa \textsf{Suprimir}
(o \textsf{Retroceso}) para borrar. Selecciona texto tanto con los
cursores como con el rat�n. \textsf{\underbar{E}}\textsf{dici�n\lyxarrow{}Deshacer}
funciona de la misma forma, al igual que cortar y pegar. Hay una cosa
con la que debes tener cuidado: si est�s fuera de la expresi�n y pulsas
\textsf{Suprimir} (o \textsf{Retroceso}) la borrar�s entera. Afortunadamente
puedes \textsf{Deshacer} para recuperarla.

�Y qu� ocurre si quieres cambiar \( E=mc^{2} \) a \( E=mc^{2.5}+1 \)?
Una vez m�s, puedes utilizar el rat�n para pinchar en el sitio que
vas a modificar. Tambi�n puedes utilizar las teclas de direcci�n.
Si el cursor se encuentra detr�s de la {}``c'' y delante del {}``2'',
pulsando \textsf{Arriba} mover�s el cursor al nivel del super�ndice,
justo antes del {}``2''. A�ade el {}``.5''. Ahora, pulsando \textsf{Abajo}
devolver�s el curso al nivel normal. De hecho, si pulsas \textsf{Abajo}
desde cualquier parte del super�ndice, el cursor se sit�a justo \emph{tras}
�ste (de manera que puedes introducir el {}``+1'').

Tambi�n puedes usar la \textsf{Barra espaciadora} para recorrer una
expresi�n. Si ya est�s en una estructura matem�tica (un sub�ndice,
super�ndice, fracci�n, ra�z cuadrada, delimitadores o matriz, todo
lo cual se describe en las siguientes secciones), pulsando \textsf{Espacio}
mover�s el cursor detr�s de la estructura, permaneciendo en modo matem�tico.
As�, si el cursor est� en cualquier parte del super�ndice, pulsando
\textsf{Espacio} el cursor volver� al nivel normal y justo detr�s
del super�ndice. Esto quiere decir que puedes escribir \( E=mc^{1+x}-2 \)
sin usar el rat�n o las teclas de direcci�n, un m�todo que seguramente
preferir�s una vez que vayas cogiendo pr�ctica. Ten cuidado y no pulses
\textsf{Espacio} estando entre el {}``1'' y el signo m�s, o saldr�s
del super�ndice. En aquellos sitios donde estas acciones no tienen
sentido (por ejemplo, entre la {}``m'' y la {}``c''), la \textsf{Barra
espaciadora} no tiene efecto alguno%
\footnote{El \textsf{Espacio} y el \textsf{Tabulador} \emph{no} se utilizan
para introducir espacios entre los elementos de una ecuaci�n. Este
espacio es parte de la composici�n, lo que significa de debes dejar
que \LyX{} se preocupe de �l (ver Sec. \ref{sec:whitespace}). Si
este espaciado no te satisface del todo, hay maneras para ajustarlo,
para lo cual puedes mirar en la \emph{Gu�a del Usuario} (pero no te
molestes en hacer ajustes hasta que hayas escrito todo el documento).
}.

Observa que si introduces una expresi�n y sales con \textsf{Esc},
no habr� ning�n espacio tras �sta. Esto viene bien si vas a escribir
un punto o una coma, pero si lo que quieres es escribir una palabra
tras la f�rmula, tienes que introducir el \textsf{Espacio} expl�citamente
despu�s de salir de ella. Un atajo consiste en pulsar \textsf{Espacio}
justo al final de la f�rmula, lo cual te saca de ella \emph{y} adem�s
a�ade un espacio. As�, puedes escribir {}``\( f=ma \) es mi ecuaci�n
favorita'' en vez de {}``\( f=ma \)es mi ecuaci�n favorita''.


\subsection{Exponentes e �ndices}

Un exponente se puede introducir desde el men� \textsf{F�r}\textsf{\underbar{m}}\textsf{ulas},
pero es m�s sencillo pulsar la tecla del acento circunflejo {}``\textasciicircum{}''.
\LyX{} coloca un punto de inserci�n (el recuadro azul, �recuerdas?)
en el super�ndice para que lo siguiente que introduzcas sea super�ndice,
con un tama�o de letra menor. Todo lo que escribas hasta que pulses
\textsf{Espacio} (o \textsf{Esc} para salir completamente de la expresi�n)
ser� puesto en super�ndice.

Escribir un sub�ndice es igual de f�cil: para comenzar uno pulsa la
tecla de subrayado, {}``\_''. Puedes introducir sub�ndices y super�ndices
tanto en sub�ndices como en super�ndices, como en esta f�rmula: \( A_{a_{0}+b^{2}}+C^{a_{0}+b^{2}} \). 

\textbf{Ejercicio}: Pon la ecuaci�n 1 del fichero \texttt{es\_ejemplo\_sin\_lyx.lyx}
en modo matem�tico.


\subsection{El \textsf{Panel de F�rmulas}}

El \textsf{Panel de F�rmulas} es una forma pr�ctica de introducir
s�mbolos o realizar complicadas funciones matem�ticas. Muchas de estas
funciones pueden llevarse a cabo con el teclado o con el men� \textsf{F�r}\textsf{\underbar{m}}\textsf{ulas}.
Sin embargo, nos vamos a centrar en el uso del panel para que conozcas
todo lo que hay; podr�s aprender atajos con el teclado m�s tarde en
otros manuales (esto es una indirecta). As� que abre el \textsf{Panel
de F�rmulas} y d�jalo abierto durante toda la secci�n.


\subsubsection{S�mbolos y letras griegas}

Si pinchas en el bot�n {}``\( \Gamma \rho \epsilon \epsilon \kappa  \)''
del panel, obtendr�s un men� desde el que podr�s elegir una letra
griega. �sta aparecer� all� donde est� situado el cursor. Observa
que hay un par de variantes de epsilon, pi, theta, y sigma. Un atajo:
si est�s escibiendo texto normal e insertas algo desde el panel, entras
autom�ticamente en modo matem�tico.

Otros cuatro botones en la parte de abajo del panel te permiten seleccionar
en una matriz gran cantidad de s�mbolos usados en matem�ticas: flechas,
relaciones, operadores, sumas e integrales. Ten en cuenta que los
super�ndices y los sub�ndices te permiten establecer los l�mites superior
e inferior en estas �ltimas. El �ltimo bot�n es el caj�n desastre,
llamado \textsf{Varios}. {}``No hay nada que no puedas hacer\ldots{}
Todo lo que necesitas es \( \heartsuit  \).''


\subsubsection{Ra�ces cuadradas, tildes y delimitadores}

Para escribir una ra�z cuadrada s�lo tienes que pulsar el bot�n con
su s�mbolo. La ra�z aparece, y el cursor va a un nuevo punto de inserci�n
dentro de ella. Aqu� puedes introducir variables, n�meros, otras ra�ces
cuadradas, fracciones y todo lo que quieras. \LyX{} ir� cambiando
autom�ticamente el tama�o de la ra�z para que se ajuste a su contenido.

La colocaci�n de tildes en una letra (\( \overrightarrow{v} \)) o
grupo de letras (\( \overrightarrow{a+b} \)) se realiza de la misma
forma. Pulsa el bot�n que tiene un cuadro azul con una tilde negra
(\textasciitilde{}) encima. Aparecer� la ventana de \textsf{Decoraci�n}.
Pincha en una de las decoraciones, y \LyX{} la imprimir� con un punto
de inserci�n debajo (o encima). Escribe lo que quieras y pulsa \textsf{Espacio}
para {}``salir'' de ella.

Los delimitadores (como par�ntesis, corchetes y llaves) funcionan
de forma similar, pero son un poco m�s complicados. Pulsa el bot�n
con un cuadro azul entre corchetes para abrir la ventana \textsf{Delimitador}.
Pincha en un delimitador izquierdo con el \emph{bot�n izquierdo del
rat�n} y en un delimitador derecho con el \emph{bot�n derecho}. (De
forma alternativa, puedes usar s�lo el bot�n izquierdo utilizando
los selectores etiquetados como {}``\underbar{I}zq.'' y {}``\underbar{D}cha''
al elegir cada delimitador). En cada instante, tu selecci�n de delimitadores
aparece en un recuadro en la parte de arriba de la ventana. Por defecto
es un par de par�ntesis, pero con este m�todo de selecci�n puedes
elegir un par de llaves, una llave y un par�ntesis, o incluso elegir
el delimitador vac�o para obtener algo as�: {}``\( a=\left\langle 7\right.  \)''
(el delimitador vac�o se muestra como una l�nea negra discontinua
en \LyX{}, pero no aparece en la salida).

Una vez que has elegido tus delimitadores, pulsa \textsf{OK} para
que aparezcan en la expresi�n (o pulsa \textsf{Aplicar} si quieres
dejar la ventana abierta). Si eres un poco vago, puedes escribir los
par�ntesis normales desde el teclado en modo matem�tico, en vez de
usar la ventana \textsf{Delimitador}. Sin embargo, estos par�ntesis
tendr�n el mismo tama�o que el texto normal, lo cual quedar� bastante
mal si encierran una fracci�n o una matriz grande. Usar la ventana
\textsf{Delimitador} te garantiza que los delimitadores se ajusten
de forma precisa a aquello que encierren.

Tambi�n puedes poner delimitadores, o una ra�z cuadrada o una tilde
sobre texto ya existente. Selecciona la porci�n de f�rmula que quieres
ajustar, y pulsa el bot�n que quieras del \textsf{Panel de F�rmulas}.
Prueba a hacer esto para cambiar la segunda ley de Newton de forma
escalar a forma vectorial (\( f=ma \) a \( \overrightarrow{f}=m\overrightarrow{a} \)).
Una vez que hayas aprendido a hacer matrices, esta ser� la manera
de encerrarlas entre par�ntesis o corchetes.


\subsubsection{Fracciones}

Las fracciones son muy sencillas en modo matem�tico. Pincha en el
bot�n \textsf{Fracci�n} del panel (el que tiene un par de cuadros
azules a modo de numerador y denominador). \LyX{} coloca dos puntos
de inserci�n. Como cabr�a esperar, puedes usar los cursores o el rat�n
para moverte a trav�s de la fracci�n. Pincha en el cuadro de arriba
y escribe {}``1''. Despu�s pulsa \textsf{Abajo} y escribe {}``2''.
�Acabas de hacer una fracci�n! Por supuesto, puedes escribir cualquier
cosa dentro de cada uno de los dos cuadros: variables con exponentes,
ra�ces cuadradas, otras fracciones, cualquier cosa.

\textbf{Ejercicio}: Pon la ecuaci�n 2 en el fichero \texttt{es\_ejemplo\_sin\_lyx.lyx}
en modo matem�tico.


\subsubsection{Modo \TeX{}: L�mites, logaritmos, senos y otros}

Como las letras en modo de f�rmulas se consideran variables, si introduces
{}``sin'' \LyX{} cree que est�s escribiendo el producto de tres
variables \( s \), \( i \), y \( n \). Las tres se imprimir�n en
cursiva, cuando lo que realmente quer�as era la palabra {}``sin''
escrita en letra normal. Adem�s, no se pondr� espacio alguno entre
{}``sin'' y la {}``x'' (pulsar \textsf{Espacio} saldr� del modo
matem�tico). As� que, �c�mo conseguir {}``\( \sin x \)'' en vez
de {}``\( sinx \)''?

Pincha en {}``sin'' en la lista \textsf{Funciones} del \textsf{Panel
de F�rmulas}. La palabra {}``sin'' se escribe en rojo, en tipo normal,
tambi�n llamado modo \TeX{}. La palabra completa se trata como un
�nico s�mbolo, de manera que si pulsas \textsf{Suprimir}, la borrar�s
entera. Ahora escribe {}``x'', que aparecer� azul y en cursiva,
como es de esperar en modo matem�tico. En el fichero \texttt{dvi},
la expresi�n se ver� correctamente. Int�ntalo.

M�s comandos que necesitas escribir en modo \TeX{} usando la lista
de \textsf{Funciones} incluyen otras funciones trigonom�tricas y sus
inversas, funciones hiperb�licas, logaritmos, l�mites, y unas pocas
m�s. Todas aceptan sub�ndices y super�ndices, lo cual es importante
para poder escribir {}``\( \cos ^{2}\theta  \)'' o {}``\( \lim _{n\rightarrow \infty } \)''.

\textbf{Ejercicio}: Pon la ecuaci�n 3 del fichero \texttt{es\_ejemplo\_sin\_lyx.lyx}
en modo matem�tico.


\subsubsection{Matrices}

\label{sec:matrices}Pulsa en el bot�n \textsf{Matriz} del panel.
Se abrir� una ventana del mismo nombre , con dos barras deslizantes
para que elijas cu�ntas filas y columnas quieres que tenga tu matriz.
Selecciona 2 filas y 3 columnas y pulsa \textsf{Aplicar} u \textsf{OK}.
\LyX{} imprime seis puntos de inserci�n formando una matriz \( 2\times 3 \).
Como siempre, puedes introducir cualquier clase de expresi�n matem�tica
(un ra�z cuadrada, otra matriz, etc) en cada uno de ellos. Tambi�n
puedes dejar alguno vac�o si quieres.

Puedes usar el \textsf{Tabulador} para moverte horizontalmente entre
las columnas de la matriz. De manera alternativa, puedes usar las
teclas de direcci�n para desplazarte (pulsando \textsf{Derecha} al
final de un cuadro mover� el cursor al siguiente, \textsf{Abajo} lo
mover� a la siguiente fila, etc).

Mira la \emph{Gu�a del Usuario} para obtener informaci�n sobre c�mo
cambiar la alineaci�n de cada columna y la posici�n vertical de la
matriz completa. Si lo que quieres es escribir una tabla que contenga
texto, deber�as usar el magn�fico soporte de tablas de \LyX{}, en
lugar de tratar de escribir texto en una matriz.


\subsubsection{El modo demostraci�n}

Todas las expresiones que hemos escrito hasta ahora estaban en la
misma l�nea que el texto que las rodeaba. Reciben pues el nombre de
expresiones en l�nea. Est�n bien para expresiones cortas y sencillas,
pero si quieres escribir cosas m�s extensas, o si quieres que queden
separadas del texto, tienes que escribirlas en modo demostraci�n.
Adem�s, s�lo las expresiones en este modo pueden ser etiquetadas o
numeradas (ver \emph{Gu�a del Usuario}). Por �ltimo, las ecuaciones
de varias l�neas (ver Sec. \ref{sec:multiline}) tambi�n deben estar
en modo demostraci�n.

Pincha en el bot�n \textsf{Mostrar} del \textsf{Panel de F�rmulas},
que representa un par de l�neas de texto rodeando un cuadro azul.
\LyX{} abre un punto de inserci�n, pero lo coloca en una nueva l�nea
y centrado. Ahora escribe una expresi�n y ejecuta \LaTeX{} para ver
el resultado. El bot�n \textsf{Mostrar} act�a como un conmutador entre
modo demostraci�n y modo matem�tico normal; �salo ahora para cambiar
un par de expresiones a modo demostraci�n y viceversa.

Este modo tiene un par de diferencias con respecto al modo normal:

\begin{itemize}
\item El tipo de letra por defecto es de menor tama�o para unos pocos s�mbolos,
como \( \sum  \) y \( \int  \)
\item Los sub�ndices y super�ndices en las sumas y l�mites (no en las integrales)
se escriben debajo, en lugar de seguir a los s�mbolos
\item El texto se centra
\end{itemize}
Aparte de estas diferencias, las expresiones en l�nea y en modo demostraci�n
son muy similares.

Un �ltimo apunte acerca del modo en que se componen las ecuaciones
en este modo: presta atenci�n a si lo que quieres es poner una ecuaci�n
en un nuevo p�rrafo o no. Si �sta se encuentra en mitad de de una
frase o de un p�rrafo, no pulses \textsf{Intro}. De lo contrario har�s
que el texto tras la ecuaci�n comience en un nuevo p�rrafo. Este texto
ser� por tanto sangrado, que probablemente no es lo que quer�as.

\textbf{Ejercicio}: Pon las ecuaciones del fichero \texttt{es\_ejemplo\_sin\_lyx.lyx}
en modo demostraci�n, y comprueba que se componen de manera diferente.

\textbf{Ejercicio}: Utilizando las herramientas que has aprendido
en esta secci�n, deber�as de ser capaz de escribir una ecuaci�n como
�sta: \[
f(x)=\left\{ \begin{array}{cc}
\log _{8}x & x>0\\
0 & x=0\\
\sum ^{5}_{i=1}\alpha _{i}+\sqrt{-\frac{1}{x}} & x<0
\end{array}\right. \]



\subsection{Ecuaciones de varias l�neas}

\label{sec:multiline}Intenta escribir las siguientes ecuaciones y
observa el resultado en el fichero \texttt{dvi}. Tendr�s que introducir
dos ecuaciones distintas en modo demostraci�n.

\[
x=y+y+y+y+y\]


\[
=5y\]


�No queda bien! Al escribir dos o m�s ecuaciones seguidas, quedan
mucho mejor si sus signos de igualdad est�n alineados; esto se hace
especialmente patente si la segunda ecuaci�n no tiene parte izquierda.
\LyX{} te permite escribir ecuaciones de varias l�neas con cierto
control sobre su alineaci�n.

\begin{eqnarray*}
x & = & y+y+y+y+y\\
 & = & 5y
\end{eqnarray*}


�Esto es otra cosa! Los signos de igualdad est�n alineados, y hay
menos espacio entre las ecuaciones.

Para empezar a escribir una ecuaci�n de varias l�neas, abre una expresi�n
en modo demostraci�n y pulsa \textsf{C-Retorno de carro}. \LyX{} imprimir�
dos l�neas, cada una con tres puntos de inserci�n. Al igual que en
las matrices, puedes usar el el rat�n, los cursores o el \textsf{Tabulador}
para desplazarte a trav�s de ellos. Intenta reproducir la ecuaci�n
anterior. Ten en cuenta que es legal dejar uno o m�s puntos de inserci�n
vac�os. Esto es �til para ejemplos como el de arriba, o para separar
ecuaciones muy largas, como:

\begin{eqnarray*}
x & = & a+b+c+d\\
 &  & +e+f+g
\end{eqnarray*}


\LyX{} alinear� el segundo campo de cada l�nea, donde usualmente pondr�s
signos de igualdad u otros operadores; de hecho, puedes poner cualquier
cosa. Pero no uses una ecuaci�n de varias l�neas para escribir una
matriz. Para eso ya est� la herramienta apropiada (see Sec. \ref{sec:matrices}).

Si quieres un conjunto mayor de ecuaciones, usa \textsf{C-Retorno
de carro} para a�adir nuevas l�neas con tres puntos de inserci�n vac�os.
Si no est�s al final de la l�nea cuando lo hagas, lo que quede de
l�nea ser� llevado a la siguiente. Si pulsas \textsf{C-Retorno de
carro} cuando ya has escrito una ecuaci�n (de una l�nea), la ecuaci�n
entera quedar� en el primer campo. Sit�a el cursor antes del signo
igual y pulsa \textsf{C-Tabulador} para moverlo al segundo campo.
Col�cate pasado el signo y pulsa de nuevo \textsf{C-Tabulador} para
mover el resto de la ecuaci�n al tercer campo. Prueba a cambiar tu
ecuaci�n \( E=mc^{2} \) a:

\begin{eqnarray*}
E & = & mc^{2}\\
 & = & mc\times c
\end{eqnarray*}


Si has escrito demasiadas l�neas, sit�a el cursor al final de una
de ellas y usa \textsf{M-e~k} para borrar la siguiente. Se eliminar�
la alimentaci�n de l�nea, concatenando la l�nea siguiente (los tres
puntos de inserci�n) con el final de l�nea en la que est�s. Si aqu�lla
est� vac�a, simplemente la borrar�. Aviso: usar \textsf{M-e~k} cuando
no est�s al final de la l�nea puede llevar a comportamientos extra�os.


\subsection{M�s cosas sobre matem�ticas}

El modo matem�tico puede hacer muchas cosas m�s. Por ahora te has
familiarizado con lo b�sico, as� que dir�gete a la \emph{Gu�a del
Usuario} si quieres buscar trucos para:

\begin{itemize}
\item Etiquetar y numerar expresiones.
\item Cambiar el tipo de letra (i.e., escribir texto en negrita dentro de
una expresi�n). Aprovechamos para decirte que pulsando en el bot�n
\textsf{Modo~de~F�rmulas} de la barra de herramientas dentro de
una expresi�n te permitir� escribir con tipo de letra normal hasta
que introduzcas un espacio (no protegido).
\item Ajustar los tama�os de letra y el espaciado en una expresi�n. (No
te preocupes de estas cosas hasta el borrador final).
\item Escribir macros. Son muy potentes, ya que te permiten definirlas al
principio del documento, y usarlas en cualquier parte de �l. Si cambias
la definici�n de una macro, las referencias a ella cambiar�n en todo
el documento. Las macros pueden incluso tomar par�metros.
\item Hacer un mont�n de cosas m�s que no tenemos tiempo de contarte en
este \emph{Tutorial}.
\end{itemize}

\section{Miscel�neo}


\subsection{Otras caracter�sticas importantes de \LyX{}}

No hemos tratado todos los posibles comandos de \LyX{}, y no tenemos
intenci�n de hacerlo. Como siempre, ve a la \emph{Gu�a del Usuario}
para m�s informaci�n. La funci�n exacta de cada comando del men� se
describe en el \emph{Manual de Referencia}. S�lo mencionaremos un
par m�s de cosas importantes que \LyX{} puede hacer\ldots{}

\begin{itemize}
\item \LyX{} posee soporte WYSIWYG para tablas. Usa el comando \textsf{\underbar{I}}\textsf{nsertar\lyxarrow{}Ta}\textsf{\underbar{b}}\textsf{la}
para crear una. Pincha en la tabla con el \emph{bot�n derecho} para
que aparezca la ventana de Formato de \textsf{Tabla}, que te ofrece
extensas posibilidades de edici�n de la tabla.
\item \LyX{} tambi�n permite incluir gr�ficos PostScript� (o \LaTeX{} puro).
Lo adivinaste: \textsf{\underbar{I}}\textsf{nsertar\lyxarrow{}Fi}\textsf{\underbar{g}}\textsf{ura}\@.
Despu�s pincha en la figura para elegir el fichero que quieres incluir,
rotarla, escalarla, etc. Tanto las tablas como las figuras tienen
etiqueta, y \LyX{} genera autom�ticamente listas de figuras y/o de
tablas.
\item Soporte de control de versiones, usando RCS (\texttt{man rcsintro}
para m�s informaci�n).
\item \LyX{} es altamente configurable. Todo, desde el aspecto de la ventana
hasta la forma de la salida, puede ser configurado de m�ltiples maneras.
Gran parte de la configuraci�n se lleva a cabo editando el fichero
\texttt{lyxrc}%
\footnote{Actualmente, tienes que editar el fichero \texttt{lyxrc} con un editor
de texto. Los programadores esperan crear una interfaz para la configuraci�n
dentro de \LyX{}.
}. Para m�s informaci�n sobre esto, echa un vistazo a \textsf{A}\textsf{\underbar{y}}\textsf{uda\lyxarrow{}Personalizaci�n\@.}
\item \LyX{} est� siendo desarrollado por un equipo de programadores en
los cinco continentes. De esta forma, tiene mejor soporte para otros
idiomas adem�s del ingl�s (como holand�s, alem�n, griego, checo, turco,
\ldots{}) que muchos procesadores de texto. Puedes escribir documentos
en otros idiomas, pero tambi�n puedes configurar \LyX{} para que muestre
los men�s y los mensajes de error en otras lenguas.
\item Los men�s de \LyX{} tienen asociadas combinaciones de teclas. Esto
significa que puedes hacer \textsf{\underbar{A}}\textsf{rchivo\lyxarrow{}}\textsf{\underbar{A}}\textsf{brir}
tecleando \textsf{M-F} seguido de \textsf{O}. Las asociaciones de
teclas tambi�n son configurables (y puede haber asociaciones incluso
para algunos de los men�s traducidos del ingl�s). Para m�s informaci�n,
busca en \textsf{A}\textsf{\underbar{y}}\textsf{uda\lyxarrow{}Personalizaci�n\@.}
\item \LyX{} puede leer documentos \LaTeX{}. Ver Sec. \ref{sec:relyx}.
\item Corrige los errores ortogr�ficos de tu documento con \textsf{\underbar{E}}\textsf{dici�n\lyxarrow{}Ortograf�a}\@.%
\footnote{Ten en cuenta que el corrector ortogr�fico s�lo comprueba desde el
cursor hasta el final del documento.
}
\end{itemize}

\subsection{\LyX{} para usuarios de \LaTeX{}}

\label{sec:latexusers}Si no sabes nada de \LaTeX{}, no tienes que
leer esta secci�n. Realmente, puede que quieras \emph{aprender} algo
sobre \LaTeX{}, y entonces leas este cap�tulo. Sin embargo, a mucha
gente que empieza a usar \LyX{}, \LaTeX{} le ser� familiar. Si ese
es tu caso, te estar�s preguntando si verdaderamente \LyX{} puede
hacer todo lo que hace \LaTeX{}. La respuesta corta es que, de una
forma u otra, puede hacer pr�cticamente todo lo que hace �ste, simplificando
en gran medida el proceso de escritura de un documento mediante \LaTeX{}.
Actualmente hay algunos problemas convirtiendo antiguos documentos
\LaTeX{} y en un par de cosas m�s, pero las versiones posteriores
subsanar�n esto.

Como s�lo se trata de un tutorial, vamos a mencionar �nicamente aquellas
cosas que los nuevos usuarios de \LyX{} vayan a encontrar interesantes.
Para mantener corto el \emph{Tutorial}, vamos a dar informaci�n m�nima.
La \emph{Gu�a del Usuario} ya posee una gran cantidad de informaci�n
sobre las diferencias entre \LyX{} y \LaTeX{}, y de c�mo hacer varios
trucos de \LaTeX{} en \LyX{}.


\subsubsection{Modo \TeX{}}

Todo lo que introduzcas en modo \TeX{} se imprimir� en rojo en la
pantalla, y ser� pasado directamente a \LaTeX{}. Entra en este modo
con \textsf{\underbar{F}}\textsf{ormato\lyxarrow{}Estilo~}\textsf{\underbar{\TeX{}}}
o pinchando en el bot�n con \TeX{} escrito en rojo de la barra de
herramientas.

En el modo matem�tico, el modo \TeX{} se maneja de forma ligeramente
diferente. En este caso para entrar tienes que introducir una contrabarra
{}``\textbackslash{}''. Esta no saldr� en pantalla, pero todo lo
que escribas de ah� en adelante aparecer� en rojo. Para salir pulsa
\textsf{Espacio} o cualquier otro car�cter no alfab�tico, como un
n�mero, el subrayado, el circunflejo o el par�ntesis. Una vez que
salgas del modo \TeX{}, si \LyX{} conoce el comando que acabas de
introducir lo convertir� a formato WYSIWYM. As�, si en modo matem�tico
escribes \texttt{\textbackslash{}gamma}, cuando pulses \textsf{Espacio},
\LyX{} convertir� el texto en rojo en una {}``\( \gamma  \)'' de
color azul. Esto funcionar� para casi todas las macros matem�ticas
que no sean muy complejas (aunque debes tener en cuenta que funciones
como \texttt{\textbackslash{}sin} permanecen en rojo, ya que esa es
su forma WYSIWYM). Este sistema puede ser m�s r�pido que usar el \textsf{Panel
de F�rmulas}, especialmente para usuarios experimentados de \LaTeX{}.

Como caso particular cabe comentar que si escribes una llave en modo
\TeX{} dentro de una expresi�n matem�tica, aparecer�n en rojo tanto
la llave de apertura como una de cierre, se te sacar� \emph{fuera}
del modo \TeX{} y se situar� el cursor entre las llaves. Esto hace
que sea m�s pr�ctico para escribir comandos \LaTeX{} que el modo matem�tico
no conoce y que toman un par�metro.

\LyX{} no puede hacer absolutamente todo lo que hace \LaTeX{} (�todav�a
no?). Algunas funciones muy elaboradas no est�n soportadas, mientras
que algunas funcionan pero no son WYSIWYG. El modo \TeX{} permite
a los usuarios disponer de la completa flexibilidad de \LaTeX{}, a
la vez que gozan de las �tiles caracter�sticas de \LyX{}, como las
matem�ticas WYSIWYG, las tablas y la edici�n de texto. \LyX{} no podr�a
abarcar nunca todos los paquetes de \LaTeX{}. Sin embargo, escribiendo
\texttt{\textbackslash{}usepackage\{nombre\_paquete\}} en el pre�mbulo
(ver Secci�n \ref{sec:preamble}), puedes utilizar el paquete que
quieras (aunque no tendr�s soporte WYSIWYG para las caracter�sticas
de ese paquete).


\subsubsection{Importar documentos \LaTeX{}: \texttt{reLyX}}

\label{sec:relyx}Puedes importar un fichero \LaTeX{} usando el comando
\textsf{\underbar{A}}\textsf{rchivo\lyxarrow{}}\textsf{\underbar{I}}\textsf{mportar\lyxarrow{}\LaTeX{}}.
�ste llamar� a un programa Perl llamado re\LyX{}, que crear� un fichero
\texttt{fich.lyx} a partir del fichero \texttt{fich.tex} y lo abrir�.
Si la traducci�n no funciona, puedes probar a ejecutar \texttt{reLyX}
desde la l�nea de comandos%
\footnote{Cuando \LyX{} se instala, se crea un fichero ejecutable separado llamado
\texttt{reLyX} en el mismo directorio que el propio \texttt{lyx} (i.e.,
\texttt{/usr/local/bin/reLyX}). \texttt{reLyX} requiere Perl (versi�n
5.002 en el momento de escribir esto).
}, posiblemente usando opciones m�s complejas.

\texttt{reLyX} traducir� la mayor�a de los comandos legales de \LaTeX{},
pero no todo. Dejar� lo que no entienda en modo \TeX{}, as� que despu�s
de la traducci�n puedes buscar el texto en rojo y editarlo a mano
para tratar de arreglarlo.

\texttt{reLyX} tiene su propia p�gina del manual. L�ela si quieres
conocer los comandos y entornos de \LaTeX{} que no admite, fallos
(y c�mo evitarlos), y c�mo usar las distintas opciones.


\subsubsection{Convertir documentos \LyX{} a \LaTeX{}}

Puede que desees convertir un documento \LyX{} en un fichero \LaTeX{}.
Por ejemplo, un compa�ero de trabajo o colaborador que no tiene \LyX{}
puede querer leerlo. Esto se soluciona muy f�cilmente con \LyX{}.
Selecciona \textsf{\underbar{A}}\textsf{rchivo\lyxarrow{}}\textsf{\underbar{E}}\textsf{xportar\lyxarrow{}como~\LaTeX{}}.
Se crear� un fichero \texttt{cualquiera.tex} a partir del fichero
\texttt{cualquiera.lyx} que est�s editando. Al fin y al cabo, el programa
siempre genera ficheros temporales \LaTeX{} cuando visualiza o imprime
los documentos, as� que no tiene ning�n problema para hacer esto.


\subsubsection{Pre�mbulo \LaTeX{}}


\subsubsection{Clase de documento }

El men� \textsf{\underbar{F}}\textsf{ormato\lyxarrow{}}\textsf{\underbar{D}}\textsf{ocumento}
se encarga de muchas de las opciones que introducir�as en un comando
\texttt{\textbackslash{}documentclass}. Cambia aqu� la clase del documento,
el tama�o de letra por defecto y el tama�o del papel. Pon cualquier
opci�n adicional en la zona de \textsf{Opciones~e}\textsf{\underbar{x}}\textsf{tra}.


\subsubsection{Otros aspectos del pre�mbulo}

\label{sec:preamble}Si tienes que poner comandos especiales en el
pre�mbulo de un fichero \LaTeX{}, tambi�n puedes hacerlo en un documento
de \LyX{}. Elige \textsf{\underbar{F}}\textsf{ormato\lyxarrow{}Pre�mbulo~}\textsf{\underbar{L}}\textsf{atex}
y escribe en la ventana que aparecer�. Todo lo que introduzcas se
le pasar� directamente a \LaTeX{} (como el modo \TeX{}).


\subsubsection{Bib\TeX{}}

\LyX{} tiene un soporte de Bib\TeX{} bueno pero incompleto, el cual
te permite construir bases de datos de referencias bibliogr�ficas
que pueden usarse en m�ltiples documentos. Selecciona en el men� \textsf{\underbar{I}}\textsf{nsertar\lyxarrow{}Listas~e~�ndice~gral.\lyxarrow{}Referencia~}\textsf{\underbar{B}}\textsf{ib\TeX{}}
para incluir un fichero \texttt{bib}. Pincha en el cuadro resultante
de {}``Referencias Bib\TeX{} generadas'' y obtendr�s un men� \textsf{Bib\TeX{}}.
En el campo \textsf{Base~de~Datos} escribe aquello que pondr�as
dentro de las llaves del comando \texttt{\textbackslash{}bibliography\{\}}%
\footnote{Como en \LaTeX{} normal, dos o m�s bibiograf�as deben separarse por
comas, sin espacios en blanco.
}. En el campo de  \textsf{Estilo}, escribe aquello que corresponda
al comando \texttt{\textbackslash{}bibliographystyle\{\}}.

Despu�s de hacer esto, puedes hacer referencia a cualquier entrada
de las bibliograf�as que hayas incluido mediante \textsf{\underbar{I}}\textsf{nsertar\lyxarrow{}Referenc}\textsf{\underbar{i}}\textsf{a~a~Cita}
(ver Secci�n \ref{sec:bibliographies}). El programa se preocupar�
de ejecutar Bib\TeX{}. La raz�n por la que decimos {}``soporte bueno
pero incompleto'' es que \LyX{} no es capaz de crear ficheros \texttt{bib},
y adem�s no te ofrece la lista con las referencias de tu fichero \texttt{bib}
en la ventana \textsf{Cita}.


\subsubsection{Miscel�neo}

Inserta un espacio protegido con \textsf{C-espacio}. Aparecer� en
pantalla como una peque�a {}``u'' rosa. Hay montones a lo largo
de este \emph{Tutorial\@.} En el men� \textsf{\underbar{I}}\textsf{nsertar\lyxarrow{}Car�cter~Especial}
ver�s otros caracteres especiales, incluyendo puntos suspensivos,
salto de l�nea forzoso y punto de ruptura de palabra.


\subsection{�Errores!}

A veces, al ejecutar \LaTeX{} habr� errores, cosas que \LyX{} o el
propio \LaTeX{} no entienden. Cuando esto sucede, \LyX{} crea un recuadro
de error (con la palabra {}``error'' dentro). Pulsando sobre el
recuadro se abrir� una ventana que muestra el mensaje de error concreto.
Si se trata de algo que has hecho mal con \LyX{}, ser� un error de
\LyX{}. Estos errores son muy raros. Si el problema es de \LaTeX{}
(la mayor�a de las veces ocurre con cosas que has escrito en modo
\TeX{}) \LyX{} se limitar� a citar el mensaje de error que �ste le
devuelve.


