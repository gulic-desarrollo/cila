%Autor: miguev

\pagestyle{empty}

\vspace*{20mm}
\begin{center}
{\LARGE Gu�a de aprendizaje}\vspace*{5mm}
\resizebox{\textwidth}{10mm}{{\LARGE\bf Curso de Introducci�n a Linux para Alumnos}}
\end{center}

\vfill
\cleardoublepage

\vspace*{20mm}
\begin{center}
{\LARGE Gu�a de aprendizaje}\vspace*{5mm}
\resizebox{\textwidth}{10mm}{{\LARGE\bf Curso de Introducci�n a Linux para Alumnos}}
\end{center}

\begin{center}
\vspace*{40mm}
{\LARGE Grupo de Usuarios de Linux de Canarias}
\vspace*{20mm}
{\large
\\ Tom�s Bautista Delgado
\\ Luis Cabrera Sa�co
\\ Carlos de la Cruz Pinto
\\ Teresa Gonz�lez de la F�
%\\ Pedro Gracia Fajardo
\\ Ed�n Kozo
\\ F�lix J. Marcelo Wirnitzer
%\\ Carlos P�rez P�rez
\\ Ren� Mart�n Rodr�guez
\\ Carlos Mestre Gonz�lez
\\ Carlos Alberto Morales D�az	
%\\ Alberto Carlos Ruiz Ruiz
%\\ V�ctor Ruiz Ruiz
\\ Jes�s Miguel Torres Jorge	
\\ Miguel �ngel Vilela Garc�a }
\end{center}

\vfill
\clearpage

\vspace*{95mm}
\noindent Copyright  \copyright~ 2001, 2002 \GULiC
\vspace{5mm}

\noindent  Se da  permiso  para copiar,  distribuir  o modificar  este
documento  en los  t�rminos que  establece la  GNU Free  Documentation
License, Versi�n 1.1 o cualquier  otra versi�n posterior publicada por
la Free Software  Foundation; las secciones que  no pueden modificarse
son  ``Presentaci�n''  y  ``Agradecimientos'',  y  no  existen  textos
cubierta ni contracubierta. Se incluye una  copia de la licencia en la
secci�n ``GNU Free Documentation  License''. Este libro es \copyright~
de sus autores.

\noindent  La forma  original de  este documento  es c�digo  fuente en
\LaTeX.  La  compilaci�n de  este  c�digo  fuente  en LaTeX  tiene  el
efecto de  generar una representaci�n del  documento independiente del
dispositivo, que puede convertirse a otros formatos o imprimirse.

\noindent  La  GNU  Free  Documentation  License  est�  disponible  en
www.gnu.org o solici�ndola por escrito a the Free Software Foundation,
Inc., 59 Temple Place - Suite 330, Boston, MA 02111-1307, USA.

\noindent  Este  documento lo  han  maquetado  sus autores  utilizando
\LaTeX y {\sf The GIMP}, que son programas abiertos y libres.

\vfill
\clearpage

\vspace*{35mm}
\begin{flushright}
A los novatos y novatas en el mundo de GNU/Linux, \\
y a sus familiares, amistades y parejas \\
{\small (ellas y ellos saben por qu�)}
\end{flushright}

\vfill
\cleardoublepage

\pagestyle{fancy}
