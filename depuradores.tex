%Autor: aplatanad

\chapter{Depuradores}

\section{Depurando la aplicaci�n}


Se suele decir que el 10\% del  tiempo de desarrollo de un programa se
dedica a la codificaci�n y el 90\% restante a la depuraci�n. Al margen
de  que sea  cierto o  no la  verdad es  que es  de vital  importancia
disponer de las  herramientas adecuadas para corregir  los errores del
software en un tiempo razonable.

La primera  recomendaci�n es  dejar que  {\tt lclint}  analice nuestro
c�digo  para  que  busque  y notifique  cualquier  inconsistencia.  Es
importante destacar que dicho programa es mucho m�s potente detectando
posibles errores que  el analizador sint�ctico del {\tt  gcc} debido a
que  el  compilador asume  que  ya  hemos  pasado nuestro  c�digo  por
un  programa  como  {\tt  lclint}.  Esa  suposici�n  permite  al  {\tt
gcc}  realizar  ciertas optimizaciones  que  mejoran  su velocidad  de
compilaci�n.

\begin{verbatim}
  $ lclint main.c holafunc.c
\end{verbatim}

Si disponemos de nuestro programa perfectamente compilado y observamos
que  presenta alg�n  error del  que no  sabemos determinar  su origen,
significa que ha llegado la  hora del depurador. GNU/Linux dispone del
GNU  Debugger bajo  el comando  {\tt  gdb}. Para  usarlo s�lo  debemos
ejecutarlo  especificando  el  nombre  del programa  en  la  l�nea  de
comandos; y haber compilado nuestro programa con la opci�n {\tt -g}.

\begin{verbatim}
  $ gdb holamundo
  GNU gdb 19990928
  Copyright 1998 Free Software Foundation, Inc.
  GDB is free software, covered by the GNU General Public License, and you are
  welcome to change it and/or distribute copies of it under certain conditions.
  Type "show copying" to see the conditions.
  There is absolutely no warranty for GDB.  Type "show warranty" for details.
  This GDB was configured as "i686-pc-linux-gnu"...
  (gdb) 
\end{verbatim}

En ese momento tendremos al {\tt gdb} esperando alguno de los comandos
de depuraci�n. A con\-ti\-nua\-ci�n disponemos de una lista de los comandos
m�s b�sicos.

\begin{description}

\item[{\tt break}]  Sit�a un punto  de ruptura  en la l�nea  o funci�n
indicada como argumento.

\item[{\tt continue}]  Contin�a la ejecuci�n  de un programa  que est�
siendo depurado y se encuentra detenido en un punto de ruptura.

\item[{\tt display  exp}] Muestra el  valor de la expresi�n  {\tt exp}
cada vez que el programa se detiene.

\item[{\tt  help}] Lista  las clases  de comandos  disponibles. Si  el
comando va  seguido por un nombre  de clase se listan  los comandos de
dicha clase.  Si va seguido  por un nombre  de comando se  muestra una
ayuda completa del comando indicado.

\item[{\tt  list}]  Lista una  l�nea  o  funci�n especificada.  Si  el
comando  va seguido  del nombre  de  una funci�n,  el comando  muestra
dicha  funci�n. Si  va  seguido de  un n�mero  de  l�nea, muestra  esa
l�nea.  En  programas  de  varios  archivos  se  puede  utilizar  {\tt
nombre\_archivo:nombre\_funcion} o {\tt nombre\_archivo:numero\_l�nea}
para  listar  los  contenidos  de  un archivo  particular.  Si  no  se
especifican argumentos se  lista desde la �ltima l�nea  mostrada; y si
se  especifican  dos n�meros  de  l�nea  separados  por una  coma,  se
muestran las l�neas comprendidas en el intervalo.

\item[{\tt next}] Ejecuci�n paso a  paso pero ignorando las llamadas a
funciones.

\item[{\tt print exp}]  Muestra el valor de la expresi�n  {\tt exp} en
el punto actual.

\item[{\tt quit}] Salir del {\tt gdb}.

\item[{\tt run}] Inicia la ejecuci�n  del programa. Los argumentos del
comando son los argumentos que se le pasan al programa.

\item[{\tt  step}] Ejecuci�n  paso  a paso  incluso  de las  funciones
llamadas por el programa.

\item[{\tt undisplay  exp}] Deja de  mostrar el valor de  la expresi�n
{\tt exp} cada vez que el programa se detiene.

\end{description}

El  n�mero  de  comandos  es  mucho m�s  grande  pero  basta  con  los
anteriores para agilizar enormemente  nuestro trabajo. Una alternativa
a todo  esto es  usar el {\tt  ddd} que se  puede considerar  como una
interfaz gr�fica para el  {\tt gdb}. Funciona sobre las X  y su uso es
semejante a de los depuradores existentes en otras plataformas.

 
