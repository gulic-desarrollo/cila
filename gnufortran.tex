%Autor: miguev
%miguev: 8

\chapter{GNU Fortran}
\label{gnufortran.tex}

\index{GNU Fortran}
\index{Fortran}

\section{¿Fortran?}

{\tt Fortran} no  es un lenguaje muy popular, ni  tiene por qué serlo.
Se  trata de  un lenguaje  para un  uso específico:  cálculo numérico.
Fortran no es un lenguaje de  propósito general como C, Pascal, Python
o Perl y por eso sólo  se utiliza en entornos científicos como centros
de cálculo.

El lenguaje  de programación {\tt  Fortran} es importante  en entornos
científicos.  Su  nombre   es  un  acrónimo  de   {\bf  For}mula  {\bf
Tran}slator, ya que su mayor uso  era traducir las fórmulas de cálculo
matemático  al lenguaje  de  las máquinas.  Desde  sus principios,  el
lenguaje Fortran ha tenido una  sintaxis muy peculiar, adaptada al uso
de  tarjetas  perforadas. En  la  actualidad,  Fortran se  utiliza  en
asignaturas de Cálculo Numérico en carreras técnicas como Matemáticas,
Física y algunas ingenierías.

Utilizaremos aquí el  compilador {\bf GNU Fortran},  compatible con la
mayoría  del  lenguaje básico  de  Fortran  77 y  algunas  extensiones
propias de Fortran  90, suficiente para las  prácticas de programación
en Fortran 77. En este libro nos mantendremos dentro del estándar ANSI
Fortran 77.

Para programación  en Fortran 95  existe un proyecto de  compilador en
marcha en {\tt  http://g95.sf.net}, aunque aún se  encuentra en estado
larval.

\section{Uso básico del compilador}

\index{Fortran!compilador}

El compilador de GNU  Fortran es muy similar al del  GNU C/C++, por lo
que los detalles  de ambos serán abarcados en el  próximo tema. Veamos
una vez más  el típico ejemplo de ``HolaMundo''. En  el editor que más
te guste, escribe el siguiente código Fortran:

\begin{ejemplo}%
{HolaMundo.for}%
{Primer "Hola Mundo" en Fortran}
Programa básico de holamundo en Fortran 77, define un formato e imprime 
un mensaje usándolo. también incluye una línea de comentario.
\end{ejemplo}

El comando para  invocar al compilador es {\tt g77},  y su sintaxis es
muy  similar  a  la  del  compilador  {\tt  gcc}  (GNU  C/C++).  Estos
compiladores producen  la salida por  defecto en un ejecutable  con el
nombre {\tt a.out}, comportamiento que  puedes modificar con la opción
{\tt -o nombreejecutable}.

\begin{verbatim}
$ ls
HolaMundo.for

$ g77 HolaMundo.for

$ ls
a.out  HolaMundo.for

$ g77 -o HolaMundo HolaMundo.for

$ ls
HolaMundo  HolaMundo.for

$ ./HolaMundo


  Hola Mundo

\end{verbatim}

\section{Dividir el código}

Veamos  ahora  también como  podemos  dividir  un programa  en  varios
ficheros  de  código  fuente.  En el  editor  escribe  los  siguientes
códigos fuente  Fortran y guárdalos  como {\tt HolaMundo2.for}  y {\tt
Saludos.for} respectivamente.

\begin{ejemplo}%
{HolaMundo2.for}%
{Segundo ``Hola Mundo'' en Fortran}
Segundo ``Hola Mundo'' en Fortran. Este fichero llama a una función
{\tt saluda()} que no está definida en él, sino en otro fichero.
\end{ejemplo}

\begin{ejemplo}%
{Saludos.for}%
{Segundo ``Hola Mundo'' en Fortran}
Segundo ``Hola  Mundo'' en Fortran.  Este fichero implementa  la función
{\tt saluda()} que es llamada desde el fichero {\tt HolaMundo2.for}.
\end{ejemplo}

Para generar ahora el ejecutable  utiliza el comando {\tt g77} dándole
ambos ficheros como argumentos. En  general, para compilar un programa
Fortran escrito en varios ficheros  bastará con utilizar el comando de
la forma {\tt g77 -o ejecutable fichero1.for ... ficheroN.for}

\begin{verbatim}
$ ls
HolaMundo2.for  Saludos.for

$ g77 -o HolaMundo2 HolaMundo2.for Saludos.for

$ ls
HolaMundo2  HolaMundo2.for  Saludos.for

$ ./HolaMundo2


  Hola Mundo

\end{verbatim}

Si lo prefieres puedes también  compilar los ficheros de código fuente
por  separado para  obtener los  ficheros  de código  objeto, y  luego
enlazarlos al final.  Esto resulta útil cuando  tienes muchos ficheros
de código fuente y sólo estás trabajando en uno de ellos, ya que puede
significar  un  ahorro de  tiempo  no  tener  que compilar  todos  los
ficheros cada  vez que quieras  compilar el programa. En  este ejemplo
los comandos serían los siguientes:

\begin{verbatim}
$ g77 -c HolaMundo2.for 
$ g77 -c Saludos.for 
$ g77 -o HolaMundo2 HolaMundo2.o Saludos.o
\end{verbatim}

En  los dos  primeros  comandos,  la opción  {\tt  -c} del  compilador
le  indica  que sólo  debe  generar  el  código objeto,  sin  intentar
enlazarlo. En el último, el compilador está recibiendo los ficheros de
código objeto  (y la  opción para modificar  el nombre  del ejecutable
resultante) e interpreta que debe  enlazarlos. Es importante notar que
no podemos utilizar el comando {\tt ld} para enlazar código en Fortran
porque faltarían muchos símbolos  (mayormente funciones) que {\tt g77}
se encarga de poner pero {\tt ld} desconoce.

Si después de  haber compilado el programa de esta  forma modificas el
fichero {\tt Saludos.for}  y quieres recompilar el  programa, sólo has
de recompilar el fichero (o los  ficheros) que has modificado y volver
a enlazar los ficheros de código objeto: 

\begin{verbatim}
$ g77 -c Saludos.for 
$ g77 -o HolaMundo2 HolaMundo2.o Saludos.o
\end{verbatim}

En  este ejemplo  no  se nota  la  ventaja, pero  en  una práctica  de
programación  en la  que estás  aprovechando tres  ficheros de  código
fuente de  prácticas anteriores y  escribiendo otros tres  ficheros de
código fuente nuevos, según la máquina en la que trabajes puede que no
te  apetezca  tener  que  compilar  los seis  ficheros  cada  vez  que
modificas uno. Si a esto le sumas el uso de GNU Make (que estudiaremos
más adelante) el proceso de  compilación y recompilación resulta mucho
más cómodo.

\section{Mezclar Fortran con C}\label{fortran+c}

\index{Fortran!mezclado con C}

Con lo que has visto en este  tema sabes más que suficiente para hacer
prácticas de programación  en Fortran 77, pero ahora vamos  a rizar el
rizo un  poco. Si  sabes algo  (un poco)  de programación  en C  no te
resultará difícil entender  lo siguiente, en caso  contrario deja este
apartado para cuando hayas terminado con  los temas de GNU C/C++ y GNU
Make.

Fortran es un lenguaje de  cálculo numérico, e intentar programar algo
que  no sea  cálculo numérico  en Fortran  puede ser  un suplicio.  Un
ejemplo  sencillo  de esto  es  un  menú, que  permanezca  preguntando
opciones hasta  que el usuario  decida salir del programa.  Es posible
programar un  menú en Fortran,  pero no  resulta tan efectivo  como en
otros lenguajes como C o Pascal.

C es  un lenguaje de  propósito general,  lo que significa  que puedes
programar  en C  casi cualquier  cosa  que te  propongas. Sin  embargo
programar cálculo numérico  en C puede no ser tan  cómodo como hacerlo
en Fortran.  Entonces, unamos  ambos lenguajes  en un  mismo programa,
aprovechando lo mejor de cada uno.

Esta curiosa mezcla es posible con  los compiladores de GNU, porque de
hecho el compilador de GNU Fortran está basado en el de GNU C/C++. Sin
embargo hay una serie de consideraciones  que debes tener en cuenta, y
es conveniente  que sepas programar algo  en C porque hay  que mirarlo
desde el punto de vista del lenguaje C.

\begin{itemize}

\item {\bf  En  Fortran las funciones reciben siempre  punteros}. Esto
implica  que si  modificas  el valor  de un  parámetro  dentro de  una
función en Fortran,  ese cambio seguirá siendo  efectivo tras terminar
la función.

\item  {\bf En  Fortran  los vectores  son  también punteros}.  Cuando
declaras en Fortran  un vector puedes especificar el  rango de índices
que será válido  dentro del vector, por ejemplo  {\tt integer a(-2:8)}
define un  vector de enteros que  puede ser indexado desde  -2 hasta 8
ambos  inclusive, de  forma  que  {\tt a(-2)}  y  {\tt  a(8)} son  los
extremos del  vector. Esto  se maneja internamente  como un  puntero a
entero a  partir del cual hay  11 posiciones reservadas. Es  decir, su
equivalente en C sería {\tt int a[11]} y {\tt a(-2)} sería {\tt a[0]}.
Incluso programando sólo en Fortran es importante entender esto.

\item {\bf Las funciónes de Fortran se renombran en el código objeto}.
Esto es, si  has definido una función en Fortran  llamada {\tt spline}
para  llamarla desde  C debes  hacerlo  con el  nombre {\tt  spline\_}
(añadiendo  dos caracteres  de  subrayado). Además  debes pasarle  los
parámetros como punteros.

\item {\bf Las funciones llamadas desde Fortran también se renombran}.
Para llamar  a la función {\tt  menu} desde el código  Fortran hay que
definirla en C con el nombre {\tt menu\_}.

\item {\bf  Los ficheros  de código fuente  se compilan  sin enlazar}.
Para  compilar un  fichero  de  código fuente  de  Fortran utiliza  el
comando {\tt g77 -c fichero.for} y  para compilar un fichero de código
fuente de C utiliza el comando {\tt gcc -c fichero.for}.

\item {\bf Los  ficheros de código objeto se enlazan  con g77}. Cuando
hayas compilado todos los fuentes  utiliza el compilador {\tt g77} (no
el {\tt gcc}) para enlazarlo y producir el ejecutable.

\end{itemize}

Veamos un ejemplo sencillo de cómo usar esto. Los siguientes ficheros,
dos  de Fortran  y  uno de  C, implementan  un  sencillo algoritmo  de
ordenación de datos (selección directa). 

\begin{ejemplo}%
{ordena.for}%
{Ejemplo de programación conjunta Fortran y C}
Declara las variables, llama al menú escrito en C y finalmente muestra
el contenido del vector después de terminar el menú. Nótese que el 
vector {\tt a} está indexado desde 1 hasta {\tt n}.
\end{ejemplo}

\begin{ejemplo}%
{menu.c}%
{Ejemplo de programación conjunta Fortran y C}
Implementa el  menú que  permite operar  con el vector  y llamar  a la
función {\tt mysort} programada en Fortran. Además protege al programa
de entradas de  datos erróneas o maliciosas. Nótese qe  el vector {\tt
a} es de tamaño  {\tt *n}, pero sólo se utilizan  las {\tt m} primeras
posiciones, indexadas desde {\tt 0} hasta {\tt m - 1}.
\end{ejemplo}

\begin{ejemplo}%
{mysort.for}%
{Ejemplo de programación conjunta Fortran y C}
Implementa en Fortran  el método de ordenación  por selección directa.
Nótese que  el tamaño del  vector se recibe  como parámetro y  sólo se
utilizan  las {\tt  m} primeras  posiciones, indexadas  desde {\tt  1}
hasta {\tt m}.
\end{ejemplo}

Para compilar este programa hazlo fichero a fichero, generando primero
los ficheros de  código objeto utilizando {\tt g77} o  {\tt gcc} según
esté cada  fichero de código fuente  escrito en Fortran o  en C. Luego
enlazalos todos con {\tt g77}.

\begin{verbatim}
$ g77 -c ordena.for 
$ gcc -c menu.c 
$ g77 -c mysort.for 
$ g77 -o ordena ordena.o menu.o mysort.o
\end{verbatim}

Con la  ayuda de GNU  Make el proceso  de recompilado se  convierte en
algo casi automático  y bastante más cómodo. Para saber  más sobre GNU
Make ve a la página \pageref{make}.

