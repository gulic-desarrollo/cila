%Autor: miguev

\chapter{GNU Fortran 77}

El lenguaje  de programaci�n {\tt Fortran}  estaba orientado puramente
al  c�lculo   matem�tico,  y  sigue  siendo   un  lenguaje  importante
en  entornos  cient�ficos.  Su  nombre viene  de  {\bf  For}mula  {\bf
Tran}slator, ya que su mayor uso  era traducir las f�rmulas de c�lculo
matem�tico  al lenguaje  de  las m�quinas.  Desde  sus principios,  el
lenguaje Fortran  ha tenido una  sintaxis particular, adaptada  al uso
de  tarjetas  perforadas. En  la  actualidad,  Fortran se  utiliza  en
asignaturas de C�lculo en carreras t�cnicas como Matem�ticas, F�sica y
algunas ingenier�as.

Utilizaremos aqu� el  compilador {\bf GNU Fortran  77}, compatible con
la  mayor�a del  lenguaje b�sico  de Fortran  77, suficiente  para las
pr�cticas de  programaci�n en  Fortran. Veamos una  vez m�s  el t�pico
ejemplo de ``HolaMundo''. En el editor que m�s nos guste, escribiremos
el siguiente c�digo Fortran:

\begin{ejemplo}%
{HolaMundo.for}%
{HolaMundo en Fortran}
Programa b�sico de HolaMundo en Fortran 77, define un formato e imprime 
un mensaje us�ndolo. Tambi�n incluye una l�nea de comentario.
\end{ejemplo}

Para  compilar el  programa utilizamos  el comando  {\tt g77}  como si
fuera el {\tt gcc}. El compilador  {\tt g77} tambi�n produce la salida
por defecto en  un ejecutable llamado {\tt a.out},  opci�n que podemos
modificar con la opci�n {\tt -o nombredelejecutable}.

\begin{verbatim}
$ ls
HolaMundo.for

$ g77 HolaMundo.for

$ ls
a.out  HolaMundo.for

$ g77 -o HolaMundo HolaMundo.for

$ ls
HolaMundo  HolaMundo.for

$ ./HolaMundo


  Hola Mundo

\end{verbatim}

Veamos  ahora  tambi�n como  podemos  dividir  un programa  en  varios
ficheros  de c�digo  fuente. En  el editor  escribimos los  siguientes
c�digos fuente  Fortran y  los guardamos  como {\tt  HolaMundo2.for} y
{\tt Saludos.for} respectivamente.

\begin{verbatim}
* Ejemplo 2 de Fortran para CILA
* Fichero: HolaMundo2.for

      Program HolaMundo2
      character(10) saludo

      saludo = 'Mundo'

      call saluda (saludo)

      stop
      end
\end{verbatim}

\begin{verbatim}
* Ejemplo 2 de Fortran para CILA
* Fichero: saludos.for

      Subroutine saluda(mensaje)
      character(10) mensaje

      print 5, 'Hola ', mensaje

 5    format (//,2x,a,a,/)

      return
      end
\end{verbatim}

Para  generar ahora  el  ejecutable utilizamos  el  comando {\tt  g77}
d�ndole ambos ficheros  como argumentos. En general,  para compilar un
programa  Fortran  escrito en  varios  ficheros  bastar� con  utilizar
el  comando  de la  forma  {\tt  g77  -o ejecutable  fichero1.for  ...
ficheroN.for}

\begin{verbatim}
$ ls
HolaMundo2.for  Saludos.for

$ g77 -o HolaMundo2 HolaMundo2.for Saludos.for

$ ls
HolaMundo2  HolaMundo2.for  Saludos.for

$ ./HolaMundo2


  Hola Mundo

\end{verbatim}

Estos pocos  comandos ser�n suficientes  para compilar y  ejecutar las
pr�cticas de Fortran 77 que podamos  necesitar, pero no piensen que es
verdad eso  de que  {\em ``Real  programers do  Artificial Inteligence
programs in Fortran''}.


