%Autor: todos
  
\chapter{Recursos en internet}
\label{recursos}  

La mayoría  de los recursos que  podamos necesitar para usar  Linux se
encuentran  distribuidos por  Internet,  y una  enorme cantidad  están
disponibles en español.

\begin{itemize}

\item {\tt http://www.debian.org} -- El Proyecto Debian es una asociación
  de  personas  que  han  hecho  causa común  para  crear  un  sistema
  operativo (SO)  libre. Este  sistema operativo  que hemos  creado se
  llama Debian GNU/Linux, o simplemente Debian para acortar.

\item {\tt http://www.redhat.es} --  La  primera  distribución Linux  en
  reorientarse hacia los usuarios finales.

\item {\tt http://www.linux-mandrake.com/es/} -- Linux-Mandrake es
  un amigable  Sistema Ope\-ra\-ti\-vo  Linux. Es  muy fácil  de usar,
  tanto en el  hogar u oficina como en servidores.  Está disponible en
  forma gratuita en varios idiomas alrededor del mundo.

\item {\tt http://www.ututo.org} -- Un GNU/Linux Simple.

\item {\tt http://www.linux-es.com} -- Existen muchos lugares en
  Internet  dedicados  a  LINUX,  pero  la mayoría de ellos están en
  inglés. Estas páginas  pretenden  ser un  punto  de  partida  para
  aquellos  que necesitan encontrar información sobre este sistema y
  en  la  medida de lo posible se ha intentado que la mayoría de los
  enlaces y contenidos sean en castellano.

\item {\tt http://www.gnu.org/home.es.html} -- El  Projecto  GNU
  comenzó en  1984 para  desarrollar un sistema  operativo tipo Unix
  completo, el cual  es  software libre:  El  sistema  GNU. Variantes
  del  sistema GNU, utilizando Linux  como  kernel,  son  ampliamente
  usadas, y aunque frecuentemente llamadas ``Linux'', dichas variantes
  deberían referirse más exactamente como sistemas GNU/Linux.

\item {\tt http://es.tldp.org}  --  Proyecto LuCAS  -  La
  mayor biblioteca en español dedicada a GNU/LiNUX de todo el planeta

\item {\tt http://www.insflug.org}  --  En  el INSFLUG  se  coordina
  la traducción ``oficial'' de documentos  breves, como los COMOs  y
  PUFs o  Preguntas de Uso Frecuente, las FAQs  en inglés. Esperamos
  que la información que encuentre aquí le sea de utilidad.

\item {\tt http://www.gnu.org/software/emacs/emacs.html} --  Es un
  editor de  pantalla y  ambiente  para cómputo  de  tiempo real,
  extensible  y  personalizable.  Ofrece Lisp (finalmente integrado al
  editor)  para escribir  extensiones  y proporciona  una interfaz  al
  sistema de ventanas X.

\item {\tt http://www.vim.org} -- VIM es  una versión mejorada del
  editor VI, uno de los editores de texto estándar en los sistemas UNIX.
  VIM añade muchas de las  características que se  esperan en  un editor:
  Deshacer ilimitado, coloreado de sintaxis, GUI, y mucho más.

\item \label{putty}{\tt http://www.chiark.greenend.org.uk/~sgtatham/putty/} --
  Implementación libre de un cliente TELNET/SSH para sistemas operativos
  Microsoft® Windows®. Escrito y mantenido por Simon Tatham.

\item {\tt http://pinsa.escomposlinux.org/sromero/linux/} -- S.O.S. Linux.

\item  {\tt http://www.geocities.com/Athens/Temple/2269/}  -- Tutorial
de C/C++

% La URL no funciona
% \item {\tt http://www.fie.us.es/docencia/publi/JAVA/} -- Tutorial de Java

\item {\tt http://www.cervantex.org} -- Información LaTeX en español

% La URL no funciona
% \item {\tt ftp://ftp.cma.ulpgc.es/pub/software/TeX/tex/latex2e/doc/ldesc2e/mix
% /ldesc2e.pdf} -- Una descripción de LaTeX, por Tomás Bautista y cia.

\item {\tt http://www.lyx.org}  --  Página del proyecto LyX 

\item {\tt http://www.octave.org}  --  Página del proyecto Octave

\item {\tt http://www.gnuplot.org}  --  Página del proyecto Gnuplot

\item {\tt http://www.r-project.org} -- Página del proyecto R

\item {\tt http://yacas.sourceforge.net} -- Página del proyecto Yacas

% La URL no funciona
% \item {\tt http://www-rocq.inria.fr/scilab/}  --  Página del proyecto SciLab

\item {\tt http://www.lysator.liu.se/\~{}alla/dia/} --  Página del proyecto DIA

\item {\tt http://www.gimp.org} --  Página del proyecto GIMP

\item {\tt http://www.qcad.org} -- Página del proyecto QCad

\item {\tt http://www.linuxfocus.org/Castellano/January2002/article132.shtml} -- Tutorial 
de {\sf QCad} en Castellano.

\end{itemize}


